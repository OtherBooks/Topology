\setcounter{subsection}{4-1}
\subsection{The Integers and the Real Numbers}

\def\parta{If $x + y = x$, then $y = 0$.}
\def\partb{$0 \cdot x = 0$. [Hint: Compute $(x+0) \cdot x$.]}
\def\partc{$-0 = 0$.}
\def\partd{$-(-x) = x$.}
\def\parte{$x(-y) = -(xy) = (-x)y$.}
\def\partf{$(-1)x = -x$.}
\def\partg{$x(y-z) = xy - xz$.}
\def\parth{$-(x+y) = -x - y$; $-(x-y) = -x + y$.}
\def\parti{If $x \neq 0$ and $x \cdot y = x$, then $y = 1$.}
\def\partj{$x/x = 1$ if $x \neq 0$.}
\def\partk{$x/1 = x$.}
\def\partl{$x \neq 0$ and $y \neq 0 \imp xy \neq 0$.}
\def\partm{$(1/y)(1/z) = 1/(yz)$ if $y,z \neq 0$.}
\def\partn{$(x/y)(w/z) = (xw)/(yz)$ if $y,z \neq 0$.}
\def\parto{$(x/y) + (w/z) = (xz + wy)/(yz)$ if $y,z \neq 0$.}
\def\partp{$x \neq 0 \imp 1/x \neq 0$.}
\def\partq{$1/(w/z) = z/w$ if $w,z \neq 0$.}
\def\partr{$(x/y)/(w/z) = (xz)/(yw)$ if $y,w,z \neq 0$.}
\def\parts{$(ax)/y = a(x/y)$ if $y \neq 0$.}
\def\partt{$(-x)/y = x/(-y) = -(x/y)$ if $y \neq 0$.}
\def\xr{\frac{1}{x}}
\def\yr{\frac{1}{y}}
\def\zr{\frac{1}{z}}

\exercise{1}{
  Prove the following ``laws of algebra'' for $\reals$, using only axioms (1)-(5):

  \begin{multicols}{2}
    \eparts{
    \item \parta
    \item \partb
    \item \partc
    \item \partd
    \item \parte
    \item \partf
    \item \partg
    \item \parth
    \item \parti
    \item \partj
    \item \partk
    \item \partl
    \item \partm
    \item \partn
    \item \parto
    \item \partp
    \item \partq
    \item \partr
    \item \parts
    \item \partt
    }
  \end{multicols}
}
\sol{
  \dwhitman

  \begin{lem}\label{lem:intreal:eqadd}
    $x + y = x + z$ if and only if $y = z$.
  \end{lem}
  \qproof{
    $(\pmi)$ Clearly if $y = z$ then $x+y = x+z$ since the $+$ operation is a function.

    $(\imp)$ If $x+y = x+z$ then we have
    \ali{
      y &= y + 0 & \text{(by (3))} \\
      &= 0 + y & \text{(by (2))} \\
      &= (x + (-x)) + y & \text{(by (4))} \\
      &= (-x + x) + y & \text{(by (2))} \\
      &= -x + (x + y) & \text{(by (1))} \\
      &= -x + (x + z) & \text{(by what was just shown for $(\pmi)$)} \\
      &= (-x + x) + z & \text{(by (1))} \\
      &= (x + (-x)) + z & \text{(by (2))} \\
      &= 0 + z & \text{(by (4))} \\
      &= z + 0 & \text{(by (2))} \\
      &= z & \text{(by (3))}
    }
    as desired.
  }

  \begin{lem}\label{lem:intreal:eqmul}
    If $x \neq 0$ then $x \cdot y = x \cdot z$ if and only if $y=z$.
  \end{lem}
  \qproof{
    $(\pmi)$ Clearly if $y=z$ then $x \cdot y = x \cdot z$ since the $\cdot$ operation is a function.

    $(\imp)$ If $x \cdot y = x \cdot z$ then we have
    \ali{
      y &= y \cdot 1 & \text{(by (3))} \\
      &= 1 \cdot y & \text{(by (2))} \\
      &= \parens{x \cdot \xr} \cdot y & \text{(by (4), noting that $x \neq 0$)} \\
      &= \parens{\xr \cdot x} \cdot y  & \text{(by (2))} \\
      &= \xr \cdot \parens{x \cdot y} & \text{(by (1))} \\
      &= \xr \cdot \parens{x \cdot z} & \text{(by what was just shown for $(\pmi)$)} \\
      &= \parens{\xr \cdot x} \cdot z & \text{(by (1))} \\
      &= \parens{x \cdot \xr} \cdot z & \text{(by (2))} \\
      &= 1 \cdot z & \text{(by (4))} \\
      &= z \cdot 1 & \text{(by (2))} \\
      &= z & \text{(by (3))}
    }
    as desired.
  }

  \begin{lem}\label{lem:intreal:recom}
    $1/(yz) = 1/(zy)$ if $y,z \neq 0$.
  \end{lem}
  \qproof{
    We have $(zy) \cdot 1/(yz) = (yz) \cdot 1/(yz) = 1$ by (2) followed by (4) so that $1/(yz)$ is a reciprocal of $zy$.
    Since this reciprocal is unique, however, it must be that $1/(yz) = 1/(zy)$ as desired.
  }

  \mainprob

  (a) \parta
  \qproof{
    Clearly by (3) we have $x + 0 = x = x + y$ so that it has to be that $y = 0$ by Lemma~\ref{lem:intreal:eqadd}.
  }

  (b) \partb
  \qproof{
    We have
    \ali{
      x \cdot x + 0 \cdot x &= x \cdot x + x \cdot 0 & \text{(since $0 \cdot x = x \cdot 0$ by (2))} \\
      &= x \cdot (x + 0) & \text{(by (5))} \\
      &= x \cdot x \,. & \text{(since $x + 0 = x$ by (3))}
    }
    Thus it must be that $0 \cdot x = 0$ by part (a).
  }

  (c) \partc
  \qproof{
    By (4) we have $0 + (-0) = 0$ so that it has to be that $-0 = 0$ by part (a).
  }

  (d) \partd
  \qproof{
    We have
    \ali{
      -(-x) &= -(-x) + 0 & \text{(by (3))} \\
      &= -(-x) + (x + (-x)) & \text{(by (4))} \\
      &= -(-x) + ((-x) + x) & \text{(by (2))} \\
      &= (-(-x) + (-x)) + x & \text{(by (1))} \\
      &= ((-x) + (-(-x))) + x & \text{(by (2))} \\
      &= 0 + x & \text{(by (4))} \\
      &= x + 0 & \text{(by (2))} \\
      &= x & \text{(by (3))}
    }
    as desired.
  }

  (e) \parte
  \qproof{
    First we have
    \ali{
      x(-y) &= x(-y) + 0 & \text{(by (3))} \\
      &= x(-y) + (xy + (-(xy))& \text{(by (4))} \\
      &= (x(-y) + xy) + (-(xy)) & \text{(by (1))} \\
      &= x(-y + y) + (-(xy)) & \text{(by (5))} \\
      &= x(y + (-y)) + (-(xy)) & \text{(by (2))} \\
      &= x \cdot 0 + (-(xy)) & \text{(by (4))} \\
      &= 0 \cdot x + (-(xy)) & \text{(by (2))} \\
      &= 0 + (-(xy)) & \text{(by part(b))} \\
      &= -(xy) + 0 & \text{(by (2))} \\
      &= -(xy) \,. & \text{(by (3))} \\
    }
    We also have
    \ali{
      (-x)y &= y(-x) & \text{(by (2))} \\
      &= -(yx) & \text{(by what was just shown)} \\
      &= -(xy) & \text{(by (2))}
    }
    so that the result follows since equality is transitive.
  }

  (f) \partf
  \qproof{
    We have
    \ali{
      (-1)x &= -(1 \cdot x) & \text{(by part(e))} \\
      &= -(x \cdot 1) & \text{(by (2))} \\
      &= -x & \text{(since $x \cdot 1 = x$ by (3))}
    }
    as desired.
  }

  (g) \partg
  \qproof{
    We have
    \ali{
      x(y-z) &= x(y + (-z)) & \text{(by the definition of subtraction)} \\
      &= xy + x(-z) & \text{(by (5))} \\
      &= xy + (-(xz)) & \text{(by part(e))} \\
      &= xy - xz & \text{(by the definition of subtraction)}
    }
    as desired.
  }

  (h) \parth
  \qproof{
    We have
    \ali{
      -(x+y) &= (-1)(x+y) & \text{(by part (f))} \\
      &= (-1)x + (-1)y & \text{(by (5))} \\
      &= -x + (-y) & \text{(by part (f) twice)} \\
      &= -x - y & \text{(by the definition of subtraction)}
    }
    and
    \ali{
      -(x-y) &= -(x + (-y)) & \text{(by the definition of subtraction)} \\
      &= -x - (-y)) & \text{(by what was just shown)} \\
      &= -x + (-(-y)) & \text{(by the definition of subtraction)} \\
      &= -x + y & \text{(by part (d))}
    }
    as desired.
  }

  (i) \parti
  \qproof{
    By (3) we have $x \cdot 1 = x = x \cdot y$ so that it has to be that $y = 1$ by Lemma~\ref{lem:intreal:eqmul}, noting that this applies since $x \neq 0$.
  }

  (j) \partj
  \qproof{
    By the definition of division we have $x/x = x \cdot (1/x) = 1$ by (4) since $x \neq 0$ and $1/x$ is defined as the reciprocal (i.e. the multiplicative inverse) of $x$.
  }

  (k) \partk
  \qproof{
    First, we have by (4) that $1 \cdot (1/1) = 1$, where $1/1$ is the reciprocal of $1$.
    We also have that $1 \cdot (1/1) = (1/1) \cdot 1 = 1/1$ by (2) and (3).
    Therefore $1/1 = 1 \cdot (1/1) = 1$ so that $1$ is its own reciprocal.
    Then, by the definition of division, we have $x/1 = x \cdot (1/1) = x \cdot 1 = x$ by (3).
  }

  (l) \partl
  \qproof{
    Suppose that $x \neq 0$ and $y \neq 0$.
    Also suppose to the contrary that $xy = 0$.
    Since $y \neq 0$ it follows from (4) that $1/y$ exists.
    So, we have $(xy) \cdot (1/y) = 0 \cdot (1/y) = 0$ by part (b).
    We also have
    \ali{
      (xy) \cdot \yr &= x \parens{y \cdot \yr} & \text{(by (1))} \\
      &= x \cdot 1 & \text{(by (4))} \\
      &= x & \text{(by (3))}
    }
    so that $x = (xy) \cdot (1/y) = 0$, which is a contradiction since we supposed that $x \neq 0$.
    Hence it must be that $xy \neq 0$ as desired.
  }

  (m) \partm
  \qproof{
    We have
    \ali{
      (yz)\parens{\yr \cdot \zr} &= (yz) \parens{\zr \cdot \yr} & \text{(by (2))} \\
      &= \parens{(yz) \cdot \zr}\yr & \text{(by (1))} \\
      &= \parens{y \parens{z \cdot \zr}} \yr & \text{(by (1))} \\
      &= \parens{y \cdot 1} \yr & \text{(by (4))} \\
      &= y \cdot \yr & \text{(by (3))} \\
      &= 1 & \text{(by (4))}
    }
    so that $(1/y)(1/z)$ is a multiplicative inverse of $yz$.
    Since this inverse is \emph{unique} by (4), however, it has to be that $(1/y)(1/z) = 1/(yz)$ as desired.
  }

  \def\yzr{\frac{1}{yz}}
  (n) \partn
  \qproof{
    We have
    \ali{
      \frac{x}{y} \cdot \frac{w}{z} &= \parens{x \cdot \yr}\parens{w \cdot \zr} & \text{(by the definition of division)} \\
      &= \parens{x \cdot \yr} \parens{\zr \cdot w} & \text{(by (2))} \\
      &= \parens{\parens{x \cdot \yr} \zr} w  & \text{(by (1))} \\
      &= \parens{x \parens{\yr \cdot \zr}} w & \text{(by (1))} \\
      &= \parens{x \cdot \yzr} w & \text{(by part (m) since $y,z \neq 0$)} \\
      &= \parens{\yzr \cdot x} w & \text{(by (2))} \\
      &= \yzr (xw) & \text{(by (1))} \\
      &=(xw) \yzr & \text{(by (2))} \\
      &= \frac{xw}{yz} & \text{(by the definition of division)}
    }
    as desired.
  }

  (o) \parto
  \qproof{
    We have
    \ali{
      \frac{x}{y} + \frac{w}{z} &= \frac{x}{y} \cdot 1 + \frac{w}{z} \cdot 1 & \text{(by (3))} \\
      &= \frac{x}{y} \cdot \frac{z}{z} + \frac{w}{z} \cdot \frac{y}{y}  & \text{(by part (j))} \\
      &= \frac{xz}{yz} + \frac{wy}{zy} & \text{(by part(n))} \\
      &= (xz)\frac{1}{yz} + (wy)\frac{1}{zy} & \text{(by the definition of division)} \\
      &= (xz)\frac{1}{yz} + (wy)\frac{1}{yz} & \text{(by Lemma~\ref{lem:intreal:recom})} \\
      &= \frac{1}{yz}(xz) + \frac{1}{yz}(wy) & \text{(by (2))} \\
      &= \frac{1}{yz}\parens{xz + wy} & \text{(by (5))} \\
      &= \parens{xz+ wy} \frac{1}{yz} & \text{(by (2))} \\
      &= \frac{xz + wy}{yz} & \text{(by the definition of division)}
    }
    as desired.
  }

  (p) \partp
  \qproof{
    Suppose that $x \neq 0$ but $1/x = 0$.
    Then we first have that $x \cdot (1/x) = x \cdot 0 = 0 \cdot x = 0$ by (2) and part (b).
    However, we also have $x \cdot (1/x) = 1$ by (4).
    Hence we have $0 = x \cdot (1/x) = 1$, which is a contradiction since we know that $0$ and $1$ are distinct by (3).
    So, if we accept that $x \neq 0$, then it must be that $1/x \neq 0$ also.
  }

  (q) \partq
  \qproof{
    We have
    \ali{
      \frac{w}{z} \cdot \frac{z}{w} &= \frac{wz}{zw} & \text{(by part (n) since $w,z \neq 0$)} \\
      &= (wz) \frac{1}{zw} & \text{(by the definition of division)} \\
      &= (wz) \frac{1}{wz} & \text{(by Lemma~\ref{lem:intreal:recom} since $w,z \neq 0$)} \\
      &= 1 & \text{(by (4))}
    }
    so that by definition $z/w$ is the reciprocal of $w/z$.
    Since this is unique by (4) we then have $z/w = 1/(w/z)$ as desired.
  }

  (r) \partr
  \qproof{
    We have
    \ali{
      \frac{x/y}{w/z} &= \frac{x}{y} \cdot \frac{1}{w/z} & \text{(by the definition of division)} \\
      &= \frac{x}{y} \cdot \frac{z}{w} & \text{(by part (q) since $w,z \neq 0$)} \\
      &= \frac{xz}{yw} & \text{(by part (n) since $y,w \neq 0$)}
    }
    as desired.
  }

  (s) \parts
  \qproof{
    We have
    \ali{
      \frac{ax}{y} &= (ax) \cdot \yr & \text{(by the definition of division)} \\
      &= a \parens{x \cdot \yr} & \text{(by (1))} \\
      &= a \cdot \frac{x}{y} & \text{(by the definition of division)}
    }
    as desired.
  }

  (t) \partt
  \qproof{
    We have
    \ali{
      \frac{-x}{y} &= (-x) \cdot \yr & \text{(by the definition of division)} \\
      &= ((-1)x) \cdot \yr & \text{(by part (f))} \\
      &= (-1) \parens{x \cdot \yr} & \text{(by (1))} \\
      &= (-1) \frac{x}{y} & \text{(by the definition of division)} \\
      &= -\parens{\frac{x}{y}} \,. & \text{(by part (f))}
    }
    Now, we have $(-1)(-1) = -(-1) = 1$ by parts (f) and (d) so that $-1$ is its own reciprocal, since the reciprocal is unique, i.e. $1/(-1) = -1$.
    We also have
    \ali{
      \frac{-x}{y} &= (-x) \cdot \yr & \text{(by the definition of division)} \\
      &= ((-1)x) \cdot \yr & \text{(by part (f))} \\
      &= (x(-1)) \cdot \yr & \text{(by (2))} \\
      &= x \parens{(-1) \yr} & \text{(by (1))} \\
      &= x \parens{\frac{1}{-1} \cdot \yr} & \text{(by what was just shown above)} \\
      &= x \frac{1}{(-1)y} & \text{(part (m) since $y \neq 0$)} \\
      &= x\frac{1}{-y} & \text{(by part (f))}
    }
    so that $-(x/y) = (-x)/y = x/(-y)$ as desired.
  }
}

\def\parta{$x > y$ and $w > z \imp x+w > y+z$.}
\def\partb{$x > 0$ and $y > 0 \imp x+y>0$ and $x \cdot y > 0$.}
\def\partc{$x > 0 \bic -x < 0$.}
\def\partd{$x > y \bic -x < -y$.}
\def\parte{$x > y$ and $z < 0 \imp xz < yz$.}
\def\partf{$x \neq 0 \imp x^2 > 0$, where $x^2 = x \cdot x$.}
\def\partg{$-1 < 0 < 1$.}
\def\parth{$xy > 0 \bic x$ and $y$ are both positive or both negative.}
\def\parti{$x > 0 \imp 1/x > 0$.}
\def\partj{$x > y > 0 \imp 1/x < 1/y$.}
\def\partk{$x < y \imp x < (x+y)/2 < y$.}

\newpage % This is just because the exercise label was on a separate page from the text, which was annoying
\exercise{2}{
  Prove the following ``laws of inequalities'' for $\reals$, using axioms (1)-(6) along with the results of Exercise~1:

  \begin{multicols}{2}
    \eparts{
    \item \parta
    \item \partb
    \item \partc
    \item \partd
    \item \parte
    \item \partf
    \item \partg
    \item \parth
    \item \parti
    \item \partj
    \item \partk
    }
  \end{multicols}
}
\sol{
  \dwhitman

  \begin{lem}\label{lem:intreal:xpx}
    $x + x = 2x$ for any real $x$.
  \end{lem}
  \qproof{
    We simply have
    \ali{
      x+x &= x \cdot 1 + x \cdot 1 & \text{(by (3))} \\
      &= x(1+1) & \text{(by (5))} \\
      &= x \cdot 2 & \text{(since $2$ is defined as $1+1$)} \\
      &= 2x & \text{(by (2))}
    }
    as desired.
  }

  \mainprob

  (a) \parta
  \qproof{
    We have
    \ali{
      x+w &> y+w & \text{(by (6) since $x>y$)} \\
      &= w + y & \text{(by (2))} \\
      &> z + y & \text{(by (6) since $w>z$)} \\
      &= y + z & \text{(by (2))}
    }
    as desired.
  }

  (b) \partb
  \qproof{
    First we have
    \ali{
      x+y &> 0 + y & \text{(by (6) since $x>0$)} \\
      &= y+0 & \text{(by (2))} \\
      &= y & \text{(by (3))} \\
      &> 0 \,.
    }
    Also
    \ali{
      x \cdot y &> 0 \cdot y & \text{(by (6) since $x>0$ and $y>0$)} \\
      &= 0 & \text{(by Exercise~4.1b)}
    }
    as desired.
  }

  (c) \partc
  \qproof{
    $(\imp)$ Suppose that $x>0$.
    Then we have
    \ali{
      -x &= -x + 0 & \text{(by (3))} \\
      &= 0 + (-x) & \text{(by (2))} \\
      &< x + (-x) & \text{(by (6) since $0 < x$)} \\
      &= 0 \,. & \text{(by (4))}
    }

    $(\pmi)$ Suppose now that $-x < 0$.
    Then we have
    \ali{
      x &= x + 0 & \text{(by (3))} \\
      &= 0 + x & \text{(by (2))} \\
      &> -x + x & \text{(by (6) since $0 > -x$)} \\
      &= x + (-x) & \text{(by (2))} \\
      &= 0 & \text{(by (4))}
    }
    as desired.
  }

  (d) \partd
  \qproof{
    $(\imp)$ Suppose that $x > y$.
    Then we have
    \ali{
      -y &= -y + 0 & \text{(by (3))} \\
      &= -y + (x + (-x)) & \text{(by (4))} \\
      &= (x + (-x)) + (-y) & \text{(by (2))} \\
      &= x + (-x + (-y)) & \text{(by (1))} \\
      &> y + (-x + (-y)) & \text{(by (6) since $x > y$)} \\
      &= y + (-y + (-x)) & \text{(by (2))} \\
      &= (y + (-y)) + (-x) & \text{(by (1))} \\
      &= 0 + (-x) & \text{(by (4))} \\
      &= -x + 0 & \text{(by (2))} \\
      &= -x \,. & \text{(by (3))}
    }

    $(\pmi)$ Now suppose that $-x < -y$.
    Then we have
    \ali{
      x &= x + 0 & \text{(by (3))} \\
      &= x + (y + (-y)) & \text{(by (4))} \\
      &= (y + (-y)) + x & \text{(by (2))} \\
      &= (-y + y) + x & \text{(by (2))} \\
      &= -y + (y + x) & \text{(by (1))} \\
      &> -x + (y + x) & \text{(by (6) since $-y > -x$)} \\
      &= -x + (x + y) & \text{(by (2))} \\
      &= (-x + x) + y & \text{(by (1))} \\
      &= (x + (-x)) + y & \text{(by (2))} \\
      &= 0 + y & \text{(by (4))} \\
      &= y + 0 & \text{(by (2))} \\
      &= y & \text{(by (3))}
    }
    as desired.
  }

  (e) \parte
  \qproof{
    First, by Exercise~4.1d, we have $-(-z) = z < 0$ so that $-z > 0$ by part (c).
    Then, since $x > y$, it follows from (6) that
    \ali{
      x(-z) &> y(-z) \\
      -(xz) &> -(yz) & \text{(by Exercise~4.1e applied to both sides)} \\
      xz &< yz & \text{(by part (d))} 
    }
    as desired.
  }

  (f) \partf
  \qproof{
    Since $x \neq 0$ we either have that $x> 0$ or $x < 0$ since the $<$ relation is an order (in particular a linear order since this is part of the definition of order in this text).
    If $x > 0$ then we have $x^2 = x \cdot x > 0 \cdot x = 0$ by (6) (since $x > 0$) and Exercise~4.1b.
    If $x < 0$ then we have $0 = 0 \cdot x < x \cdot x = x^2$ by part (e) (since $0 > x$) and Exercise~4.1b.
    Together these show the desired result.
  }

  (g) \partg
  \qproof{
    By (4) we know that $1 \neq 0$ so that $1^2 > 0$ by part (f).
    However, we have $1^2 = 1 \cdot 1 = 1$ by (3).
    Hence $1 = 1^2 > 0$.
    It then follows from part (c) that $-1 < 0$ so that we have $-1 < 0 < 1$ as desired.
  }

  (h) \parth
  \qproof{
    $(\imp)$ Suppose that $xy > 0$.
    It cannot be that $x = 0$, for then we would have $0 = 0 \cdot y = xy > 0$ by Exercise~4.1b, which is impossible by the definition of an order.
    Hence we have $x \neq 0$, and an analogous argument shows that $y \neq 0$ as well.
    We then have the following:

    Case: $x > 0$.
    Suppose that $y < 0$.
    Then, by part (e) and Exercise~4.1b, we have $xy < 0 \cdot y = 0$ since $x > 0$ and $y < 0$, which contradicts our initial supposition.
    Thus, since we know that $y \neq 0$, it has to be that $y > 0$ as well.

    Case: $x < 0$.
    Suppose that $y > 0$.
    Then, by (6) and Exercise~4.1b, we have $0 = 0 \cdot y > xy$ since $0 > x$ and $y > 0$, which again contradicts the initial supposition.
    So it must be that $y < 0$ also since $y \neq 0$.

    Therefore in every case either both $x$ and $y$ are positive or they are both negative.
    Since $x \neq 0$, these cases are exhaustive so that this shows the result.

    $(\pmi)$ Suppose that either $x > 0, y > 0$ or $x < 0, y < 0$.
    In the case where both $x>0$  and $y>0$ we clearly have $xy > 0 \cdot y = 0$ by (6) and Exercise~4.1b.
    In the other case in which $x<0$ and $y<0$ we have $0 = 0 \cdot y < xy$ by part (e) and Exercise~4.1b since $0 > x$ and $y < 0$.
    Hence $xy > 0$ in both cases.
  }

  (i) \parti
  \qproof{
    First, it cannot be that $1/x=0$ because then we would have $1 = x(1/x) = x \cdot 0 = 0 \cdot x = 0$ by (4), (2), and Exercise~4.1b.
    This is clearly a contradiction since we know that $1 \neq 0$ by (3).
    Hence $1/x \neq 0$.
    Now suppose that $1/x<0$ so that $1 = x(1/x) < 0 \cdot (1/x) = 0$ by part (e) since $x>0$ and $1/x<0$, and we have also used Exercise~4.1b.
    This is also a contradiction since it was proved in part (g) that $1>0$.
    Hence the only remaining possibility is that $1/x>0$ as desired.
  }

  (j) \partj
  \qproof{
    First, since the order is transitive, we have $x,y > 0$.
    It then follows from part (i) that $1/x,1/y > 0$.
    Then $(1/x)(1/y) > 0$ by part (h).
    We then have
    \ali{
      \xr &= \xr \cdot 1 & \text{(by (3))} \\
      &= \xr \parens{y \cdot \yr} & \text{(by (4))} \\
      &= \parens{\xr \cdot y} \yr & \text{(by (1))} \\
      &= \parens{y \cdot \xr} \yr & \text{(by (2))} \\
      &= y \parens{\xr \cdot \yr} & \text{(by (1))} \\
      &< x \parens{\xr \cdot \yr} & \text{(by (6) since $y<x$ and $(1/x)(1/y)>0$)} \\
      &= \parens{x \cdot \xr} \yr & \text{(by (1))} \\
      &= 1 \cdot \yr & \text{(by (4))} \\
      &= \yr \cdot 1 & \text{(by (2))} \\
      &= \yr & \text{(by (3))}
    }
    as desired.
  }

  (k) \partk
  \qproof{
    First, we know by part (g) that $1 > 0$ so that
    \ali{
      2 &= 1 + 1 & \text{(by the definition of 2)} \\
      &> 0 + 1 & \text{(by (6) since $1>0$ )} \\
      &= 1 + 0 & \text{(by (2))} \\
      &= 1 & \text{(by (3))} \\
      &> 0 \,. & \text{(by part (g))}
    }
    To summarize, $0 < 1 < 2$.
    It then follows from part (i) that $1/2 > 0$.
    We then have
    \ali{
      x &< y \\
      x + x &< x + y & \text{(by (6))} \\
      2x &< x + y & \text{(by Lemma~\ref{lem:intreal:xpx})} \\
      (2x)\frac{1}{2} &< (x+y)\frac{1}{2} & \text{(by (6) since $1/2>0$)} \\
      (x \cdot 2) \frac{1}{2} &< \frac{x+y}{2} & \text{(by (2) and the definition of division)} \\
      x \parens{2 \cdot \frac{1}{2}} &< \frac{x+y}{2} & \text{(by (1))} \\\
      x \cdot 1 &< \frac{x+y}{2} & \text{(by (4))} \\
      x &< \frac{x+y}{2}\,. & \text{(by (3))}
    }
    Similarly, we have
    \ali{
      x &< y \\
      x + y &< y + y & \text{(by (6))} \\
      x + y &< 2y & \text{(by Lemma~\ref{lem:intreal:xpx})} \\
      (x+y)\frac{1}{2} &< (2y)\frac{1}{2} & \text{(by (6) since $1/2>0$)} \\
      \frac{x+y}{2} &< (y \cdot 2) \frac{1}{2} & \text{(by the definition of division and (2))} \\
      \frac{x+y}{2} &< y \parens{2 \cdot \frac{1}{2}} & \text{(by (1))} \\
      \frac{x+y}{2} &< y \cdot 1 & \text{(by (4))} \\
      \frac{x+y}{2} &< y \,. & \text{(by (3))}
    }
    This shows that $x < (x+y)/2 < y$ as desired.
  }
}

\def\inds{\mathcal{A}}
\def\intA{\bigcap_{A \in \inds} A}
\def\intB{\bigcap_{B \in \inds} B}
\exercise{3}{
  \eparts{
  \item Show that if $\inds$ is a collection of inductive sets, then the intersection of the elements of $\inds$ is an inductive set.
  \item Prove the basic properties (1) and (2) of $\pints$.
  }
}
\sol{
  \dwhitman

  (a) We must show that $\intA$ is inductive.
  \qproof{
    First, consider any $A \in \inds$.
    Then, since $A$ is inductive, $1 \in A$.
    Since $A$ was arbitrary, this shows that $1 \in \intA$.
    Now suppose that $x \in \intA$ and again consider arbitrary $A \in \inds$.
    Then $x \in A$ so that $x+1 \in A$ also since $A$ is inductive.
    Since $A$ was arbitrary, this shows that $x+1 \in \intA$.
    Hence by definition $\intA$ is inductive.
  }

  (b)
  \qproof{
    Let $\inds$ be the collection of all inductive sets of $\reals$ so that by definition $\pints = \intA$.
    It then follows immediately from part (a) that $\pints$ is inductive since $\inds$ is a collection of inductive sets.
    This shows property (1).

    Now suppose that $A$ is an inductive set of positive integers.
    That is, $A$ is inductive and $A \ss \pints$.
    Consider any $x \in \pints = \intB$, where again $\inds$ is the the collection of all inductive subsets of $\reals$.
    Clearly we have that $A \ss \pints \ss \reals$ so that $A \in \inds$ since $A$ is an inductive subset of $\reals$.
    Hence $x \in A$ (since $x \in \intB$ and $A \in \inds$) so that $\pints \ss A$ since $x$ was arbitrary.
    This shows that $A = \pints$ as desired since also $A \ss \pints$.
    This shows property (2).
  }
}

\exercise{4}{
  \eparts{
  \item Prove by induction that given $n \in \pints$, every nonempty subset of $\braces{1,\ldots,n}$ has a largest element.
  \item Explain why you cannot conclude from (a) that every nonempty subset of $\pints$ has a largest element.
  }
}
\sol{
  \dwhitman

  (a)
  \qproof{
    Let $A$ be the set of integers such that the hypothesis is true.
    Clearly the result is then shown if we can prove that $A = \pints$.
    So first, clearly $1 \in A$ since the set $\braces{1}$ has only a single nonempty subset, i.e. $\braces{1}$ itself, in which 1 is clearly the largest element.
    Now suppose that $n \in A$ so that every nonempty subset of $S_{n+1} = \braces{1,\ldots,n}$ has a largest element.
    Consider any nonempty subset $B$ of $S_{n+2} = \braces{1,\ldots,n+1}$, noting that $S_{n+2} = S_{n+1} \cup \braces{n+1}$.

    Case: $n+1 \in B$.
    Then, for any other $k \in B$, $k \in S_{n+2}$ so that either $k = n+1$ or $k \in S_{n+1}$ so that $k < n+1$ by the definition of $S_{n+1}$.
    Thus in either case $k \leq n+1$ so that $n+1$ is the greatest element of $B$ since $k$ was arbitrary.

    Case: $n+1 \notin B$.
    Then clearly $B \ss S_{n+1}$ so that $B$ has a greatest element by the induction hypothesis since $B$ is nonempty.

    Hence in either case $B$ has a greatest element so that $n+1 \in A$ since $B$ was an arbitrary nonempty subset of $S_{n+2} = \braces{1, \ldots, n+1}$.
    This shows that $A$ is an inductive set of positive integers so that $A = \pints$ as desired by the Principle of Induction.
  }

  (b) There could be nonempty subsets of $\pints$ that are \emph{not} subsets of $S_{n+1} = \braces{1, \ldots, n}$ for any $n \in \pints$, in which cases the hypothesis of part (a) is not satisfied so that the conclusion does not necessarily apply.
  In fact, $\pints$ itself is an example of such a set.
}
