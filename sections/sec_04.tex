\setcounter{subsection}{4-1}
\subsection{The Integers and the Real Numbers}

% Macros used in this section
\def\multcom{since multiplication is commutative}
\def\multass{since multiplication is associative}

\def\parta{If $x + y = x$, then $y = 0$.}
\def\partb{$0 \cdot x = 0$. [Hint: Compute $(x+0) \cdot x$.]}
\def\partc{$-0 = 0$.}
\def\partd{$-(-x) = x$.}
\def\parte{$x(-y) = -(xy) = (-x)y$.}
\def\partf{$(-1)x = -x$.}
\def\partg{$x(y-z) = xy - xz$.}
\def\parth{$-(x+y) = -x - y$; $-(x-y) = -x + y$.}
\def\parti{If $x \neq 0$ and $x \cdot y = x$, then $y = 1$.}
\def\partj{$x/x = 1$ if $x \neq 0$.}
\def\partk{$x/1 = x$.}
\def\partl{$x \neq 0$ and $y \neq 0 \imp xy \neq 0$.}
\def\partm{$(1/y)(1/z) = 1/(yz)$ if $y,z \neq 0$.}
\def\partn{$(x/y)(w/z) = (xw)/(yz)$ if $y,z \neq 0$.}
\def\parto{$(x/y) + (w/z) = (xz + wy)/(yz)$ if $y,z \neq 0$.}
\def\partp{$x \neq 0 \imp 1/x \neq 0$.}
\def\partq{$1/(w/z) = z/w$ if $w,z \neq 0$.}
\def\partr{$(x/y)/(w/z) = (xz)/(yw)$ if $y,w,z \neq 0$.}
\def\parts{$(ax)/y = a(x/y)$ if $y \neq 0$.}
\def\partt{$(-x)/y = x/(-y) = -(x/y)$ if $y \neq 0$.}
\def\xr{\frac{1}{x}}
\def\yr{\frac{1}{y}}
\def\zr{\frac{1}{z}}

\exercise{1}{
  Prove the following ``laws of algebra'' for $\reals$, using only axioms (1)-(5):

  \begin{multicols}{2}
    \eparts{
    \item \parta
    \item \partb
    \item \partc
    \item \partd
    \item \parte
    \item \partf
    \item \partg
    \item \parth
    \item \parti
    \item \partj
    \item \partk
    \item \partl
    \item \partm
    \item \partn
    \item \parto
    \item \partp
    \item \partq
    \item \partr
    \item \parts
    \item \partt
    }
  \end{multicols}
}
\sol{
  \dwhitman

  \begin{lem}\label{lem:intreal:eqadd}
    $x + y = x + z$ if and only if $y = z$.
  \end{lem}
  \qproof{
    $(\pmi)$ Clearly if $y = z$ then $x+y = x+z$ since the $+$ operation is a function.

    $(\imp)$ If $x+y = x+z$ then we have
    \ali{
      y &= y + 0 & \text{(by (3))} \\
      &= 0 + y & \text{(by (2))} \\
      &= (x + (-x)) + y & \text{(by (4))} \\
      &= (-x + x) + y & \text{(by (2))} \\
      &= -x + (x + y) & \text{(by (1))} \\
      &= -x + (x + z) & \text{(by what was just shown for $(\pmi)$)} \\
      &= (-x + x) + z & \text{(by (1))} \\
      &= (x + (-x)) + z & \text{(by (2))} \\
      &= 0 + z & \text{(by (4))} \\
      &= z + 0 & \text{(by (2))} \\
      &= z & \text{(by (3))}
    }
    as desired.
  }

  \begin{lem}\label{lem:intreal:eqmul}
    If $x \neq 0$ then $x \cdot y = x \cdot z$ if and only if $y=z$.
  \end{lem}
  \qproof{
    $(\pmi)$ Clearly if $y=z$ then $x \cdot y = x \cdot z$ since the $\cdot$ operation is a function.

    $(\imp)$ If $x \cdot y = x \cdot z$ then we have
    \ali{
      y &= y \cdot 1 & \text{(by (3))} \\
      &= 1 \cdot y & \text{(by (2))} \\
      &= \parens{x \cdot \xr} \cdot y & \text{(by (4), noting that $x \neq 0$)} \\
      &= \parens{\xr \cdot x} \cdot y  & \text{(by (2))} \\
      &= \xr \cdot \parens{x \cdot y} & \text{(by (1))} \\
      &= \xr \cdot \parens{x \cdot z} & \text{(by what was just shown for $(\pmi)$)} \\
      &= \parens{\xr \cdot x} \cdot z & \text{(by (1))} \\
      &= \parens{x \cdot \xr} \cdot z & \text{(by (2))} \\
      &= 1 \cdot z & \text{(by (4))} \\
      &= z \cdot 1 & \text{(by (2))} \\
      &= z & \text{(by (3))}
    }
    as desired.
  }

  \begin{lem}\label{lem:intreal:recom}
    $1/(yz) = 1/(zy)$ if $y,z \neq 0$.
  \end{lem}
  \qproof{
    We have $(zy) \cdot 1/(yz) = (yz) \cdot 1/(yz) = 1$ by (2) followed by (4) so that $1/(yz)$ is a reciprocal of $zy$.
    Since this reciprocal is unique, however, it must be that $1/(yz) = 1/(zy)$ as desired.
  }

  \mainprob

  (a) \parta
  \qproof{
    Clearly by (3) we have $x + 0 = x = x + y$ so that it has to be that $y = 0$ by Lemma~\ref{lem:intreal:eqadd}.
  }

  (b) \partb
  \qproof{
    We have
    \ali{
      x \cdot x + 0 \cdot x &= x \cdot x + x \cdot 0 & \text{(since $0 \cdot x = x \cdot 0$ by (2))} \\
      &= x \cdot (x + 0) & \text{(by (5))} \\
      &= x \cdot x \,. & \text{(since $x + 0 = x$ by (3))}
    }
    Thus it must be that $0 \cdot x = 0$ by part~(a).
  }

  (c) \partc
  \qproof{
    By (4) we have $0 + (-0) = 0$ so that it has to be that $-0 = 0$ by part~(a).
  }

  (d) \partd
  \qproof{
    We have
    \ali{
      -(-x) &= -(-x) + 0 & \text{(by (3))} \\
      &= -(-x) + (x + (-x)) & \text{(by (4))} \\
      &= -(-x) + ((-x) + x) & \text{(by (2))} \\
      &= (-(-x) + (-x)) + x & \text{(by (1))} \\
      &= ((-x) + (-(-x))) + x & \text{(by (2))} \\
      &= 0 + x & \text{(by (4))} \\
      &= x + 0 & \text{(by (2))} \\
      &= x & \text{(by (3))}
    }
    as desired.
  }

  (e) \parte
  \qproof{
    First we have
    \ali{
      x(-y) &= x(-y) + 0 & \text{(by (3))} \\
      &= x(-y) + (xy + (-(xy))& \text{(by (4))} \\
      &= (x(-y) + xy) + (-(xy)) & \text{(by (1))} \\
      &= x(-y + y) + (-(xy)) & \text{(by (5))} \\
      &= x(y + (-y)) + (-(xy)) & \text{(by (2))} \\
      &= x \cdot 0 + (-(xy)) & \text{(by (4))} \\
      &= 0 \cdot x + (-(xy)) & \text{(by (2))} \\
      &= 0 + (-(xy)) & \text{(by part(b))} \\
      &= -(xy) + 0 & \text{(by (2))} \\
      &= -(xy) \,. & \text{(by (3))} \\
    }
    We also have
    \ali{
      (-x)y &= y(-x) & \text{(by (2))} \\
      &= -(yx) & \text{(by what was just shown)} \\
      &= -(xy) & \text{(by (2))}
    }
    so that the result follows since equality is transitive.
  }

  (f) \partf
  \qproof{
    We have
    \ali{
      (-1)x &= -(1 \cdot x) & \text{(by part(e))} \\
      &= -(x \cdot 1) & \text{(by (2))} \\
      &= -x & \text{(since $x \cdot 1 = x$ by (3))}
    }
    as desired.
  }

  (g) \partg
  \qproof{
    We have
    \ali{
      x(y-z) &= x(y + (-z)) & \text{(by the definition of subtraction)} \\
      &= xy + x(-z) & \text{(by (5))} \\
      &= xy + (-(xz)) & \text{(by part(e))} \\
      &= xy - xz & \text{(by the definition of subtraction)}
    }
    as desired.
  }

  (h) \parth
  \qproof{
    We have
    \ali{
      -(x+y) &= (-1)(x+y) & \text{(by part~(f))} \\
      &= (-1)x + (-1)y & \text{(by (5))} \\
      &= -x + (-y) & \text{(by part~(f) twice)} \\
      &= -x - y & \text{(by the definition of subtraction)}
    }
    and
    \ali{
      -(x-y) &= -(x + (-y)) & \text{(by the definition of subtraction)} \\
      &= -x - (-y)) & \text{(by what was just shown)} \\
      &= -x + (-(-y)) & \text{(by the definition of subtraction)} \\
      &= -x + y & \text{(by part~(d))}
    }
    as desired.
  }

  (i) \parti
  \qproof{
    By (3) we have $x \cdot 1 = x = x \cdot y$ so that it has to be that $y = 1$ by Lemma~\ref{lem:intreal:eqmul}, noting that this applies since $x \neq 0$.
  }

  (j) \partj
  \qproof{
    By the definition of division we have $x/x = x \cdot (1/x) = 1$ by (4) since $x \neq 0$ and $1/x$ is defined as the reciprocal (i.e. the multiplicative inverse) of $x$.
  }

  (k) \partk
  \qproof{
    First, we have by (4) that $1 \cdot (1/1) = 1$, where $1/1$ is the reciprocal of $1$.
    We also have that $1 \cdot (1/1) = (1/1) \cdot 1 = 1/1$ by (2) and (3).
    Therefore $1/1 = 1 \cdot (1/1) = 1$ so that $1$ is its own reciprocal.
    Then, by the definition of division, we have $x/1 = x \cdot (1/1) = x \cdot 1 = x$ by (3).
  }

  (l) \partl
  \qproof{
    Suppose that $x \neq 0$ and $y \neq 0$.
    Also suppose to the contrary that $xy = 0$.
    Since $y \neq 0$ it follows from (4) that $1/y$ exists.
    So, we have $(xy) \cdot (1/y) = 0 \cdot (1/y) = 0$ by part~(b).
    We also have
    \ali{
      (xy) \cdot \yr &= x \parens{y \cdot \yr} & \text{(by (1))} \\
      &= x \cdot 1 & \text{(by (4))} \\
      &= x & \text{(by (3))}
    }
    so that $x = (xy) \cdot (1/y) = 0$, which is a contradiction since we supposed that $x \neq 0$.
    Hence it must be that $xy \neq 0$ as desired.
  }

  (m) \partm
  \qproof{
    We have
    \ali{
      (yz)\parens{\yr \cdot \zr} &= (yz) \parens{\zr \cdot \yr} & \text{(by (2))} \\
      &= \parens{(yz) \cdot \zr}\yr & \text{(by (1))} \\
      &= \parens{y \parens{z \cdot \zr}} \yr & \text{(by (1))} \\
      &= \parens{y \cdot 1} \yr & \text{(by (4))} \\
      &= y \cdot \yr & \text{(by (3))} \\
      &= 1 & \text{(by (4))}
    }
    so that $(1/y)(1/z)$ is a multiplicative inverse of $yz$.
    Since this inverse is \emph{unique} by (4), however, it has to be that $(1/y)(1/z) = 1/(yz)$ as desired.
  }

  \def\yzr{\frac{1}{yz}}
  (n) \partn
  \qproof{
    We have
    \ali{
      \frac{x}{y} \cdot \frac{w}{z} &= \parens{x \cdot \yr}\parens{w \cdot \zr} & \text{(by the definition of division)} \\
      &= \parens{x \cdot \yr} \parens{\zr \cdot w} & \text{(by (2))} \\
      &= \parens{\parens{x \cdot \yr} \zr} w  & \text{(by (1))} \\
      &= \parens{x \parens{\yr \cdot \zr}} w & \text{(by (1))} \\
      &= \parens{x \cdot \yzr} w & \text{(by part~(m) since $y,z \neq 0$)} \\
      &= \parens{\yzr \cdot x} w & \text{(by (2))} \\
      &= \yzr (xw) & \text{(by (1))} \\
      &=(xw) \yzr & \text{(by (2))} \\
      &= \frac{xw}{yz} & \text{(by the definition of division)}
    }
    as desired.
  }

  (o) \parto
  \qproof{
    We have
    \ali{
      \frac{x}{y} + \frac{w}{z} &= \frac{x}{y} \cdot 1 + \frac{w}{z} \cdot 1 & \text{(by (3))} \\
      &= \frac{x}{y} \cdot \frac{z}{z} + \frac{w}{z} \cdot \frac{y}{y}  & \text{(by part~(j))} \\
      &= \frac{xz}{yz} + \frac{wy}{zy} & \text{(by part(n))} \\
      &= (xz)\frac{1}{yz} + (wy)\frac{1}{zy} & \text{(by the definition of division)} \\
      &= (xz)\frac{1}{yz} + (wy)\frac{1}{yz} & \text{(by Lemma~\ref{lem:intreal:recom})} \\
      &= \frac{1}{yz}(xz) + \frac{1}{yz}(wy) & \text{(by (2))} \\
      &= \frac{1}{yz}\parens{xz + wy} & \text{(by (5))} \\
      &= \parens{xz+ wy} \frac{1}{yz} & \text{(by (2))} \\
      &= \frac{xz + wy}{yz} & \text{(by the definition of division)}
    }
    as desired.
  }

  (p) \partp
  \qproof{
    Suppose that $x \neq 0$ but $1/x = 0$.
    Then we first have that $x \cdot (1/x) = x \cdot 0 = 0 \cdot x = 0$ by (2) and part~(b).
    However, we also have $x \cdot (1/x) = 1$ by (4).
    Hence we have $0 = x \cdot (1/x) = 1$, which is a contradiction since we know that $0$ and $1$ are distinct by (3).
    So, if we accept that $x \neq 0$, then it must be that $1/x \neq 0$ also.
  }

  (q) \partq
  \qproof{
    We have
    \ali{
      \frac{w}{z} \cdot \frac{z}{w} &= \frac{wz}{zw} & \text{(by part~(n) since $w,z \neq 0$)} \\
      &= (wz) \frac{1}{zw} & \text{(by the definition of division)} \\
      &= (wz) \frac{1}{wz} & \text{(by Lemma~\ref{lem:intreal:recom} since $w,z \neq 0$)} \\
      &= 1 & \text{(by (4))}
    }
    so that by definition $z/w$ is the reciprocal of $w/z$.
    Since this is unique by (4) we then have $z/w = 1/(w/z)$ as desired.
  }

  (r) \partr
  \qproof{
    We have
    \ali{
      \frac{x/y}{w/z} &= \frac{x}{y} \cdot \frac{1}{w/z} & \text{(by the definition of division)} \\
      &= \frac{x}{y} \cdot \frac{z}{w} & \text{(by part~(q) since $w,z \neq 0$)} \\
      &= \frac{xz}{yw} & \text{(by part~(n) since $y,w \neq 0$)}
    }
    as desired.
  }

  (s) \parts
  \qproof{
    We have
    \ali{
      \frac{ax}{y} &= (ax) \cdot \yr & \text{(by the definition of division)} \\
      &= a \parens{x \cdot \yr} & \text{(by (1))} \\
      &= a \cdot \frac{x}{y} & \text{(by the definition of division)}
    }
    as desired.
  }

  (t) \partt
  \qproof{
    We have
    \ali{
      \frac{-x}{y} &= (-x) \cdot \yr & \text{(by the definition of division)} \\
      &= ((-1)x) \cdot \yr & \text{(by part~(f))} \\
      &= (-1) \parens{x \cdot \yr} & \text{(by (1))} \\
      &= (-1) \frac{x}{y} & \text{(by the definition of division)} \\
      &= -\parens{\frac{x}{y}} \,. & \text{(by part~(f))}
    }
    Now, we have $(-1)(-1) = -(-1) = 1$ by parts (f) and (d) so that $-1$ is its own reciprocal, since the reciprocal is unique, i.e. $1/(-1) = -1$.
    We also have
    \ali{
      \frac{-x}{y} &= (-x) \cdot \yr & \text{(by the definition of division)} \\
      &= ((-1)x) \cdot \yr & \text{(by part~(f))} \\
      &= (x(-1)) \cdot \yr & \text{(by (2))} \\
      &= x \parens{(-1) \yr} & \text{(by (1))} \\
      &= x \parens{\frac{1}{-1} \cdot \yr} & \text{(by what was just shown above)} \\
      &= x \frac{1}{(-1)y} & \text{(part~(m) since $y \neq 0$)} \\
      &= x\frac{1}{-y} & \text{(by part~(f))}
    }
    so that $-(x/y) = (-x)/y = x/(-y)$ as desired.
  }
}

\def\parta{$x > y$ and $w > z \imp x+w > y+z$.}
\def\partb{$x > 0$ and $y > 0 \imp x+y>0$ and $x \cdot y > 0$.}
\def\partc{$x > 0 \bic -x < 0$.}
\def\partd{$x > y \bic -x < -y$.}
\def\parte{$x > y$ and $z < 0 \imp xz < yz$.}
\def\partf{$x \neq 0 \imp x^2 > 0$, where $x^2 = x \cdot x$.}
\def\partg{$-1 < 0 < 1$.}
\def\parth{$xy > 0 \bic x$ and $y$ are both positive or both negative.}
\def\parti{$x > 0 \imp 1/x > 0$.}
\def\partj{$x > y > 0 \imp 1/x < 1/y$.}
\def\partk{$x < y \imp x < (x+y)/2 < y$.}

\exercise{2}{
  Prove the following ``laws of inequalities'' for $\reals$, using axioms (1)-(6) along with the results of Exercise~1:

  \begin{multicols}{2}
    \eparts{
    \item \parta
    \item \partb
    \item \partc
    \item \partd
    \item \parte
    \item \partf
    \item \partg
    \item \parth
    \item \parti
    \item \partj
    \item \partk
    }
  \end{multicols}
}
\sol{
  \dwhitman

  \begin{lem}\label{lem:intreal:xpx}
    $x + x = 2x$ for any real $x$.
  \end{lem}
  \qproof{
    We simply have
    \ali{
      x+x &= x \cdot 1 + x \cdot 1 & \text{(by (3))} \\
      &= x(1+1) & \text{(by (5))} \\
      &= x \cdot 2 & \text{(since $2$ is defined as $1+1$)} \\
      &= 2x & \text{(by (2))}
    }
    as desired.
  }

  \mainprob

  (a) \parta
  \qproof{
    We have
    \ali{
      x+w &> y+w & \text{(by (6) since $x>y$)} \\
      &= w + y & \text{(by (2))} \\
      &> z + y & \text{(by (6) since $w>z$)} \\
      &= y + z & \text{(by (2))}
    }
    as desired.
  }

  (b) \partb
  \qproof{
    First we have
    \ali{
      x+y &> 0 + y & \text{(by (6) since $x>0$)} \\
      &= y+0 & \text{(by (2))} \\
      &= y & \text{(by (3))} \\
      &> 0 \,.
    }
    Also
    \ali{
      x \cdot y &> 0 \cdot y & \text{(by (6) since $x>0$ and $y>0$)} \\
      &= 0 & \text{(by Exercise~4.1b)}
    }
    as desired.
  }

  (c) \partc
  \qproof{
    $(\imp)$ Suppose that $x>0$.
    Then we have
    \ali{
      -x &= -x + 0 & \text{(by (3))} \\
      &= 0 + (-x) & \text{(by (2))} \\
      &< x + (-x) & \text{(by (6) since $0 < x$)} \\
      &= 0 \,. & \text{(by (4))}
    }

    $(\pmi)$ Suppose now that $-x < 0$.
    Then we have
    \ali{
      x &= x + 0 & \text{(by (3))} \\
      &= 0 + x & \text{(by (2))} \\
      &> -x + x & \text{(by (6) since $0 > -x$)} \\
      &= x + (-x) & \text{(by (2))} \\
      &= 0 & \text{(by (4))}
    }
    as desired.
  }

  (d) \partd
  \qproof{
    $(\imp)$ Suppose that $x > y$.
    Then we have
    \ali{
      -y &= -y + 0 & \text{(by (3))} \\
      &= -y + (x + (-x)) & \text{(by (4))} \\
      &= (x + (-x)) + (-y) & \text{(by (2))} \\
      &= x + (-x + (-y)) & \text{(by (1))} \\
      &> y + (-x + (-y)) & \text{(by (6) since $x > y$)} \\
      &= y + (-y + (-x)) & \text{(by (2))} \\
      &= (y + (-y)) + (-x) & \text{(by (1))} \\
      &= 0 + (-x) & \text{(by (4))} \\
      &= -x + 0 & \text{(by (2))} \\
      &= -x \,. & \text{(by (3))}
    }

    $(\pmi)$ Now suppose that $-x < -y$.
    Then we have
    \ali{
      x &= x + 0 & \text{(by (3))} \\
      &= x + (y + (-y)) & \text{(by (4))} \\
      &= (y + (-y)) + x & \text{(by (2))} \\
      &= (-y + y) + x & \text{(by (2))} \\
      &= -y + (y + x) & \text{(by (1))} \\
      &> -x + (y + x) & \text{(by (6) since $-y > -x$)} \\
      &= -x + (x + y) & \text{(by (2))} \\
      &= (-x + x) + y & \text{(by (1))} \\
      &= (x + (-x)) + y & \text{(by (2))} \\
      &= 0 + y & \text{(by (4))} \\
      &= y + 0 & \text{(by (2))} \\
      &= y & \text{(by (3))}
    }
    as desired.
  }

  (e) \parte
  \qproof{
    First, by Exercise~4.1d, we have $-(-z) = z < 0$ so that $-z > 0$ by part~(c).
    Then, since $x > y$, it follows from (6) that
    \ali{
      x(-z) &> y(-z) \\
      -(xz) &> -(yz) & \text{(by Exercise~4.1e applied to both sides)} \\
      xz &< yz & \text{(by part~(d))} 
    }
    as desired.
  }

  (f) \partf
  \qproof{
    Since $x \neq 0$ we either have that $x> 0$ or $x < 0$ since the $<$ relation is an order (in particular a linear order since this is part of the definition of order in this text).
    If $x > 0$ then we have $x^2 = x \cdot x > 0 \cdot x = 0$ by (6) (since $x > 0$) and Exercise~4.1b.
    If $x < 0$ then we have $0 = 0 \cdot x < x \cdot x = x^2$ by part~(e) (since $0 > x$) and Exercise~4.1b.
    Together these show the desired result.
  }

  (g) \partg
  \qproof{
    By (4) we know that $1 \neq 0$ so that $1^2 > 0$ by part~(f).
    However, we have $1^2 = 1 \cdot 1 = 1$ by (3).
    Hence $1 = 1^2 > 0$.
    It then follows from part~(c) that $-1 < 0$ so that we have $-1 < 0 < 1$ as desired.
  }

  (h) \parth
  \qproof{
    $(\imp)$ Suppose that $xy > 0$.
    It cannot be that $x = 0$, for then we would have $0 = 0 \cdot y = xy > 0$ by Exercise~4.1b, which is impossible by the definition of an order.
    Hence we have $x \neq 0$, and an analogous argument shows that $y \neq 0$ as well.
    We then have the following:

    Case: $x > 0$.
    Suppose that $y < 0$.
    Then, by part~(e) and Exercise~4.1b, we have $xy < 0 \cdot y = 0$ since $x > 0$ and $y < 0$, which contradicts our initial supposition.
    Thus, since we know that $y \neq 0$, it has to be that $y > 0$ as well.

    Case: $x < 0$.
    Suppose that $y > 0$.
    Then, by (6) and Exercise~4.1b, we have $0 = 0 \cdot y > xy$ since $0 > x$ and $y > 0$, which again contradicts the initial supposition.
    So it must be that $y < 0$ also since $y \neq 0$.

    Therefore in every case either both $x$ and $y$ are positive or they are both negative.
    Since $x \neq 0$, these cases are exhaustive so that this shows the result.

    $(\pmi)$ Suppose that either $x > 0, y > 0$ or $x < 0, y < 0$.
    In the case where both $x>0$  and $y>0$ we clearly have $xy > 0 \cdot y = 0$ by (6) and Exercise~4.1b.
    In the other case in which $x<0$ and $y<0$ we have $0 = 0 \cdot y < xy$ by part~(e) and Exercise~4.1b since $0 > x$ and $y < 0$.
    Hence $xy > 0$ in both cases.
  }

  (i) \parti
  \qproof{
    First, it cannot be that $1/x=0$ because then we would have $1 = x(1/x) = x \cdot 0 = 0 \cdot x = 0$ by (4), (2), and Exercise~4.1b.
    This is clearly a contradiction since we know that $1 \neq 0$ by (3).
    Hence $1/x \neq 0$.
    Now suppose that $1/x<0$ so that $1 = x(1/x) < 0 \cdot (1/x) = 0$ by part~(e) since $x>0$ and $1/x<0$, and we have also used Exercise~4.1b.
    This is also a contradiction since it was proved in part~(g) that $1>0$.
    Hence the only remaining possibility is that $1/x>0$ as desired.
  }

  (j) \partj
  \qproof{
    First, since the order is transitive, we have $x,y > 0$.
    It then follows from part~(i) that $1/x,1/y > 0$.
    Then $(1/x)(1/y) > 0$ by part~(h).
    We then have
    \ali{
      \xr &= \xr \cdot 1 & \text{(by (3))} \\
      &= \xr \parens{y \cdot \yr} & \text{(by (4))} \\
      &= \parens{\xr \cdot y} \yr & \text{(by (1))} \\
      &= \parens{y \cdot \xr} \yr & \text{(by (2))} \\
      &= y \parens{\xr \cdot \yr} & \text{(by (1))} \\
      &< x \parens{\xr \cdot \yr} & \text{(by (6) since $y<x$ and $(1/x)(1/y)>0$)} \\
      &= \parens{x \cdot \xr} \yr & \text{(by (1))} \\
      &= 1 \cdot \yr & \text{(by (4))} \\
      &= \yr \cdot 1 & \text{(by (2))} \\
      &= \yr & \text{(by (3))}
    }
    as desired.
  }

  (k) \partk
  \qproof{
    First, we know by part~(g) that $1 > 0$ so that
    \ali{
      2 &= 1 + 1 & \text{(by the definition of 2)} \\
      &> 0 + 1 & \text{(by (6) since $1>0$ )} \\
      &= 1 + 0 & \text{(by (2))} \\
      &= 1 & \text{(by (3))} \\
      &> 0 \,. & \text{(by part~(g))}
    }
    To summarize, $0 < 1 < 2$.
    It then follows from part~(i) that $1/2 > 0$.
    We then have
    \ali{
      x &< y \\
      x + x &< x + y & \text{(by (6))} \\
      2x &< x + y & \text{(by Lemma~\ref{lem:intreal:xpx})} \\
      (2x)\frac{1}{2} &< (x+y)\frac{1}{2} & \text{(by (6) since $1/2>0$)} \\
      (x \cdot 2) \frac{1}{2} &< \frac{x+y}{2} & \text{(by (2) and the definition of division)} \\
      x \parens{2 \cdot \frac{1}{2}} &< \frac{x+y}{2} & \text{(by (1))} \\\
      x \cdot 1 &< \frac{x+y}{2} & \text{(by (4))} \\
      x &< \frac{x+y}{2}\,. & \text{(by (3))}
    }
    Similarly, we have
    \ali{
      x &< y \\
      x + y &< y + y & \text{(by (6))} \\
      x + y &< 2y & \text{(by Lemma~\ref{lem:intreal:xpx})} \\
      (x+y)\frac{1}{2} &< (2y)\frac{1}{2} & \text{(by (6) since $1/2>0$)} \\
      \frac{x+y}{2} &< (y \cdot 2) \frac{1}{2} & \text{(by the definition of division and (2))} \\
      \frac{x+y}{2} &< y \parens{2 \cdot \frac{1}{2}} & \text{(by (1))} \\
      \frac{x+y}{2} &< y \cdot 1 & \text{(by (4))} \\
      \frac{x+y}{2} &< y \,. & \text{(by (3))}
    }
    This shows that $x < (x+y)/2 < y$ as desired.
  }
}

\def\inds{\mathcal{A}}
\def\intA{\bigcap_{A \in \inds} A}
\def\intB{\bigcap_{B \in \inds} B}
\exercise{3}{
  \eparts{
  \item Show that if $\inds$ is a collection of inductive sets, then the intersection of the elements of $\inds$ is an inductive set.
  \item Prove the basic properties (1) and (2) of $\pints$.
  }
}
\sol{
  \dwhitman

  (a) We must show that $\intA$ is inductive.
  \qproof{
    First, consider any $A \in \inds$.
    Then, since $A$ is inductive, $1 \in A$.
    Since $A$ was arbitrary, this shows that $1 \in \intA$.
    Now suppose that $x \in \intA$ and again consider arbitrary $A \in \inds$.
    Then $x \in A$ so that $x+1 \in A$ also since $A$ is inductive.
    Since $A$ was arbitrary, this shows that $x+1 \in \intA$.
    Hence by definition $\intA$ is inductive.
  }

  (b)
  \qproof{
    Let $\inds$ be the collection of all inductive sets of $\reals$ so that by definition $\pints = \intA$.
    It then follows immediately from part~(a) that $\pints$ is inductive since $\inds$ is a collection of inductive sets.
    This shows property (1).

    Now suppose that $A$ is an inductive set of positive integers.
    That is, $A$ is inductive and $A \ss \pints$.
    Consider any $x \in \pints = \intB$, where again $\inds$ is the the collection of all inductive subsets of $\reals$.
    Clearly we have that $A \ss \pints \ss \reals$ so that $A \in \inds$ since $A$ is an inductive subset of $\reals$.
    Hence $x \in A$ (since $x \in \intB$ and $A \in \inds$) so that $\pints \ss A$ since $x$ was arbitrary.
    This shows that $A = \pints$ as desired since also $A \ss \pints$.
    This shows property (2).
  }
}

\exercise{4}{
  \eparts{
  \item Prove by induction that given $n \in \pints$, every nonempty subset of $\intsfin{n}$ has a largest element.
  \item Explain why you cannot conclude from (a) that every nonempty subset of $\pints$ has a largest element.
  }
}
\sol{
  \dwhitman

  (a)
  \qproof{
    Let $A$ be the set of integers such that the hypothesis is true.
    Clearly the result is then shown if we can prove that $A = \pints$.
    So first, clearly $1 \in A$ since the set $\braces{1}$ has only a single nonempty subset, i.e. $\braces{1}$ itself, in which 1 is clearly the largest element.
    Now suppose that $n \in A$ so that every nonempty subset of $S_{n+1} = \intsfin{n}$ has a largest element.
    Consider any nonempty subset $B$ of $S_{n+2} = \intsfin{n+1}$, noting that $S_{n+2} = S_{n+1} \cup \braces{n+1}$.

    Case: $n+1 \in B$.
    Then, for any other $k \in B$, $k \in S_{n+2}$ so that either $k = n+1$ or $k \in S_{n+1}$ so that $k < n+1$ by the definition of $S_{n+1}$.
    Thus in either case $k \leq n+1$ so that $n+1$ is the largest element of $B$ since $k$ was arbitrary.

    Case: $n+1 \notin B$.
    Then clearly $B \ss S_{n+1}$ so that $B$ has a largest element by the induction hypothesis since $B$ is nonempty.

    Hence in either case $B$ has a largest element so that $n+1 \in A$ since $B$ was an arbitrary nonempty subset of $S_{n+2} = \intsfin{n+1}$.
    This shows that $A$ is an inductive set of positive integers so that $A = \pints$ as desired by the Principle of Induction.
  }

  (b) There could be nonempty subsets of $\pints$ that are \emph{not} subsets of $S_{n+1} = \intsfin{n}$ for any $n \in \pints$, in which cases the hypothesis of part~(a) is not satisfied so that the conclusion does not necessarily apply.
  In fact, $\pints$ itself is an example of such a set where both the hypothesis and the conclusion are false.
}

\def\pintsz{\pints \cup \braces{0}}
\exercise{5}{
  Prove the following properties of $\ints$ and $\pints$:
  \eparts{
  \item $a,b \in \pints \imp a + b \in \pints$.
    [Hint: Show that given $a \in \pints$, the set $X = \braces{x \where x \in \reals \text{ and } a + x \in \pints}$ is inductive.]
  \item $a,b \in \pints \imp a \cdot b \in \pints$.
  \item Show that $a \in \pints \imp a-1 \in \pints \cup \braces{0}$.
    [Hint: Let $X = \braces{x \where x \in \reals \text{ and } x-1 \in \pints \cup \braces{0}}$; show that $X$ is inductive.]
  \item $c,d \in \ints \imp c+d \in \ints$ and $c-d \in \ints$.
    [Hint: Prove it first for $d=1$.]
  \item $c,d \in \ints \imp c \cdot d \in \ints$.
  }
}
\sol{
  \dwhitman

  \begin{lem}\label{lem:intreal:negz}
    If $x \in \ints$ then $-x \in \ints$.
  \end{lem}
  \qproof{
    Let $\nints = \braces{-x \where x \in \pints}$ so that by definition $\ints = \pintsz \cup \nints$.
    Suppose that $x \in \ints$ so that $x \in \pintsz \cup \nints$.

    Case: $x \in \pints$.
    Then $-x \in \nints$ by definition.

    Case: $x = 0$.
    Then by Exercise~4.1c we have $-x = -0 = 0 \in \braces{0}$.

    Case: $x \in \nints$.
    Then by definition there is a $y \in \pints$ such that $x = -y$.
    Then $-x = -(-y) = y \in \pints$ by Exercise~4.1d.

    Hence in all cases either $-x \in \pints$, $-x \in \braces{0}$, or $-x \in \nints$ so that $-x \in \pintsz \cup \nints = \ints$ as desired.
  }

  \mainprob
  
  (a)
  \qproof{
    Consider any $a \in \pints$ and define $X_a = \braces{x \in \reals \where a+x \in \pints}$.
    We show that $X_a$ is inductive.
    First, since $a \in \pints$ we have that $a+1 \in \pints$ since $\pints$ is inductive.
    Hence $1 \in X_a$ by definition.
    Now suppose that $x \in X_a$ so that $a+x \in \pints$.
    Then we have $a+(x+1) = (a+x)+1 \in \pints$ since $a+x \in \pints$ and $\pints$ is inductive.
    This shows by definition that $x+1 \in X_a$ and therefore that $X_a$ is inductive.
    It follows that $\pints \ss X_a$ since $\pints$ is defined as the intersection of all inductive subsets of reals, of which $X_a$ is one.

    Therefore, for any $a,b \in \pints$, we have that $b \in X_a$ since $\pints \ss X_a$.
    Thus by definition $a + b \in \pints$ as desired
  }

  (b)
  \qproof{
    Consider any $a \in \pints$ and define $X_a = \braces{x \in \reals \where a \cdot x \in \pints}$.
    We show that $X_a$ is inductive.
    To this end, we first have that $a \cdot 1 = a \in \pints$ so that $1 \in X_a$ by definition.
    Now suppose that $x \in X_a$ so that $ax \in \pints$.
    Then we have $a \cdot (x+1) = a \cdot x + a \cdot 1 = ax + a \in \pints$ by part~(a) since we know both $ax$ and $a$ are in $\pints$.
    Hence $x+1 \in X_a$ by definition.
    This shows that $X_a$ is inductive so that again $\pints \ss X_a$.

    Hence for any $a,b \in \pints$ we have that $b \in X_a$ since $\pints \ss X_a$.
    It then follows by definition that $a \cdot b \in \pints$ as desired.
  }

  (c)
  \qproof{
    Let $X = \braces{x \in \reals \where x-1 \in \pintsz}$, which we show is inductive.
    First, we have $1 - 1 = 1 + (-1) = 0$ so that clearly $1 \in \pintsz$ and hence $1 \in X$.
    Now suppose that $x \in X$ so that $x-1 \in \pintsz$.

    Case: $x-1 \in \braces{0}$.
    Then it must be that $x - 1 = 0$, which clearly implies $x = 1 \in \pints$ since $\pints$ is inductive.
    Then $(x+1) - 1 = x + (1 - 1) = x + 0 = x \in \pints$ so that $(x+1)-1 \in \pintsz$ and therefore $x+1 \in X$.

    Case: $x-1 \in \pints$.
    Then $(x+1) - 1 = x + (1 -1) = x + ((-1) + 1) = (x - 1) + 1 \in \pints$ since $x-1 \in \pints$ and $\pints$ is inductive.
    Thus clearly $(x+1)-1 \in \pintsz$ so that $x+1 \in X$ by definition.

    Hence in both cases $x+1 \in X$, which shows that $X$ is inductive, and so $\pints \ss X$.
    Therefore, for any $a \in \pints$, we have that also $x \in X$ since $\pints \ss X$.
    Then, by the definition of $X$, it follows that $a-1 \in \pintsz$ as desired.
  }

  (d)
  \qproof{
    First we show that the set $X_c = \braces{x \in \reals \where c+x \in \ints \text{ and } c-x \in \ints}$ is inductive for any $c \in \ints$.
    So consider any $c$ and $b$ in $\ints$ so that $c,b \in \pintsz \cup \nints$.

    Case: $b \in \pints$.
    Then $b+1 \in \pints$ since $\pints$ is inductive and $b-1 \in \pintsz$ by part~(c).

    Case: $b = 0$.
    Then $b + 1 = 0 + 1 = 1 \in \pints$ since it is inductive, and $b - 1 = 0 - 1 = -1 \in \nints$ since $1 \in \pints$.

    Case: $b \in \nints$.
    Then $b = -a$ for $a \in \pints$, and we then have that $a+1 \in \pints$ since $\pints$ is inductive.
    Hence $b - 1 = -a - 1 = -(a+1) \in \nints$.
    We also have that $a-1 \in \pintsz$ by part~(c), from which it is trivial to show that $-(a-1) \in \nints \cup \braces{0}$.
    Therefore $b + 1 = -a + 1 = -(a - 1) \in \nints \cup \braces{0}$.

    Thus in all cases we have that $b+1$ and $b-1$ are in $\pints$ or $\braces{0}$ or $\nints$ so that they are both in $\ints$, and so $1 \in X_b$.
    Note that this is the case for any $b \in \ints$ so that it is clearly true for $c$, i.e. $1 \in X_c$.
    Now suppose that $x \in X_c$ so that $c+x$ and $c-x$ are both in $\ints$.
    It then follows that $1 \in X_{c+x}$ and $1 \in X_{c-x}$ so that $c + (x+1) = (c+x) + 1 \in \ints$ and $c - (x + 1) = (c -x) - 1 \in \ints$.
    This then shows that $x+1 \in X_c$.
    Hence $X_c$ is inductive for any $c \in \ints$ so that $\pints \ss X_c$.

    Now consider $c,d \in \ints$.

    Case: $d \in \pints$.
    Then clearly $d \in X_c$ since $\pints \ss X_c$.
    Hence by definition $c+d$ and $c-d$ are both in $\ints$.

    Case: $d = 0$.
    Then $c + d = c + 0 = c \in \ints$ and $c - d = c - 0 = c \in \ints$.

    Case: $d \in \nints$.
    Then by definition $d = -a$ for $a \in \pints$ so that $a \in X_c$ since $\pints \ss X_c$.
    Then $c + a$ and $c - a$ are both in $\ints$ by the definition of $X_c$.
    Hence $c + d = c + (-a) = c - a \in \ints$ and $c - d = c - (-a) = c + a \in \ints$.

    Therefore we have shown that $c+d$ and $c-d$ are both integers in all cases, which is the desired result.
  }

  (e)
  \qproof{
    For any $c \in \ints$, define $X_c = \braces{x \in \reals \where c \cdot x \in \ints}$.
    We first show that $X_c$ is inductive for any such $c \in \ints$.
    We have $c \cdot 1 = c \in \ints$ so that $1 \in X_c$.
    Now suppose that $x \in X_c$ so that $c \cdot x \in \ints$.
    Then $c \cdot(x+1) = c \cdot x + c \cdot 1 = c \cdot x + c \in \ints$ by part~(d) since both $c \cdot x$ and $c$ are integers.
    This shows that $X_c$ is inductive so that $\pints \ss X_c$.
    
    Now consider any $c,d \in \ints$.

    Case: $d \in \pints$.
    Then $d \in X_c$ since $\pints \ss X_c$.
    Thus $c \cdot d \in \ints$.

    Case: $d = 0$.
    The $c \cdot d = c \cdot 0 = 0 \in \ints$.

    Case: $d \in \nints$.
    Then there is an $a \in \pints$ such that $d = -a$.
    Hence $a \in X_c$ since $\pints \ss X_c$, from which it follows that $c \cdot a \in \ints$.
    We then have $c \cdot d = c \cdot (-a) = -(c \cdot a) \in \ints$ as well by Lemma~\ref{lem:intreal:negz}.

    Thus in all cases $c \cdot d \in \ints$ as desired.
  }
}

\exercise{6}{
  Let $a \in \reals$.
  Define inductively
  \ali{
    a^1 &= a \,, \\
    a^{n+1} &= a^n \cdot a
  }
  for $n \in \pints$.
  (See \S 7 for a discussion of the process of inductive definition.)
  Show that for $n,m \in \pints$ and $a,b \in \reals$,
  \ali{
    a^n a^m &= a^{n+m} \\
    \parens{a^n}^m &= a^{nm} \\
    a^m b^m &= (ab)^m \,.
  }
  These are called the \textbf{\emph{laws of exponents}}.
  [Hint: For fixed $n$, prove the formulas by induction on $m$.]
}
\sol{
  \dwhitman

  The following Lemma is the familiar proof by induction, which is more straightforward than having to frame everything in terms of inductive sets.
  Henceforth we use this whenever induction is required.
  \begin{lem}
    (Proof by Induction)
    Suppose that $P(x)$ is a statement with parameter $x$.
    Suppose also that $P(1)$ is true and that $P(x)$ implies $P(x+1)$.
    Then $P(n)$ is true for all $n \in \pints$.
  \end{lem}
  \qproof{
    Define the set $X = \braces{x \in \reals \where P(x)}$.
    We show that $X$ is inductive.
    Clearly since $P(1)$ is true we have $1 \in X$.
    Now suppose that $x \in X$ so that $P(x)$ is true.
    Then $P(x+1)$ is also true so that $x+1 \in X$.
    This shows that $X$ is inductive so that $\pints \ss X$.
    So, for any positive integer $n$ we have that $n \in X$ since $\pints \ss X$.
    Therefore $P(n)$ is true.
    Since $n$ was arbitrary, this shows the desired result.
  }

  \mainprob

  In what follows, suppose that $a,b \in \reals$.

  First we show that $a^n a^m = a^{n+m}$ for all $n,m \in \pints$.
  \qproof{
    Fix $n \in \pints$.
    We show the result by induction on $m$.
    First, we clearly have $a^n a^1 = a^n \cdot a = a^{n+1}$ by the inductive definition.
    Now suppose that $a^n a^m = a^{n+m}$.
    Then
    \ali{
      a^n a^{m+1} &= a^n \cdot (a^m \cdot a) & \text{(by the inductive definition)} \\
      &= (a^n a^m) \cdot a & \text{(\multass)} \\
      &= a^{n+m} \cdot a & \text{(by the induction hypothesis)} \\
      &= a^{(n+m)+1} & \text{(by the inductive definition)} \\
      &= a^{n + (m+1)} \,, & \text{(since addition is associative)}
    }
    which completes the induction step.
    Therefore the result holds for all $m \in \pints$ by induction.
  }

  Next we show that $\parens{a^n}^m = a^{nm}$ for all $n,m \in \pints$.
  \qproof{
    We again fix $n \in \pints$ and use induction on $m$.
    First, we have $\parens{a^n}^1 = a^n = a^{n \cdot 1}$ by the inductive definition.
    Supposing now that $\parens{a^n}^m = a^{n \cdot m}$, we have
    \ali{
      \parens{a^n}^{m+1} &= \parens{a^n}^m \cdot a^n & \text{(by the inductive definition)} \\
      &= a^{n \cdot m} a^n & \text{(by the induction hypothesis)} \\
      &= a^{n \cdot m + n} & \text{(by what was shown above)} \\
      &= a^{n \cdot m + n \cdot 1} \\
      &= a^{n \cdot (m+1)} \,. & \text{(by the distributive property)}
    }
    This completes the induction so that the result holds for all $m \in \pints$.
  }

  Lastly, we show that $a^m b^m = (ab)^m$ for all $m \in \pints$.
  \qproof{
    We show this by induction on $m$.
    First, we have $a^1 b^1 = ab = (ab)^1$ by the inductive definition.
    Now suppose that $a^m b^m = (ab)^m$ so that
    \ali{
      a^{m+1} b^{m+1} &= (a^m \cdot a)(b^m \cdot b) & \text{(by the inductive definition)} \\
      &= (a \cdot a^m)(b^m \cdot b) & \text{(\multcom)} \\
      &= ((a \cdot a^m) b^m) \cdot b & \text{(\multass)} \\
      &= (a \cdot (a^m b^m)) \cdot b & \text{(\multass)} \\
      &= (a (ab)^m) \cdot b & \text{(by the induction hypothesis)} \\
      &= ((ab)^m a) \cdot b & \text{(\multcom)} \\
      &= (ab)^m (ab) & \text{(\multass)} \\
      &= (ab)^{m+1} \,. & \text{(by the inductive definition)}
    }
    This completes the induction.
  }
}

\exercise{7}{
  Let $a \in \reals$ and $a \neq 0$.
  Define $a^0 = 1$, and for $n \in \pints$, $a^{-n} = 1/a^n$.
  Show that the laws of exponents hold for $a,b \neq 0$ and $n,m \in \ints$.
}
\sol{
  \dwhitman

  \begin{lem}\label{lem:intreal:expone}
    For any $n \in \ints$, $1^n = 1$.
  \end{lem}
  \qproof{
    We show this for $n \in \pints$ by simple induction on $n$.
    First, clearly $1^1 = 1$ by the inductive definition of exponentiation.
    Next, if $1^n = 1$, then we have $1^{n+1} = 1^n \cdot 1 = 1^n = 1$ by the inductive definition of exponentiation and the inductive hypothesis.
    This completes the induction so that the result holds for all $n \in \pints$.

    Clearly if $n = 0$ then, by the definition of 0 as an exponent, $1^n = 1^0 = 1$.

    Lastly, if $n \in \nints$ then there is a $k \in \pints$ where $n = -k$.
    Then we have
    \ali{
      1^n &= 1^{-k} \\
      &= \frac{1}{1^k} & \text{(by the definition of negative exponentiation)} \\
      &= \frac{1}{1} & \text{(by what was just shown by induction since $k \in \pints$)} \\
      &= 1\,. & \text{(since 1 is its own reciprocal)}
    }
    Thus the result has been shown for all the resulting cases when $n \in \ints$.
  }
  
  \begin{lem}\label{lem:intreal:recpow}
    $1/a^n = (1/a)^n$ for any real $a \neq 0$ and $n \in \pints$.
  \end{lem}
  \qproof{
    We have
    \ali{
      \parens{\frac{1}{a}}^n a^n &= \parens{\frac{1}{a} \cdot a}^n & \text{(by Exercise~4.6 since $n \in \pints$)} \\
      &= 1^n & \text{(by the definition of the reciprocal)} \\
      &= 1\,. & \text{(by Lemma~\ref{lem:intreal:expone})}
    }
    Thus $(1/a)^n$ must be the unique reciprocal of $a^n$, that is $(1/a)^n = 1/a^n$ as desired.
  }

  \begin{lem}\label{lem:intreal:negexp}
    $a^n a^{-n} = 1$ for any real $a \neq 0$ and $n \in \pints$.
  \end{lem}
  \qproof{
    We have
    \ali{
      a^n a^{-n} &= a^n \parens{\frac{1}{a^n}} & \text{(by the definition of negative exponentiation)} \\
      &= a^n \parens{\frac{1}{a}}^n & \text{(by Lemma~\ref{lem:intreal:recpow})} \\
      &= \parens{a \cdot \frac{1}{a}}^n & \text{(by Exercise~4.6 since $n \in \pints$)} \\
      &= 1^n & \text{(by the definition of the reciprocal)} \\
      &= 1 & \text{(by Lemma~\ref{lem:intreal:expone})}
    }
    as desired.
  }

  \mainprob

  First we show that $a^n a^m = a^{n+m}$ for all real $a \neq 0$ and $n,m \in \ints$.
  \qproof{
    Consider any real $a \neq 0$ and $n,m \in \ints$.
    We number the following cases for reference:
    \begin{enumerate}
    \item Case: $n \in \pints$.
      \begin{enumerate}
      \item Case: $m \in \pints$.
        Then the result immediately applies by Exercise~4.6.
      \item Case: $m = 0$.
        Then we have $a^n a^m = a^n a^0 = a^n \cdot 1 = a^n = a^{n+0} = a^{n+m}$.
      \item Case: $m \in \nints$.
        Then $m = -k$ for some $k \in \pints$.
        \begin{enumerate}
        \item Case: $n > k$.
          Then $n-k > 0$ so that $n-k \in \pints$ since $n-k \in \ints$ by Exercise~4.5d.
          We then have
          \ali{
            a^n a^m &= a^n a^{-k} \\
            &= a^{n-k+k} a^{-k} & \text{(since $n = n + 0 = n -k + k$)} \\
            &= (a^{n-k} a^k) a^{-k} & \text{(by Exercise~4.6 since $k,n-k \in \pints$)} \\
            &= a^{n-k} (a^k a^{-k}) & \text{(\multass)} \\
            &= a^{n-k} \cdot 1 & \text{(by Lemma~\ref{lem:intreal:negexp} since $k \in \pints$)} \\
            &= a^{n-k} \\
            &= a^{n+m} \,.
          }
        \item Case: $n = k$.
          Then clearly $n+m = n-k = k-k = 0$, so that we have $a^n a^m = a^k a^{-k} = 1 = a^0 = a^{n+m}$ by Lemma~\ref{lem:intreal:negexp} and the definition of 0 as an exponent.
        \item Case: $n < k$.
          Then $n-k < 0$ so that $n-k \in \nints$ since $n-k \in \ints$ by Exercise~4.5d.
          Also, clearly $-n \in \nints$ since $n \in \pints$.
          Then we have
          \ali{
            a^n a^m &= a^n a^{-k} \\
            &= a^n a^{-k+n-n} & \text{(since $-k = -k + 0 = -k + n - n$)} \\
            &= a^n a^{n-k-n} & \text{(since addition is commutative)} \\
            &= a^n (a^{n-k} a^{-n}) & \text{(by case 3c below since $n-k,-n \in \nints$)} \\
            &= a^n (a^{-n} a^{n-k}) & \text{(\multcom)} \\
            &= (a^n a^{-n}) a^{n-k} & \text{(\multass)} \\
            &= 1 \cdot a^{n-k} & \text{(by Lemma~\ref{lem:intreal:negexp})} \\
            &= a^{n-k} \\
            &= a^{n+m} \,.
          }
        \end{enumerate}
      \end{enumerate}
    \item Case: $n = 0$.
      \begin{enumerate}
      \item Case: $m \in \pints$.
        Since $a^n a^m = a^m a^n$ and $a^{n+m} = a^{m+n}$, this the same as case 1b above.
      \item Case: $m = 0$.
        Then we have $a^n a^m = a^0 a^0 = 1 \cdot 1 = 1 = a^0 = a^{0+0} = a^{n+m}$.
      \item Case: $m \in \nints$.
        Then there is a $k \in \pints$ such that $m = -k$, and $a^n a^m = a^0 a^{-k} = 1 \cdot (1/a^k) =  1/a^k = a^{-k} = a^m = a^{0+m} = a^{n+m}$.
      \end{enumerate}
    \item Case: $n \in \nints$.
      \begin{enumerate}
      \item Case: $m \in \pints$.
        This is the same as case 1c above.
      \item Case: $m = 0$.
        This is the same as case 2c above.
      \item Case: $m \in \nints$.
        Here we have that $n = -k$ and $m = -l$ for some $k,l \in \pints$.
        Hence we have
        \ali{
          a^n a^m &= a^{-k} a^{-l} \\
          &= \parens{\frac{1}{a^k}}\parens{\frac{1}{a^l}} & \text{(by the definition of negative exponents)} \\
          &= \parens{\frac{1}{a}}^k \parens{\frac{1}{a}}^l & \text{(by Lemma~\ref{lem:intreal:recpow})} \\
          &= \parens{\frac{1}{a}}^{k+l} & \text{(by Exercise~4.6 since $k,l \in \pints$)} \\
          &= \frac{1}{a^{k+l}} & \text{(by Lemma~\ref{lem:intreal:recpow})} \\
          &= a^{-(k+l)} & \text{(by the definition of negative exponents)} \\
          &= a^{-k-l} \\
          &= a^{n+m} \,.
        }
      \end{enumerate}
    \end{enumerate}

    Thus in all cases we have shown the result.
  }

  Next we show that $(a^n)^m = a^{n m}$ for all real $a \neq 0$ and $n,m \in \ints$.
  \qproof{
    Consider any real $a \neq 0$ and $n,m \in \ints$.
    We again number the cases for reference:
    \begin{enumerate}
    \item Case: $n \in \pints$.
      \begin{enumerate}
      \item Case: $m \in \pints$.
        Then the result immediately applies by Exercise~4.6.
      \item Case: $m = 0$.
        Then we have $(a^n)^m = (a^n)^0 = 1 = a^0 = a^{n \cdot 0} = a^{nm}$ by the definition of a 0 exponent.
      \item Case: $m \in \nints$.
        Then there is a $k \in \pints$ such that $m = -k$.
        Then we have
        \ali{
          (a^n)^m &= (a^n)^{-k} \\
          &= \frac{1}{(a^n)^k} & \text{(by the definition of negative exponents)} \\
          &= \frac{1}{a^{nk}} & \text{(by Exercise~4.6 since $n,k \in \pints$)} \\
          &= a^{-(nk)} & \text{(by the definition of negative exponents)} \\
          &= a^{n(-k)} \\
          &= a^{nm} \,.
        }
      \end{enumerate}
    \item Case: $n = 0$.
      Then we have $(a^n)^m = (a^0)^m = 1^m = 1 = a^0 = a^{0 \cdot m} = a^{nm}$ by the definition of 0 as an exponent and Lemma~\ref{lem:intreal:expone}.
    \item Case: $n \in \nints$.
      Then $n = -k$ for some $k \in \pints$.
      \begin{enumerate}
      \item Case: $m \in \pints$.
        Then we have
        \ali{
          (a^n)^m &= (a^{-k})^m \\
          &= \parens{\frac{1}{a^k}}^m & \text{(by the definition of negative exponents)} \\
          &= \squares{\parens{\frac{1}{a}}^k}^m & \text{(by Lemma~\ref{lem:intreal:recpow})} \\
          &= \parens{\frac{1}{a}}^{km} & \text{(by Exercise~4.6 since $k,m \in \pints$)} \\
          &= \frac{1}{a^{km}} & \text{(by Lemma~\ref{lem:intreal:recpow})} \\
          &= a^{-(km)} & \text{(by the definition of negative exponents)} \\
          &= a^{(-k)m} \\
          &= a^{nm} \,.
        }
      \item Case: $m = 0$.
        The same argument as in case 1b above applies here as it does not depend on $n$ being positive.
      \item Case: $m \in \nints$.
        Then $m = -l$ for some $l \in \pints$, and we have
        \ali{
          (a^n)^m &= (a^{-k})^{-l} \\
          &= \parens{\frac{1}{a^k}}^{-l} & \text{(by the definition of negative exponents)} \\
          &= \frac{1}{(1/a^k)^l} & \text{(by the definition of negative exponents)} \\
          &= \frac{1}{[(1/a)^k]^l} & \text{(by Lemma~\ref{lem:intreal:recpow})} \\
          &= \frac{1}{(1/a)^{kl}} & \text{(by Exercise~4.6 since $k,l \in \pints$)} \\
          &= \parens{\frac{1}{1/a}}^{kl} & \text{(by Lemma~\ref{lem:intreal:recpow})} \\
          &= a^{kl} \\
          &= a^{(-k)(-l)} \\
          &= a^{nm} \,.
        }
      \end{enumerate}
    \end{enumerate}

    Thus in all cases we have shown the result.
  }

  Lastly, we show that $a^m b^m = (ab)^m$ for all real $a,b \neq 0$ and $m \in \ints$.
  \qproof{
    We have the following cases:

    Case: $m \in \pints$.
    The result then follows immediately from Exercise~4.6.

    Case: $m = 0$.
    Then we have $a^m b^m = a^0 b^0 = 1 \cdot 1 = 1 = (ab)^0 = (ab)^m$ by the definition of a 0 exponent.

    Case: $m \in \nints$.
    Then there is a $k \in \pints$ such that $m = -k$.
    Then we have
    \ali{
      a^m b^m &= a^{-k} b^{-k} \\
      &= \frac{1}{a^k} \cdot \frac{1}{b^k} & \text{(by the definition of negative exponents)} \\
      &= \parens{\frac{1}{a}}^k \parens{\frac{1}{b}}^k & \text{(by Lemma~\ref{lem:intreal:recpow})} \\
      &= \parens{\frac{1}{a} \cdot \frac{1}{b}}^k & \text{(by Exercise~4.6 since $k \in \pints$)} \\
      &= \parens{\frac{1}{ab}}^k & \text{(by Exercise~4.1m)} \\
      &= \frac{1}{(ab)^k} & \text{(by Lemma~\ref{lem:intreal:recpow})} \\
      &= (ab)^{-k} & \text{(by the definition of negative exponents)} \\
      &= (ab)^m \,.
    }
    Therefore in all cases the result has been shown.
  }
}

\exercise{8}{
  \eparts{
  \item Show that $\reals$ has the greatest lower bound property.
  \item Show that $\inf\braces{1/n \where n \in \pints} = 0$.
  \item Show that given $a$ with  $0 < a < 1$, $\inf\braces{a^n \where n \in \pints} = 0$.
    [Hint: Let $h=(1-a)/a$, and show that $(1+h)^n \geq 1 + nh$.]
  }
}
\sol{
  \dwhitman

  (a)
  \qproof{
    Suppose that $A$ is an arbitrary nonempty set of real number that is bounded below by $a$.
    Now let $B = \braces{-x \where x \in A}$ and $b = -a$.
    First, we claim that $b$ is an upper bound of $B$.
    So consider any $y \in B$ so that $y = -x$ for some $x \in A$.
    Then $a \leq x$ since $a$ a lower bound of $A$.
    It then follows from Exercise~4.2d that $y = -x \leq -a = b$.
    Since $y \in B$ was arbitrary, this shows that $b$ is an upper bound of $B$.

    Since $B$ is clearly nonempty (since $A$ is), we have that $B$ has a least upper bound $d = \sup B$ since the reals have the least upper bound property.
    We claim that $c = -d$ is the greatest lower bound of $A$.
    So first consider any $x \in A$ so that $y = -x \in B$.
    Then we have $y \leq d$ since $d = \sup B$.
    Hence $c = -d \leq -y = x$ again by Exercise~4.2d.
    Since $x \in A$ was arbitrary, this shows that $c$ is in fact a lower bound of $A$.

    Now suppose that $x$ is any lower bound of $A$.
    Then, by the same argument as above for $b = -a$, we have that $y = -x$ is an upper bound of $B$.
    It then follows that $d \leq y$ since $d$ is the \emph{least} upper bound of $B$.
    Then, again by Exercise~4.2d, we have $x = -(-x) = -y \leq -d = c$, which shows that $c$ is in fact the greatest lower bound since $x$ was arbitrary.
    This completes the proof.
  }

  (b)
  \qproof{
    First, let $A = \braces{1/n \where n \in \pints}$ so that we must show that $\inf{A} = 0$.
    For any $x \in A$ we have that $x = 1/n$ for some $n \in \pints$.
    Then $n > 0$ so that $x = 1/n > 0$ also by Exercise~4.2i.
    Hence $0 \leq x$ is true, which shows that $0$ is a lower bound of $A$ since $x$ was arbitrary.

    Now consider any $x > 0$ so that also $1/x > 0$ by Exercise~4.2i.
    Then, by the Archimedean ordering property there is an $n \in \pints$ such that $n > 1/x > 0$ (since otherwise $1/x$ would be an upper bound of $\pints$).
    It then follows from Exercise~4.2j that $1/n < 1/(1/x) = x$.
    Since clearly $1/n \in A$ we have that $x$ is \emph{not} a lower bound of $A$.
    Since $x>0$ was arbitrary, this shows that $0$ is the greatest lower bound of $A$ since, by the contrapositive, $x$ being a lower bound of $A$ implies that $x \leq 0$.
  }

  (c)
  \qproof{
    Consider any real $a$ where $0 < a < 1$.
    First we show that the set $\braces{1/a^n \where n \in \pints}$ has no upper bound.
    To this end define $h = (1-a)/a = 1/a - 1$ so that $1+h = 1 + (1/a - 1) = 1/a$.
    Clearly we have
    \ali{
      a &< 1 \\
      -a &> -1 \\
      1-a &> 1 -1  = 0 \\
      \frac{1-a}{a} &> \frac{0}{a} = 0 & \text{(since $a>0$)} \\
      h &> 0
    }
    so that $1+h > 1 > 0$ and
    \ali{
      h &> 0 \\
      h^2 &> h \cdot 0 = 0 & \text{(since $h>0$)} \\
      n h^2 &> n \cdot 0 = 0
    }
    for any $n \in \pints$ since $n > 0$.
    
    We show by induction that $(1+h)^n \geq 1 + nh$ for all $n \in \pints$.
    For $n = 1$ we clearly have $(1+h)^n = (1+h)^1 = 1+h \geq 1+h = 1+1\cdot h = 1+nh$.
    Now, supposing that $(1+h)^n \geq 1 + nh$,we have
    \ali{
      (1+h)^{n+1} &= (1+h)^n (1+h) \\
      &\geq (1+nh)(1+h) & \text{(since $1+h > 0$)} \\
      &= 1 + nh + h + nh^2 \\
      &\geq 1 + nh + h & \text{(since $nh^2 > 0$)} \\
      &= 1 + (n+1)h \,,
    }
    which completes the induction.
    So consider any real $x$.
    Then, since we know that $\pints$ has no upper bound, there is an $n \in \pints$ where $n > x/h$ (noting that $h > 0$) so that
    \ali{
      n &> x/h \\
      nh &> (x/h)h = x & \text{(since $h > 0$)} \\
      1 + nh &> 1 + x > x \,. 
    }
    Then we have $1/a^n = (1/a)^n = (1+h)^n \geq 1 + nh > x$, which shows that the set $\braces{1/a^n \where n \in \pints}$ is unbounded above since $x$ was arbitrary.

    Now we show the main result.
    Let $A = \braces{a^n \where n \in \pints}$ so that we must show that $\inf{A} = 0$.
    First we show by induction that 0 is a lower bound of $A$.
    For $n=1$ we clearly have $a^n = a^1 = a \geq 0$.
    Then, if $a^n \geq 0$, we have $a^{n+1} = a^n \cdot a \geq 0 \cdot a = 0$ since $a > 0$.
    This completes the induction so that clearly $0$ is indeed a lower bound of $A$.

    Now consider any real $x > 0$ so that $1/x > 0$ also.
    Then, by what was shown above, we know that there is an $n \in \pints$ such that $1/a^n > 1/x > 0$.
    We then have $a^n = 1/(1/a^n) < 1/(1/x) = x$ by Exercise~4.2j.
    This shows that $x$ is \emph{not} a lower bound of $A$ since obviously $a^n \in A$.
    It then follows that $0$ is the \emph{greatest} lower bound of $A$ since $x>0$ was arbitrary, because, by the contrapositive, $x$ being a lower bound of $A$ implies that $x \leq 0$.
    Hence $0 = \inf{A}$ as desired.
  }
}

\exercise{9}{
  \eparts{
  \item Show that every nonempty subset of $\ints$ that is bounded above has a largest element.
  \item If $x \notin \ints$, show that there is exactly one $n \in \ints$ such that $n < x < n+1$.
  \item If $x-y > 1$, show there is at least one $n \in \ints$ such that  $y < n < x$.
  \item If $y < x$, show there is a rational number $z$ such that $y < z < x$.
  }
}
\sol{
  \dwhitman

  \begin{lem}\label{lem:intreal:intlub}
    The set of integers $\ints$ is an inductive set that has no lower or upper bounds in $\reals$.
  \end{lem}
  \qproof{
    First we show that $\ints$ is inductive.
    Clearly $1 \in \ints$ since $1 \in \pints \ss \ints$.
    Now suppose that $n \in \ints$ so that clearly $n+1 \in \ints$ by Exercise~4.5d since $1 \in \ints$.

    Next, consider any $x \in \reals$.
    Then we know that $\pints$ has no upper bound so that there is an $n \in \pints$ such that $n > x$, and clearly $n \in \ints$ since $\pints \ss \ints$.
    By the same token there is an $m \in \pints$ such that $m > -x$.
    But then we have $-m < -(-x) = x$ by Exercise~4.2d, and $-m \in \nints$ so that also $-m \in \ints$ since $\nints \ss \ints$.
    Since $x$ was arbitrary, this shows that $\ints$ is not bounded above or below.
  }

  \begin{lem}\label{lem:intreal:zerone}
    There is no integer $n$ such that $0 < n < 1$.
  \end{lem}
  \qproof{
    Suppose to the contrary that $n \in \ints$ and $0 < n < 1$.
    Let $S = \braces{k \in \ints \where 0 < k < 1}$ so that clearly $n \in S$ so that $S \neq \es$.
    Also since $0 < k$ and $k \in \ints$ for any $k \in S$, clearly $S \ss \pints$.
    Thus $S$ is a nonempty subset of positive integers so that it has a smallest element $m$ by the well-ordering property.
    Since $m \in S$ we have $0 < m < 1$ and hence $m^2 = m \cdot m < 1 \cdot m = m < 1$ by property (6) since $m > 0$.
    By the same property clearly $0 = 0 \cdot m < m \cdot m = m^2$ as well so that $0 < m^2 < 1$.
    Also, clearly $m^2 = m \cdot m \in \ints$ by Exercise~4.5e since $m \in \ints$, and so $m^2 \in S$.
    However, this cannot be since $m$ is the smallest element of $S$ and yet $m^2 < m$.
    Therefore we have a contradiction, which proves the result.
  }

  \begin{cor}\label{cor:intreal:intbet}
    For any integer $n$, there is no integer $a$ such that $n < a < n+1$.
  \end{cor}
  \qproof{
    Consider any $n \in \ints$ and suppose to the contrary that there is an $a \in \ints$ such that $n < a < n+1$.
    First, we have $n-a \in \ints$ by Exercise~4.5d since $a,n \in \ints$.
    Also, $n < a$ clearly implies that $0 < a-n$.
    Similarly, $a < n+1$ means that $a-n < 1$.
    But then we have that $a-n$ is an integer where $0 < a-n < 1$, which contradicts Lemma~\ref{lem:intreal:zerone}.
    Thus it must be the case that there is no such integer $a$.
  }

  \mainprob
  
  (a)
  \qproof{
    Suppose that $A$ is a nonempty subset of $\ints$ and that it is bounded above by $\a \in \reals$.
    Since $A \neq \es$, there is an $a \in A$, so define $A' = \braces{n - a + 1 \where n \in A}$.
    First we claim that $\a' = \a - a + 1$ is an upper bound of $A'$
    So consider any $n' \in A'$ so that $n' = n-a+1$ for some $n \in A$.
    Since $\a$ is an upper bound of $A$ we have
    \ali{
      n &\leq \a \\
      n-a &\leq \a - a \\
      n-a+1 & \leq \a - a + 1 \\
      n' &\leq \a' \,,
    }
    which shows that $\a'$ is an upper bound of $A'$ since $n'$ was an arbitrary element.
    We also have that there is an $N' \in \pints$ such that $\a' < N'$ since $\pints$ has no upper bound.

    Now let $B' = A' \cap \pints$.
    Then, for any $n' \in B'$, we have that $n' \in A'$ so that $n' \leq \a' < N'$.
    Since also clearly $n' \in \pints$, we have that $n' \in S_{N'} = \braces{k \in \pints \where k < N'} = \intsfin{N'-1}$.
    Hence $B' \ss S_{N'}$ since $n'$ was arbitrary.
    We also have that $1 \in A'$ since $a \in A$ and $a - a + 1 = 1$.
    Hence $1 \in B'$ since clearly also $1 \in \pints$ since it is inductive.
    Thus $B'$ is a nonempty subset of $S_{N'}$ so that it has a largest element $b'$ by Exercise~4.4a.

    Since $b' \in B'$, we have that $b' \in A'$ so that there is a $b \in A$ such that $b' = b-a+1$.
    We claim that $b$ is the largest element of $A$.
    We already know that $b \in A$ so we need only show that it is also an upper bound of $A$.
    So consider any $n \in A$ so that clearly $n' = n - a + 1 \in A'$.
    Now, it follows from Exercise~4.5d that $n' \in \ints$ since $n,a,1 \in \ints$.
    Thus we have the following:

    Case: $n' \in \pints$.
    Then, clearly $n' \in A' \cap \pints = B'$ so that $n' \leq b'$ since $b'$ is the largest element of $B'$.

    Case: $n' \in \nints \cup \braces{0}$.
    Then $n' < 1 \leq b'$ since $1 \in B'$ and $b'$ is the largest element of $B'$.

    Thus in either case $n' \leq b'$ is true so that
    \ali{
      n' &\leq b' \\
      n - a + 1 &\leq b - a + 1 \\
      n - a &\leq b - a \\
      n &\leq b \,,
    }
    which shows that $b$ is an upper bound and thus the largest element of $A$ since $n$ was arbitrary.
  }

  (b)
  \qproof{
    Suppose an $x \in \reals$ where $x \notin \ints$ and let $A = \braces{n \in \ints \where n < x}$.
    It follows from Lemma~\ref{lem:intreal:intlub} that there is an $m \in \ints$ where $m < x$ since $\ints$ has no lower bounds.
    Hence by definition $m \in A$ so that $A \neq \es$.
    Clearly also $x$ is an upper bound of $A$ so that $A$ is a nonempty subset of $\ints$ that is bounded above.
    It then follows from part~(a) that $A$ has a largest element $n$, where clearly $n < x$ since $n \in A$.

    Now, suppose for the moment that $n+1 \leq x$.
    Then, since $\ints$ is inductive (again by Lemma~\ref{lem:intreal:intlub}) and $n \in \ints$, we have that $n+1 \in \ints$ as well.
    But $x \notin \ints$ so that it must be that $n+1 \neq x$, and hence $n+1 < x$.
    Then $n+1 \in A$ so that $n+1 \leq n$ since $n$ is the largest element of $A$.
    However, this contradicts the obvious fact that $n+1 > n$ so that it must be that $n+1 \leq x$ is not true.
    Hence $n+1 > x$ and thus we have shown that $n < x < n+1$.

    Lastly, suppose that there is an integer $m$ such that $m < x < m+1$.
    Then $m \in A$ so that $m \leq n$ since $n$ is the largest element of $A$.
    Suppose for a moment that $m < n$.
    Then we would have $m < n < x < m+1$ so that $n$ is an integer between $m$ and $m+1$, which violates Corollary~\ref{cor:intreal:intbet}.
    Thus is has to be that $m = n$ (since $m \leq n$), which shows that $n$ is the unique integer such that $n < x < n+1$.
  }

  (c)
  \qproof{
    Suppose that $x,y \in \reals$ and $x-y > 1$.
    If $x \in \ints$ then let $n = x-1$ so that clearly $n \in \ints$ by Exercise~4.5d.
    First, we have
    \ali{
      x - y &> 1 \\
      x &> 1 + y \\
      x - 1 &> y \\
      n &> y \,.
    }
    We also clearly have $n = x - 1 < x$ so that $y < n < x$.

    On the other hand, if $x \notin \ints$, then we know from part~(b) that there is a unique integer $n$ such that $n < x < n+1$.
    We also have that
    \ali{
      x &< n + 1 \\
      1 < x - y &< n + 1 - y \\
      0 &< n - y \\
      y &< n
    }
    so that again $y < n < x$.

    Hence in both cases we have found an integer $n$ such that $y < n < x$, which proves the result.
  }

  (d)
  \qproof{
    Suppose that $x,y \in \reals$ where $y < x$.
    Then $0 < x-y$ so that $1/(x-y)$ exists..
    Since $\pints$ is unbounded above there is a $b \in \pints$ where $b > 1/(x-y)$.
    Hence
    \ali{
      b &> \frac{1}{x-y} \\
      b(x-y) &> 1 & \text{(since $x-y > 0$)} \\
      bx - by &> 1 \,.
    }
    It then follows from part~(c) that there is an integer $a$ such that $by < a < bx$.
    We then have that $y < a/b < x$ since $b > 0$ (since $b \in \pints$).
    This shows the result since clearly $a/b$ is rational because $a,b \in \ints$.
  }
}

\exercise{10}{
  Show that every positive number $a$ has exactly one positive square root, as follows:
  \eparts{
  \item Show that if $x > 0$ and $0 \leq h < 1$, then
    \ali{
      (x+h)^2 &\leq x^2 + h(2x+1) \,, \\
      (x-h)^2 &\geq x^2 - h(2x) \,.
    }
  \item Let $x > 0$.
    Show that if $x^2 < a$, then $(x+h)^2 < a$ for some $h > 0$; and if $x^2 > a$, then $(x-h)^2 > a$ for some $h>0$.
  \item Given $a>0$, let $B$ be the set of all real numbers $x$ such that $x^2 < a$.
    Show that $B$ is bounded above and contains at least one positive number.
    Let $b = \sup{B}$; show that $b^2 = a$.
  \item Show that if $b$ and $c$ are positive and $b^2 = c^2$, then $b=c$.
  }
}
\sol{
  \dwhitman

  \begin{lem}\label{lem:intreal:xsx}
    If $x \in \reals$ and $x^2 < 1$, then $x < 1$ also.
  \end{lem}
  \qproof{
    Suppose that $x \geq 1$.
    If $x = 1$ then clearly $x^2 = 1^1 = 1$.
    On the other hand, if $x  > 1$ then clearly $x^2 = x \cdot x > 1 \cdot x = x > 1$ by property (6) since $x > 1 > 0$.
    Thus in either case $x^2 \geq 1$ so that we have shown that $x \geq 1$ implies that $x^2 \geq 1$.
    It then follows that $x^2 < 1$ implies $x < 1$ by the contrapositive.
  }

  \begin{lem}\label{lem:intreal:ysx}
    If $0 < y < x$ then $0 < y^2  < x^2$.
  \end{lem}
  \qproof{
    Supposing that $0 < y < x$, we have $0 = 0 \cdot y < y \cdot y = y^2 = y \cdot y < x \cdot y = y \cdot x < x \cdot x = x^2$ all by property (6) since both $x$ and $y$ are positive.
  }

  \mainprob
  
  (a)
  \qproof{
    First, we know that $0 \leq h < 1$.
    If $h = 0$ then clearly $h = 0 = 0^2 = h^2$ so that $0 \leq h^2 \leq h$ is true.
    If $h \neq 0$ then $0 < h < 1$ so that $0 = 0 \cdot h < h \cdot h = h^2 < 1 \cdot h = h$ by property (6) since $h > 0$ so that again $0 \leq h^2 \leq h$ is true.

    We then have
    \ali{
      (x+h)^2 &= (x+h)(x+h) \\
      &= x^2 + 2xh + h^2 \\
      &\leq x^2 + 2xh + h & \text{(since $h^2 \leq h$)} \\
      &= x^2 + h(2x+ 1) \,.
    }
    Also
    \ali{
      (x-h)^2 &= (x-h)(x-h) \\
      &= x^2 - 2xh + h^2 \\
      &\geq x^2 - 2xh + 0 & \text{(since $h^2 \geq 0$)} \\
      &= x^2 - h(2x) \,,
    }
    which show the desired results.
  }

  (b) We modify this result so that the $h$ in the second part is not just positive but also $h < x$.
  In fact, without this stipulation, the theorem becomes obvious since any arbitrarily large $h$ will suffice.
  Because then then $x-h$ is arbitrarily large in magnitude (but negative) so that $(x-h)^2$ can be made arbitrarily large so that of course $(x-h)^2 > a$.
  Adding the stipulation that $0 < h < x$ makes the theorem more useful and is necessary for it to be of use in part~(c) below.
  
  \qproof{
    Suppose that $x>0$.
    Then clearly $2x > 2\cdot 0 = 0$ as well.
    Also it then follows that $2x + 1 > 1 > 0$.

    If $x^2 < a$ then clearly $0 < a - x^2$.
    Hence we have that $0 < (a-x^2) / (2x+1)$ by Exercise~4.2 parts (i) and (h) since both $a-x^2$ and $2x+1$ are positive.
    So let $y = \min(1, (a-x^2)/(2x+1))$ so that clearly both $y \leq 1$ and $y \leq (a-x^2)/(2x+1)$.
    Since $0 < 1$ and $0 < (a-x^2)/(2x+1)$, we have that $0 < y$ so that it follows from Exercise~4.9d that there is a rational $h$ such that $0 < h < y$.
    Hence $0 < h < y \leq 1$ so that, by part~(a), we have
    \ali{
      (x+h)^2 &\leq x^2 + h(2x+1) \\
      &< x^2 + \parens{\frac{a-x^2}{2x+1}}\parens{2x+1} & \text{(since $h < y \leq (a-x^2)/(2x+1)$ and $2x+1 > 0$)} \\
      &= x^2 + (a - x^2) \\
      &= a \,.
    }

    If $x^2 > a$ then clearly $x^2 - a > 0$.
    Then we have again that $(x^2-a)/(2x)$ is positive since we showed previously that $2x$ is.
    So let $y = \min(1, (x^2-a)/(2x), x)$ so that clearly $y \leq 1$,  $y \leq (x^2 - a)/(2x)$, and $y \leq x$.
    Since both 1, $(x^2 - a)/(2x)$, and $x$ are all positive it follows that $0 < y$ so that there is a rational $h$ such that $0 < h < y$ by Exercise~4.9d.
    Therefore $0 > -h > -y$.
    Since $0 < h < y \leq 1$  we have by part~(a) that
    \ali{
      (x-h)^2 &\geq x^2 - h(2x) \\
      &> x^2 - \parens{\frac{x^2 - a}{2x}} \parens{2x} & \text{(since $-h > -y \geq -(x^2-a)/(2x)$ and $2x > 0$)} \\
      &= x^2 - (x^2 - a) \\
      &= a \,,
    }
    which show the desired results since clearly $0 < h < y \leq x$.
  }

  (c)
  \qproof{
    Suppose that $a > 0$ and let $B = \braces{x \in \reals \where x^2 < a}$.

    If $a < 1$ then $0 < a < 1$ so that $a^2 = a \cdot a < 1 \cdot a = a$ so that $a$ itself is in $B$ (and of course $a$ is positive).
    Now consider any $x \in B$ so that $x^2 < a$.
    Then $x^2 < a < 1$ so that also $x < 1$ by Lemma~\ref{lem:intreal:xsx}.
    Since $x \in B$ was arbitrary, this shows that $1$ is an upper bound of $B$.

    If $a \geq 1$ then $(1/2)^2 = 1/2^2 = 1/4 < 1 \leq a$ so that $1/2 \in B$ (and of course $1/2$ is positive).
    Now consider any $x \in B$ so that $x^2 < a$.
    If $x \leq 1$ then $x \leq 1 \leq a$.
    On the other hand, if $x > 1$ then $x^2 = x \cdot x > 1 \cdot x = x$ since $x > 1 > 0$ so that $x < x^2 < a$.
    Thus in both cases $x \leq a$ so that $a$ is an upper bound of $B$ since $x$ was arbitrary.

    Therefore in each case $B$ contains a positive element (so that $b \neq \es$) and $B$ is bounded above.
    It then follows that $B$ has a least upper bound $b$ (so that $b = \sup{B}$).
    Clearly since $B$ has a positive element $x$, it follows that $0 < x \leq b$ so that $b$ is positive.

    Now suppose that $b^2 < a$.
    Then by definition $b \in B$ so that $b$ has to be the largest element of $b$ since it is the least upper bound.
    Since we know that $b$ is positive and $b^2 < a$, it follows from part~(b) that there is an $h > 0$ where $(b+h)^2 < a$ and hence $b+h \in B$.
    However, since $h>0$, it follows that $b < b+h$, which contradicts the fact that $b$ is the greatest element of $B$.
    Hence it cannot be that $b^2 < a$.

    So suppose that $b^2 > a$.
    Then again by part~(b) there is an $h$ where $0 < h < b$ such that $(b-h)^2 > a$.
    Now, since $h> 0$, it follows that $b-h < b$ so that $n-h$ is not an upper bound of $B$ (since then $b$ would not be the least upper bound).
    Hence there is an $x \in B$ such that $b-h < x$, noting that $x^2 < a$ by the definition of $B$.
    Since $h < b$, we have that $0 < b - h < x$ so that $(b-h)^2 < x^2 < a$ by Lemma~\ref{lem:intreal:ysx}.
    But this contradicts the established fact that $(b-h)^2 > a$ so that it cannot be that $b^2 > a$.

    Thus the only possibility remaining is that $b^2 = a$ as desired.
  }

  (d)
  \qproof{
    Suppose that $b$ and $c$ are positive and that $b^2 = c^2$.
    If it were the case that $b < c$ then $0 < b < c$ so that $0 < b^2 < c^2$ by Lemma~\ref{lem:intreal:ysx} so that clearly $b^2 \neq c^2$.
    As this is a contradiction, it has to be that $b \geq c$.
    An analogous argument shows that $b > c$ also leads to a contradiction so that $b \leq c$.
    Hence it must be that $b = c$ as desired.
  }
}

\exercise{11}{
  Given $m \in \ints$, we say that $m$ is \textbf{\emph{even}} if $m/2 \in \ints$, and $m$ is \textbf{\emph{odd}} otherwise.
  \eparts{
  \item Show that if $m$ is odd, $m = 2n+1$ for some $n \in \ints$.
    [Hint: Choose $n$ so that $n < m/2 < n+1$.]
  \item Show that if $p$ and $q$ are odd, so are $p \cdot q$ and $p^n$, for any $n \in \pints$.
  \item Show that if $a > 0$ is rational, then $a=m/n$ for some $m,n \in \pints$ where not both $n$ and $m$ are even.
    [Hint: Let $n$ be the smallest element of the set $\braces{x \where x \in \pints \text{ and } x \cdot a \in \pints}$.]
  \item \emph{Theorem: $\sqrt{2}$ is irrational.}
  }
}
\sol{
  \dwhitman

  \begin{lem}\label{lem:intreal:ltleq}
    If $n,m \in \ints$ and $n < m$, then $n+1 \leq m$ and $n \leq m-1$.
  \end{lem}
  \qproof{
    Suppose that $n+1 > m$ so that $n < m < n+1$, which violates Corollary~\ref{cor:intreal:intbet} since $m \in \ints$.
    Thus it has to be that $n+1 \leq m$.
    From this it immediately follows that $n = n+1-1 \leq m-1$ by simply subtracting 1 from both sides of the previous inequality.
  }

  \begin{lem}\label{lem:intreal:even}
    An integer $m$ is even if and only if $m = 2n$ for some integer $n$.
  \end{lem}
  \qproof{
    $(\imp)$ Supposing that $m$ is even, then $n = m/2 \in \ints$. Then clearly $m = 2n$.

    $(\pmi)$ Now suppose that $m = 2n$ for some integer $n$.
    Then clearly $m/2 = n$ is an integer so that $m$ is even by definition.
  }

  \begin{lem}\label{lem:intreal:squodd}
    An integer $a$ is odd if and only if $a^2$ is also odd.
  \end{lem}
  \qproof{
    $(\imp)$ Suppose that $a$ is odd so that $a = 2n+1$ for some integer $n$ (this is shown in part (a) below, which does not depend on this lemma).
    Then
    \gath{
      a^2 = a \cdot a = (2n+1)(2n+1) = 4n^2 + 2n + 2n + 1 = 4n^2 + 4n + 1 = 2\squares{2(n^2 + n)} + 1 \,
    }
    noting that clearly $2(n^2+n)$ is an integer since $n$ is.
    Hence $a^2$ is odd again by what will be shown in part (a).

    $(\pmi)$ We prove this by contrapositive, so suppose that $a$ is not odd so that it must be even.
    Therefore $a = 2n$ for some integer $n$ by Lemma~\ref{lem:intreal:even}.
    Then $a^2 = a \cdot a = (2n)(2n) = 4n^2 = 2(2n^2)$ so that $a^2$ is even since clearly $2n^2$ is an integer since $n$ is.
    Thus $a^2$ is not odd.
  }

  \mainprob

  (a) Here we show the converse as well, i.e. we show that $m$ is odd if and only if $m = 2n+1$ for some $n \in \ints$.
  \qproof{
    $(\imp)$ Suppose that $m$ is odd so that by definition $m/2 \notin \ints$.
    It then follows from Exercise~4.9b that there is a unique integer $n$ such that $n < m/2 < n+1$.
    We then have that $2n < m < 2(n+1) = 2n+2$ since obviously $2 > 0$.
    Hence by Lemma~\ref{lem:intreal:ltleq} we have that $2n + 1 \leq m$ and also $m \leq 2n+2-1 = 2n+1$.
    Therefore it has to be that $m = 2n+1$ as desired.

    $(\pmi)$ Now suppose that there is an $n \in \ints$ such that $m = 2n+1$.
    Then we have that
    \gath{
      \frac{m}{2} = \frac{2n+1}{2} = n + \frac{1}{2} \,.
    }
    We then clearly have that $n = n+0 < n+1/2 < n+1$ since $0 < 1/2 < 1$ so that $m/2 = n+1/2$ cannot be an integer by Corollary~\ref{cor:intreal:intbet}.
    Hence $m$ is odd by definition.
  }

  (b)
  \qproof{
    Suppose that $p$ and $q$ are odd so that $p = 2k+1$ and $q = 2m+1$ for some $k,m \in \ints$ by part~(a).
    We then have that
    \gath{
      p \cdot q = (2k+1)(2m+1) = 4km + 2m + 2k + 1 = 2(2km + m + k) + 1
    }
    so that $p \cdot q$ is odd by what was shown in part~(a) since clearly $2km+m+k \in \ints$ by Exercise~4.5 since $k$ and $m$ are integers.

    Now we show by induction on $n$ that $p^n$ is odd for any $n \in \pints$.
    First, for $n=1$ we clearly have $p^n = p^1 = p$ is odd by supposition.
    Then, if we assume that $p^n$ is odd, we have that the product $p^{n+1} = p^n \cdot p$ is odd as well by what was just shown since both $p^n$ and $p$ are odd.
    This completes the induction.
  }

  (c)
  \qproof{
    Suppose that $a > 0$ is rational.
    Then $a = p/q$ for some integers $p$ and $q$.
    Clearly it cannot be that $q = 0$, and if $q < 0$ then $q = -b$ for some $b \in \pints$.
    Then we have $a = p/q = p/(-b) = (-p)/b$ so that $ab = -p$.
    Furthermore, since $a$ and $b$ are both positive, we have that $ab = -p$ is positive by Exercise~4.2h.
    Thus clearly $-p \in \pints$ since $p \in \ints$.

    Now, let $X = \braces{x \in \pints \where a x \in \pints}$.
    Since we just showed that $b \in \pints$ and $a b = -p \in \pints$ it follows that $b \in X$.
    Since clearly $X \ss \pints$ and $X$ is nonempty (since $b \in X$), it has a smallest element $n$ by the well-ordering property.
    Letting $m = an$, we clearly have that $m \in \pints$ since $n \in X$.
    Then, we have $a = m/n$, noting again that $m,n \in \pints$.

    To show that not both $m$ and $n$ are even, suppose to the contrary that they \emph{are} both even.
    Then by Lemma~\ref{lem:intreal:even} we have that $m = 2k$ and $n = 2l$ for some $k,l \in \ints$.
    Clearly then $k = m/2$ and $l = n/2$ so that both $k$ and $l$ are positive by Exercise~4.2h since $m$ and $n$ (and $1/2$) are.
    Hence $k,l \in \pints$.
    We have $a = m/n = 2k/2l = k/l$ so that $al = k$, which implies that $l \in X$ since $l$ and $al=k$ are both in $\pints$.
    However, we also have that $l = n/2 < n$ since $n > 0$, which contradicts the fact that $n$ is the smallest element of $X$.
    Thus it has to be the case that not both $m$ and $n$ are even.
  }

  (d) This is one of the most famous proofs in all of mathematics, and is often used as an example of mathematical proofs since it can be understood by most laymen.
  \qproof{
    Obviously we take $\sqrt{2}$ to be the unique positive real number such that $(\sqrt{2})^2 = 2$ as was shown to exist in Exercise~4.10.
    Suppose to the contrary that $\sqrt{2}$ is rational so that $\sqrt{2} = a/b$ for $a,b \in \pints$ where not both $a$ and $b$ are even by part (c) since $\sqrt{2} > 0$.
    We therefore have that $2 = (\sqrt{2})^2 = (a/b)^2 = a^2 / b^2$ so that $2b^2 = a^2$.
    Since $b^2$ is an integer (clearly, since $b$ is and $b^2 = b \cdot b$) it follows from Lemma~\ref{lem:intreal:even} that $a^2$ is even.
    This means that $a$ itself is even by Lemma~\ref{lem:intreal:squodd}.
    Hence $a = 2n$ for some integer $n$ so that $a^2 = (2n)^2 = 4n^2$.
    From before, we then have $2b^2 = a^2 = 4n^2$ so that clearly $b^2 = 2n^2$, from which it follows as before that $b^2$ and therefore $b$ itself is even by Lemmas~\ref{lem:intreal:even} and \ref{lem:intreal:squodd}.
    However, this is a contradiction since we previously established that $a$ and $b$ cannot both be even!
    So it has to be that $\sqrt{2}$ is not rational and is therefore irrational as desired.
  }
}
