\setcounter{subsection}{4-1}
\subsection{The Integers and the Real Numbers}

% Macros used in this section
\def\multcom{since multiplication is commutative}
\def\multass{since multiplication is associative}

\def\parta{If $x + y = x$, then $y = 0$.}
\def\partb{$0 \cdot x = 0$. [Hint: Compute $(x+0) \cdot x$.]}
\def\partc{$-0 = 0$.}
\def\partd{$-(-x) = x$.}
\def\parte{$x(-y) = -(xy) = (-x)y$.}
\def\partf{$(-1)x = -x$.}
\def\partg{$x(y-z) = xy - xz$.}
\def\parth{$-(x+y) = -x - y$; $-(x-y) = -x + y$.}
\def\parti{If $x \neq 0$ and $x \cdot y = x$, then $y = 1$.}
\def\partj{$x/x = 1$ if $x \neq 0$.}
\def\partk{$x/1 = x$.}
\def\partl{$x \neq 0$ and $y \neq 0 \imp xy \neq 0$.}
\def\partm{$(1/y)(1/z) = 1/(yz)$ if $y,z \neq 0$.}
\def\partn{$(x/y)(w/z) = (xw)/(yz)$ if $y,z \neq 0$.}
\def\parto{$(x/y) + (w/z) = (xz + wy)/(yz)$ if $y,z \neq 0$.}
\def\partp{$x \neq 0 \imp 1/x \neq 0$.}
\def\partq{$1/(w/z) = z/w$ if $w,z \neq 0$.}
\def\partr{$(x/y)/(w/z) = (xz)/(yw)$ if $y,w,z \neq 0$.}
\def\parts{$(ax)/y = a(x/y)$ if $y \neq 0$.}
\def\partt{$(-x)/y = x/(-y) = -(x/y)$ if $y \neq 0$.}
\def\xr{\frac{1}{x}}
\def\yr{\frac{1}{y}}
\def\zr{\frac{1}{z}}

\exercise{1}{
  Prove the following ``laws of algebra'' for $\reals$, using only axioms (1)-(5):

  \begin{multicols}{2}
    \eparts{
    \item \parta
    \item \partb
    \item \partc
    \item \partd
    \item \parte
    \item \partf
    \item \partg
    \item \parth
    \item \parti
    \item \partj
    \item \partk
    \item \partl
    \item \partm
    \item \partn
    \item \parto
    \item \partp
    \item \partq
    \item \partr
    \item \parts
    \item \partt
    }
  \end{multicols}
}
\sol{
  \dwhitman

  \begin{lem}\label{lem:intreal:eqadd}
    $x + y = x + z$ if and only if $y = z$.
  \end{lem}
  \qproof{
    $(\pmi)$ Clearly if $y = z$ then $x+y = x+z$ since the $+$ operation is a function.

    $(\imp)$ If $x+y = x+z$ then we have
    \ali{
      y &= y + 0 & \text{(by (3))} \\
      &= 0 + y & \text{(by (2))} \\
      &= (x + (-x)) + y & \text{(by (4))} \\
      &= (-x + x) + y & \text{(by (2))} \\
      &= -x + (x + y) & \text{(by (1))} \\
      &= -x + (x + z) & \text{(by what was just shown for $(\pmi)$)} \\
      &= (-x + x) + z & \text{(by (1))} \\
      &= (x + (-x)) + z & \text{(by (2))} \\
      &= 0 + z & \text{(by (4))} \\
      &= z + 0 & \text{(by (2))} \\
      &= z & \text{(by (3))}
    }
    as desired.
  }

  \begin{lem}\label{lem:intreal:eqmul}
    If $x \neq 0$ then $x \cdot y = x \cdot z$ if and only if $y=z$.
  \end{lem}
  \qproof{
    $(\pmi)$ Clearly if $y=z$ then $x \cdot y = x \cdot z$ since the $\cdot$ operation is a function.

    $(\imp)$ If $x \cdot y = x \cdot z$ then we have
    \ali{
      y &= y \cdot 1 & \text{(by (3))} \\
      &= 1 \cdot y & \text{(by (2))} \\
      &= \parens{x \cdot \xr} \cdot y & \text{(by (4), noting that $x \neq 0$)} \\
      &= \parens{\xr \cdot x} \cdot y  & \text{(by (2))} \\
      &= \xr \cdot \parens{x \cdot y} & \text{(by (1))} \\
      &= \xr \cdot \parens{x \cdot z} & \text{(by what was just shown for $(\pmi)$)} \\
      &= \parens{\xr \cdot x} \cdot z & \text{(by (1))} \\
      &= \parens{x \cdot \xr} \cdot z & \text{(by (2))} \\
      &= 1 \cdot z & \text{(by (4))} \\
      &= z \cdot 1 & \text{(by (2))} \\
      &= z & \text{(by (3))}
    }
    as desired.
  }

  \begin{lem}\label{lem:intreal:recom}
    $1/(yz) = 1/(zy)$ if $y,z \neq 0$.
  \end{lem}
  \qproof{
    We have $(zy) \cdot 1/(yz) = (yz) \cdot 1/(yz) = 1$ by (2) followed by (4) so that $1/(yz)$ is a reciprocal of $zy$.
    Since this reciprocal is unique, however, it must be that $1/(yz) = 1/(zy)$ as desired.
  }

  \mainprob

  (a) \parta
  \qproof{
    Clearly by (3) we have $x + 0 = x = x + y$ so that it has to be that $y = 0$ by Lemma~\ref{lem:intreal:eqadd}.
  }

  (b) \partb
  \qproof{
    We have
    \ali{
      x \cdot x + 0 \cdot x &= x \cdot x + x \cdot 0 & \text{(since $0 \cdot x = x \cdot 0$ by (2))} \\
      &= x \cdot (x + 0) & \text{(by (5))} \\
      &= x \cdot x \,. & \text{(since $x + 0 = x$ by (3))}
    }
    Thus it must be that $0 \cdot x = 0$ by part (a).
  }

  (c) \partc
  \qproof{
    By (4) we have $0 + (-0) = 0$ so that it has to be that $-0 = 0$ by part (a).
  }

  (d) \partd
  \qproof{
    We have
    \ali{
      -(-x) &= -(-x) + 0 & \text{(by (3))} \\
      &= -(-x) + (x + (-x)) & \text{(by (4))} \\
      &= -(-x) + ((-x) + x) & \text{(by (2))} \\
      &= (-(-x) + (-x)) + x & \text{(by (1))} \\
      &= ((-x) + (-(-x))) + x & \text{(by (2))} \\
      &= 0 + x & \text{(by (4))} \\
      &= x + 0 & \text{(by (2))} \\
      &= x & \text{(by (3))}
    }
    as desired.
  }

  (e) \parte
  \qproof{
    First we have
    \ali{
      x(-y) &= x(-y) + 0 & \text{(by (3))} \\
      &= x(-y) + (xy + (-(xy))& \text{(by (4))} \\
      &= (x(-y) + xy) + (-(xy)) & \text{(by (1))} \\
      &= x(-y + y) + (-(xy)) & \text{(by (5))} \\
      &= x(y + (-y)) + (-(xy)) & \text{(by (2))} \\
      &= x \cdot 0 + (-(xy)) & \text{(by (4))} \\
      &= 0 \cdot x + (-(xy)) & \text{(by (2))} \\
      &= 0 + (-(xy)) & \text{(by part(b))} \\
      &= -(xy) + 0 & \text{(by (2))} \\
      &= -(xy) \,. & \text{(by (3))} \\
    }
    We also have
    \ali{
      (-x)y &= y(-x) & \text{(by (2))} \\
      &= -(yx) & \text{(by what was just shown)} \\
      &= -(xy) & \text{(by (2))}
    }
    so that the result follows since equality is transitive.
  }

  (f) \partf
  \qproof{
    We have
    \ali{
      (-1)x &= -(1 \cdot x) & \text{(by part(e))} \\
      &= -(x \cdot 1) & \text{(by (2))} \\
      &= -x & \text{(since $x \cdot 1 = x$ by (3))}
    }
    as desired.
  }

  (g) \partg
  \qproof{
    We have
    \ali{
      x(y-z) &= x(y + (-z)) & \text{(by the definition of subtraction)} \\
      &= xy + x(-z) & \text{(by (5))} \\
      &= xy + (-(xz)) & \text{(by part(e))} \\
      &= xy - xz & \text{(by the definition of subtraction)}
    }
    as desired.
  }

  (h) \parth
  \qproof{
    We have
    \ali{
      -(x+y) &= (-1)(x+y) & \text{(by part (f))} \\
      &= (-1)x + (-1)y & \text{(by (5))} \\
      &= -x + (-y) & \text{(by part (f) twice)} \\
      &= -x - y & \text{(by the definition of subtraction)}
    }
    and
    \ali{
      -(x-y) &= -(x + (-y)) & \text{(by the definition of subtraction)} \\
      &= -x - (-y)) & \text{(by what was just shown)} \\
      &= -x + (-(-y)) & \text{(by the definition of subtraction)} \\
      &= -x + y & \text{(by part (d))}
    }
    as desired.
  }

  (i) \parti
  \qproof{
    By (3) we have $x \cdot 1 = x = x \cdot y$ so that it has to be that $y = 1$ by Lemma~\ref{lem:intreal:eqmul}, noting that this applies since $x \neq 0$.
  }

  (j) \partj
  \qproof{
    By the definition of division we have $x/x = x \cdot (1/x) = 1$ by (4) since $x \neq 0$ and $1/x$ is defined as the reciprocal (i.e. the multiplicative inverse) of $x$.
  }

  (k) \partk
  \qproof{
    First, we have by (4) that $1 \cdot (1/1) = 1$, where $1/1$ is the reciprocal of $1$.
    We also have that $1 \cdot (1/1) = (1/1) \cdot 1 = 1/1$ by (2) and (3).
    Therefore $1/1 = 1 \cdot (1/1) = 1$ so that $1$ is its own reciprocal.
    Then, by the definition of division, we have $x/1 = x \cdot (1/1) = x \cdot 1 = x$ by (3).
  }

  (l) \partl
  \qproof{
    Suppose that $x \neq 0$ and $y \neq 0$.
    Also suppose to the contrary that $xy = 0$.
    Since $y \neq 0$ it follows from (4) that $1/y$ exists.
    So, we have $(xy) \cdot (1/y) = 0 \cdot (1/y) = 0$ by part (b).
    We also have
    \ali{
      (xy) \cdot \yr &= x \parens{y \cdot \yr} & \text{(by (1))} \\
      &= x \cdot 1 & \text{(by (4))} \\
      &= x & \text{(by (3))}
    }
    so that $x = (xy) \cdot (1/y) = 0$, which is a contradiction since we supposed that $x \neq 0$.
    Hence it must be that $xy \neq 0$ as desired.
  }

  (m) \partm
  \qproof{
    We have
    \ali{
      (yz)\parens{\yr \cdot \zr} &= (yz) \parens{\zr \cdot \yr} & \text{(by (2))} \\
      &= \parens{(yz) \cdot \zr}\yr & \text{(by (1))} \\
      &= \parens{y \parens{z \cdot \zr}} \yr & \text{(by (1))} \\
      &= \parens{y \cdot 1} \yr & \text{(by (4))} \\
      &= y \cdot \yr & \text{(by (3))} \\
      &= 1 & \text{(by (4))}
    }
    so that $(1/y)(1/z)$ is a multiplicative inverse of $yz$.
    Since this inverse is \emph{unique} by (4), however, it has to be that $(1/y)(1/z) = 1/(yz)$ as desired.
  }

  \def\yzr{\frac{1}{yz}}
  (n) \partn
  \qproof{
    We have
    \ali{
      \frac{x}{y} \cdot \frac{w}{z} &= \parens{x \cdot \yr}\parens{w \cdot \zr} & \text{(by the definition of division)} \\
      &= \parens{x \cdot \yr} \parens{\zr \cdot w} & \text{(by (2))} \\
      &= \parens{\parens{x \cdot \yr} \zr} w  & \text{(by (1))} \\
      &= \parens{x \parens{\yr \cdot \zr}} w & \text{(by (1))} \\
      &= \parens{x \cdot \yzr} w & \text{(by part (m) since $y,z \neq 0$)} \\
      &= \parens{\yzr \cdot x} w & \text{(by (2))} \\
      &= \yzr (xw) & \text{(by (1))} \\
      &=(xw) \yzr & \text{(by (2))} \\
      &= \frac{xw}{yz} & \text{(by the definition of division)}
    }
    as desired.
  }

  (o) \parto
  \qproof{
    We have
    \ali{
      \frac{x}{y} + \frac{w}{z} &= \frac{x}{y} \cdot 1 + \frac{w}{z} \cdot 1 & \text{(by (3))} \\
      &= \frac{x}{y} \cdot \frac{z}{z} + \frac{w}{z} \cdot \frac{y}{y}  & \text{(by part (j))} \\
      &= \frac{xz}{yz} + \frac{wy}{zy} & \text{(by part(n))} \\
      &= (xz)\frac{1}{yz} + (wy)\frac{1}{zy} & \text{(by the definition of division)} \\
      &= (xz)\frac{1}{yz} + (wy)\frac{1}{yz} & \text{(by Lemma~\ref{lem:intreal:recom})} \\
      &= \frac{1}{yz}(xz) + \frac{1}{yz}(wy) & \text{(by (2))} \\
      &= \frac{1}{yz}\parens{xz + wy} & \text{(by (5))} \\
      &= \parens{xz+ wy} \frac{1}{yz} & \text{(by (2))} \\
      &= \frac{xz + wy}{yz} & \text{(by the definition of division)}
    }
    as desired.
  }

  (p) \partp
  \qproof{
    Suppose that $x \neq 0$ but $1/x = 0$.
    Then we first have that $x \cdot (1/x) = x \cdot 0 = 0 \cdot x = 0$ by (2) and part (b).
    However, we also have $x \cdot (1/x) = 1$ by (4).
    Hence we have $0 = x \cdot (1/x) = 1$, which is a contradiction since we know that $0$ and $1$ are distinct by (3).
    So, if we accept that $x \neq 0$, then it must be that $1/x \neq 0$ also.
  }

  (q) \partq
  \qproof{
    We have
    \ali{
      \frac{w}{z} \cdot \frac{z}{w} &= \frac{wz}{zw} & \text{(by part (n) since $w,z \neq 0$)} \\
      &= (wz) \frac{1}{zw} & \text{(by the definition of division)} \\
      &= (wz) \frac{1}{wz} & \text{(by Lemma~\ref{lem:intreal:recom} since $w,z \neq 0$)} \\
      &= 1 & \text{(by (4))}
    }
    so that by definition $z/w$ is the reciprocal of $w/z$.
    Since this is unique by (4) we then have $z/w = 1/(w/z)$ as desired.
  }

  (r) \partr
  \qproof{
    We have
    \ali{
      \frac{x/y}{w/z} &= \frac{x}{y} \cdot \frac{1}{w/z} & \text{(by the definition of division)} \\
      &= \frac{x}{y} \cdot \frac{z}{w} & \text{(by part (q) since $w,z \neq 0$)} \\
      &= \frac{xz}{yw} & \text{(by part (n) since $y,w \neq 0$)}
    }
    as desired.
  }

  (s) \parts
  \qproof{
    We have
    \ali{
      \frac{ax}{y} &= (ax) \cdot \yr & \text{(by the definition of division)} \\
      &= a \parens{x \cdot \yr} & \text{(by (1))} \\
      &= a \cdot \frac{x}{y} & \text{(by the definition of division)}
    }
    as desired.
  }

  (t) \partt
  \qproof{
    We have
    \ali{
      \frac{-x}{y} &= (-x) \cdot \yr & \text{(by the definition of division)} \\
      &= ((-1)x) \cdot \yr & \text{(by part (f))} \\
      &= (-1) \parens{x \cdot \yr} & \text{(by (1))} \\
      &= (-1) \frac{x}{y} & \text{(by the definition of division)} \\
      &= -\parens{\frac{x}{y}} \,. & \text{(by part (f))}
    }
    Now, we have $(-1)(-1) = -(-1) = 1$ by parts (f) and (d) so that $-1$ is its own reciprocal, since the reciprocal is unique, i.e. $1/(-1) = -1$.
    We also have
    \ali{
      \frac{-x}{y} &= (-x) \cdot \yr & \text{(by the definition of division)} \\
      &= ((-1)x) \cdot \yr & \text{(by part (f))} \\
      &= (x(-1)) \cdot \yr & \text{(by (2))} \\
      &= x \parens{(-1) \yr} & \text{(by (1))} \\
      &= x \parens{\frac{1}{-1} \cdot \yr} & \text{(by what was just shown above)} \\
      &= x \frac{1}{(-1)y} & \text{(part (m) since $y \neq 0$)} \\
      &= x\frac{1}{-y} & \text{(by part (f))}
    }
    so that $-(x/y) = (-x)/y = x/(-y)$ as desired.
  }
}

\def\parta{$x > y$ and $w > z \imp x+w > y+z$.}
\def\partb{$x > 0$ and $y > 0 \imp x+y>0$ and $x \cdot y > 0$.}
\def\partc{$x > 0 \bic -x < 0$.}
\def\partd{$x > y \bic -x < -y$.}
\def\parte{$x > y$ and $z < 0 \imp xz < yz$.}
\def\partf{$x \neq 0 \imp x^2 > 0$, where $x^2 = x \cdot x$.}
\def\partg{$-1 < 0 < 1$.}
\def\parth{$xy > 0 \bic x$ and $y$ are both positive or both negative.}
\def\parti{$x > 0 \imp 1/x > 0$.}
\def\partj{$x > y > 0 \imp 1/x < 1/y$.}
\def\partk{$x < y \imp x < (x+y)/2 < y$.}

\newpage % This is just because the exercise label was on a separate page from the text, which was annoying
\exercise{2}{
  Prove the following ``laws of inequalities'' for $\reals$, using axioms (1)-(6) along with the results of Exercise~1:

  \begin{multicols}{2}
    \eparts{
    \item \parta
    \item \partb
    \item \partc
    \item \partd
    \item \parte
    \item \partf
    \item \partg
    \item \parth
    \item \parti
    \item \partj
    \item \partk
    }
  \end{multicols}
}
\sol{
  \dwhitman

  \begin{lem}\label{lem:intreal:xpx}
    $x + x = 2x$ for any real $x$.
  \end{lem}
  \qproof{
    We simply have
    \ali{
      x+x &= x \cdot 1 + x \cdot 1 & \text{(by (3))} \\
      &= x(1+1) & \text{(by (5))} \\
      &= x \cdot 2 & \text{(since $2$ is defined as $1+1$)} \\
      &= 2x & \text{(by (2))}
    }
    as desired.
  }

  \mainprob

  (a) \parta
  \qproof{
    We have
    \ali{
      x+w &> y+w & \text{(by (6) since $x>y$)} \\
      &= w + y & \text{(by (2))} \\
      &> z + y & \text{(by (6) since $w>z$)} \\
      &= y + z & \text{(by (2))}
    }
    as desired.
  }

  (b) \partb
  \qproof{
    First we have
    \ali{
      x+y &> 0 + y & \text{(by (6) since $x>0$)} \\
      &= y+0 & \text{(by (2))} \\
      &= y & \text{(by (3))} \\
      &> 0 \,.
    }
    Also
    \ali{
      x \cdot y &> 0 \cdot y & \text{(by (6) since $x>0$ and $y>0$)} \\
      &= 0 & \text{(by Exercise~4.1b)}
    }
    as desired.
  }

  (c) \partc
  \qproof{
    $(\imp)$ Suppose that $x>0$.
    Then we have
    \ali{
      -x &= -x + 0 & \text{(by (3))} \\
      &= 0 + (-x) & \text{(by (2))} \\
      &< x + (-x) & \text{(by (6) since $0 < x$)} \\
      &= 0 \,. & \text{(by (4))}
    }

    $(\pmi)$ Suppose now that $-x < 0$.
    Then we have
    \ali{
      x &= x + 0 & \text{(by (3))} \\
      &= 0 + x & \text{(by (2))} \\
      &> -x + x & \text{(by (6) since $0 > -x$)} \\
      &= x + (-x) & \text{(by (2))} \\
      &= 0 & \text{(by (4))}
    }
    as desired.
  }

  (d) \partd
  \qproof{
    $(\imp)$ Suppose that $x > y$.
    Then we have
    \ali{
      -y &= -y + 0 & \text{(by (3))} \\
      &= -y + (x + (-x)) & \text{(by (4))} \\
      &= (x + (-x)) + (-y) & \text{(by (2))} \\
      &= x + (-x + (-y)) & \text{(by (1))} \\
      &> y + (-x + (-y)) & \text{(by (6) since $x > y$)} \\
      &= y + (-y + (-x)) & \text{(by (2))} \\
      &= (y + (-y)) + (-x) & \text{(by (1))} \\
      &= 0 + (-x) & \text{(by (4))} \\
      &= -x + 0 & \text{(by (2))} \\
      &= -x \,. & \text{(by (3))}
    }

    $(\pmi)$ Now suppose that $-x < -y$.
    Then we have
    \ali{
      x &= x + 0 & \text{(by (3))} \\
      &= x + (y + (-y)) & \text{(by (4))} \\
      &= (y + (-y)) + x & \text{(by (2))} \\
      &= (-y + y) + x & \text{(by (2))} \\
      &= -y + (y + x) & \text{(by (1))} \\
      &> -x + (y + x) & \text{(by (6) since $-y > -x$)} \\
      &= -x + (x + y) & \text{(by (2))} \\
      &= (-x + x) + y & \text{(by (1))} \\
      &= (x + (-x)) + y & \text{(by (2))} \\
      &= 0 + y & \text{(by (4))} \\
      &= y + 0 & \text{(by (2))} \\
      &= y & \text{(by (3))}
    }
    as desired.
  }

  (e) \parte
  \qproof{
    First, by Exercise~4.1d, we have $-(-z) = z < 0$ so that $-z > 0$ by part (c).
    Then, since $x > y$, it follows from (6) that
    \ali{
      x(-z) &> y(-z) \\
      -(xz) &> -(yz) & \text{(by Exercise~4.1e applied to both sides)} \\
      xz &< yz & \text{(by part (d))} 
    }
    as desired.
  }

  (f) \partf
  \qproof{
    Since $x \neq 0$ we either have that $x> 0$ or $x < 0$ since the $<$ relation is an order (in particular a linear order since this is part of the definition of order in this text).
    If $x > 0$ then we have $x^2 = x \cdot x > 0 \cdot x = 0$ by (6) (since $x > 0$) and Exercise~4.1b.
    If $x < 0$ then we have $0 = 0 \cdot x < x \cdot x = x^2$ by part (e) (since $0 > x$) and Exercise~4.1b.
    Together these show the desired result.
  }

  (g) \partg
  \qproof{
    By (4) we know that $1 \neq 0$ so that $1^2 > 0$ by part (f).
    However, we have $1^2 = 1 \cdot 1 = 1$ by (3).
    Hence $1 = 1^2 > 0$.
    It then follows from part (c) that $-1 < 0$ so that we have $-1 < 0 < 1$ as desired.
  }

  (h) \parth
  \qproof{
    $(\imp)$ Suppose that $xy > 0$.
    It cannot be that $x = 0$, for then we would have $0 = 0 \cdot y = xy > 0$ by Exercise~4.1b, which is impossible by the definition of an order.
    Hence we have $x \neq 0$, and an analogous argument shows that $y \neq 0$ as well.
    We then have the following:

    Case: $x > 0$.
    Suppose that $y < 0$.
    Then, by part (e) and Exercise~4.1b, we have $xy < 0 \cdot y = 0$ since $x > 0$ and $y < 0$, which contradicts our initial supposition.
    Thus, since we know that $y \neq 0$, it has to be that $y > 0$ as well.

    Case: $x < 0$.
    Suppose that $y > 0$.
    Then, by (6) and Exercise~4.1b, we have $0 = 0 \cdot y > xy$ since $0 > x$ and $y > 0$, which again contradicts the initial supposition.
    So it must be that $y < 0$ also since $y \neq 0$.

    Therefore in every case either both $x$ and $y$ are positive or they are both negative.
    Since $x \neq 0$, these cases are exhaustive so that this shows the result.

    $(\pmi)$ Suppose that either $x > 0, y > 0$ or $x < 0, y < 0$.
    In the case where both $x>0$  and $y>0$ we clearly have $xy > 0 \cdot y = 0$ by (6) and Exercise~4.1b.
    In the other case in which $x<0$ and $y<0$ we have $0 = 0 \cdot y < xy$ by part (e) and Exercise~4.1b since $0 > x$ and $y < 0$.
    Hence $xy > 0$ in both cases.
  }

  (i) \parti
  \qproof{
    First, it cannot be that $1/x=0$ because then we would have $1 = x(1/x) = x \cdot 0 = 0 \cdot x = 0$ by (4), (2), and Exercise~4.1b.
    This is clearly a contradiction since we know that $1 \neq 0$ by (3).
    Hence $1/x \neq 0$.
    Now suppose that $1/x<0$ so that $1 = x(1/x) < 0 \cdot (1/x) = 0$ by part (e) since $x>0$ and $1/x<0$, and we have also used Exercise~4.1b.
    This is also a contradiction since it was proved in part (g) that $1>0$.
    Hence the only remaining possibility is that $1/x>0$ as desired.
  }

  (j) \partj
  \qproof{
    First, since the order is transitive, we have $x,y > 0$.
    It then follows from part (i) that $1/x,1/y > 0$.
    Then $(1/x)(1/y) > 0$ by part (h).
    We then have
    \ali{
      \xr &= \xr \cdot 1 & \text{(by (3))} \\
      &= \xr \parens{y \cdot \yr} & \text{(by (4))} \\
      &= \parens{\xr \cdot y} \yr & \text{(by (1))} \\
      &= \parens{y \cdot \xr} \yr & \text{(by (2))} \\
      &= y \parens{\xr \cdot \yr} & \text{(by (1))} \\
      &< x \parens{\xr \cdot \yr} & \text{(by (6) since $y<x$ and $(1/x)(1/y)>0$)} \\
      &= \parens{x \cdot \xr} \yr & \text{(by (1))} \\
      &= 1 \cdot \yr & \text{(by (4))} \\
      &= \yr \cdot 1 & \text{(by (2))} \\
      &= \yr & \text{(by (3))}
    }
    as desired.
  }

  (k) \partk
  \qproof{
    First, we know by part (g) that $1 > 0$ so that
    \ali{
      2 &= 1 + 1 & \text{(by the definition of 2)} \\
      &> 0 + 1 & \text{(by (6) since $1>0$ )} \\
      &= 1 + 0 & \text{(by (2))} \\
      &= 1 & \text{(by (3))} \\
      &> 0 \,. & \text{(by part (g))}
    }
    To summarize, $0 < 1 < 2$.
    It then follows from part (i) that $1/2 > 0$.
    We then have
    \ali{
      x &< y \\
      x + x &< x + y & \text{(by (6))} \\
      2x &< x + y & \text{(by Lemma~\ref{lem:intreal:xpx})} \\
      (2x)\frac{1}{2} &< (x+y)\frac{1}{2} & \text{(by (6) since $1/2>0$)} \\
      (x \cdot 2) \frac{1}{2} &< \frac{x+y}{2} & \text{(by (2) and the definition of division)} \\
      x \parens{2 \cdot \frac{1}{2}} &< \frac{x+y}{2} & \text{(by (1))} \\\
      x \cdot 1 &< \frac{x+y}{2} & \text{(by (4))} \\
      x &< \frac{x+y}{2}\,. & \text{(by (3))}
    }
    Similarly, we have
    \ali{
      x &< y \\
      x + y &< y + y & \text{(by (6))} \\
      x + y &< 2y & \text{(by Lemma~\ref{lem:intreal:xpx})} \\
      (x+y)\frac{1}{2} &< (2y)\frac{1}{2} & \text{(by (6) since $1/2>0$)} \\
      \frac{x+y}{2} &< (y \cdot 2) \frac{1}{2} & \text{(by the definition of division and (2))} \\
      \frac{x+y}{2} &< y \parens{2 \cdot \frac{1}{2}} & \text{(by (1))} \\
      \frac{x+y}{2} &< y \cdot 1 & \text{(by (4))} \\
      \frac{x+y}{2} &< y \,. & \text{(by (3))}
    }
    This shows that $x < (x+y)/2 < y$ as desired.
  }
}

\def\inds{\mathcal{A}}
\def\intA{\bigcap_{A \in \inds} A}
\def\intB{\bigcap_{B \in \inds} B}
\exercise{3}{
  \eparts{
  \item Show that if $\inds$ is a collection of inductive sets, then the intersection of the elements of $\inds$ is an inductive set.
  \item Prove the basic properties (1) and (2) of $\pints$.
  }
}
\sol{
  \dwhitman

  (a) We must show that $\intA$ is inductive.
  \qproof{
    First, consider any $A \in \inds$.
    Then, since $A$ is inductive, $1 \in A$.
    Since $A$ was arbitrary, this shows that $1 \in \intA$.
    Now suppose that $x \in \intA$ and again consider arbitrary $A \in \inds$.
    Then $x \in A$ so that $x+1 \in A$ also since $A$ is inductive.
    Since $A$ was arbitrary, this shows that $x+1 \in \intA$.
    Hence by definition $\intA$ is inductive.
  }

  (b)
  \qproof{
    Let $\inds$ be the collection of all inductive sets of $\reals$ so that by definition $\pints = \intA$.
    It then follows immediately from part (a) that $\pints$ is inductive since $\inds$ is a collection of inductive sets.
    This shows property (1).

    Now suppose that $A$ is an inductive set of positive integers.
    That is, $A$ is inductive and $A \ss \pints$.
    Consider any $x \in \pints = \intB$, where again $\inds$ is the the collection of all inductive subsets of $\reals$.
    Clearly we have that $A \ss \pints \ss \reals$ so that $A \in \inds$ since $A$ is an inductive subset of $\reals$.
    Hence $x \in A$ (since $x \in \intB$ and $A \in \inds$) so that $\pints \ss A$ since $x$ was arbitrary.
    This shows that $A = \pints$ as desired since also $A \ss \pints$.
    This shows property (2).
  }
}

\exercise{4}{
  \eparts{
  \item Prove by induction that given $n \in \pints$, every nonempty subset of $\braces{1,\ldots,n}$ has a largest element.
  \item Explain why you cannot conclude from (a) that every nonempty subset of $\pints$ has a largest element.
  }
}
\sol{
  \dwhitman

  (a)
  \qproof{
    Let $A$ be the set of integers such that the hypothesis is true.
    Clearly the result is then shown if we can prove that $A = \pints$.
    So first, clearly $1 \in A$ since the set $\braces{1}$ has only a single nonempty subset, i.e. $\braces{1}$ itself, in which 1 is clearly the largest element.
    Now suppose that $n \in A$ so that every nonempty subset of $S_{n+1} = \braces{1,\ldots,n}$ has a largest element.
    Consider any nonempty subset $B$ of $S_{n+2} = \braces{1,\ldots,n+1}$, noting that $S_{n+2} = S_{n+1} \cup \braces{n+1}$.

    Case: $n+1 \in B$.
    Then, for any other $k \in B$, $k \in S_{n+2}$ so that either $k = n+1$ or $k \in S_{n+1}$ so that $k < n+1$ by the definition of $S_{n+1}$.
    Thus in either case $k \leq n+1$ so that $n+1$ is the greatest element of $B$ since $k$ was arbitrary.

    Case: $n+1 \notin B$.
    Then clearly $B \ss S_{n+1}$ so that $B$ has a greatest element by the induction hypothesis since $B$ is nonempty.

    Hence in either case $B$ has a greatest element so that $n+1 \in A$ since $B$ was an arbitrary nonempty subset of $S_{n+2} = \braces{1, \ldots, n+1}$.
    This shows that $A$ is an inductive set of positive integers so that $A = \pints$ as desired by the Principle of Induction.
  }

  (b) There could be nonempty subsets of $\pints$ that are \emph{not} subsets of $S_{n+1} = \braces{1, \ldots, n}$ for any $n \in \pints$, in which cases the hypothesis of part (a) is not satisfied so that the conclusion does not necessarily apply.
  In fact, $\pints$ itself is an example of such a set where both the hypothesis and the conclusion are false.
}

\def\pintsz{\pints \cup \braces{0}}
\exercise{5}{
  Prove the following properties of $\ints$ and $\pints$:
  \eparts{
  \item $a,b \in \pints \imp a + b \in \pints$.
    [Hint: Show that given $a \in \pints$, the set $X = \braces{x \where x \in \reals \text{ and } a + x \in \pints}$ is inductive.]
  \item $a,b \in \pints \imp a \cdot b \in \pints$.
  \item Show that $a \in \pints \imp a-1 \in \pints \cup \braces{0}$.
    [Hint: Let $X = \braces{x \where x \in \reals \text{ and } x-1 \in \pints \cup \braces{0}}$; show that $X$ is inductive.]
  \item $c,d \in \ints \imp c+d \in \ints$ and $c-d \in \ints$.
    [Hint: Prove it first for $d=1$.]
  \item $c,d \in \ints \imp c \cdot d \in \ints$.
  }
}
\sol{
  \dwhitman

  \begin{lem}\label{lem:intreal:negz}
    If $x \in \ints$ then $-x \in \ints$.
  \end{lem}
  \qproof{
    Let $\nints = \braces{-x \where x \in \pints}$ so that by definition $\ints = \pintsz \cup \nints$.
    Suppose that $x \in \ints$ so that $x \in \pintsz \cup \nints$.

    Case: $x \in \pints$.
    Then $-x \in \nints$ by definition.

    Case: $x = 0$.
    Then by Exercise~4.1c we have $-x = -0 = 0 \in \braces{0}$.

    Case: $x \in \nints$.
    Then by definition there is a $y \in \pints$ such that $x = -y$.
    Then $-x = -(-y) = y \in \pints$ by Exercise~4.1d.

    Hence in all cases either $-x \in \pints$, $-x \in \braces{0}$, or $-x \in \nints$ so that $-x \in \pintsz \cup \nints = \ints$ as desired.
  }

  \mainprob
  
  (a)
  \qproof{
    Consider any $a \in \pints$ and define $X_a = \braces{x \in \reals \where a+x \in \pints}$.
    We show that $X_a$ is inductive.
    First, since $a \in \pints$ we have that $a+1 \in \pints$ since $\pints$ is inductive.
    Hence $1 \in X_a$ by definition.
    Now suppose that $x \in X_a$ so that $a+x \in \pints$.
    Then we have $a+(x+1) = (a+x)+1 \in \pints$ since $a+x \in \pints$ and $\pints$ is inductive.
    This shows by definition that $x+1 \in X_a$ and therefore that $X_a$ is inductive.
    It follows that $\pints \ss X_a$ since $\pints$ is defined as the intersection of all inductive subsets of reals, of which $X_a$ is one.

    Therefore, for any $a,b \in \pints$, we have that $b \in X_a$ since $\pints \ss X_a$.
    Thus by definition $a + b \in \pints$ as desired
  }

  (b)
  \qproof{
    Consider any $a \in \pints$ and define $X_a = \braces{x \in \reals \where a \cdot x \in \pints}$.
    We show that $X_a$ is inductive.
    To this end, we first have that $a \cdot 1 = a \in \pints$ so that $1 \in X_a$ by definition.
    Now suppose that $x \in X_a$ so that $ax \in \pints$.
    Then we have $a \cdot (x+1) = a \cdot x + a \cdot 1 = ax + a \in \pints$ by part (a) since we know both $ax$ and $a$ are in $\pints$.
    Hence $x+1 \in X_a$ by definition.
    This shows that $X_a$ is inductive so that again $\pints \ss X_a$.

    Hence for any $a,b \in \pints$ we have that $b \in X_a$ since $\pints \ss X_a$.
    It then follows by definition that $a \cdot b \in \pints$ as desired.
  }

  (c)
  \qproof{
    Let $X = \braces{x \in \reals \where x-1 \in \pintsz}$, which we show is inductive.
    First, we have $1 - 1 = 1 + (-1) = 0$ so that clearly $1 \in \pintsz$ and hence $1 \in X$.
    Now suppose that $x \in X$ so that $x-1 \in \pintsz$.

    Case: $x-1 \in \braces{0}$.
    Then it must be that $x - 1 = 0$, which clearly implies $x = 1 \in \pints$ since $\pints$ is inductive.
    Then $(x+1) - 1 = x + (1 - 1) = x + 0 = x \in \pints$ so that $(x+1)-1 \in \pintsz$ and therefore $x+1 \in X$.

    Case: $x-1 \in \pints$.
    Then $(x+1) - 1 = x + (1 -1) = x + ((-1) + 1) = (x - 1) + 1 \in \pints$ since $x-1 \in \pints$ and $\pints$ is inductive.
    Thus clearly $(x+1)-1 \in \pintsz$ so that $x+1 \in X$ by definition.

    Hence in both cases $x+1 \in X$, which shows that $X$ is inductive, and so $\pints \ss X$.
    Therefore, for any $a \in \pints$, we have that also $x \in X$ since $\pints \ss X$.
    Then, by the definition of $X$, it follows that $a-1 \in \pintsz$ as desired.
  }

  (d)
  \qproof{
    First we show that the set $X_c = \braces{x \in \reals \where c+x \in \ints \text{ and } c-x \in \ints}$ is inductive for any $c \in \ints$.
    So consider any $c$ and $b$ in $\ints$ so that $c,b \in \pintsz \cup \nints$.

    Case: $b \in \pints$.
    Then $b+1 \in \pints$ since $\pints$ is inductive and $b-1 \in \pintsz$ by part (c).

    Case: $b = 0$.
    Then $b + 1 = 0 + 1 = 1 \in \pints$ since it is inductive, and $b - 1 = 0 - 1 = -1 \in \nints$ since $1 \in \pints$.

    Case: $b \in \nints$.
    Then $b = -a$ for $a \in \pints$, and we then have that $a+1 \in \pints$ since $\pints$ is inductive.
    Hence $b - 1 = -a - 1 = -(a+1) \in \nints$.
    We also have that $a-1 \in \pintsz$ by part (c), from which it is trivial to show that $-(a-1) \in \nints \cup \braces{0}$.
    Therefore $b + 1 = -a + 1 = -(a - 1) \in \nints \cup \braces{0}$.

    Thus in all cases we have that $b+1$ and $b-1$ are in $\pints$ or $\braces{0}$ or $\nints$ so that they are both in $\ints$, and so $1 \in X_b$.
    Note that this is the case for any $b \in \ints$ so that it is clearly true for $c$, i.e. $1 \in X_c$.
    Now suppose that $x \in X_c$ so that $c+x$ and $c-x$ are both in $\ints$.
    It then follows that $1 \in X_{c+x}$ and $1 \in X_{c-x}$ so that $c + (x+1) = (c+x) + 1 \in \ints$ and $c - (x + 1) = (c -x) - 1 \in \ints$.
    This then shows that $x+1 \in X_c$.
    Hence $X_c$ is inductive for any $c \in \ints$ so that $\pints \ss X_c$.

    Now consider $c,d \in \ints$.

    Case: $d \in \pints$.
    Then clearly $d \in X_c$ since $\pints \ss X_c$.
    Hence by definition $c+d$ and $c-d$ are both in $\ints$.

    Case: $d = 0$.
    Then $c + d = c + 0 = c \in \ints$ and $c - d = c - 0 = c \in \ints$.

    Case: $d \in \nints$.
    Then by definition $d = -a$ for $a \in \pints$ so that $a \in X_c$ since $\pints \ss X_c$.
    Then $c + a$ and $c - a$ are both in $\ints$ by the definition of $X_c$.
    Hence $c + d = c + (-a) = c - a \in \ints$ and $c - d = c - (-a) = c + a \in \ints$.

    Therefore we have shown that $c+d$ and $c-d$ are both integers in all cases, which is the desired result.
  }

  (e)
  \qproof{
    For any $c \in \ints$, define $X_c = \braces{x \in \reals \where c \cdot x \in \ints}$.
    We first show that $X_c$ is inductive for any such $c \in \ints$.
    We have $c \cdot 1 = c \in \ints$ so that $1 \in X_c$.
    Now suppose that $x \in X_c$ so that $c \cdot x \in \ints$.
    Then $c \cdot(x+1) = c \cdot x + c \cdot 1 = c \cdot x + c \in \ints$ by part (d) since both $c \cdot x$ and $c$ are integers.
    This shows that $X_c$ is inductive so that $\pints \ss X_c$.
    
    Now consider any $c,d \in \ints$.

    Case: $d \in \pints$.
    Then $d \in X_c$ since $\pints \ss X_c$.
    Thus $c \cdot d \in \ints$.

    Case: $d = 0$.
    The $c \cdot d = c \cdot 0 = 0 \in \ints$.

    Case: $d \in \nints$.
    Then there is an $a \in \pints$ such that $d = -a$.
    Hence $a \in X_c$ since $\pints \ss X_c$, from which it follows that $c \cdot a \in \ints$.
    We then have $c \cdot d = c \cdot (-a) = -(c \cdot a) \in \ints$ as well by Lemma~\ref{lem:intreal:negz}.

    Thus in all cases $c \cdot d \in \ints$ as desired.
  }
}

\exercise{6}{
  Let $a \in \reals$.
  Define inductively
  \ali{
    a^1 &= a \,, \\
    a^{n+1} &= a^n \cdot a
  }
  for $n \in \pints$.
  (See \S 7 for a discussion of the process of inductive definition.)
  Show that for $n,m \in \pints$ and $a,b \in \reals$,
  \ali{
    a^n a^m &= a^{n+m} \\
    \parens{a^n}^m &= a^{nm} \\
    a^m b^m &= (ab)^m \,.
  }
  These are called the \textbf{\emph{laws of exponents}}.
  [Hint: For fixed $n$, prove the formulas by induction on $m$.]
}
\sol{
  \dwhitman

  The following Lemma is the familiar proof by induction, which is more straightforward than having to frame everything in terms of inductive sets.
  Henceforth we use this whenever induction is required.
  \begin{lem}
    (Proof by Induction)
    Suppose that $P(x)$ is a statement with parameter $x$.
    Suppose also that $P(1)$ is true and that $P(x)$ implies $P(x+1)$.
    Then $P(n)$ is true for all $n \in \pints$.
  \end{lem}
  \qproof{
    Define the set $X = \braces{x \in \reals \where P(x)}$.
    We show that $X$ is inductive.
    Clearly since $P(1)$ is true we have $1 \in X$.
    Now suppose that $x \in X$ so that $P(x)$ is true.
    Then $P(x+1)$ is also true so that $x+1 \in X$.
    This shows that $X$ is inductive so that $\pints \ss X$.
    So, for any positive integer $n$ we have that $n \in X$ since $\pints \ss X$.
    Therefore $P(n)$ is true.
    Since $n$ was arbitrary, this shows the desired result.
  }

  \mainprob

  In what follows, suppose that $a,b \in \reals$.

  First we show that $a^n a^m = a^{n+m}$ for all $n,m \in \pints$.
  \qproof{
    Fix $n \in \pints$.
    We show the result by induction on $m$.
    First, we clearly have $a^n a^1 = a^n \cdot a = a^{n+1}$ by the inductive definition.
    Now suppose that $a^n a^m = a^{n+m}$.
    Then
    \ali{
      a^n a^{m+1} &= a^n \cdot (a^m \cdot a) & \text{(by the inductive definition)} \\
      &= (a^n a^m) \cdot a & \text{(\multass)} \\
      &= a^{n+m} \cdot a & \text{(by the induction hypothesis)} \\
      &= a^{(n+m)+1} & \text{(by the inductive definition)} \\
      &= a^{n + (m+1)} \,, & \text{(since addition is associative)}
    }
    which completes the induction step.
    Therefore the result holds for all $m \in \pints$ by induction.
  }

  Next we show that $\parens{a^n}^m = a^{nm}$ for all $n,m \in \pints$.
  \qproof{
    We again fix $n \in \pints$ and use induction on $m$.
    First, we have $\parens{a^n}^1 = a^n = a^{n \cdot 1}$ by the inductive definition.
    Supposing now that $\parens{a^n}^m = a^{n \cdot m}$, we have
    \ali{
      \parens{a^n}^{m+1} &= \parens{a^n}^m \cdot a^n & \text{(by the inductive definition)} \\
      &= a^{n \cdot m} a^n & \text{(by the induction hypothesis)} \\
      &= a^{n \cdot m + n} & \text{(by what was shown above)} \\
      &= a^{n \cdot m + n \cdot 1} \\
      &= a^{n \cdot (m+1)} \,. & \text{(by the distributive property)}
    }
    This completes the induction so that the result holds for all $m \in \pints$.
  }

  Lastly, we show that $a^m b^m = (ab)^m$ for all $m \in \pints$.
  \qproof{
    We show this by induction on $m$.
    First, we have $a^1 b^1 = ab = (ab)^1$ by the inductive definition.
    Now suppose that $a^m b^m = (ab)^m$ so that
    \ali{
      a^{m+1} b^{m+1} &= (a^m \cdot a)(b^m \cdot b) & \text{(by the inductive definition)} \\
      &= (a \cdot a^m)(b^m \cdot b) & \text{(\multcom)} \\
      &= ((a \cdot a^m) b^m) \cdot b & \text{(\multass)} \\
      &= (a \cdot (a^m b^m)) \cdot b & \text{(\multass)} \\
      &= (a (ab)^m) \cdot b & \text{(by the induction hypothesis)} \\
      &= ((ab)^m a) \cdot b & \text{(\multcom)} \\
      &= (ab)^m (ab) & \text{(\multass)} \\
      &= (ab)^{m+1} \,. & \text{(by the inductive definition)}
    }
    This completes the induction.
  }
}

\exercise{7}{
  Let $a \in \reals$ and $a \neq 0$.
  Define $a^0 = 1$, and for $n \in \pints$, $a^{-n} = 1/a^n$.
  Show that the laws of exponents hold for $a,b \neq 0$ and $n,m \in \ints$.
}
\sol{
  \dwhitman

  \begin{lem}\label{lem:intreal:expone}
    For any $n \in \ints$, $1^n = 1$.
  \end{lem}
  \qproof{
    We show this for $n \in \pints$ by simple induction on $n$.
    First, clearly $1^1 = 1$ by the inductive definition of exponentiation.
    Next, if $1^n = 1$, then we have $1^{n+1} = 1^n \cdot 1 = 1^n = 1$ by the inductive definition of exponentiation and the inductive hypothesis.
    This completes the induction so that the result holds for all $n \in \pints$.

    Clearly if $n = 0$ then, by the definition of 0 as an exponent, $1^n = 1^0 = 1$.

    Lastly, if $n \in \nints$ then there is a $k \in \pints$ where $n = -k$.
    Then we have
    \ali{
      1^n &= 1^{-k} \\
      &= \frac{1}{1^k} & \text{(by the definition of negative exponentiation)} \\
      &= \frac{1}{1} & \text{(by what was just shown by induction since $k \in \pints$)} \\
      &= 1\,. & \text{(since 1 is its own reciprocal)}
    }
    Thus the result has been shown for all the resulting cases when $n \in \ints$.
  }
  
  \begin{lem}\label{lem:intreal:recpow}
    $1/a^n = (1/a)^n$ for any real $a \neq 0$ and $n \in \pints$.
  \end{lem}
  \qproof{
    We have
    \ali{
      \parens{\frac{1}{a}}^n a^n &= \parens{\frac{1}{a} \cdot a}^n & \text{(by Exercise~4.6 since $n \in \pints$)} \\
      &= 1^n & \text{(by the definition of the reciprocal)} \\
      &= 1\,. & \text{(by Lemma~\ref{lem:intreal:expone})}
    }
    Thus $(1/a)^n$ must be the unique reciprocal of $a^n$, that is $(1/a)^n = 1/a^n$ as desired.
  }

  \begin{lem}\label{lem:intreal:negexp}
    $a^n a^{-n} = 1$ for any real $a \neq 0$ and $n \in \pints$.
  \end{lem}
  \qproof{
    We have
    \ali{
      a^n a^{-n} &= a^n \parens{\frac{1}{a^n}} & \text{(by the definition of negative exponentiation)} \\
      &= a^n \parens{\frac{1}{a}}^n & \text{(by Lemma~\ref{lem:intreal:recpow})} \\
      &= \parens{a \cdot \frac{1}{a}}^n & \text{(by Exercise~4.6 since $n \in \pints$)} \\
      &= 1^n & \text{(by the definition of the reciprocal)} \\
      &= 1 & \text{(by Lemma~\ref{lem:intreal:expone})}
    }
    as desired.
  }

  \mainprob

  First we show that $a^n a^m = a^{n+m}$ for all real $a \neq 0$ and $n,m \in \ints$.
  \qproof{
    Consider any real $a \neq 0$ and $n,m \in \ints$.
    We number the following cases for reference:
    \begin{enumerate}
    \item Case: $n \in \pints$.
      \begin{enumerate}
      \item Case: $m \in \pints$.
        Then the result immediately applies by Exercise~4.6.
      \item Case: $m = 0$.
        Then we have $a^n a^m = a^n a^0 = a^n \cdot 1 = a^n = a^{n+0} = a^{n+m}$.
      \item Case: $m \in \nints$.
        Then $m = -k$ for some $k \in \pints$.
        \begin{enumerate}
        \item Case: $n > k$.
          Then $n-k > 0$ so that $n-k \in \pints$ since $n-k \in \ints$ by Exercise~4.5d.
          We then have
          \ali{
            a^n a^m &= a^n a^{-k} \\
            &= a^{n-k+k} a^{-k} & \text{(since $n = n + 0 = n -k + k$)} \\
            &= (a^{n-k} a^k) a^{-k} & \text{(by Exercise~4.6 since $k,n-k \in \pints$)} \\
            &= a^{n-k} (a^k a^{-k}) & \text{(\multass)} \\
            &= a^{n-k} \cdot 1 & \text{(by Lemma~\ref{lem:intreal:negexp} since $k \in \pints$)} \\
            &= a^{n-k} \\
            &= a^{n+m} \,.
          }
        \item Case: $n = k$.
          Then clearly $n+m = n-k = k-k = 0$, so that we have $a^n a^m = a^k a^{-k} = 1 = a^0 = a^{n+m}$ by Lemma~\ref{lem:intreal:negexp} and the definition of 0 as an exponent.
        \item Case: $n < k$.
          Then $n-k < 0$ so that $n-k \in \nints$ since $n-k \in \ints$ by Exercise~4.5d.
          Also, clearly $-n \in \nints$ since $n \in \pints$.
          Then we have
          \ali{
            a^n a^m &= a^n a^{-k} \\
            &= a^n a^{-k+n-n} & \text{(since $-k = -k + 0 = -k + n - n$)} \\
            &= a^n a^{n-k-n} & \text{(since addition is commutative)} \\
            &= a^n (a^{n-k} a^{-n}) & \text{(by case 3c below since $n-k,-n \in \nints$)} \\
            &= a^n (a^{-n} a^{n-k}) & \text{(\multcom)} \\
            &= (a^n a^{-n}) a^{n-k} & \text{(\multass)} \\
            &= 1 \cdot a^{n-k} & \text{(by Lemma~\ref{lem:intreal:negexp})} \\
            &= a^{n-k} \\
            &= a^{n+m} \,.
          }
        \end{enumerate}
      \end{enumerate}
    \item Case: $n = 0$.
      \begin{enumerate}
      \item Case: $m \in \pints$.
        Since $a^n a^m = a^m a^n$ and $a^{n+m} = a^{m+n}$, this the same as case 1b above.
      \item Case: $m = 0$.
        Then we have $a^n a^m = a^0 a^0 = 1 \cdot 1 = 1 = a^0 = a^{0+0} = a^{n+m}$.
      \item Case: $m \in \nints$.
        Then there is a $k \in \pints$ such that $m = -k$, and $a^n a^m = a^0 a^{-k} = 1 \cdot (1/a^k) =  1/a^k = a^{-k} = a^m = a^{0+m} = a^{n+m}$.
      \end{enumerate}
    \item Case: $n \in \nints$.
      \begin{enumerate}
      \item Case: $m \in \pints$.
        This is the same as case 1c above.
      \item Case: $m = 0$.
        This is the same as case 2c above.
      \item Case: $m \in \nints$.
        Here we have that $n = -k$ and $m = -l$ for some $k,l \in \pints$.
        Hence we have
        \ali{
          a^n a^m &= a^{-k} a^{-l} \\
          &= \parens{\frac{1}{a^k}}\parens{\frac{1}{a^l}} & \text{(by the definition of negative exponents)} \\
          &= \parens{\frac{1}{a}}^k \parens{\frac{1}{a}}^l & \text{(by Lemma~\ref{lem:intreal:recpow})} \\
          &= \parens{\frac{1}{a}}^{k+l} & \text{(by Exercise~4.6 since $k,l \in \pints$)} \\
          &= \frac{1}{a^{k+l}} & \text{(by Lemma~\ref{lem:intreal:recpow})} \\
          &= a^{-(k+l)} & \text{(by the definition of negative exponents)} \\
          &= a^{-k-l} \\
          &= a^{n+m} \,.
        }
      \end{enumerate}
    \end{enumerate}

    Thus in all cases we have shown the result.
  }

  Next we show that $(a^n)^m = a^{n m}$ for all real $a \neq 0$ and $n,m \in \ints$.
  \qproof{
    Consider any real $a \neq 0$ and $n,m \in \ints$.
    We again number the cases for reference:
    \begin{enumerate}
    \item Case: $n \in \pints$.
      \begin{enumerate}
      \item Case: $m \in \pints$.
        Then the result immediately applies by Exercise~4.6.
      \item Case: $m = 0$.
        Then we have $(a^n)^m = (a^n)^0 = 1 = a^0 = a^{n \cdot 0} = a^{nm}$ by the definition of a 0 exponent.
      \item Case: $m \in \nints$.
        Then there is a $k \in \pints$ such that $m = -k$.
        Then we have
        \ali{
          (a^n)^m &= (a^n)^{-k} \\
          &= \frac{1}{(a^n)^k} & \text{(by the definition of negative exponents)} \\
          &= \frac{1}{a^{nk}} & \text{(by Exercise~4.6 since $n,k \in \pints$)} \\
          &= a^{-(nk)} & \text{(by the definition of negative exponents)} \\
          &= a^{n(-k)} \\
          &= a^{nm} \,.
        }
      \end{enumerate}
    \item Case: $n = 0$.
      Then we have $(a^n)^m = (a^0)^m = 1^m = 1 = a^0 = a^{0 \cdot m} = a^{nm}$ by the definition of 0 as an exponent and Lemma~\ref{lem:intreal:expone}.
    \item Case: $n \in \nints$.
      Then $n = -k$ for some $k \in \pints$.
      \begin{enumerate}
      \item Case: $m \in \pints$.
        Then we have
        \ali{
          (a^n)^m &= (a^{-k})^m \\
          &= \parens{\frac{1}{a^k}}^m & \text{(by the definition of negative exponents)} \\
          &= \squares{\parens{\frac{1}{a}}^k}^m & \text{(by Lemma~\ref{lem:intreal:recpow})} \\
          &= \parens{\frac{1}{a}}^{km} & \text{(by Exercise~4.6 since $k,m \in \pints$)} \\
          &= \frac{1}{a^{km}} & \text{(by Lemma~\ref{lem:intreal:recpow})} \\
          &= a^{-(km)} & \text{(by the definition of negative exponents)} \\
          &= a^{(-k)m} \\
          &= a^{nm} \,.
        }
      \item Case: $m = 0$.
        The same argument as in case 1b above applies here as it does not depend on $n$ being positive.
      \item Case: $m \in \nints$.
        Then $m = -l$ for some $l \in \pints$, and we have
        \ali{
          (a^n)^m &= (a^{-k})^{-l} \\
          &= \parens{\frac{1}{a^k}}^{-l} & \text{(by the definition of negative exponents)} \\
          &= \frac{1}{(1/a^k)^l} & \text{(by the definition of negative exponents)} \\
          &= \frac{1}{[(1/a)^k]^l} & \text{(by Lemma~\ref{lem:intreal:recpow})} \\
          &= \frac{1}{(1/a)^{kl}} & \text{(by Exercise~4.6 since $k,l \in \pints$)} \\
          &= \parens{\frac{1}{1/a}}^{kl} & \text{(by Lemma~\ref{lem:intreal:recpow})} \\
          &= a^{kl} \\
          &= a^{(-k)(-l)} \\
          &= a^{nm} \,.
        }
      \end{enumerate}
    \end{enumerate}

    Thus in all cases we have shown the result.
  }

  Lastly, we show that $a^m b^m = (ab)^m$ for all real $a,b \neq 0$ and $m \in \ints$.
  \qproof{
    We have the following cases:

    Case: $m \in \pints$.
    The result then follows immediately from Exercise~4.6.

    Case: $m = 0$.
    Then we have $a^m b^m = a^0 b^0 = 1 \cdot 1 = 1 = (ab)^0 = (ab)^m$ by the definition of a 0 exponent.

    Case: $m \in \nints$.
    Then there is a $k \in \pints$ such that $m = -k$.
    Then we have
    \ali{
      a^m b^m &= a^{-k} b^{-k} \\
      &= \frac{1}{a^k} \cdot \frac{1}{b^k} & \text{(by the definition of negative exponents)} \\
      &= \parens{\frac{1}{a}}^k \parens{\frac{1}{b}}^k & \text{(by Lemma~\ref{lem:intreal:recpow})} \\
      &= \parens{\frac{1}{a} \cdot \frac{1}{b}}^k & \text{(by Exercise~4.6 since $k \in \pints$)} \\
      &= \parens{\frac{1}{ab}}^k & \text{(by Exercise~4.1m)} \\
      &= \frac{1}{(ab)^k} & \text{(by Lemma~\ref{lem:intreal:recpow})} \\
      &= (ab)^{-k} & \text{(by the definition of negative exponents)} \\
      &= (ab)^m \,.
    }
    Therefore in all cases the result has been shown.
  }
}
