\setcounter{subsection}{5-1}
\subsection{Cartesian Products}

\exercise{1}{
  Show that there is a bijective correspondence of $A \times B$ with $B \times A$.
}
\sol{
  \dwhitman

  \qproof{
    We define a function $f: A \times B \to B \times A$.
    For any element $(a,b) \in A \times B$ we set $f(a,b) = (b,a)$, noting that of course $a \in A$ and $b \in B$.
    It should be obvious then that $f(a,b) = (b,a) \in B \times A$ so that $B \times A$ can be the range of $f$.

    \def\opx{(a_1, b_1)}
    \def\opy{(a_2, b_2)}
    \def\pox{(b_1, a_1)}
    \def\poy{(b_2, a_2)}
    First we show that $f$ is injective.
    To this end consider $\opx$ and $\opy$ in $A \times B$ where $\opx \neq \opy$.
    Of course we have that $f\opx = \pox$ and $f\opy = \poy$.
    Since $\opx \neq \opy$ clearly either $a_1 \neq a_2$ or $b_1 \neq b_2$.
    In either case it should be clear that $f\opx = \pox \neq \poy = f\opy$, which shows that $f$ is injective since $\opx$ and $\opy$ were arbitrary.

    It is very easy to that $f$ is also surjective since, for any $(b,a) \in B \times A$, clearly $(a,b) \in A \times B$ and $f(a,b) = (b,a)$.
    Hence $f$ is a bijection as desired.
    Note that if $A \times B = \es$ then $f = \es$ as well, which is vacuously a bijective function since it must be that $B \times A = \es$ as well (because either $A = \es$ or $B = \es$).
  }
}

\exercise{2}{
  \eparts{
  \item Show that if $n > 1$ there is a bijective correspondence of
    \gath{
      A_1 \times \cdots \times A_n \hspace{1cm}
      \text{with} \hspace{1cm}
      \parens{A_1 \times \cdots \times A_{n-1}} \times A_n \,.
    }
  \item Given the indexed family $\braces{A_1, A_2, \ldots}$, let $B_i = A_{2i-1} \times A_{2i}$ for each positive integer $i$.
    Show that there is a bijective correspondence of $A_1 \times A_2 \times \cdots$ with $B_1 \times B_2 \times \cdots$
  }
}
\sol{
  \dwhitman

  \begin{lem}\label{lem:cartp:eopints}
    If $n \in \pints$ is even, then $n/2 \in \pints$.
    If $n \in \pints$ is odd, then $(n+1)/2 \in \pints$.
  \end{lem}
  \qproof{
    First, suppose that $n \in \pints$ is even.
    Then by definition $n/2$ is an integer.
    However, since both $n$ and $1/2$ are positive, it follows from Exercise~4.2h that $n \cdot (1/2) = n/2$ is positive also so that $n/2 \in \pints$.

    Now, suppose that $n \in \pints$ is odd so that $n = 2k+1$ for some integer $k$ by Exercise~4.11a.
    Then
    \gath{
      \frac{n+1}{2} = \frac{(2k + 1) + 1}{2} = \frac{2k + 2}{2} = \frac{2(k+1)}{2} = k+1 \,,
    }
    which is clearly an integer since $k$ is.
    Moreover, we have $n+1 > n > 0$ since $n \in \pints$ and again $1/2 > 0$ so that $(n+1) \cdot (1/2) = (n+1)/2$ is positive by Exercise~4.2h.
    Thus $(n+1)/2 \in \pints$.
  }

  \mainprob
  
  (a)
  \qproof{
    For brevity, let $X = A_1 \times \cdots \times A_n$ and $Y = \parens{A_1 \times \cdots \times A_{n-1}} \times A_n$.
    Suppose that $n > 1$ so that $X$ and $Y$ make sense.
    We construct a bijective function $f: X \to Y$.
    For any $x = (x_1, \ldots x_n) \in X$ we have that $x_i \in A_i$ for $1 \leq i \leq n$.
    So set $f(x) = ((x_1, \ldots, x_{n-1}), x_n)$, which is clearly an element of $Y$.

    To see that $f$ is injective consider $x = (x_1, \ldots, x_n)$ and $y = (y_1, \ldots, y_n)$ in $X$ where $x \neq y$.
    It then follows that there must be an $i \in \braces{1, \ldots, n}$ where $x_i \neq y_i$.
    Let $x' = (x_1, \ldots, x_{n-1})$ and $y' = (y_1, \ldots, y_{n-1})$ so that clearly $f(x) = (x', x_n)$ and $f(y) = (y', y_n)$.
    Now, if $i = n$, then clearly $f(x) = (x', x_n) \neq (y', y_n) = f(y)$ since $x_n = x_i \neq y_i = y_n$.
    On the other hand, if $i \neq n$ then it has to be that $i < n$, and hence $i \leq n-1$.
    It then follows that $x' = (x_1, \ldots, x_{n-1}) \neq (y_1, \ldots y_{n-1}) = y'$ so that then $f(x) = (x', x_n) \neq (y', y_n) = f(y)$ again.
    Since $x$ and $y$ were arbitrary, this shows that $f$ is indeed injective.

    Now consider any $y = ((x_1, \ldots, x_{n-1}), x_n) \in Y$ and let $x = (x_1, \ldots x_n)$.
    It should be obvious that both $x \in X$ and $f(x) = y$ so that $f$ is surjective.
    Hence $f$ is a bijective function as desired.
  }

  (b)
  \qproof{
    First let $A = A_1 \times A_2 \times \cdots$ and $B = B_1 \times B_2 \times \cdots$.
    We construct a bijective $f: A \to B$.
    So, for any $a \in A$, we have that $a = (a_1, a_2, \ldots)$, where $a_i \in A_i$ for any $i \in \pints$.
    Then, for any $i \in \pints$, define $b_i = (a_{2i-1}, a_{2i})$ so that clearly $b_i \in A_{2i-1} \times A_{2i} = B_i$.
    We then have that $b = (b_1, b_2, \ldots) \in B_1 \times B_2 \times \cdots = B$.
    So set $f(a) = b$ so that $f$ is a function from $A$ to $B$.

    To show that $f$ is injective, consider $a = (a_1, a_2, \ldots)$ and $a' = (a_1', a_2', \ldots)$ in $A$ where $a \neq a'$.
    For each $i \in \pints$, define $b_i = (a_{2i-1}, a_{2i})$ and $b_i' = (a_{2i-1}', a_{2i}')$ as above and set $b = (b_1, b_2, \ldots)$ and $b' = (b_1', b_2', \ldots)$ so that clearly $f(a) = b$ and $f(a') = b'$.
    Since $a \neq a'$, it follows that there must be an $i \in \pints$ where $a_i \neq a_i'$.

    Case: $i$ is even.
    Then let $j = i/2$ so that $j \in \pints$ by Lemma~\ref{lem:cartp:eopints}.
    We also clearly have that $i = 2j$ so that $b_j = (a_{2j-1}, a_{2j}) \neq (a_{2j-1}', a_{2j}') = b_j'$ since $a_{2j} = a_i \neq a_i' = a_{2j}'$.

    Case: $i$ is odd.
    Then let $j = (i+1)/2$ so that $j \in \pints$ by Lemma~\ref{lem:cartp:eopints}.
    We then clearly have that $i = 2j-1$ so that $b_j = (a_{2j-1}, a_{2j}) \neq (a_{2j-1}', a_{2j}') = b_j'$ since $a_{2j-1} = a_i \neq a_i' = a_{2j-1}'$.

    Hence in all cases we have that there is a $j \in \pints$ where $b_j \neq b_j'$.
    It then follows that $f(a) = b = (b_1, b_2, \ldots) \neq (b_1', b_2', \ldots) = b' = f(a')$ so that $f$ is injective since $a$ and $a'$ were arbitrary.

    Lastly, to show that $f$ is surjective, consider any $b \in B$ so that $b = (b_1, b_2, \ldots)$ where $b_i \in B_i = A_{2i-1} \times A_{2i}$ for every $i \in \pints$.
    Then, for any $i \in \pints$,  $b_i = (a_i', a_i'')$ where $a_i' \in A_{2i-1}$ and $a_i'' \in A_{2i}$.
    So consider any $j \in \pints$.
    If $j$ is even, then $i = j/2 \in \pints$ by Lemma~\ref{lem:cartp:eopints}.
    Clearly also $j = 2i$.
    So, define $a_j = a_i''$ so that $a_j = a_{2i} = a_i'' \in A_{2i} = A_j$.
    On the other hand, if $j$ is odd, then $i = (j+1)/2 \in \pints$ again by Lemma~\ref{lem:cartp:eopints}.
    Then clearly $j = 2i-1$.
    So, here let $a_j = a_i'$ so that $a_j = a_{2i-1} = a_i' \in A_{2i-1} = A_j$.
    Hence $a_j \in A_j$ for all $j \in \pints$ so that $a = (a_1, a_2, \ldots) \in A$.
    Then, for any $i \in \pints$, we have $b_i = (a_i', a_i'') = (a_{2i-1}, a_{2i}) \in A_{2i-1} \times A_{2i} = B_i$ so that by definition $f(a) = b = (b_1, b_2, \ldots)$.
    This shows that $f$ is surjective since $b$ was arbitrary.

    This completes the proof that $f$ is bijective so that the desired result follows.
  }
}

\exercise{3}{
  Let $A = A_1 \times A_2 \times \cdots$ and $B = B_1 \times B_2 \times \cdots$.
  \eparts{
  \item Show that if $B_i \ss A_i$ for all $i$, then $B \ss A$.
    (Strictly speaking, if we are given a function mapping the index set $\pints$ into the union of the sets $B_i$, we must change its range before it can be considered as a function mapping $\pints$ into the union of the sets $A_i$.
    We shall ignore this technicality when dealing with cartesian products)
  \item Show the converse of (a) holds if $B$ is nonempty.
  \item Show that if $A$ is nonempty, each $A_i$ is nonempty.
    Does the converse hold?
    (We will return to this question in the exercises of \S 19.)
  \item What is the relation between the set $A \cup B$ and the cartesian product of the sets $A_i \cup B_i$?
    What is the relation between the set $A \cap B$ and the cartesian product of the sets $A_i \cap B_i$?
  }
}
\sol{
  \dwhitman

  (a)
  \qproof{
    Suppose that $b \in B$ so that $b = (b_1, b_2, \ldots)$ where $b_i \in B_i$ for every $i \in \pints$.
    Consider any such $i \in \pints$ so that $b_i \in B_i$.
    Then also $b_i \in A_i$ since $B_i \ss A_i$.
    Since $i$ was arbitrary, $b_i \in A_i$ for every $i \in \pints$ so that $b = (b_1, b_2, \ldots) \in A_1 \times A_2 \times \cdots = A$.
    Since $b$ was arbitrary, this shows that $B \ss A$.
    Note that we ignore the function range technicality issue mentioned above.
  }

  (b)
  \qproof{
    Suppose that $B \ss A$.
    Since $B \neq \es$, there is a $b' \in B$ so that $b' = (b_1', b_2', \ldots)$ where $b_i' \in B_i$ for every $i \in \pints$.
    Now consider any $i \in \pints$ and $b_0 \in B_i$.
    Then define
    \gath{
      b_j = \begin{cases}
        b_0 & j = i \\
        b_j' & j \neq i
      \end{cases}
    }
    for any $j \in \pints$.
    Clearly we have that $b_j \in B_j$ for any $j \in \pints$ so that $b = (b_1, b_2, \ldots) \in B_1 \times B_2 \times \cdots = B$.
    It then follows that also $b \in A$ since $B \ss A$.
    Hence $b_j \in A_j$ for every $j \in \pints$.
    In particular, we have $b_0 = b_i \in A_i$.
    Since $b_0$ was arbitrary, this shows that $B_i \ss A_i$, and since $i$ was arbitrary, this shows the desired result.
  }

  (c)
  \qproof{
    Suppose that $A$ is nonempty so that there is an $a \in A$.
    Then, since $A = A_1 \times A_2 \times \cdots$, it follows that $a = (a_1, a_2, \ldots)$ where $a_i \in A_i$ for every $i \in \pints$.
    Therefore, for any such $i \in \pints$, we have that $a_i \in A_i$ so that $A_i \neq \es$.
    Hence every $A_i$ is nonempty as desired since $i$ was arbitrary.
  }

  Consider the converse.
  Suppose that each $A_i$ is nonempty (for $i \in \pints$).
  Then there is an $a_i \in A_i$ for every $i \in \pints$ so that $a = (a_1, a_2, \ldots) \in A_1 \times A_2 \times \cdots = A$ so that then $A \neq \es$.
  While this may seem like an innocuous argument, especially out of the context of axiomatic set theory, it actually requires the Axiom of Choice.
  The reason is that, in the general case when each $A_i$ may have more than one, or even an infinite number, of elements, we have to choose a specific $a_i$ in each $A_i$.
  Since the index set $\pints$ is infinite, an infinite number of these choices must be made, which is precisely when the Axiom of Choice is required.
  If the index set was finite, then the axiom would not be needed.

  (d)
  First, let $C_i = A_i \cup B_i$ for every $i \in \pints$, and let $C = C_1 \times C_2 \times \cdots$,  so that we are asked to compare $C$ with $A \cup B$.

  We claim that $A \cup B \ss C$ but that $C$ is \emph{not} generally a subset of $A \cup B$.
  
  \qproof{
    First consider any $x \in A \cup B$ so that $x \in A$ or $x \in B$.
    If $x \in A$ then it has to be that $x = (x_1, x_2, \ldots)$ where $x_i \in A_i$ for every $i \in \pints$.
    Consider then any such $i \in \pints$.
    Then $x_i \in A_i$ so that clearly $x_i \in A_i \cup B_i = C_i$.
    Since $i$ was arbitrary, we conclude that $x = (x_1, x_2, \ldots) \in C_1 \times C_2 \times \cdots = C$.
    An analogous argument shows that $x \in C$ when $x \in B$ as well.
    Hence $A \cup B \ss C$ since $x$ was arbitrary.

    To show that $C$ is \emph{not} a subset of $A \cup B$ in general, consider the following counterexample.
    Let $A_1 = \es$ and $A_i = \braces{1}$ for every $i \in \pints$ where $i > 1$.
    Also let $B_i = \braces{2}$ for every $i \in \pints$.
    Now, it follows from the contrapositive of part (c) that $A = \es$ since $A_1 = \es$.
    We also clearly have $B = B_1 \times B_2 \times \cdots = \braces{(2, 2, \ldots)}$ so that $A \cup B = \es \cup B = B = \braces{(2, 2, \ldots)}$.
    Clearly $C_1 = A_1 \cup B_1 = \es \cup \braces{2} = \braces{2}$ while, for $i > 1$ we have $C_i = A_i \cup B_i = \braces{1} \cup \braces{2} = \braces{1,2}$.
    It then follows that, for $a_1 = 2$ and $a_i = 1$ for $i > 1$, we have $a = (a_1, a_2, \ldots) = (2, 1, 1, \ldots) \in C_1 \times C_2 \times \cdots = C$.
    However, clearly $a \notin A \cup B$, which suffices to show that $C$ cannot be a subset of $A \cup B$ in general.
  }

  Now let $C_i = A_i \cap B_i$ for every $i \in \pints$ so that we are asked to compare $C = C_1 \times C_2 \times \cdots$ and $A \cap B$.

  Here we claim that in fact $A \cap B = C$.

  \qproof{
    First consider any $x \in A \cap B$ so that $x \in A$ and $x \in B$.
    It then follows that $x = (x_1, x_2, \ldots)$ where $x_i \in A_i$ for every $i \in \pints$ and $x_i \in B_i$ for every $i \in \pints$.
    Then, for any such $i \in \pints$, clearly $x_i \in A_i$ and $x_i \in B_i$ so that $x_i \in A_i \cap B_i = C_i$.
    We then have that $x = (x_1, x_2, \ldots) \in C_1 \times C_2 \times \cdots = C$.
    Since $x$ was arbitrary, this shows that $A \cap B \ss C$.

    Now consider any $x \in C$ so that $x = (x_1, x_2, \ldots)$ where $x_i \in C_i$ for any $i \in \pints$.
    Then, for any such $i \in \pints$, we have $x_i \in C_i = A_i \cap B_i$ so that $x_i \in A_i$ and $x_i \in B_i$.
    Since $i$ was arbitrary, this shows that both $x = (x_1, x_2, \ldots) \in A_1 \times A_2 \times \cdots = A$ and $x = (x_1, x_2, \ldots) \in B_1 \times B_2 \times \cdots = B$.
    Hence $x \in A \cap B$, which shows that $C \ss A \cap B$ since $x$ was arbitrary.

    Therefore it must be that $A \cap B = C$ as desired.
  }
}
