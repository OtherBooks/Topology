\setcounter{subsection}{19-1}
\subsection{The Product Topology}

\exercise{1}{
  Prove Theorem~19.2
}
\sol{
  \dwhitman

  Let $\cC$ be the collection of sets that are alleged to be a basis for the box or product topologies in Theorem~19.2.
  \qproof{
    We show that $\cC$ is a basis of the box or product topology using Lemma~13.2.
    First, it is easy to see that $\cC$ is a collection of open sets.
    Consider any $B \in \cC$ so that $B = \prod  B_\a$ where each $B_\a \in \cB_\a$ (for a finitely many $\a \in J$ and $B_\a = X_\a$ for the rest in the product topology).
    Since each $B_\a$ is a basis element of $X_\a$ (or $X_\a$ itself), they are open so that $B$ is a basis element of the box or product topology by definition and therefore open.
    Note that the basis for the product topology is given directly by Theorem~19.1.
    
    Now suppose that $U$ is an any open set of the box topology and consider any $x \in U$.
    Then it follows that there is a basis element $\prod_{\a \in J} U_\a$  of the box or product topology containing $x$ where $\prod_{\a \in J} U_\a \ss U$.
    Thus each $U_\a$ is an open set of $X_\a$ (or $U_\a = X_\a$ for all but finitely many $\a \in J$ for the product topology).
    Also $x \in \prod_{\a \in J} U_\a$ so that $x = (x_\a)_{\a \in J}$ where each $x_\a \in U_\a$.
    It then follows that there is basis element $B_\a \in \cB_\a$ of $X_\a$ containing $x_\a$ where $B_\a \ss U_\a$ (for $U_\a = X_\a$ we simply set $B_\a = X_\a$ as well).
    
    Then clearly $x \in \prod_{\a \in J} B_\a$ and $\prod_{\a \in J} B_\a \in \cC$.
    Consider also any $y \in \prod_{\a \in J} B_\a$ so that $y = (y_\a)_{\a \in J}$ where each $y_\a \in B_\a$.
    Then also each $y_\a \in U_\a$ since $B_\a \ss U_\a$.
    This suffices to show that $y \in \prod_{\a \in J} U_\a \ss U$.
    Since $y$ was arbitrary this shows that $\prod_{\a \in J} B_\a \ss U$.
    Therefore $\cC$ is a basis of the box topology by Lemma~13.2.
  }
}

\exercise{2}{
  Prove Theorem~19.3.
}
\sol{
  \dwhitman

  \qproof{
    The basis of the box or product topologies on $\prod A_\a$ is the collection of sets $\prod V_\a$, where each $V_\a$ is open in $A_\a$ and, in the case of the product topology, $V_\a = A_\a$ for all but finitely many $\a \in J$ (by Theorem~19.1).
    Denote this basis collection by $\cC$.
    By Lemma~16.1, the collection
    \gath{
      \cB_A = \braces{B \cap \prod A_\a \where B \in \cB}
    }
    is a basis of the subspace topology on $\prod A_\a$, where $\cB$ is the basis of $\prod X_\a$.
    To prove that $\prod A_\a$ is a subspace of $\prod X_\a$, it therefore suffices to show that $\cC = \cB_A$.

    $(\ss)$ First consider any element $B \in \cC$ so that $B = \prod V_\a$ for open sets $V_\a$ in $A_\a$ (and $V_\a = A_\a$ for all but finite many $\a \in J$ for the product topology).
    For each $\a \in J$, we then have that $V_\a = U_\a \cap A_\a$ for some open set $U_\a$ in $X_\a$ since $A_\a$ is a subspace of $X_\a$.
    Note that this is true even for those $\a$ where $V_\a = A_\a$ in the product topology since then $V_\a = A_\a = X_\a \cap A_\a$.
    In fact, for these $\a$ we need to choose $U_\a = X_\a$ as will become apparent.
    We then have the following:
    \ali{
      x \in B &\bic x \in \prod V_\a \\
      &\bic \forall \a \in J (x_\a \in V_\a) \\
      &\bic \forall \a \in J (x_\a \in U_\a \cap A_\a) \\
      &\bic \forall \a \in J (x_\a \in U_\a \land x_\a \in A_\a) \\
      &\bic \forall \a \in J (x_\a \in U_\a) \land \forall \a \in J (x_\a \in A_\a) \\
      &\bic x \in \prod U_\a \land x \in \prod A_\a \\
      &\bic x \in \parens{\prod U_\a} \cap \parens{\prod A_\a} \,,
    }
    Since $U_\a = X_\a$ for all but a finitely many $\a \in J$ for the product topology, we have that $\prod U_\a$ is a basis element of $\prod X_\a$, i.e. $\prod U_\a \in \cB$.
    This shows that $B \in \cB_A$ so that $\cC \ss \cB_A$ since $B$ was arbitrary.

    $(\sps)$ Now suppose that $B \in \cB_A$ so that $B = B_X \cap \prod A_\a$ for some basis element $B_X \in \cB$ of $\prod X_\a$.
    We then have that $B_X = \prod U_\a$ where each $U_\a$ is an open set of $X_\a$ (and $U_\a = X_\a$ for all but finitely many $\a \in J$ for the product topology).
    Then let $V_\a = U_\a \cap A_\a$ for each $\a \in J$, noting that $V_\a = X_\a \cap A_\a = A_\a$ when $U_\a = X_\a$.
    Following the above chain of logical equivalences in reverse order then shows that $B = \prod V_\a$ so that $B \in \cC$ since clearly each $V_\a$ is open in the subspace topology $A_\a$.
    Hence $\cC \sps \cB_A$ since $B$ was arbitrary.
  }
}

\exercise{3}{
  Prove Theorem~19.4.
}
\sol{
  \dwhitman

  \qproof{
    Suppose that $x$ and $y$ are distinct points of $\prod X_\a$.
    Then $x = (x_a)$ and $y = (y_\a)$ where each $x_\a, y_\a \in X_\a$, and there must be a $\b$ where $x_\b \neq y_\b$ since $x \neq y$.
    Thus $x_\b$ and $y_\b$ are distinct points of $X_\b$, so that there are neighborhoods $W_x$ and $W_y$ of $x_\b$ and $y_\b$, respectively, that are disjoint since $X_\b$ is a Hausdorff space.
    So define the sets
    \ali{
      U_\a &= \begin{cases}
        W_x & \a = \b \\
        X_\a & \a \neq \b
      \end{cases}
      &
      V_\a &= \begin{cases}
        W_y & \a = \b \\
        X_\a & \a \neq \b
      \end{cases}
    }
    so that clearly $x \in \prod U_\a$ and $y \in \prod V_\a$.
    Then since each $U_\a$ and $V_\a$ are open, we have that $\prod U_\a$ and $\prod V_\a$ are both basis elements of $\prod X_\a$ and therefore open.
    Note that this is true for both the box and product topologies since, in the case of the latter, $U_\a$ and $V_\a$ are not all of $X_\a$ for only one $\a$, namely $\a = \b$.
    Thus $\prod U_\a$ is a neighborhood of $x$ and $\prod V_\a$ is a neighborhood of $y$ in $\prod X_\a$.

    We also assert that $\prod U_\a$ and $\prod V_\a$ are disjoint, which of course completes the proof that $\prod X_\a$ is Hausdorff.
    To see this, suppose to the contrary that there is a $z$ in both $\prod U_\a$ and $\prod V_\a$.
    Then $z = (z_\a)$ and in particular we would have that $z_\b \in U_\b = W_x$ and $z_\b \in V_\b = W_y$.
    But then $z_\b \in W_x \cap W_y$, which contradicts the fact that $W_x$ and $W_y$ are disjoint!
    So it must be that in fact $\prod U_\a$ and $\prod V_\a$ are disjoint.
  }
}

\def\doms{\cpfin{X}{n}}
\def\rans{(\cpfin{X}{n-1}) \times X_n}
\exercise{4}{
  Show that $\rans$ is homeomorphic to $\doms$.
}
\sol{
  \dwhitman

  \qproof{
    First we note that since we are dealing with finite products, the box and product topologies are the same; we shall find it most convenient to use the box topology definition.
    Also, as there are no intervals involved here, we use the traditional tuple notation using parentheses.
    So define $f: \doms  \to \rans$ by
    \gath{
      f(x_1, \ldots, x_{n-1}) = ((x_1, \ldots, x_{n-1}), x_n) \,.
    }
    It is obvious that this is a bijection, and it is trivial to prove.
    Also obvious and trivial to prove based on the definition of $f$ is that $f(\cpfin{A}{n}) = (\cpfin{A}{n-1}) \times A_n$ when each $A_k \ss X_k$.
    
    First we show that $f$ is continuous by showing that the inverse image of every basis element in $\rans$ is open in $\doms$.
    So consider any basis element $C$ of $\rans$ and let $U = \inv{f}(C)$ so that of course $f(U) = C$ and $U \ss \doms$.
    We then have that $C = V' \times V_n$ where $V'$ is open in $\cpfin{X}{n-1}$ and $V_n$ is open in $X_n$ by the definition of the box/product topology.
    Now consider any $x \in U$ so that $x = (x_k)_{k=1}^n$ and we have that $f(x) = ((x_1, \ldots, x_{n-1}), x_n) \in f(U) = C$.
    Hence $x' = (x_1, \ldots, x_{n-1}) \in V'$ and $x_n \in V_n$.
    Since $V'$ is open in $\cpfin{X}{n-1}$ there is a basis element $C'$ containing $x'$ that is a subset of $V'$.
    By the definition of the box topology, we then have that $C' = \cpfin{V}{n-1}$ where each $V_k$ is open in $X_k$.

    We then have that $B = \cpfin{V}{n}$ is a basis element of $\doms$ and also clearly $B$ contains $x$ since $(x_1, \ldots, x_{n-1}) = x' \in C' = \cpfin{V}{n-1}$ and $x_n \in V_n$.
    Now suppose that $y = (y_k)_{k=1}^n \in B$ so that each $y_k \in V_k$.
    Then we have that $y' = (y_1, \ldots, y_{n-1}) \in C'$ so that also $y' \in V'$ since $C' \ss V'$.
    Since also of course $y_n \in V_n$, we have that $(y', y_n) \in V' \times V_n = C$.
    Also clearly $f(y) = (y', y_n) \in C = f(U)$ so that $y \in U$.
    Since $y$ was arbitrary this shows that $B \ss U$, which suffices to show that $U$ is open since $x$ was arbitrary.
    This completes the proof that $f$ is continuous.

    Next we show that $\inv{f}$ is continuous, which is a little simpler.
    Let $B$ be any basis element of $\doms$ so that $B = \cpfin{U}{n}$ where each $U_k$ is open in $X_k$ by the definition of the box topology.
    Then we have that $f(B) = (\cpfin{U}{n-1}) \times U_n$.
    By the definition of the box topology, we then have that $U' = \cpfin{U}{n-1}$ is a basis element of $\cpfin{X}{n-1}$ and is therefore open.
    Since $U_n$ is also open, we have that $f(B) = U' \times U_n$ is a basis element of $\rans$ by the definition of the box/product topology, and is therefore open.
    Since $f(B) = \inv{(\inv{f})}(B)$ is the inverse image of $B$ under $\inv{f}$, this shows that $\inv{f}$ is also continuous.

    We have shown that both $f$ and $\inv{f}$ are continuous, which proves that $f$ is a homeomorphism by definition.
  }
}

\exercise{5}{
  One of the implications stated in Theorem~19.6 holds for the box topology. Which one?
}
\sol{
  \dwhitman

  Example 19.2 gives a function $f$ that is not continuous in the box topology even though all of its constituent functions $f_\a$ are continuous.
  Hence the only implication that can be generally true in the box topology is that $f$ being continuous implies that each $f_\a$ is continuous.
  A proof of this is straightforward.
  \qproof{
    As in Theorem~19.6 suppose that $f: A \to \prod_{\a \in J} X_\a$ be given by
    \gath{
      f(a) = (f_\a(a))_{\a \in J} \,,
    }
    where $f_\a: A \to X_\a$ for each $\a \in J$.
    Here $\prod X_\a$ has the box topology.
    Suppose that $f$ is continuous and consider any $\b \in J$.
    We show that $f_\b$ is continuous, which of course shows the desired result.

    So let $V$ be any open set of $X_\b$ and define
    \gath{
      B_\a = \begin{cases}
        V & \a = \b \\
        X_\a & \a \neq \b \,.
      \end{cases}
    }
    Then, since each $B_\a$ is clearly open in $X_\a$, we have that $B = \prod B_\a$ is a basis element of the box topology by definition and is therefore open.
    Hence $U = \inv{f}(B)$ is open in $A$ since $f$ is continuous.
    We claim that $U = \inv{f_\b}(V)$, which shows that $f_\b$ is continuous since $U$ is open in $A$ and $V$ was an arbitrary open set of $X_\b$.

    $(\ss)$ If $x \in U = \inv{f}(B)$ then of course $f(x) \in B$ so that each $f_\a(x) \in B_\a$ since $f(x) = (f_\a(x))_{\a \in J}$ and $B = \prod B_\a$.
    In particular $f_\b(x) \in B_\b = V$ so that $x \in \inv{f_\b}(V)$.
    Hence $U \ss \inv{f_\b}(V)$ since $x$ was arbitrary.

    $(\sps)$ If $x \in \inv{f_\b}(V)$ then $f_\b(x) \in V = B_\b$.
    Since of course every other $f_\a(x) \in X_\a = B_\a$ we have that $f(x) \in \prod B_\a = B$.
    Hence $x \in \inv{f}(B) = U$ so that $\inv{f_\b}(V) \ss U$ since $x$ was arbitrary.
  }
}

\exercise{6}{
  Let $\vx_1, \vx_2, \ldots$ be a sequence of the points of the product space $\prod X_\a$.
  Show that the sequence converges to the point $\vx$ if and only if the sequence $\pi_\a(\vx_1), \pi_\a(\vx_2), \ldots$ converges to $\pi_\a(\vx)$ for each $\a$.
  Is this fact true if one uses the box topology instead of the product topology?
}
\sol{
  \dwhitman

  \qproof{
    $(\imp)$ First suppose that the sequence $\vx_1, \vx_2, \ldots$ converges to $\vx$ and consider any $\b$.
    Also suppose that $U$ is any neighborhood of $\pi_\b(\vx)$.
    Define
    \gath{
      B_a = \begin{cases}
        U & \a = \b \\
        X_\a & \a \neq \b
      \end{cases}
    }
    so that $B = \prod B_\a$ is a basis element of $\prod X_\a$ since each $B_\a$ is open.
    Note that $B$ is a basis element of both the box and product topologies since possibly $B_\a \neq X_\a$ for only one $\a$ (i.e. for $\a = \b$).
    We also clearly have that $\vx \in B$ so that $B$ is a neighborhood of $\vx$ in $\prod X_\a$.
    Since the sequence $\vx_1, \vx_2, \ldots$ converges to $\vx$, we have that there is an $N \in \pints$ where $\vx_n \in B$ for all $n \geq N$.
    So consider any such $n \geq N$ so that $\vx_n \in B = \prod B_\a$.
    Hence $\pi_\a(\vx_n) \in B_\a$ for all $\a$, and in particular $\pi_\b(\vx_n) \in B_\b = U$.
    This suffices to show that the sequence $\pi_\b(\vx_1), \pi_\b(\vx_2), \ldots$ converges to $\pi_\b(\vx)$ as desired since $U$ was an arbitrary neighborhood.

    $(\pmi)$ Now suppose that the sequence $\pi_\a(\vx_1), \pi_\a(\vx_2), \ldots$ converges to $\pi_\a(\vx)$ for every $\a$.
    Let $U$ be any neighborhood of $\vx$ in $\prod X_\a$.
    Then there is a basis element $B = \prod U_\a$ of $\prod X_\a$ where $\vx \in B$ and $B \ss U$.
    Since $\prod X_\a$ is the product topology, each $U_\a$ is open but only a finite number of them are different from $X_\a$.
    Suppose then that $J$ is the index set of $\a$ and that $I \ss J$ is the finite subset where $U_\a = X_\a$ for all $\a \notin I$.

    Then for any $\b \in I$ we have that $\pi_\b(\vx) \in U_\b$ since $\vx \in B = \prod U_\a$, hence $U_\b$ is a neighborhood of $\pi_\b(\vx)$.
    Then, since  $\pi_\b(\vx_1), \pi_\b(\vx_2), \ldots$ converges to $\pi_\b(\vx)$, there is an $N_\b \in \pints$ where $\pi_\b(\vx_n) \in U_\b$ for all $n \geq N_\b$.
    So let $N = \max_{\a \in I} N_\a$, noting that this exists since $I$ is finite.
    Consider any $n \geq N$ and any $\a \in J$.
    If $\a \in I$ then we have that $n \geq N \geq N_\a$ so that $\pi_\a(\vx_n) \in U_\a$.
    If $\a \notin J$ then of course we have that $\pi_a(\vx_n) \in X_\a = U_\a$.
    Hence either way we have that $\pi_\a(\vx_n) \in U_\a$ so that $\vx_n \in \prod U_\a = B$ and hence also $\vx_n \in U$ since $B \ss U$.
    Since $n \geq N$ was arbitrary and $U$ was an arbitrary neighborhood of $\vx$, this shows that $\vx_1, \vx_2, \ldots$ converges to $\vx$ as desired.
  }

  As noted there, the forward direction of the preceding proof works for the product or the box topology.
  However, then reverse direction was proved only for the product topology, with the critical point being where we took $\max_{\a \in I} N_\a$, which was only guaranteed to exist since $I$ is finite in the product topology.
  The provides a hint as to how to construct a counterexample that proves that this direction is not generally true for the box topology.
  \qproof{
    Define
    \gath{
      x_{ij} = \begin{cases}
        1 & j \leq i \\
        \frac{1}{j-i} & j > i
      \end{cases}
    }
    for $i,j \in \pints$.
    Now define a sequence $\vx_1, \vx_2, \ldots$ in $\prod_{i \in \pints} \reals = \reals^\w$ by $\pi_i(\vx_j) = x_{ij}$.
    With the box topology on $\reals^\w$ we claim that each coordinate sequence $\pi_i(\vx_1), \pi_i(\vx_2), \ldots$ converges to $0$ but that the sequence $\vx_1, \vx_2, \ldots$ does not converge to the point $\vect{0} = (0, 0, \ldots)$.

    First, it is easy to see that each coordinate sequence $\pi_i(\vx_1), \pi_i(\vx_2), \ldots$ converges to $0$ since, for fixed $i$, there is always an $N \in \pints$ large enough such that $j > i$ and $\pi_i(\vx_j) = x_{ij} = 1/(j-i)$ is small enough to be within any fixed neighborhood of $0$ for all $j \geq N$.
    To show that the sequence $\vx_1, \vx_2, \ldots$ does not converge to $\vect{0}$ though, consider the neighborhood $U = \prod U_k$ of $\vect{0}$ where every $U_k = (-1,1)$.
    We note that clearly $U$ is open in the box topology since each $U_k$ is a basis element of $\reals$ and therefore open.
    For any $N \in \pints$ we then have that $\pi_{N}(\vx_N) = x_{NN} = 1$ so that clearly $\pi_N(\vx_N) \notin (-1, 1) = U_N$ and hence $\vx_N \notin \prod U_k = U$.
    This suffices to show that the sequence does not converge, but it does not even come close to converging since there are actually no points in the sequence that are even in this quite large neighborhood of $\vect{0}$!
  }
}

\exercise{7}{
  Let $\reals^\infty$ be the subset of $\reals^\w$ consisting of all sequences that are ``eventually zero,'' that is all sequences $(x_1, x_2, \ldots)$ such that $x_i \neq 0$ for only finitely many values of $i$.
  What is the closure of $\reals^\infty$ in $\reals^\w$ in the box and product topologies?
  Justify your answer.
}
\sol{
  \dwhitman

  First we claim that $\reals^\infty$ is dense in $\reals^\w$ in the product topology in the sense that its closure is all of $\reals^\w$.
  \qproof{
    We show that any point of $\reals^\w$ is in $\closure{\reals^\w}$.
    So consider any point $x = (x_n)_{n \in \pints} \in \reals^\w$ and any neighborhood $U$ of $x$.
    Then there is a basis element $B = \prod U_n$ containing $x$ where $B \ss U$.
    By the definition of the product topology each $U_n$ is open and $U_n = \reals$ for all but finitely many values of $n$.
    So let $I$ be a finite subset of $\pints$ such that $U_n = \reals$ for all $n \notin I$ and $U_n$ is merely just open for $n \in I$.

    Consider now the sequence $y = (y_n)_{n \in \pints}$ defined by
    \gath{
      y_n = \begin{cases}
        x_n & n \in I \\
        0 & n \notin I
      \end{cases}
    }
    for $n \in \pints$.
    Since $I$ is finite clearly $y \in \reals^\infty$.
    Also $y_n = x_n \in U_n$ when $n \in I$ since $B = \prod U_n $ contains $x$.
    We also have $y_n = 0 \in \reals = U_n$ when $n \notin I$ so that either way $y_n \in U_n$ and hence $y \in \prod U_n = B$.
    Thus also $y \in U$ since $B \ss U$.
    Since $U$ was an arbitrary neighborhood and $U$ intersects $\reals^\infty$ (with $y$ being a point in the intersection), this shows that $x \in \closure{\reals^\infty}$ by Theorem~17.5.
    This of course shows the desired result since $x$ was any element of $\reals^\w$.
  }

  For the box topology, we claim that $\reals^\infty$ is already closed.
  \qproof{
    We show this by showing that any point not in $\reals^\infty$ is not a limit point of $\reals^\infty$ so that $\reals^\infty$ must already contain all its limit points.
    So consider any $x = (x_n)_{n \in \pints} \notin \reals^\infty$ so that $x_n \neq 0$ for infinitely many values of $n$.
    Now define the sets
    \gath{
      U_n = \begin{cases}
        (-1, 1) & x_n = 0 \\
        (x_n/2, 2x_n) & x_n > 0 \\
        (2x_n, x_n/2) & x_n < 0
      \end{cases}
    }
    for $n \in \pints$.
    Clearly each $U_n$ is a basis element of $\reals$ and is therefore open.
    Also clearly each $x_n \in U_n$.
    It therefore follows that $B = \prod U_n$ is a basis element of $\reals^\w$ and is therefore open, and that $x \in B$.
    Hence $B$ is a neighborhood of $x$.

    Then, for any $y = (y_n)_{n \in \pints} \in B$ we have that each $y_n \in U_n$.
    For infinitely many $n \in \pints$ we then have that $x_n \neq 0$ and hence $x_n > 0$ or $x_n < 0$.
    In the former case $y_n \in U_n = (x_n/2, 2x_n)$ so that $0 < x_n/2 < y_n$.
    In the latter case $y_n \in U_n = (2x_n, x_n/2)$ so that $y_n < x_n / 2 < 0$.
    Hence either way $y_n \neq 0$ so that $y \notin \reals^\infty$ since this is true for infinitely many $n$.
    Since $y \in B$ was arbitrary, this shows that $B$ cannot not intersect $\reals^\infty$.
    Therefore $x$ is not a limit point of $\reals^\infty$ since $B$ is a neighborhood of $x$.
  }
}

\exercise{8}{
  Given sequences $(a_1, a_2, \ldots)$ and $(b_1, b_2, \ldots)$ of reals numbers with $a_i > 0$ for all $i$, define $h: \reals^\w \to \reals^\w$ by the equation
  \gath{
    h((x_1, x_2, \ldots)) = (a_1 x_1 + b_1, a_2 x_2 + b_2, \ldots) \,.
  }
  Show that if $\reals^\w$ is given the product topology, $h$ is a homeomorphism of $\reals^\w$ with itself.
  What happens if $\reals^\w$ is given the box topology?
}
\sol{
  \dwhitman

  \begin{lem}\label{lem:prodtop:box}
    Consider the spaces $\prod X_\a$ and $\prod Y_\a$ in the box topologies over the index set $J$.
    If $f: \prod X_\a \to \prod Y_\a$ is defined by
    \gath{
      f((x_\a)_{\a \in J}) = (f_\a(x_\a))_{\a \in J}
    }
    and each $f_\a : X_\a \to Y_\a$ is continuous, then $f$ is continuous.
  \end{lem}
  \qproof{
    Consider any basis element $B = \prod V_\a$ in $\prod Y_\a$ so that each $V_\a$ is open in $Y_\a$ since we are in the box topology.
    For each $\a \in J$ then define $U_\a = \inv{f_\a}(V_\a)$, which is open in $X_\a$ since $f_\a$ is continuous.
    Hence the set $U = \prod U_\a$ is a basis element of $\prod X_\a$ in the box topology and is therefore open.
    We claim that $U = \inv{f}(B)$, which shows that $f$ is continuous since $U$ is open and $B$ was arbitrary.

    $(\ss)$ Consider any $\vx \in U = \prod U_\a$.
    Then, for any $\a \in J$, we have $x_\a \in U_\a = \inv{f_\a}(V_\a)$ so that $f(x_\a) \in V_\a$.
    Hence $f(\vx) = (f_\a(x_\a))_{\a \in J} \in \prod V_\a = B$ so that $\vx \in \inv{f}(B)$.
    this shows that $U \ss \inv{f}(B)$ since $\vx$ was arbitrary.

    $(\sps)$ Now consider any $\vx \in \inv{f}(B)$ so that $f(\vx) \in B = \prod V_\a$ and hence each $f_\a(x_\a) \in V_\a$ by the definition of $f$.
    Then $x_\a \in \inv{f_\a}(V_\a) = U_\a$ so that clearly $\vx \in \prod U_\a = U$.
    Since $\vx$ was arbitrary this shows that $\inv{f}(B) \ss U$ as well.
  }

  \mainprob
  \qproof{
    First note that clearly $h(\vx) = (h_1(\vx), h_2(\vx), \ldots)$ for $\vx \in \reals^\w$, where each $h_i : \reals^\w \to \reals$ is defined by
    \gath{
      h_i(\vx) = a_i \pi_i(\vx) + b_i \,.
    }
    This can further be broken down as $h_i(\vx) = f_i(\pi_i(\vx)) = (f_i \circ \pi_i)(\vx)$, where each $f_i : \reals \to \reals$ is defined by $f_i(x) = a_i x + b_i$.
    As discussed in the proof of Theorem~19.6, each $\pi_i$ is continuous and we have that each $f_i$ is continuous by elementary calculus, noting that this is true whether each $a_i > 0$ or not.
    It then follows from Theorem~18.2c that each $f_i \circ \pi_i = h_i$ is continuous.
    Then we have that $h$ is continuous by Theorem~19.6 since each coordinate function is continuous and we are using the product topology.

    Now define the functions $g_i : \reals \to \reals$ by $g_i(x) = (x - b_i) / a_i$ for $i \in \pints$, noting that this is defined since each $a_i > 0$.
    Define also the functions $k_i : \reals^\w \to \reals$ by $k_i = g_i \circ \pi_i$, and finally define $k : \reals^\w \to \reals^\w$ by $k(\vx) = (k_1(\vx), k_2(\vx), \ldots)$.
    Now again we have that each $\pi_i$ and $g_i$ are continuous by the proof of Theorem~19.6 and elementary calculus.
    Hence $k_i = g_i \circ \pi_i$ and $k$ are continuous by Theorem~18.2c, and Theorem~19.6, respectively, as before.

    Now consider any $\vx = (x_i)_{i \in \pints} \in \reals^\w$ so that we have, for any $i \in \pints$,
    \ali{
      k_i(h(\vx)) &=  [g_i \circ \pi_i](h(\vx)) = g_i(\pi_i(h(\vx))) = g_i(h_i(\vx)) \\
      &= g_i([f_i \circ \pi_i](\vx)) = g_i(f_i(\pi_i(\vx))) = g_i(f_i(x_i)) \\
      &= \frac{f_i(x_i) - b_i}{a_i} = \frac{(a_i x_i + b_i) - b_i}{a_i} = \frac{a_i x_i}{a_i} \\
      &= x_i \,.
    }
    Therefore
    \gath{
      k(h(\vx)) = (k_1(h(\vx)), k_2(h(\vx)), \ldots) = (x_1, x_2, \ldots) = \vx \,.
    }
    We also have that
    \ali{
      h_i(k(\vx)) &= [f_i \circ \pi_i](k(\vx)) = f_i(\pi_i(k(\vx))) = f_i(k_i(\vx)) \\
      &= f_i([g_i \circ \pi_i](\vx)) = f_i(g_i(\pi_i(\vx))) = f_i(g_i(x_i)) \\
      &= a_i g_i(x_i) + b_i = a_i\parens{\frac{x_i - b_i}{a_i}} + b_i = (x_i - b_i) + b_i\\
      &= x_i \,.
    }
    for each $i \in \pints$ so that
    \gath{
      h(k(\vx)) = (h_1(k(\vx)), h_2(k(\vx)), \ldots) = (x_1, x_2, \ldots) = \vx \,.
    }
    Since $\vx$ was arbitrary, it thus follows from Lemma~2.1 that $h$ is bijective and $k = \inv{h}$.
    Since we have already shown that $h$ and $k = \inv{h}$ are continuous, this suffices to prove that $h$ is a homeomorphism as desired.
  }

  We claim that $h$ is also a homeomorphism in the box topology.
  \qproof{
    First, $h$ is still a bijection as the proof of this above does not depend on the topology at all.
    However, Theorem~19.6 was used in the proofs that $h$ and $\inv{h}$ are continuous, and we know that this theorem is not generally true for the box topology.
    On the other hand $h$ can be formulated as $h(\vx) = (f_1(x_1), f_2(x_2), \ldots)$, where as before each $f_i(x) = a_i x + b_i$.
    Since each $f_i$ is continuous by elementary calculus, it follows from Lemma~\ref{lem:prodtop:box} that $h$ is continuous in the box topology.
    The same argument applies to the inverse function $\inv{h}$ since $\inv{h}(\vx) = (g_1(x_1), g_2(x_2), \ldots)$ and each $g_i$ is continuous.
  }
}

\exercise{9}{
  Show that the choice axiom is equivalent to the statement that for any indexed family $\braces{A_\a}_{\a \in J}$ of nonempty sets, with $J \neq 0$, the cartesian product
  \gath{
    \prod_{\a \in J} A_\a
  }
  is not empty.
}
\sol{
  \dwhitman

  \qproof{
    For the following denote the collection $\braces{A_\a}_{\a \in J}$ by $\cA$.

    $(\imp)$ First suppose that the choice axiom is true.
    Then by Lemma~9.2 there exists a choice function
    \gath{
      c : \cA \to \bigcup_{A \in \cA} A
    }
    where $c(A) \in A$ for each $A \in \cA$, noting that this is true since $\cA$ is a collection of nonempty sets.
    Then consider, set $x_\a = c(A_\a)$ for each $\a \in J$ so that $x_\a = c(A_\a) \in A_\a$.
    Therefore clearly $\vx = (x_\a)_{\a \in J} \in \prod A_\a$ so that $\prod A_\a$ is not empty.

    $(\pmi)$ Now suppose that $\prod_{\a \in J} A_\a$ is nonempty for any indexed family $\braces{A_\a}_{\a \in J}$ of nonempty sets when $J \neq \es$.
    Let $\cA$ be a collection of disjoint nonempty sets where $\cA \neq \es$.
    Then the $\braces{A}_{A \in \cA}$ is a nonempty family of nonempty sets.
    Hence $\prod_{A \in \cA} A$ is nonempty so that there is an $\vx = (x_A)_{A \in \cA} \in \prod_{A \in \cA} A$, and thus $x_A \in A$ for every $A \in \cA$.
    Now let $C = \braces{x_A}_{A \in \cA}$ so that clearly $C \ss \bigcup \cA$.
    Consider any $A \in \cA$ so that $x_A \in C$ and $x_A \in A$, and hence $x_A \in C \cap A$.
    Suppose that $y \in C \cap A$ so that $y \in C$ and hence there is a $B \in \cA$ where $y = x_B$.
    We also have that $x_B = y \in A$.
    If $B \neq A$ then $x_B \in B$ and $x_B \in A$, which is not possible since $B$ and $A$ are disjoint as they are distinct elements of $\cA$.
    So it must be that $B = A$ and hence $y = x_B = x_A$.
    Since $y$ was arbitrary, this shows that $C \cap A$ has only a single element $x_A$.
    This suffices to show the choice axiom.
  }
}

\exercise{10}{
  Let $A$ be a set; let $\braces{X_\a}_{\a \in J}$ be an indexed family of spaces; and let $\braces{f_\a}_{\a \in J}$ be an indexed family of functions $f_\a: A \to X_\a$.
  \eparts{
  \item Show there is a unique coarsest topology $\cT$ on $A$ relative to which each of the functions $f_\a$ is continuous.
  \item Let
    \gath{
      \cS_\b = \braces{\inv{f_\b}(U_\b) \where \text{$U_\b$ is open in $X_\b$}}\,,
    }
    and let $\cS = \bigcup \cS_\b$.
    Show that $\cS$ is a subbasis for $\cT$.
  \item Show that a map $g: Y \to A$ is continuous relative to $\cT$ if and only if each map $f_\a \circ g$ is continuous.
  \item Let $f: A \to \prod X_\a$ be defined by the equation
    \gath{
      f(a) = (f_\a(a))_{\a \in J} \,;
    }
    let $Z$ denote the subspace $f(A)$ of the product space $\prod X_\a$.
    Show that the image under $f$ of each element of $\cT$ is an open set of $Z$.
  }
}
\sol{
  \dwhitman

  (a)
  \qproof{
    Let $\cC$ be the collection of topologies on $A$ relative to which each of the functions $f_\a$ is continuous.
    Clearly $\cC$ is nonempty as the discrete topology is in $\cC$ since every subset of $A$ is open in it so that $f_\a(V_\a)$ is always open when $V_\a$ is open in $X_\a$.
    Let $\cT = \bigcap \cC$, which is a topology on $A$ by what was shown in Exercise~13.4a.
    We claim that this is the unique coarsest topology such that each $f_\a$ is continuous relative to it.
    To see this suppose that $\cT'$ is any topology in such that each $f_\a$ is continuous relative to it, hence $\cT' \in \cC$.
    Then, for any open $U \in \cT = \bigcap \cC$ we of course have that $U \in \cT'$ since $\cT' \in \cC$.
    Hence $\cT \ss \cT'$ since $U$ was arbitrary so that $\cT$ is courser than $\cT'$, noting that it could of course be that $\cT = \cT'$ as well.
    Since $\cT'$ was artbitary, this shows the desired result.

    Of course it also must be that $\cT$ is unique since, for any other $\cT'$ that is a coarsest element of $\cC$, we just showed above that $\cT \ss \cT'$ since $\cT' \in \cC$.
    But also $\cT \sps \cT'$ since $\cT'$ must be coarser than $\cT$ since $\cT \in \cC$.
    This shows that $\cT = \cT'$ so that $\cT$ is unique since $\cT'$ was arbitrary.
    This also follows from the more general fact that any smallest element in an order or partial order is always unique, and inclusion is always at least a partial order.
  }

  (b)
  \qproof{
    We show that $\cC$ from part (a) is exactly the set of topologies on $A$ that contain the subbasis $\cS$.
    That is, we show that $\cT' \in \cC$ if and only if $\cS \ss \cT'$ when $\cT'$ is a topology on $A$.
    Since the coarsest topology $\cT$ from part (a) is defined as $\bigcap \cC$, this shows that $\cT$ is the topology generated from the subbasis $\cS$ by Exercise~13.5.

    $(\imp)$ Suppose that $\cT' \in \cC$ so that every $f_\a$ is continuous relative to $\cT'$.
    Now consider any subbasis element $S \in \cS$ so that $S = \inv{f_\b}(U_\b)$ for some $\b \in J$ and some open set $U_\b$ in $X_\b$.
    Then $f_\b$ is continuous relative to $\cT'$ so that $S$ is open with respect to $\cT'$, and hence $S \in \cT'$.
    This shows that $\cS \ss \cT'$ since $S$ was arbitrary, hence $\cT'$ contains $\cS$.

    $(\pmi)$ Now suppose that $\cT'$ is a topology on $A$ that contains $\cS$ so that $\cS \ss \cT'$.
    Consider any $\a \in J$ and any open set $U_\a$ of $X_\a$.
    Then clearly $\inv{f_\a}(U_\a)$ is in $\cS_\a$ so that it is also clearly in $\cS = \bigcup \cS_\b$.
    Hence also $\inv{f_\a}(U_\a) \in \cT'$ since $\cS \ss \cT'$.
    Therefore $\inv{f_\a}(U_\a)$ is open with respect to $\cT'$, which shows that $f_\a$ is continuous relative to $\cT'$ since $U_\a$ was an arbitrary open set of $X_\a$.
    Since $\a \in J$ was also arbitrary, this shows that every $f_\a$ is continuous relative to $\cT'$ so that $\cT' \in \cC$ by definition.
  }

  (c)
  \qproof{
    $(\imp)$ Suppose that $g : Y \to A$ is continuous relative to $\cT$.
    Consider any $\a \in J$ and any open set $U_\a$ of $X_\a$.
    Then $\inv{f_\a}(U_\a)$ is open with respect to $\cT$ since $f_\a$ is continuous relative to $\cT$ since every $f_\a$ is.
    It then follows that $\inv{g}(\inv{f_\a}(U_\a))$ is open in $Y$ since $g$ is continuous relative to $\cT$.
    From Exercise~2.4a we have that $\inv{g}(\inv{f_\a}(U_\a)) = \inv{(f_\a \circ g)}(U_\a)$, which shows that $f_\a \circ g$ is continuous since $U_\a$ was an arbitrary open set of $X_\a$.
    Since $\a \in J$ was arbitrary, this shows the desired result.

    $(\pmi)$ Now suppose that every $f_\a \circ g$ is continuous and consider any open set $U$ of $A$ with respect to $\cT$.
    Then by part (b) we have that $U$ is an arbitrary union of finite intersections of subbasis elements $\inv{f_\a}(U_\a)$ for $\a \in J$ and open $U_\a$ in $X_\a$.
    It then follows from Exercise~2.2 parts (b) and (c) that $\inv{g}(U)$ is an arbitrary union of finite intersections of sets $\inv{g}(\inv{f_\a}(U_\a))$.
    Again we have that each $\inv{g}(\inv{f_\a}(U_\a)) = \inv{(f_\a \circ g)}(U_\a)$ by Exercise~2.4a so that each of these sets is open in $Y$ since every $f_\a \circ g$ is continuous.
    Hence $\inv{g}(U)$ is open as well since it is the arbitrary union of finite intersections of these open sets and $Y$ is a topological space.
    Since $U$ was an arbitrary open set of $A$ with respect to $\cT$, this shows that $g$ is continuous relative to $\cT$ as desired.
  }

  (d)
  \qproof{
    Suppose that $U$ is any open set of $A$ with respect to $\cT$.
    Consider any $\vy = (y_\a)_{\a \in J} \in f(U)$ so that there is an $a \in U$ where $f(a) = \vy$.
    Since $a \in U$ and $U$ is open in $A$, we have that there is a basis element $B_A$ containing $a$ where $B_A \ss U$.
    It then follows from part (b) that this basis element is a finite intersection of subbasis elements, hence $B_A = \bigcap_{\b \in I} \inv{f_\b}(U_\b)$, where $I \ss J$ is finite and each $U_\b$ is open in $X_\b$.
    Now define
    \gath{
      V_\a = \begin{cases}
        U_\b & \a \in I \\
        X_\b & \a \notin I
      \end{cases}
    }
    so that clearly the set $B_p = \prod V_\a$ is a basis element of $\prod X_\a$ in the product topology by Theorem~19.1 since $I$ is finite.
    We then have that $B_Z = Z \cap B_p$ is a basis element of the subspace $Z$ by Lemma~16.1.

    Now, we have that $a \in U$ and $U \ss A$ so that $a \in A$ as well.
    It then follows that $\vy = f(a) \in f(A) = Z$.
    For $\b \in I$, we also have that $a \in \inv{f_\b}(U_\b)$ since the basis element $B_A = \bigcap_{\b \in I} \inv{f_\b}(U_\b)$ contains $a$.
    Hence $f_\b(a) \in U_\b$.
    Since of course every other $f_\a(a) \in X_\a$ when $\a \notin I$, we have that $f_\a(a) \in V_\a$ for all $\a \in J$ and thus $\vy = f(a) = (f_\a(a))_{\a \in J} \in \prod V_\a = B_p$.
    We therefore have that $\vy \in Z \cap B_p = B_Z$ so that $B_Z$ contains $\vy$.

    Lastly, consider any $\vz = (z_\a)_{\a \in J} \in B_Z = Z \cap B_p$.
    Then $\vz \in Z = f(A)$ so that there is an $x \in A$ where $f(x) = (f_\a(x))_{\a \in J} = \vz$ and hence each $f_\a(x) = z_\a$.
    We also have that $\vz \in B_p = \prod V_\a$ so that $z_\a \in V_\a$ for every $\a \in J$.
    In particular $f_\b(x) = z_\b \in V_\b = U_\b$ for all $\b \in I$ so that $x \in \inv{f_\b}(U_\b)$.
    Therefore $x \in \bigcap_{\b \in I} \inv{f_\b}(U_\b) = B_A$ so that also $x \in U$ since $B_A \ss U$.
    Then we have that $\vz = f(x) \in f(U)$.
    Since $\vz$ was arbitrary this shows that $B_Z \ss f(U)$.

    We have thus shown that $B_Z$ is a basis element of the subspace $Z$ that contains $\vy$ where $B_Z \ss f(U)$.
    Since $\vy$ was an arbitrary element of $f(U)$, this suffices to show that $f(U)$ is open in the subspace $Z$ as desired.
  }
}
