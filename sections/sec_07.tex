\setcounter{subsection}{7-1}
\subsection{Countable and Uncountable Sets}

% Macros for this section
\def\zs{\ints \times \ints}
\def\zps{\pints \times \pints}

\exercise{1}{
  Show that $\rats$ is countably infinite.
}
\sol{
  \dwhitman

  \begin{lem}\label{lem:countun:zs}
    The set $\zs$ is countably infinite.
  \end{lem}
  \qproof{
    First, by Example 7.1, the set of integers $\ints$ is countably infinite so that there is a bijection $f$ from $\ints$ to $\pints$.
    We construct a bijection $g$ from $\zs$ to $\zps$.
    For any $(a,b) \in \zs$ define $g(a,b) = (f(a), f(b))$, noting that clearly $g(a,b) \in \zps$ since $\pints$ is the range of $f$.
    Hence $g$ is a function from $\zs$ to $\zps$.

    It is easy to show that $g$ is bijective.
    First, consider any $(a,b)$ and $(a',b')$ in $\zs$ where $(a,b) \neq (a',b')$ so that $a \neq a'$ or  $b \neq b'$.
    If $a \neq a'$ then $f(a) \neq f(a')$ since $f$ is bijective (and therefore injective).
    Thus we have that $g(a,b) = (f(a),f(b)) \neq (f(a'),f(b')) = g(a',b')$.
    A similar argument shows the same result when $b \neq b'$.
    Since $(a,b)$ and $(a',b')$ were arbitrary, this shows that $g$ is injective.

    Now consider any $(c,d) \in \zps$ so that $c,d \in \pints$.
    Since $f$ is surjective (since it is a bijection) there are $a,b \in \ints$ where $f(a) = c$ and $f(b) = d$.
    We then clearly have that $g(a,b) = (f(a),f(b)) = (c,d)$ so that $g$ is surjective $(c,d)$ was arbitrary.

    Therefore $g$ is a bijection.
    Now, we know from Corollary~7.4 that $\zps$ is countably infinite so that there must be a bijection $h$ from $\zps$ to $\pints$.
    It then follows that $h \circ g$ is bijection from $\zs$ to $\pints$, which shows the desired result by definition.
  }

  \mainprob
  \qproof{
    First we define a straightforward function $f$ from $\zs$ to $\rats$.
    First consider any $(m,n) \in \zs$.
    If $n \neq 0$ then let $q = m/n$.
    If $n = 0$ then set $q = 0$.
    Setting $f(m,n) = q$ we clearly have that $f$ is a function from $\zs$ to $\rats$.
    Now consider any rational $q$ so that by definition there are integers $m$ and $n$ where $q=m/n$.
    It then of course follows that $f(m,n) = m/n = q$, which shows that $f$ is surjective since $q$ was arbitrary.

    Now, from Lemma~\ref{lem:countun:zs} we know that $\zs$ is countably infinite so that there is a bijection $g$ from $\zs$ to $\pints$.
    Hence $\inv{g}$ is a bijection from $\pints$ to $\zs$.
    It then follows that the function $f \circ \inv{g}$ is a surjective function from $\pints$ to $\rats$.
    From this it follows from Theorem~7.1 that $\rats$ is countable.
    Since $\pints$ is a subset of $\rats$, it has to be that $\rats$ is infinite, and hence must be countably infinite.
  }
}

\exercise{2}{
  Show that the maps $f$ and $g$ of Examples~1 and 2  are bijections.
}
\sol{
  \dwhitman

  It is claimed in Example~7.1 that the function
  \gath{
    f(n) = \begin{cases}
      2n & n > 0 \\
      -2n+1 & n \leq 0
    \end{cases}
  }
  is a bijection from $\ints$ to $\pints$.
  \qproof{
    To show that $f$ is injective, consider $n,m \in \ints$ where $n \neq m$.

    Case: $n > 0$.
    Then $f(n) = 2n$, which is clearly even.
    If $m > 0$, then clearly $f(n) = 2n \neq 2m = f(m)$ since $n \neq m$.
    If $m \leq 0$ then $f(m) = -2m + 1 = 2(-m) + 1$ is clearly odd so that it must be that $f(n) \neq f(m)$.

    Case: $n \leq 0$.
    Then $f(n) = -2n+1 = 2(-n)+1$, which is clearly odd.
    If $m > 0$ then $f(m) = 2m$ is even so that it has to be that $f(n) \neq f(m)$.
    If $m \leq 0$ then $f(m) = -2m+1 \neq -2n+1 = f(n)$ since $n \neq m$.

    Thus in every case $f(n) \neq f(m)$, which shows that $f$ is injective since $n$ and $m$ were arbitrary.

    To show that $f$ is surjective, consider any $k \in \pints$.
    If $k$ is even then $k = 2n$ for some $n \in \pints$.
    Hence $n > 0$ (since $k > 0$ and $n=k/2$) so that $f(n) = 2n = k$, noting that $n \in \ints$ since $\pints \ss \ints$.
    If $k$ is odd then $k = 2m-1$ for some $m \in \pints$.
    So let $n=1-m$ so that clearly $n$ is an integer and
    \ali{
        m &\geq 1 & \text{(since $m \in \pints$)} \\
        -m &\leq -1 \\
        1-m &\leq 0 \\
        n &\leq 0 \,.
    }
    Thus $f(n) = -2n+1 = -2(1-m) + 1 = -2 + 2m + 1 = 2m-1 = k$.
    This shows that $f$ is surjective since $k$ was arbitrary.
    Therefore we have shown that $f$ is a bijection as desired.
  }

  Regarding Example~7.2, the following set is defined:
  \gath{
    A = \braces{(x,y) \in \zps \where y \leq x} \,.
  }
  Then the function $f$ is defined from $\zps$ to $A$ by
  \gath{
    f(x,y) = (x+y-1, y)
  }
  for $(x,y) \in \zps$.
  It is claimed that $f$ is a bijection.
  \qproof{
    First, it is not even clear that the range of $f$ is constrained to $A$, so let us show this.
    Consider any $(x,y) \in \zps$ so that $f(x,y) = (x+y-1, y)$.
    Since $x \geq 1$ and $y \geq 1$, we have that $x+y \geq 1 + 1 = 2 > 1$ so that $x+y-1 > 0$ and hence $x+y-1 \in \pints$.
    Thus clearly $f(x,y) = (x+y-1,y) \in \zps$.
    We also have
    \ali{
      1 &\leq x \\
      0 &\leq x-1 \\
      y &\leq x+y-1 \,.
    }
    Therefore it is clear that $f(x,y) = (x+y-1,y) \in A$ by definition.

    To show that $f$ is injective consider $(x,y)$ and $(x',y')$ in $\zps$ where $f(x,y) = (x+y-1,y) = (x'+y'-1,y') = f(x',y')$.
    Thus $x+y-1 = x'+y'-1$ and $y = y'$.
    Therefore $x+y-1 = x'+y'-1 = x'+y-1$, from which it obviously follows that $x = x'$ as well.
    Then $(x,y) = (x',y')$, which shows that $f$ is injective since $(x,y)$ and $(x',y')$ were arbitrary.

    Now consider any $(z,y) \in A$ so that $(z,y) \in \zps$ and $y \leq z$.
    Let $x = z-y+1$ so that clearly $z = x+y-1$.
    We also have
    \ali{
      y &\leq z = x+y-1 \\
      0 &\leq x-1 \\
      1 &\leq x
    }
    so that $(x,y) \in \zps$.
    Since also we have $f(x,y) = (x+y-1,y) = (z,y)$, $f$ is surjective since $(z,y)$ was arbitrary.
    This completes the proof that $f$ is a bijection.
  }

  The function $g$ is then defined from $A$ to $\pints$ by
  \gath{
    g(x,y) = \frac{1}{2}(x-1)x + y
  }
  for $(x,y) \in A$.
  This is also claimed to be a bijection.
  \qproof{
    First we show that the range of $g$ is indeed $\pints$ since this is not obvious.
    Consider any $(x,y) \in A$ so that $(x,y) \in \zps$ and $y \leq x$.
    First, if $x$ is even then $x = 2n$ for some $n \in \ints$.
    Then $g(x,y) = (x-1)x/2 + y = (2n-1)(2n)/2 + y = (2n-1)n+y$, which is clearly an integer.
    If $x$ is odd then $x = 2n+1$ for some integer $n$ so that
    \gath{
      g(x,y) = (x-1)x/2+y = (2n+1-1)(2n+1)/2+y = (2n)(2n+1)/2+y = n(2n+1)+y \,,
    }
    which is also clearly an integer.
    We also have that $-y < 0$ since $y > 0$ so that
    \ali{
      x &\geq 1 \\
      x-1 &\geq 0 \\
      \frac{1}{2}(x-1) &\geq 0 & \text{(since $1/2 > 0$)} \\
      \frac{1}{2}(x-1)x &\geq 0 > -y & \text{(since $x>0$)} \\
      \frac{1}{2}(x-1)x + y &> 0 \\
      g(x,y) &> 0 \,.
    }
    Since we have shown that $g(x,y) \in \ints$ as well, it follows that $g(x,y) \in \pints$.

    Consider any $(x,y) \in A$ so that $(x,y) \in \zps$ and $y \leq x$.
    Then clearly
    \ali{
      g(x,y) &= \frac{1}{2}(x-1)x + y \leq \frac{1}{2}(x-1)x + x \\
      &< \frac{1}{2}(x-1)x+x+1 = \frac{1}{2}(x^2 - x + 2x) + 1 \\
      &= \frac{1}{2}(x^2 + x) + 1 = \frac{1}{2}x(x+1) + 1 \\
      &= \frac{1}{2}(x+1-1)(x+1) + 1 \\
      &= g(x+1,1) \,.
    }
    A simple inductive argument shows that $g(x,y) < g(x+n,1)$ for any $n \in \pints$.
    This was just shown for $n=1$.
    Then, assuming it true for $n$, we have that $g(x,y) < g(x+n,1) < g((x+n)+1,1) = g(x+(n+1),1)$, which completes the induction.

    So consider any $(x,y)$ and $(x',y')$ in $A$ so that $(x,y)$ and $(x',y')$ are in $\zps$, $y \leq x$, and $y' \leq x'$.
    Also suppose that $(x,y) \neq (x',y')$ so that either $x \neq x'$ or $y \neq y$.
    If $x=x'$ then it has to be that $y \neq y'$ so that clearly
    \ali{
      y &\neq y' \\
      \frac{1}{2}(x-1)x + y &\neq \frac{1}{2}(x-1)x + y' \\
      \frac{1}{2}(x-1)x + y &\neq \frac{1}{2}(x'-1)x' + y' \\
      g(x,y) &\neq g(x',y') \,.
    }
    If $x \neq x'$ then we can assume that $x < x'$.
    Then let $n = x'-x$ so that clearly $n > 0$ and $x' = x+n$.
    By what was just shown, we have
    \ali{
      g(x,y) < g(x+n,1) = g(x',1) = \frac{1}{2}(x'-1)x' + 1 \leq \frac{1}{2}(x'-1)x' + y' = g(x',y')
    }
    since $1 \leq y'$.
    Thus $g(x,y) \neq g(x',y')$.
    Since this is true in both cases, this shows that $g$ is injective since $(x,y)$ and $(x',y')$ were arbitrary.

    To show that $g$ is also surjective, consider any $z \in \pints$.
    Define the set $B = \braces{x \in \pints \where g(x,1) \leq z}$.
    First, we have that $g(1,1) = 1 \leq z$ since $z \in \pints$ so that $1 \in B$ and therefore $B \neq \es$.
    If $z = 1$ then clearly $z = 1 \leq 1 = g(1,1) = g(z,1)$.
    If $z \neq 1$ then we have
    \ali{
      2 &\leq z \\
      1 &\leq \frac{1}{2}z \\
      z-1 &\leq \frac{1}{2}(z-1)z \\
      z &\leq \frac{1}{2}(z-1)z + 1 \\
      z &\leq g(z,1)
    }
    Now consider any $x,y \in \pints$ where $x < y$.
    It then follows from what was shown above that $g(x,1) \leq g(x,y) < g(x+1,1)$.
    From this we clearly have that the function $g(x,1)$ is monotonically increasing in $x$, i.e. for $x,y \in \pints$, $x < y$ implies that $g(x,1) < g(y,1)$.
    By the contrapositive of this, $g(x,1) \geq g(y,1)$ implies that $x \geq y$.
    With this in mind, consider any $x \in B$ so $g(x,1) \leq z \leq g(z,1)$.
    Then this implies that $x \leq z$, which shows that $z$ is an upper bound of $B$ since $x$ was arbitrary.

    We have thus shown that $B$ is a nonempty set of integers that is bounded above.
    It then follows from Exercise~4.9a that $B$ has a largest element $x$.
    Now let $y = z - g(x,1) + 1$, noting that, since $x \in B$,
    \ali{
      g(x,1) &\leq z \\
      0 &\leq z - g(x,1) \\
      1 &\leq z - g(x,1) + 1 \\
      1 &\leq y
    }
    and hence $y \in \pints$ so that $(x,y) \in \zps$.
    We also must have that $z < g(x+1,1)$ since otherwise we would have that $x+1 \in B$, which would violate the definition of $x$ as being the largest element of $B$.
    Thus we have
    \ali{
      z &\leq g(x+1,1) - 1 \\
      z &\leq \frac{1}{2}(x+1-1)(x+1) + 1 - 1 \\
      z &\leq \frac{1}{2}(x+1)x \\
      z &\leq x + \frac{1}{2}(x-1)x \\
      z &\leq x + \frac{1}{2}(x-1)x + 1 - 1 \\
      z &\leq x + g(x,1) - 1 \\
      z - g(x,1) + 1 &\leq x \\
      y &\leq x
    }
    so that $(x,y) \in A$.
    
    Lastly, since $y = z - g(x,1) + 1$, we clearly have
    \gath{
      z = y + g(x,1) - 1 = y + \frac{1}{2}(x-1)x + 1 - 1 = \frac{1}{2}(x-1)x + y = g(x,y) \,.
    }
    This shows that $g$ is surjective since $z$ was arbitrary, thereby completing the long and arduous proof that $g$ is a bijection.
  }
}

\exercise{3}{
  Let $X$ be the two-element set $\braces{0,1}$.
  Show there is a bijective correspondence between the set $\pset{\pints}$ and the cartesian product $X^\w$.
}
\sol{
  \dwhitman

  \qproof{
    Similar to Exercise~6.6a, we construct such a bijection $f$ from $\pset{\pints}$ to $X^\w$.
    For any $A \in \pset{\pints}$ we have that $A \ss \pints$.
    Then, for $i \in \pints$, set
    \gath{
      x_i = \begin{cases}
        1 & i \in A \\
        0 & i \notin A
      \end{cases}
    }
    and set $f(A) = (x_1,x_2,\ldots)$ so that clearly $f(A) \in X^\w$.

    To show that $f$ is injective consider $A$ and $A'$ in $\pset{\pints}$ where $A \neq A'$.
    Without loss of generality, we can assume that there is an $i \in A$ where $i \notin A'$, noting that of course $i \in \pints$ since $A \ss \pints$.
    Let $\vx = (x_1, x_2, \ldots) = f(A)$ and $\vx' = (x_1', x_2', \ldots) = f(A')$.
    Then $x_i = 1 \neq 0 = x_i'$ by the definition of $f$ since $i \in A$ but $i \notin A'$.
    Thus clearly $f(A) = \vx \neq \vx' = f(A')$, which shows that $f$ is injective since $A$ and $A'$ were arbitrary.

    Now consider any $\vx = (x_1,x_2,\ldots) \in X^\w$ and define the set $A = \braces{i \in \pints \where x_i = 1}$ so that clearly $A \ss \pints$ and hence $A \in \pset{\pints}$.
    Let $\vx' = (x_1',x_2',\ldots) = f(A)$ and consider $i \in \pints$.
    If $i \in A$ then $x_i' = 1 = x_i$ by the definitions of $A$ and $f$.
    If $i \notin A$ then $x_i \neq 1$ since otherwise $i \in A$ by definition.
    Since $x_i \in X = \braces{0,1}$ it clearly must be that $x_i = 0$.
    We then also have that $x_i' = 0$ by the definition of $f$, and thus $x_i = 0 = x_i'$.
    Since $x_i = x_i'$ in both cases and $i$ was arbitrary, it follows that $\vx = \vx' = f(A)$.
    This proves that $f$ is surjective since $\vx$ was arbitrary.

    Hence it has been shown that $f$ is a bijection as desired.
  }
}

\exercise{4}{
  \eparts{
  \item A real number $x$ is said to be \boldit{algebraic} (over the rationals) if it satisfies some polynomial equation of positive degree
    \gath{
      x^n + a_{n-1}x^{n-1} + \cdots + a_1 x + a_0 = 0
    }
    with rational coefficients $a_i$.
    Assuming that each polynomial equation has only finitely many roots, show that the set of algebraic numbers is countable.
  \item A real number is said to be \boldit{transcendental} if it is not algebraic.
    Assuming the reals are uncountable, show that the transcendental numbers are uncountable.
    (It is a somewhat surprising fact that only two transcendental numbers are familiar to us: $e$ and $\pi$.
    Even proving these two numbers transcendental is highly nontrivial.)
  }
}
\sol{
  \dwhitman

  \def\vq{\vect{q}}
  (a)
  \qproof{
    First consider arbitrary degree $n \in \pints$.
    Then for each $\vq = (q_1, \ldots, q_n) \in \rats^n$, there is a corresponding polynomial equation in $x$:
    \gath{
      x^n + \sum_{i = 1}^n q_i x^{i-1}  = x^n + q_n x^{n-1} + \cdots + q_2 x + q_1 = 0 \,,
    }
    which is assumed to have a finite number of solutions.
    So let $X_\vq$ be the finite set real numbers that are solutions.
    (We note that the polynomial corresponding to the vector $\vq = (0,\ldots,0) \in \rats^n$ becomes $0=0$ so that any real number $x$ satisfies it.
    Similarly the polynomial corresponding to $\vq = (q_1, 0, \ldots, 0) \in \rats^n$ for nonzero $q_1$ corresponds to the equation $q_1 = 0$, which has no solutions.
    Of course $X_\vq = \es$ is still finite in this case.
    For the infinite-solution case we could simply remove the zero vector from $\rats^n$ without changing the argument in any substantial way.
    This is also taken care of if we really do assume that \emph{any} polynomial has a finite number of solutions as we are evidently doing here.)

    Now, we clearly have that $\rats^n$ is countable by Theorem~7.6 since it is a finite product of countable sets (since it was shown in Exercise~7.1 that $\rats$ is countable).
    Thus the set $A_n = \bigcup_{\vq \in \rats^n} X_\vq$ is countable by Theorem~7.5 since it is a countable union of finite (and therefore countable) sets.
    Of course, this is the set of all algebraic numbers from polynomials of degree $n$.
    Then $A = \bigcup_{n \in \pints} A_n$ is the set of all algebraic numbers, which is also then countable by Theorem~7.5 since each $A_n$ was shown to be countable.
  }

  (b)
  \qproof{
    As in part (a), let $A \ss \reals$ be the set of algebraic numbers so that clearly, by definition, $T = \reals - A$ is the set of transcendental numbers.
    Note that clearly $\reals = A \cup T$ so that, if $T$ were countable, then $\reals$ would be too since it is a finite union of countable sets.
    This of course contradicts the (hitherto unproven) fact that $\reals$ is uncountable so that it must be that $T$ is also uncountable as desired.
  }
}

\def\parta{The set $A$ of all functions $f: \braces{0,1} \to \pints$.}
\def\partb{The set $B_n$ of all functions $f: \intsfin{n} \to \pints$.}
\def\partc{The set $C = \bigcup_{n \in \pints} B_n$.}
\def\partd{The set $D$ of all functions $f: \pints \to \pints$.}
\def\parte{The set $E$ of all functions $f: \pints \to \braces{0,1}$.}
\def\partf{
  The set $F$ of all functions $f: \pints \to \braces{0,1}$ that are ``eventually zero.''
  [We say that $f$ is \boldit{eventually zero} if there is a positive integer $N$ such that $f(n) = 0$ for all $n \geq N$.]
}
\def\partg{The set $G$ of all functions $f: \pints \to \pints$ that are eventually 1.}
\def\parth{The set $H$ of all functions $f: \pints \to \pints$ that are eventually constant.}
\def\parti{The set $I$ of all two-element subsets of $\pints$.}
\def\partj{The set $J$ of all finite subsets of $\pints$.}
\exercise{5}{
  Determine, for each of the following sets, whether or not it is countable.
  Justify your answers.
  \eparts{
  \item \parta
  \item \partb
  \item \partc
  \item \partd
  \item \parte
  \item \partf
  \item \partg
  \item \parth
  \item \parti
  \item \partj
  }
}
\sol{
  \dwhitman

  (a) \parta

  We claim that $A$ is countable.
  \qproof{
    For any $f \in A$, clearly the mapping $g(f) = (f(0), f(1))$ is a bijection from $A$ to $\pints^2$.
    Since $\pints^2$ is a finite cartesian product of countable sets, it follows that it is also countable by Theorem~7.6.
    Hence there is a bijection $h :\pints^2 \to \pints$.
    It is then obvious that $h \circ g$ is a bijection from $A$ to $\pints$ so that $A$ is countable.
  }

  (b) \partb

  We claim that $B_n$ (for some $n \in \pints$) is also countable.
  \qproof{
    By the definition of $\pints^n$, $B_n = \pints^n$, which is clearly a finite cartesian product of countable sets.
    Thus $B_n$ is countable by Theorem~7.6.
  }

  (c) \partc

  We claim that $C$ is countable.
  \qproof{
    Since $n$ was arbitrary in part (b), we showed that $B_n$ is countable for any $n \in \pints$.
    Thus $C = \bigcup_{n \in \pints} B_n$ is a countable union of countable sets, which is itself also countable by Theorem~7.5 as desired.
  }

  (d) \partd

  Clearly $D = \pints^\w$, which we claim is uncountable.
  \qproof{
    We proceed to show, as in Theorem~7.7, that any function $g: \pints \to D$ is not surjective.
    So denote
    \gath{
      g(n) = \vx_n = (x_{n1}, x_{n2}, \ldots) \,,
    }
    where of course each $x_{nm} \in \pints$ since $\vx_n \in D$ and so is a function from $\pints$ to $\pints$.
    Now set
    \gath{
      y_n = \begin{cases}
        0 & x_{nn} \neq 0 \\
        1 & x_{nn} = 0
      \end{cases}
    }
    so that clearly $\vy = (y_1, y_2, \ldots)$ is an element of $D$.
    Now consider any $n \in \pints$.
    If $x_{nn} = 0$ then $y_n = 1 \neq 0 = x_{nn}$, and if $x_{nn} \neq 0$ then $y_n = 0 \neq x_{nn}$.
    Thus clearly $g(n) = \vx_n\neq \vy$ since the $n$th element of each differs.
    This shows that $g$ cannot be surjective since $\vy \in D$ and $n$ was arbitrary.
    It then follows from Theorem~7.1 that $D$ is not countable.
  }

  (e) \parte

  This is exactly the set $X^\w$ in Theorem~7.7, wherein it was shown to be uncountable.

  (f) \partf

  We claim that $F$ is countable.
  \qproof{
    For brevity define $X = \braces{0,1}$.
    First let $F_N$ be the set of all eventually zero functions $f:\pints \to X$ that are zero for $n \geq N$, where of course $N \in \pints$.
    Then clearly $F = \bigcup_{N \in \pints} F_N$.

    We show that each $F_N$ is countable.
    So consider any $N \in \pints$.
    If $N=1$ then clearly $f:\pints \to X$ defined by $f(n)=0$ for $n \in \pints$ (which could be denoted $(0,0,\ldots)$) is the only element of $F_N = F_1$ so that $f_N$ is clearly finite and therefore countable.
    If $N > 1$ then for $\vx = (x_1, \ldots, x_{N-1}) \in X^{N-1}$ define
    \gath{
      y_n = \begin{cases}
        x_n & n < N \\
        0 & n \geq N
      \end{cases}
    }
    for $n \in \pints$.
    It then trivial to show that $g$ defined by $g(\vx) = \vy = (y_1, y_2, \ldots)$ is a bijection from $X^{N-1}$ to $F_N$.
    Now, since $X = \braces{0,1}$ is finite, $X^{N-1}$ is finite by Corollary~6.8.
    Since this is in bijective correspondence with $F_N$, it follows that it must also be finite and therefore countable.

    Thus $F = \bigcup_{N \in \pints} F_N$ is a countable union of countable sets, and so is countable by Theorem~7.5 as desired
  }

  (g) \partg

  Since $G$ is clearly a subset of $H$ in part (h) below, it is countable by Corollary~7.3 since $H$ is.

  (h) \parth

  We claim that $H$ is countable.
  \qproof{
    For $N \in \pints$, let $H_N$ be the set of functions $f: \pints \to \pints$ such that $f(n)$ is constant for $n \geq N$.
    Thus clearly $H = \bigcup_{N \in \pints} H_N$.

    We show that each $H_N$ is countable.
    So consider $N \in \pints$.
    For any $\vx = (x_1, \ldots, x_N) \in \pints^N$ define
    \gath{
      y_n = \begin{cases}
        x_n & n < N \\
        x_N & n \geq N
      \end{cases}
    }
    for $n \in \pints$, and set $g(\vx) = \vy = (y_1, y_2, \ldots)$.
    It is then a simple matter to show that $g$ is a bijection from $\pints^N$ to $H_N$.
    Then, since $\pints^N$ is a finite product of countable sets, it is countable by Theorem~7.6.
    Hence $H_N$ must also be countable since there is a bijective correspondence between them.

    Thus $H = \bigcup_{N \in \pints} H_N$ is the countable union of countable sets so that it must also be countable by Theorem~7.5.
  }

  (i) \parti

  In part (j) below it is shown that the set $J$ of all finite subsets of $\pints$ is countable.
  Since clearly $I \ss J$, it follows that $I$ is also countable by Corollary~7.3.

  (j) \partj

  We claim that $J$ is countable.
  \qproof{
    First, let $J_n$ denote the set of $n$-element subsets of $\pints$ (for $n \in pints$), and let $J_0 = \braces{\es}$ since $\es$ is the only ``zero-element'' subset of $\pints$.
    Clearly then $J = \bigcup_{n \in \pints \cup \braces{0}} J_n$.
    Obviously $J_0$ is finite and therefore countable.
    Next, we show that $J_n$ is countable for any $n \in \pints$.

    So consider any such $n \in \pints$.
    Clearly $\pints^n$ is countable by Theorem~7.6 since it is a finite product of countable sets.
    Hence there is a bijection $f : \pints^n \to \pints$.
    We now construct an injective function $g : J_n \to \pints^n$.
    For any $X \in J_n$, we can choose a bijection $h : X \to \intsfin{n}$ since it has $n$ elements.
    Since $X \ss \pints$, clearly $\inv{h} \in \pints^n$, so set $g(X) = \inv{h}$.
    To show that $g$ is injective, consider $X$ and $X'$ in $J_n$ where $X \neq X'$.
    Without loss of generality we can assume that there is an $x \in X$ where $x \notin X'$.
    Let $h$ and $h'$ be the chosen bijections from $X$ and $X'$, respectively, to $\intsfin{n}$ so that by definition $g(X) = \inv{h}$ and $g(X') = \inv{h'}$.
    Now let $k = h(x)$ so that $\inv{h}(k) = x$.
    It has to be that $\inv{h'}(k) \neq x$ since otherwise $x$ would be in $X'$.
    Hence $\inv{h}(k) = x \neq \inv{h'}(k)$, which shows that $g(X) = \inv{h} \neq \inv{h'} = g(X')$.
    Thus $g$ is injective since $X$ and $X'$ were arbitrary.
    It then follows that $f \circ g$ is an injective function from $J_n$ to $\pints$ so that $J_n$ must be countable by Theorem~7.1.

    Since $n$ was arbitrary, this shows that $J_n$ is countable for any $n \in \pints$.
    From this it follows from Theorem~7.5 that $J = \bigcup_{n \in \pints \cup \braces{0}} J_n$ is also countable since it is clearly a countable union of countable sets.
  }
}
