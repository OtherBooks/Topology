\setcounter{subsection}{16-1}
\subsection{The Subspace Topology}

\exercise{1}{
  Show that if $Y$ is a subspace of $X$ and $A$ is a subspace of $Y$, then the topology $A$ inherits as a subspace of $Y$ is the same as the topology it inherits as a subspace of $X$.
}
\sol{
  \dwhitman

  \qproof{
    Let $\cT$ be the topology on $X$ and  $\cT_Y$ be the subspace topology that $Y$ inherits from $X$.
    Also let $\cT_A$ and $\cT_A'$ be the topologies that $A$ inherits as a subspace of $Y$ and $X$, respectively.
    Therefore we must show that $\cT_A = \cT_A'$.
    Now, by definition of subspace topologies we have that,
    \ali{
      \cT_Y &= \braces{Y \cap U \where U \in \cT} &
      \cT_A &= \braces{A \cap U \where U \in \cT_Y} &
      \cT_A' &= \braces{A \cap U \where U \in \cT} \,.
    }
    Now suppose that $W \in \cT_A$ so that $W = A \cap V$ for some $V \in \cT_Y$.
    Then we have that $V = Y \cap U$ for some $U \in \cT$, and hence
    \gath{
      W = A \cap V = A \cap (Y \cap U) = (A \cap Y) \cap U = A \cap U
    }
    since we have that $A \cap Y = A$ since $A \ss Y$.
    Since $U \in \cT$ this clearly shows that $W \in \cT_A'$ so that $\cT_A \ss \cT_A'$ since $W$ was arbitrary.

    Then, for any $W \in \cT_A'$, we have that $W = A \cap U$ for some $U \in \cT$.
    Let $V = Y \cap U$ so that clearly $V \in \cT_Y$.
    Then as before we have that $A = A \cap Y$ since $A \ss Y$ so that
    \gath{
      W = A \cap U = (A \cap Y) \cap U = A \cap (Y \cap U) = A \cap V \,,
    }
    and thus $W \in \cT_A$ since $V \in \cT_Y$.
    Since $W$ was arbitrary this shows that $\cT_A' \ss \cT_A$, which completes the proof that $\cT_A = \cT_A'$.
  }
}

\exercise{2}{
  If $\cT$ and $\cT'$ are topologies on $X$ and $\cT'$ is strictly finer than $\cT$, what can you say about the corresponding subspace topologies on the subset $Y$ of $X$?
}
\sol{
  \dwhitman

  Let $\cT_Y$ and $\cT_Y'$ be the subspace topologies on $Y$ corresponding to $\cT$ and $\cT'$, respectively.
  We claim that $\cT_Y$ is finer than $\cT_Y$ but not necessarily strictly finer.
  \qproof{
    First, we have that
    \ali{
      \cT_Y &= \braces{Y \cap U \where U \in \cT} &
      \cT_Y' &= \braces{Y \cap U \where U \in \cT'}
    }
    by the definition of subspace topologies.
    So for any $V \in \cT_Y$ we have that $V = Y \cap U$ where $U \in \cT$.
    Then also $U \in \cT'$ since $\cT'$ is finer than $\cT$.
    This shows that $V \in \cT_Y'$ since $V = Y \cap U$ where $U \in \cT'$.
    Hence $\cT_Y'$ is finer than $\cT_Y$ since $V$ was arbitrary.

    To show that it is not necessarily strictly finer, consider the sets $X = \braces{a,b,c}$ and $Y = \braces{a,b}$ so that clearly $Y \ss X$.
    Consider also the topologies
    \ali{
      \cT &= \braces{\es, X, \braces{a,b}} &
      \cT' &= \braces{\es, X, \braces{a,b}, \braces{c}} 
    }
    on $X$ so that clearly $\cT'$ is strictly finer than $\cT$.
    This results in the subspace topologies
    \ali{
      \cT_Y &= \braces{\es, Y} &
      \cT_Y' &= \braces{\es, Y} \,,
    }
    which are clearly the same so that $\cT_Y'$ is not strictly finer than $\cT_Y'$, noting that it is technically still finer.
    However, if we instead have the topologies
    \ali{
      \cT &= \braces{\es, X, \braces{a,b}} &
      \cT' &= \braces{\es, X, \braces{a,b}, \braces{b}} 
    }
    then 
    \ali{
      \cT_Y &= \braces{\es, Y} &
      \cT_Y' &= \braces{\es, Y, \braces{b}}
    }
    so that $\cT_Y'$ \emph{is} strictly finer than $\cT_Y$.
    Thus we can say nothing about the strictness of relation of the subspace topologies.
  }
}

\def\oh{\tfrac{1}{2}}
\exercise{3}{
  Consider the set $Y = [-1, 1]$ as a subspace of $\reals$.
  Which of the following sets are open in $Y$?
  Which are open in $\reals$?
  \ali{
    A &= \braces{x \where \oh < \abs{x} < 1}, \\
    B &= \braces{x \where \oh < \abs{x} \leq 1}, \\
    C &= \braces{x \where \oh \leq \abs{x} < 1}, \\
    D &= \braces{x \where \oh \leq \abs{x} \leq 1}, \\
    E &= \braces{x \where 0 < \abs{x} < 1 \text{ and } 1/x \notin \pints}.
  }
}
\sol{
  \dwhitman
  
  \begin{lem}\label{lem:subset:abs}
    If $a,b \in \reals$ such that $0 \leq a < b$ then the following are true:
    \ali{
      \braces{x \in \reals \where a < \abs{x} < b} &= (-b, -a) \cup (a, b) &
      \braces{x \in \reals \where a \leq \abs{x} \leq b} &= [-b, -a] \cup [a, b] \\
      \braces{x \in \reals \where a \leq \abs{x} < b} &= \opcl{-b, -a} \cup \clop{a, b} &
      \braces{x \in \reals \where a < \abs{x} \leq b} &= \clop{-b, -a} \cup \opcl{a, b} \,.
    }
  \end{lem}
  \qproof{
    We prove only the first of these as the rest follow from nearly identical arguments.
    Let $A = \braces{x \in \reals \where a < \abs{x} < b}$ and $B = (-b, -a) \cup (a, b)$ so that we must show that $A = B$.

    So consider $x \in A$ so that $a < \abs{x} < b$.
    If $x \geq 0$ then $\abs{x} = x$ so that $a < x < b$ and hence $x \in (a,b)$.
    If $x < 0$ then $\abs{x} = -x$ so that $a < -x < b$, and thus $-a > x > -b$ so that $x \in (-b, -a)$.
    Thus in either case $x \in B$ so that $A \ss B$.

    Now let $x \in B$ so that either $x \in (-b,-a)$ or $x \in (a,b)$.
    In the former case we have that $x < -a \leq 0$ since $a \geq 0$ so that $\abs{x} = -x$ and therefore
    \gath{
      x \in (-b, -a) \imp -b < x < -a \imp b > -x = \abs{x} > a \imp x \in A \,.
    }
    In the  latter case we have that $x > a \geq 0$ so that $\abs{x} = x$ and therefore
    \gath{
      x \in (a, b) \imp a < x = \abs{x} < b \imp x \in A \,.
    }
    This shows that $B \ss A$ since $x$ was arbitrary, and thus $A = B$ as desired.
  }

  \begin{lem}\label{lem:subset:open}
    Suppose that $X$ is a topological space and $Y \ss X$ with the subspace topology.
    Then, if a set $U \ss Y$ is open in $X$, then it is also open in $Y$.
  \end{lem}
  \qproof{
    So suppose that $U \ss Y$ is open in $X$.
    Then we have that $Y \cap U = U$ is also open in $Y$ by the definition of the subspace topology.
  }

  \mainprob

  First we claim that $A$ is open in both $\reals$ and $Y$.
  \qproof{
    We have from Lemma~\ref{lem:subset:abs} that $A = (-1, -\oh) \cup (\oh, 1)$ which is clearly the union of basis elements so that $A$ is open in $\reals$.
    We also have that $A \ss Y$ so that $A$ is open in $Y$ by Lemma~\ref{lem:subset:open} since it is open in $\reals$.
  }

  Next we claim that $B$ is open in $Y$ but not in $\reals$.
  \qproof{
    By Lemmma~\ref{lem:subset:abs} we have that $B = \clop{-1, -\oh} \cup \opcl{\oh, 1}$.
    First, consider the sets $(-2,-\oh)$ and $(\oh,2)$, which are clearly both basis elements and therefore open in $\reals$.
    We then have that $(-2,-\oh) \cap Y = \clop{-1,-\oh}$ and $(\oh,2) \cap Y = \opcl{\oh, 1}$ so that these sets are open in $Y$ by the definition of the subspace topology.
    Clearly then their union $B = \clop{-1, -\oh} \cup \opcl{\oh, 1}$ is then also open in $Y$.

    It is also easy to see that $B$ is not open in $\reals$.
    For example, $-1 \in B$ but for any basis element $B' = (a,b)$ containing $-1$ we have that $a < -1 < b$ so that $a < (a-1)/2 < -1 < b$ and hence $(a-1)/2 \in B'$.
    Clearly though $(a-1)/2 \notin B$ so that $B'$ cannot be a subset of $B$.
    Thus suffices to show that $B$ is not open by the definition of the topology of $\reals$ generated by its basis.
  }

  We claim that $C$ is open neither in $\reals$ nor $Y$.
  \qproof{
    By Lemmma~\ref{lem:subset:abs} we have that $C = \opcl{-1,-\oh} \cup \clop{\oh, 1}$.
    If $\cB$ is the standard basis on $\reals$, then, by Lemma~16.1, the set $\cB_Y = \braces{B \cap Y \where B \in \cB}$ is a basis for the subspace $Y$.
    So consider the point $x = \oh$ and any basis element $B_Y \in \cB_Y$ containing $x$.
    Then we have that $B_Y = Y \cap B_X$ for some basis element $B_X = (a,b)$ in $\cB$, and thus $a < x < b$ since $x \in B_X$.
    Let $a' = \max(a, -\oh)$ and set $y = (a'+x)/2$ so that
    \gath{
      a \leq a' < (a'+x)/2 = y < x < b \,,
    }
    and hence $y \in (a,b) = B_X$.
    Also we have
    \gath{
      -1 < -\oh \leq a' < (a'+x)/2 = y < x = \oh < 1
    }
    so that $y \in [-1,1] = Y$.
    Therefore $y \in Y \cap B_X = B_Y$.
    However, since $-\oh < y < \oh$, clearly $y \notin C$ so that $B_Y$ cannot be a subset of $C$.
    Since the basis element $B_Y \in \cB_Y$ was arbitrary, this suffices to show that $C$ cannot be open in $Y$ since $\cB_Y$ is a basis.
    Since also clearly $C \ss Y$, it follows from the contrapositive of Lemma~\ref{lem:subset:open} that $C$ is not open in $\reals$ either.
  }

  Next we claim that $D$ is also not open in $\reals$ or $Y$.
  \qproof{
    This follows from basically the same argument as the previous proof, again using the point $x = \oh$ to show that any basis element of $Y$ that contains $x$ cannot be a subset of $D$.
  }

  Lastly, we claim that $E$ is open in both $\reals$ and $Y$.
  \qproof{
    First, it is trivial to show that
    \gath{
      E = \braces{x \in \reals \where 0 < \abs{x} < 1} - K = \squares{(-1,0) \cup (0,1)} - K \,,
    }
    where we have used Lemma~\ref{lem:subset:abs}.
    Now consider any $x \in E$ so that $x \in (-1,0) \cup (0,1)$ and $x \notin K$.
    If $x \in (-1,0)$ then clearly the basis element $(-1,0)$ contains $x$ and is a subset of $E$ since $(-1,0) \cap K = \es$.

    On the other hand, if $x \in (0,1)$ then $x \notin K$ so that $1/x \notin \pints$.
    From this it follows from Exercise~4.9b that there is exactly one positive integer $n$ such that $n < 1/x < n+1$.
    We then have that $1/(n+1) < x < 1/n$.
    So let $B = (1/(n+1), 1/n)$ so that clearly $x \in B$, $B \cap K = \es$, and $B$ is a basis element of the standard topology on $\reals$.
    Since $B \cap K = \es$ and clearly $0 < 1/(n+1) < 1/n \leq 1$, it also follows that $B \ss E$.

    Hence in either case there is a basis element of $\reals$ that contains $x$ and is a subset of $E$.
    This suffices to show that $E$ is open in $\reals$.
    Since clearly $E \ss Y$, we also clearly have that $E$ is open in $Y$ by Lemma~\ref{lem:subset:open}.
  }
}

\exercise{4}{
  A map $f: X \to Y$ is said to be an \boldit{open map} if for every open set $U$ of $X$, the set $f(U)$ is open in $Y$.
  Show that $\pi_1 : X \times Y \to X$ and $\pi_2 : X \times Y \to Y$ are open maps.
}
\sol{
  \dwhitman

  \qproof{
    Suppose that $U$ is an open subset of $X \times Y$.
    Consider any $x \in \pi_1(U)$ so that there is a $y \in Y$ such that $(x,y) \in U$.
    Then there is a basis element $A \times B$ of the product topology on $X \times Y$ where $(x,y) \in A \times B \ss U$.
    Then $A$ and $B$ are open sets of $X$ and $Y$, respectively, since $A \times B$ is a basis element of the product topology.
    Clearly we have that $x \in A$ since $(x,y) \in A \times B$.
    Now, for any $x' \in A$, we have that $(x',y) \in A \times B$ so that $(x',y) \in U$.
    Hence $x' = \pi_1(x',y) \in \pi_1(U)$, which shows that $A \ss \pi_1(U)$ since $x'$ was arbitrary.
    Then, since $A$ is an open subset of $X$, there is a basis element $A'$ where $x \in A' \ss A \ss \pi_1(U)$.
    This suffices to show that $\pi_1(U)$ is an open subset of $X$ since $x$ was arbitrary.
    An analogous argument shows that $\pi_2$ is also an open map.
  }
}

\exercise{5}{
  Let $X$ and $X'$ denote a single set in the topologies $\cT$ and $\cT'$, respectively; let $Y$ and $Y'$ denote a single set in the topologies $\cU$ and $\cU'$, respectively.
  Assume these sets are nonempty.
  \eparts{
  \item Show that if $\cT' \sps \cT$ and $\cU' \sps \cU$, then the product topology on $X' \times Y'$ is finer than the product topology on $X \times Y$.
  \item Does the converse of (a) hold? Justify your answer.
  }
}
\sol{
  \dwhitman

  In what follows let $\cT_p'$ and $\cT_p$ denote the product topologies on $X' \times Y'$ and $X \times Y$, respectively.
  
  (a)
  \qproof{
    Consider any $W \in \cT_p$ and any $(x,y) \in W$, noting that obviously $W \ss X \times Y$.
    Then there is a basis element $U \times V$ of $\cT_p$ such that $(x,y) \in U \times V$ and $U \times V \ss W$.
    By the definition of the product topology, we have that $U$ and $V$ are open sets in $\cT$ and $\cU$, respectively.
    Then we also have that $U \in \cT'$ and $V \in \cU'$ since $\cT \ss \cT'$ and $\cU \ss \cU'$.
    Hence $U \times V$ is also a basis element of $\cT_p'$.
    Since we know that $(x,y) \in U \times V$, $U \times V \ss W$, and $(x,y) \in W$ was arbitrary, this suffices to show that $W$ is an open subset of $X' \times Y'$ and hence $W \in \cT_p'$.
    This in turn shows that $\cT_p \ss \cT_p'$ since $W$ was arbitrary.
  }

  (b) We claim that the converse does not always hold.
  \qproof{
    As a counterexample consider $A = \braces{a, b, c, d}$ so that clearly
    \ali{
      \cT' &= \braces{\es, A, \braces{a,b}, \braces{c,d}} \\
      \cT &= \braces{\es, A, \braces{a,b}, \braces{c,d}, \braces{c}, \braces{d}, \braces{a,b,c}, \braces{a,b,d}} 
    }
    are topologies on $A$.
    Clearly also $\cT'$ is not finer that $\cT$.
    Similarly let $B = \braces{1, 2, 3, 4}$ so that
    \ali{
      \cU' &= \braces{\es, B, \braces{1,2}, \braces{3,4}} \\
      \cU &= \braces{\es, B, \braces{1,2}, \braces{3,4}, \braces{3}, \braces{4}, \braces{1,2,3}, \braces{1,2,4}} 
    }
    are topologies on $B$, also noting that clearly $\cU'$ is not finer that $\cU$.
    Now let $X = X' = \braces{a,b}$ and $Y = Y' = \braces{1,2}$ so that clearly $X$ and $X'$ are in topologies $\cT$ and $\cT'$, respectively, and $Y$ and $Y'$ are in $\cU$ and $\cU'$, respectively.

    Then the bases for the product topologies $\cT_p$ on $X \times Y$ and $\cT_p'$ on $X' \times Y'$ are then
    \ali{
      \cB &= \braces{\es, X \times Y} &
      \cB' &= \braces{\es, X' \times Y'} = \braces{\es, X \times Y} = \cB \,,
    }
    respectively, since there are no subsets of $X$ in $\cT$ or $\cT'$ other than $\es$ and $X$ itself, and similarly no subsets of $Y$ in $\cU$ or $\cU'$ other than $\es$ and $Y$.
    Since their bases are the same, clearly $\cT_p = \cT_p'$ so that it is true that $\cT_p'$ is finer than $\cT_p$ (though not strictly so).
  }
}

\exercise{6}{
  Show that the countable collection
  \gath{
    \braces{(a,b) \times (c,d) \where \text{$a < b$ and $c < d$ and $a,b,c,d$ are rational}}
  }
  is a basis for $\reals^2$.
}
\sol{
  \dwhitman

  \qproof{
    It was proven in Exercise~13.8a that the set
    \gath{
      \cB = \braces{(a,b) \where \text{$a < b$, $a$ and $b$ rational}}
    }
    is a basis for the standard topology on $\reals$.
    It then follows that
    \gath{
      \col{D} = \braces{B \times C \where B,C \in \cB}
    }
    is a basis for the standard topology on $\reals^2$ by Theorem~15.1.
    Clearly we have
    \gath{
      \col{D} = \braces{(a,b) \times (c,d) \where \text{$a < b$ and $c < d$ and $a,b,c,d$ are rational}} \,,
    }
    which shows the desired result.
  }
}
