\setcounter{subsection}{18-1}
\subsection{Continuous Functions} 

\exercise{1}{
  Prove that for functions $f: \reals \to \reals$, the $\e$-$\d$ definition of continuity implies the open set definition.
}
\sol{
  \dwhitman

  Recall that that $f: \reals \to \reals$ is continuous at a point $x \in \reals$ if, for every real $\e > 0$, there is a real $\d > 0$ such that $\abs{f(y) - f(x)} < \e$ for every real $y$ where $\abs{y - x} < \d$.
  We say that $f$ itself is continuous if it is continuous at every $p \in \reals$.
  \qproof{
    Suppose that $f: \reals \to \reals$ is continuous by the $\e$-$\d$ definition above.
    We show that this implies the open set definition by showing that $f$ satisfies (4) in Theorem~18.1.
    So consider any $x \in \reals$ and any neighborhood $V$ of $f(x)$.
    Then of course there is a basis element $(c,d)$ containing $f(x)$ such that $(c,d) \ss V$.
    Let $\e = \min(f(x)-c, d-f(x))$, noting that $\e > 0$ since $c < f(x) < d$.
    It is then trivial to show that $(f(x)-\e, f(x)+\e) \ss (c,d) \ss V$ and contains $x$.

    Then, since $f$ is continuous at $x$, there is $\d > 0$ such that $\abs{y - x} < \d$ implies that $\abs{f(y) - f(x)} < \e$ for any real $y$.
    Let $U = (x-\d, x+\d)$, which is clearly a neighborhood of $x$.
    Now consider any $z \in f(U)$ so that $z = f(y)$ for some $y \in U$.
    Then we have that $x - \d < y < x+ \d$ so that clearly $-\d < y - x < \d$, from which it follows that $\abs{y-x} < \d$.
    We then know that $\abs{z - f(x)} = \abs{f(y) - f(x)} < \e$ since $f$ is continuous.
    Hence $-\e < z - f(x) < \e$ so that $f(x) - \e < z < f(x) + \e$, and thus $z \in V$ since $(f(x)-\e, f(x)+\e) \ss V$.
    Since $z \in f(U)$ was arbitrary, this shows that $f(U) \ss V$, which shows that (4) holds for $f$ since $x$ was also arbitrary.
  }
}

\exercise{2}{
  Suppose that $f: X \to Y$ is continuous.
  If $x$ is a limit point of the subset $A$ of $X$, is it necessarily true the $f(x)$ is a limit point of $f(A)$?
}
\sol{
  \dwhitman

  This is not necessarily true.
  \qproof{
    As a counterexample consider a constant function $f: X \to Y$ defined by $f(x) = y_0$ for any $x \in X$ and some $y_0 \in Y$.
    It was shown in Theorem~18.2a that this is continuous.
    However, clearly $f(A) = \braces{y_0}$ for any subset $A$ of $X$.
    So even if $x$ is a limit point of $A$, no neighborhood of $f(x)$ can intersect $f(A)$ in a point other than $f(x) = y_0$ since $y_0$ is the only point in $f(A)$!
    Therefore $f(x)$ is not a limit point of $f(A)$.
  }
}

\exercise{3}{
  Let $X$ and $X'$ denote a single set in two topologies $\cT$ and $\cT'$, respectively.
  Let $i : X' \to X$ be the identity function.
  \eparts{
  \item Show that $i$ is continuous $\bic \cT' \text{ is finer than } \cT$.
  \item Show that $i$ is a homeomorphism $\bic \cT' = \cT$.
  }
}
\sol{
  \dwhitman

  (a)
  \qproof{
    First note that clearly the inverse of the identity function is itself with the domain and range reversed, and that for any subset $A \ss X = X'$ we have $i(A) = \inv{i}(A) = A$.
    
    $(\imp)$ Suppose that $i$ is continuous and consider any open set $U \in \cT$.
    Then we have that $\inv{i}(U) = U$ is open in $\cT'$ since $i$ is continuous.
    Since $U$ was arbitrary, this shows that $\cT \ss \cT'$ so that $\cT'$ is finer.

    $(\pmi)$ Now suppose that $\cT'$ is finer so that $\cT \ss \cT'$.
    Consider any open set $U \in \cT$ so that also clearly $U \in \cT'$, i.e. $U$ is also open in $\cT'$.
    Since $\inv{i}(U) = U$, this shows that $i$ is continuous by the definition of continuity.
  }

  (b)
  \qproof{
    Clearly $i$ is a bijection since its domain and range are the same set, and $\inv{i} = i$.
    We then have that
    \ali{
      \text{$i$ is a homeomorphism } &\bic \text{$i$ and $\inv{i}$ are both continuous} \\
      &\bic \text{$\cT'$ is finer than $\cT$ and $\cT$ is finer than $\cT'$} & \text{(by part (a) applied twice)} \\
      &\bic \cT \ss \cT' \text{ and } \cT' \ss \cT \\
      &\bic \cT' = \cT
    }
    as desired.
  }
}

\exercise{4}{
  Given $x_0 \in X$ and $y_0 \in Y$, show that the maps $f: X \to X \times Y$ and $g: Y \to X \times Y$ defined by
  \gath{
    f(x) = x \times y_0 \condgap \text{and} \condgap g(y) = x_0 \times y
  }
  are imbeddings.
}
\sol{
  \dwhitman

  We only show that $f$ is an imbedding of $X$ in $X \times Y$ as the argument for $g$ is entirely analogous.
  \qproof{
    First, it is easy to see and trivial to formally show that $f$ is injective.
    The function $f$ can be of course be defined as $f(x) = f_1(x) \times f_2(x)$ where $f_1 : X \to X$ is the identity function and $f_2 : X \to Y$ is the constant function that maps every element of $X$ to $y_0$.
    Since these have both been proven to be continuous in the text, it follows that $f$ is continuous by Theorem~18.4.

    Now let $f'$ be the function obtained by restricting the range of $f$ to $f(X) = \braces{x \times y_0 \where x \in X}$.
    Since $f$ is injective, it follows that $f'$ is a bijection.
    It follows from Theorem~18.2e that $f'$ is continuous.
    Clearly the inverse function $\inv{f'}$ is equal to the projection function $\pi_1$ so that $\inv{f'}(x,y) = x$.
    This was shown to be continuous in the proof of Theorem~18.4.
    This suffices to show that $f'$ is a homeomorphism, which shows the $f$ is an imbedding of $X$ in $X \times Y$.
  }
}

\exercise{5}{
  Show that the subspace $(a,b)$ of $\reals$ is homeomorphic with $(0,1)$ and the subspace $[a,b]$ of $\reals$ is homeomorphic with $[0,1]$.
}
\sol{
  \dwhitman

  First we show that $(a,b)$ is homeomorphic to $(0,1)$.
  \qproof{
    First let $X = (a,b)$ and $Y = (0,1)$, and define the map $f: X \to Y$ by
    \gath{
      f(x) = \frac{x - a}{b - a}
    }
    for any $x \in X$, noting that this is defined since $a < b$ so that $b - a > 0$.
    It is trivial to show that $f$ is a bijection.

    Now, $f$ is a linear function that could just as well be defined as a map from $\reals$ to $\reals$, and clearly this would be continuous by basic calculus.
    It then follows from Theorem~18.2d that restricting its domain to $X$ means that it is still continuous.
    We also clearly have from basic algebra that its inverse is the function $\inv{f}: Y \to X$ defined by
    \gath{
      \inv{f}(y) = a + y(b-a)
    }
    for $y \in Y$.
    As this is also linear, it too is continuous by the same argument.
    This suffices to show that $f$ is a homeomorphism.
  }

  The exact same argument shows that $[a,b]$ is homeomorphic to $[0,1]$ by simply setting $X = [a,b]$ and $Y = [0,1]$ in the above proof.
  It is assumed that here again $a < b$ even though the interval $[a,b]$ is valid if $a = b$ and simply becomes $[a,b] = [a,a] = \braces{a}$.
  However, clearly this set cannot be homeomorphic to $[0,1]$ since it is finite whereas $[0,1]$ is uncountable.
}

\exercise{6}{
  Find a function $f: \reals \to \reals$ that is continuous at precisely one point.
}
\sol{
  \dwhitman

  For any real $x$ define
  \gath{
    f(x) = \begin{cases}
      0 & x \in \rats \\
      x & x \notin \rats \,.
    \end{cases}
  }
  We claim that this is continuous only at $x = 0$.
  \qproof{
    As it is easier to do so, we show this using the $\e$-$\d$ definition of continuity, which we know implies the open set definition by Exercise~18.1.
    First we note that $f(0) = 0$ since $0$ is rational.
    Now consider any $\e > 0$ and let $\d = \e$.
    Suppose real $y$ where $\abs{y - 0} = \abs{y} < \d$.
    If $y$ is rational then $y = 0$ so that $\abs{y - f(0)} = \abs{0 - 0} = \abs{0} = 0 < \e$.
    If $y$ is irrational then $\abs{y - f(0)} = \abs{y - 0} = \abs{y} < \d = \e$ again.
    Since $\e$ was arbitrary this shows that $f$ is continuous at $x = 0$.

    Now consider any $x \neq 0$.
    Let $\e = \abs{x} / 2$, noting that $\e > 0$ since $x \neq 0$.
    Now consider any $\d > 0$.

    Case: $x \in \rats$.
    Then $f(x) = 0$ but there is clearly an irrational $y$ close enough to $x$ so that $\abs{y - x} < \min(\e, \d)$, and hence both $\abs{y-x} < \e$ and $\abs{y-x} < \d$.
    We also have that $f(y) = y$. We then have that
    \gath{
      2\e = \abs{x} \leq \abs{y-x} + \abs{y} < \e + \abs{y}
    }
    so that
    \gath{
      \e < \abs{y} = \abs{f(y)} = \abs{f(y) - 0} = \abs{f(y) - f(x)} \,.
    }

    Case: $x \notin \rats$.
    Then $f(x) = x$, and there is clearly a rational $y$ close enough to $x$ that $\abs{y-x} < \d$.
    We then also have $f(y) = 0$ so that
    \gath{
      \abs{f(y) - f(x)} = \abs{0 - x} = \abs{x} = 2\e > \e
    }
    since $\e > 0$.

    Hence in either case there is a $y$ such that $\abs{y-x} < \d$ but $\abs{f(y) - f(x)} \geq \e$.
    This suffices to show that $f$ is not continuous at $x$.
  }
}

\exercise{7}{
  \eparts{
  \item Suppose that $f: \reals \to \reals$ is ``continuous from the right,'' that is
    \gath{
      \lim_{x \to a^+} f(x) = f(a) \,,
    }
    for each $a \in \reals$.
    Show that $f$ is continuous when considered as a function from $\reals_l$ to $\reals$.
  \item Can you conjecture what functions $f: \reals \to \reals$ are continuous when considered as maps from $\reals$ to $\reals_l$?
    We shall return to this question in Chapter~3.
  }
}
\sol{
  \dwhitman

  \begin{lem}\label{lem:cont:Rlopcl}
    In the topology $\reals_l$, every basis element is both open and closed.
  \end{lem}
  \qproof{
    Consider any basis element $B = \clop{a,b}$, which is clearly open since basis elements are always open.
    We then have that the complement of this set is $C = \reals - B = (-\infty, a) \cup \clop{b, \infty}$.
    We claim that this complement is also open so that $B$ is closed by definition.
    To see this, define the sets $C_n = \clop{a-n-1, a-n+1} \cup \clop{b+n-1, b+n+1}$ for $n \in \pints$.
    Clearly each $C_n$ is open since it is the union of two basis elements.
    It is also trivial to show that $C = \bigcup_{n \in \pints} C_n$, which is then also open since it is a union of open sets.
  }

  \begin{lem}\label{lem:cont:Ropcl}
    The only open sets in the standard topology on $\reals$ that are both open and closed are $\es$ and $\reals$ itself.
  \end{lem}
  \qproof{
    First, clearly both $\es$ and $\reals$ are both open and closed since they are compliments of each other and are both open by the definition of a topology.
    Now suppose that $U$ is a nonempty subset of $\reals$ that is both open and closed.
    Suppose also that $U \neq \reals$ so that $U \pss \reals$ and hence there is a $y \in \reals$ where $y \notin U$.
    We show that the existence of such a $U$ results in a contradiction, which of course shows the desired result since it implies that $U = \reals$ if $U \neq \es$.
    Since $U$ is nonempty we have that there is an $x \in U$ and it must be that $x \neq y$ since $x \in U$ but $y \notin U$.

    If $x < y$ then define the set $A = \braces{z > x \where z \notin U}$.
    Clearly we have that $A$ is nonempty since $y \in A$, and that $x$ is a lower bound of $A$.
    It then follows that $A$ has a largest lower bound $a$ since this is a fundamental property of $\reals$.
    It could be that $a \in U$ or $a \notin U$.
    In the former case we have that any basis element $(c,d)$ containing $a$ is not a subset of $U$.
    To see this, we have that $c < a < d$, which means that $d$ is not a lower bound of $A$ since $a$ is the largest lower bound.
    Hence there is a $z \in A$ where $d > z$.
    We then have $c < a \leq z < d$ (noting that $a \leq z$ since $a$ is a lower bound of $A$) so $z \in (c,d)$ and $z \in A$ so that $z \notin U$.
    Hence $(c,d)$ is not a subset of $U$, which contradicts the fact that $U$ is open since the basis element $(c,d)$ was arbitrary.

    In the latter case where $a \notin U$ then it has to be that $x < a$ since $x$ is a lower bound of $A$ and $a$ is the largest lower bound (and it cannot be that $a = x$ since $x \in U$ but $a \notin U$).
    We clearly have that $a \in \reals - U$, which is open since $U$ is closed.
    Now consider any basis element $(c,d)$ containing $a$ so that $c < a < d$.
    Let $b = \max(x,c)$ so that $b < a$ and hence there is a real $z$ where $c \leq b < z < a < d$ and hence $z \in (c,d)$.
    Now, since $z < a$ it has to be that $z \notin A$ since otherwise $a$ would not be a lower bound of $A$.
    We also have that $x \leq b < z$ so that it has to be that $z \in U$ since otherwise it would be that $z \in A$.
    Thus $z \notin \reals - U$, which shows that $(c,d)$ is not a subset of $\reals - U$ since $z \in (c,d)$.
    Since $(c,d)$ was an arbitrary basis element, this contradicts the fact that $\reals - U$ is open.

    It was thus shown that in either case a contradiction arises.
    Analogous arguments also show contradictions when $x > y$, this time using the set $A = \braces{z < x \where z \notin U}$ and its least upper bound.
    Hence it has to be that $U = \reals$, which shows the desired result.
  }

  \mainprob
  
  (a) Recall that by the definition of the one-sided limit, $f: \reals \to \reals$ is continuous from the right if, for every $a \in \reals$ and every $\e > 0$, there is a $\d > 0$ such that $\abs{f(x) - f(a)} < \e$ for every $x > a$ where $\abs{x-a} < \d$.
  \qproof{
    So suppose that $f$ is continuous from the right and consider any $a \in \reals$.
    Let $V$ be neighborhood of $f(a)$ in $\reals$.
    Then there is a basis element $(c,d)$ of $\reals$ that contains $f(a)$ and is a subset of $V$.
    Hence $c < f(a) < d$, so let $\e = \min[f(a)-c, d - f(a)]$ so that clearly $\e > 0$ and if $\abs{y - f(a)} < \e$, then $y \in (c,d)$ so that also $y \in V$.
    Now, since $f$ is continuous from the right, there is a $\d > 0$ such that $\abs{f(x) - f(a)} < \e$ for every $x > a$ where $\abs{x-a} < \d$.
    So let $U = \clop{a, a+\d}$ which is clearly a basis element of $\reals_l$ and contains $a$ so that it is a neighborhood of $a$.

    Now consider any $y \in f(U)$ so that there is an $x \in U$ where $y = f(x)$.
    If $x = a$ then clearly $\abs{f(x) - f(a)} = \abs{f(a) - f(a)} = \abs{0} = 0 < \e$ so that $f(x) \in V$.
    If $x \neq a$ then it has to be that $x > a$ and also that $\abs{x-a} = x - a < \d$ since $U = \clop{a,a+\d}$.
    It then follows that $\abs{f(x) - f(a)} < \e$ so that again $f(x) \in V$.
    Hence in both cases $y = f(x) \in V$, which shows that $f(U) \ss V$ since $y$ was arbitrary.
    We have thus shown part (4) of Theorem~18.1, from which the topological continuity of $f$ follows.
  }

  (b) We claim that only constant functions are continuous from $\reals$ to $\reals_l$.
  \qproof{
    First, it was shown in Theorem~18.2a that constant functions are always continuous regardless of the topologies.
    Hence we must show that any continuous function from $\reals$ to $\reals_l$ is constant.
    So suppose that $f$ is such a function.
    Now consider any real $x$ where $x \neq 0$.
    Clearly if $f(x) = f(0)$ then $f$ is a constant function since $x$ was arbitrary.
    So suppose that this is not the case so that $f(x) \neq f(0)$.
    Without loss of generality we can assume that $f(0) < f(x)$.
    So consider the basis element $B = \clop{f(0), f(x)}$ of $\reals_l$, which clearly contains $f(0)$ but not $f(x)$.

    Since $f$ is continuous and $B$ is both open and closed by Lemma~\ref{lem:cont:Rlopcl}, it follows from the definition of continuity and from Theorem~18.1 part (3) that $\inv{f}(B)$ must be both open and closed in $\reals$.
    However the only sets that are both open and closed in $\reals$ are $\es$ and $\reals$ itself by Lemma~\ref{lem:cont:Ropcl}.
    Thus either $\inv{f}(B) = \es$ or $\inv{f}(B) = \reals$.
    It cannot be that $\inv{f}(B) = \es$ since we have that $f(0) \in B$ so that $0 \in \inv{f}(B)$.
    Hence it must be that $\inv{f}(B) = \reals$, but then we would have $x \in \inv{f}(B)$ so that $f(x) \in B$, which we know it not the case.
    We therefore have a contradiction so that it must be that $f(x) = f(0)$ so that $f$ is constant.
  }

  Lastly, we claim that  the only functions that are continuous from $\reals_l$ to $\reals_l$ are those that are \emph{continuous and non-decreasing from the right}.
  For a function $f:\reals \to \reals$ this means that for every $x \in \reals$ and every $\e > 0$ there is a $\d > 0$ such that $\abs{f(y) - f(x)} < \e$ and $f(y) \geq f(x)$ for every $x \leq y < x + \d$.
  \qproof{
    First we show that such functions are in fact continuous.
    So suppose that $f$ is continuous and non-decreasing from the right and consider any real $x$.
    Let $V$ be any neighborhood of $f(x)$ so that there is a basis element $B = \clop{c,d}$ containing $f(x)$ such that $B \ss V$.
    Let $\e = d - f(x)$ so that $\e > 0$ since $f(x) < d$.
    Hence there is a $\d > 0$ such that $x < y < x+\d$ implies that $\abs{f(y)-f(x)} < \e$ and $f(y) \geq f(x)$.
    We then have that $U = \clop{x, x+\d}$ is a basis element and therefore an open set of $\reals_l$ that contains $x$.
    Consider any $z \in f(U)$ so that $z = f(y)$ for some $y \in U$.
    Then $x \leq y < x+\d$ so that $z = f(y) \geq f(x)$ and $\abs{z - f(x)} = \abs{f(y)-f(x)} < \e$.
    It then follows that $0 \leq z - f(x) < \e$ so that $c \leq f(x) \leq z < f(x) + \e = d$, and hence $z \in \clop{c,d} = B$.
    Thus also $z \in V$ since $B \ss V$.
    This shows that $f(U) \ss V$ since $z$ was arbitrary, and hence that $f$ is continuous by Theorem~18.1.

    Now we show that a continuous function \emph{must} be continuous and non-decreasing from the right by showing the contrapositive.
    So suppose that $f$ is not continuous and non-decreasing from the right.
    Then there exists a real $x$ and an $\e > 0$ such that, for any $\d > 0$, there is a $x \leq y <x+\d$ where $f(y) < f(x)$ or $\abs{f(y)-f(x)} \geq \e$.
    Clearly we have that $V = \clop{f(x), f(x)+\e}$ is basis element and therefore open set of $\reals_l$ that contains $f(x)$.
    Consider any neighborhood $U$ of $x$ so that there is a basis element $B = \clop{a,b}$ containing $x$ where $B \ss U$.
    Then $x < b$ so that $\d = b - x > 0$.
    It then follows that there is a $x \leq y < x+\d = b$ such that $f(y) < f(x)$ or $\abs{f(y) - f(x)} \geq \e$.
    Clearly we have that $y \in B$ so that also $y \in U$ and $f(y) \in f(U)$.
    However, if $f(y) < f(x)$ then clearly $f(y) \notin V$.
    On the other hand if $f(y) \geq f(x)$ then it has to be that $\abs{f(y) - f(x)} \geq \e$.
    Then we have that $f(y) - f(x) \geq 0$ so that $f(y) - f(x) = \abs{f(y) - f(x)} \geq \e$, and hence $f(y) \geq f(x) + \e$ so that again $f(y) \notin V$.
    This suffices to show that $f(U)$ is not a subset of $V$, which shows that $f$ is not continuous by Theorem~18.1 since $U$ was an arbitrary neighborhood of $x$.
  }
}

\exercise{8}{
  Let $Y$ be an ordered set in the order topology.
  Let $f,g : X \to Y$ be continuous.
  \eparts{
  \item Show that the set $\braces{x \where f(x) \leq g(x)}$ is closed in $X$.
  \item Let $h: X \to Y$ be the function
    \gath{
      h(x) = \min\braces{f(x), g(x)} \,.
    }
    Show that $h$ is continuous.
    [Hint: Use the pasting lemma.]
  }
}
\sol{
  \dwhitman

  (a) First let $C = \braces{x \in X \where f(x) \leq g(x)}$ so that we must show that $C$ is closed in $X$.
  \qproof{
    We prove this by showing that the complement $X - C$ is open in $X$.
    So first let $S$ be the set of all $y \in Y$ where $y$ has an immediate successor, and denote that successor by $y+1$.
    Then clearly $y+1$ is well defined for all $y \in S$.
    Now define
    \ali{
      A_{>y} &= \braces{z \in Y \where z > y} &
      A_{<y} &= \braces{z \in Y \where z < y+1}
    }
    for $y \in S$.
    As these are both rays in the order topology $Y$, they are both basis elements and therefore open.
    It then follows that $\inv{f}(A_{>y})$ and $\inv{g}(A_{<y})$ are both open in $X$ since $f$ and $g$ are continuous.
    Hence their intersection $U_y = \inv{f}(A_{>y}) \cap \inv{g}(A_{<y})$ is also open in $X$.

    Similarly the rays
    \ali{
      B_{>y} &= \braces{z \in Y \where z > y} &
      B_{<y} &= \braces{z \in Y \where z < y}
    }
    for $y \in Y$ are also open so that the intersection $V_y = \inv{f}(B_{>y}) \cap \inv{g}(B_{<y})$ is open in $X$.
    Then clearly the union of unions
    \gath{
      D = \bigcup_{y \in S} U_y \cup \bigcup_{y \in Y} V_y
    }
    is also open in $X$.
    We claim that $X -C = D$ so that the complement is open in $X$ and hence $C$ is closed as desired.

    $(\ss)$ First consider any $x \in X - C$ so that clearly $f(x) > g(x)$.
    If $g(x)$ has an immediate successor $g(x)+1$ then $g(x) \in S$ and we have $f(x) \in A_{>g(x)}$ so that $x \in \inv{f}(A_{>g(x)})$.
    We also have that $g(x) \in A_{<g(x)}$ since $g(x) < g(x)+1$, and hence $x \in \inv{g}(A_{<g(x)})$.
    It then follows that $x \in U_{g(x)}$ and hence $\bigcup_{y \in S} U_y$ and $x \in D$ since $g(x) \in S$.
    If $g(x)$ does not have an immediate successor then there must be a $y \in Y$ where $g(x) < y < f(x)$.
    We then have that clearly $f(x) \in B_{>y}$ and $g(x) \in B_{<y}$ so that $x \in \inv{f}(B_{>y})$ and $x \in \inv{g}(B_{<y})$.
    Thus $x \in V_y$ so that $x \in \bigcup_{y \in Y} V_y$ and $x \in D$.
    This shows that $X - C \ss D$ since either way $x \in D$ and $x$ was arbitrary.

    $(\sps)$ Now suppose that $x \in D$.
    If $x \in \bigcup_{y \in S} U_y$ when there is a $y \in S$ where $x \in U_y$.
    Hence $x \in \inv{f}(A_{>y})$ and $x \in \inv{g}(A_{<y})$ so that $f(x) \in A_{>y}$ and $g(x) \in A_{<y}$.
    From this it follows that $f(x) > y$ and $g(x) < y+1$.
    Then it has to be that $g(x) \leq y$ so that $f(x) > y \geq g(x)$.
    If $x \in \bigcup_{y \in Y} V_y$ then there is a $y \in Y$ where $x \in V_y$.
    Hence $x \in \inv{f}(B_{>y})$ and $x \in \inv{g}(B_{<y})$ so that $f(x) \in B_{>y}$ and $g(x) \in B_{<y}$.
    It then clearly follows that $f(x) > y$  and $g(x) < y$ so that $f(x) > y > g(x)$.
    Therefore in either case we have $f(x) > g(x)$ so that $x \in X - C$.
    This of course shows that $X - C \sps D$ since $x$ was arbitrary.
  }

  (b)
  \qproof{
    Let $A = \braces{x \in X \where f(x) \leq g(x)}$ and $B = \braces{x \in X \where g(x) \leq f(x)}$, which are clearly both closed by part (a).
    It is easy to see that $X = A \cup B$.
    First, clearly $X \sps A \cup B$ since both $A \ss X$ and $B \ss X$.
    Then, for any $x \in X$, it has to be that either $f(x) \leq g(x)$ or $f(x) > g(x)$ since $<$ is a total order on $Y$.
    In the former case of course $x \in A$, and in the latter $x \in B$ so that either way $x \in A \cup B$.
    Hence $X \ss A \cup B$.
    It is also easy to see that $f(x) = g(x)$ for every $x \in A \cap B$.
    For any such $x$, we have that $x \in A$ so that $f(x) \leq g(x)$, and $x \in B$ so that $g(x) \leq f(x)$.
    From this it clearly must be that $f(x) = g(x)$.

    Since $f$ and $g$ are continuous, it then follows from the pasting lemma that the function
    \gath{
      h(x) = \begin{cases}
        f(x) & x \in A \\
        g(x) & x \in B
      \end{cases}
    }
    for $x \in X$ is continuous as well.
    Based on the definitions of $A$ and $B$ it is then easy to see and trivial to show that $h(x) = \min\braces{f(x), g(x)}$ for all $x \in X$, which of course shows the desired result.
  }
}

\exercise{9}{
  Let $\braces{A_\a}$ be a collection of subsets of $X$; let $X = \bigcup_\a A_\a$.
  Let $f: X \to Y$; suppose that $f \rest A_\a$, is continuous for each $\a$.
  \eparts{
  \item Show that if the collection $\braces{A_\a}$ is finite and each set $A_\a$ is closed, then $f$ is continuous.
  \item Find an example where the collection $\braces{A_\a}$ countable and each $A_\a$ is closed, but $f$ is not continuous.
  \item An indexed family of sets $\braces{A_\a}$ is said to be \boldit{locally finite} if each point $x$ of $X$ has a neighborhood that intersects $A_\a$ for only finitely many values of $\a$.
    Show that if the family $\braces{A_\a}$ is locally finite and each $A_\a$ is closed, then $f$ is continuous.
  }
}
\sol{
  \dwhitman

  (a)
  \qproof{
    We show using induction that $f$ is continuous for any collection $\braces{A_\a}_{\a=1}^n$, for any $n \in \pints$,  where each $A_\a$ is closed.
    This of course shows the desired result since the collection is $\braces{A_\a}_{\a=1}^n$ for some $n \in \pints$ if it is finite.
    So first, for $n=1$, we have that $A_1 = \bigcup_{\a=1}^n A_\a = X$ so that of course $f = f \rest X = f \rest A_1$ is continuous.

    Now suppose that $f$ is continuous for any collection of size $n$ and suppose we have the collection $\braces{A_\a}_{\a=1}^{n+1}$ of size $n+1$.
    Let $A = \bigcup_{\a=1}^n A_\a$, which is closed by Theorem~17.1 since each $A_\a$ is closed and it is a finite union, and let $B = A_{n+1}$ so that $B$ is also closed.
    We then have that $A \cup B = \bigcup_{\a=1}^n A_\a \cup A_{n+1} = \bigcup_{\a=1}^{n+1} A_\a = X$.
    We know that $g = f \rest B = f \rest A_{n+1}$ is continuous.
    Considering the set $A$ as a subspace of $X$, then each $A_\a$ for $\a \in \intsfin{n}$ is closed in $A$ by Theorem~17.2 since they are subsets of $A$ and closed in $X$.
    Since by definition $\bigcup_{\a=1}^n A_\a = A$, it follows from the induction hypothesis that $f' = f \rest A$ is continuous.
    Clearly also for any $x \in A \cap B$ we have that $x \in A$ and $x \in B = A_{n+1}$ so that $f'(x) = (f \rest A)(x) = f(x) = (f \rest A_{n+1})(x) = g(x)$.

    Then by the pasting lemma the function $h : X \to Y$ defined by
    \gath{
      h(x) = \begin{cases}
        f'(x) & x \in A \\
        g(x) & x \in B
      \end{cases}
    }
    is continuous.
    However, consider any $x \in X$.
    If $x \in A$ then $h(x) = f'(x) = (f \rest A)(x) = f(x)$.
    Similarly if $x \in B$ then $h(x) = g(x) = (f \rest B)(x) = f(x)$ as well.
    This suffices to show that $h = f$ since $x$ was arbitrary.
    Thus $f$ is continuous, which completes the induction.
  }

  (b) Consider the standard topology on $\reals$ and define the countable collection of set $\braces{A_n}$ by
  \gath{
    A_n = \begin{cases}
      \opcl{-\infty, 0} & n = 1 \\
      \clop{1, \infty} & n = 2 \\
      \squares{\frac{1}{n-1}, \frac{1}{n-2}} & n > 2
    \end{cases}
  }
  for $n \in \pints$.
  Also define $f: \reals \to \reals$ by
  \gath{
    f(x) = \begin{cases}
      1 & x \leq 0 \\
      0 & x > 0
    \end{cases}
  }
  for real $x$.
  We claim that this collection and function have the desired properties.
  \qproof{
    First, it is trivial to show that the collection covers all of $\reals$, i.e. that $\bigcup_{n=1}^\infty A_n = \reals$.
    It is also obvious by this point that each $A_n$ is closed in the standard topology.
    Clearly $f$ is not a continuous function since there is a discontinuity at $x = 0$, which is trivial to prove.
    Lastly, consider any $n \in \pints$.
    If $n = 1$ then for any $x \in A_n = A_1 = \opcl{-\infty, 0}$ we have that $x \leq 0$ and hence $f(x) = 1$.
    Likewise if $n = 2$ then for any $x \in A_n = A_2 = \clop{1, \infty}$ it follows that $x \geq 1 > 0$ , and hence $f(x) = 0$.
    Lastly, if $n > 2$ then for any $x \in A_n = [1/(n-1), 1/(n-2)]$ we have that $0 < 1/(n-1) \leq x$ so that $f(x) = 0$ again.
    Thus in all cases $f \rest A_n$ is constant and therefore continuous.
    This shows the desired properties.
  }

  (c)
  \qproof{
    Consider any $x \in X$ so that there is a neighborhood $U'$ of $x$ that intersects a finite subcollection $\braces{A_k}_{k=1}^n$ of the full collection $\braces{A_\a}$.
    Consider $A = \bigcup_{k=1}^n A_k$ as a subspace of $X$, from which it follows from Theorem~17.2 that each $A_k$ is closed in $A$ since it is a subset of $A$ and closed in $X$.
    It is then easy to show that $U' \ss A$.
    It also follows from part (a) that $f \rest A$ is continuous with the domain being the subspace topology on $A$.

    Now consider any neighborhood $V$ of $f(x)$, noting that of course $x \in A$ since $x \in U'$ and $U' \ss A$.
    Thus $f(x)$ is in the range of $f \rest A$ so that there is a neighborhood $U_A$ of $x$ in the subspace topology such that $(f \rest A)(U_A) \ss V$ by Theorem~18.1 since $f \rest A$ is continuous.
    Since $U_A$ is open in the subspace topology, there is an open set $U_X$ in $X$ where $U_A = A \cap U_X$.
    Now let $U = U' \cap U_X$, which is open in $X$ since $U'$ and $U_X$ are both open in $X$.
    Then also $x \in U_X$ since $x \in U_A$ and $U_A = A \cap U_X$, and hence $x \in U$ since also $x \in U'$ and $U = U' \cap U_X$.
    Thus $U$ is a neighborhood of $x$ in $X$.

    Let $z$ be any element of $f(U)$ so that $z = f(y)$ for some $y \in U$.
    Then $y \in U'$ and $y \in U_X$ since $U = U' \cap U_X$.
    Then also $y \in A$ since $U' \ss A$, and hence $y \in A \cap U_X = U_A$.
    From this it follows that $z = f(y) = (f \rest A)(y) \in (f \rest A)(U_A)$ so that $z \in V$ since $(f \rest A)(U_A) \ss V$.
    Since $z$ was arbitrary, this shows that $f(U) \ss V$, which in turn shows that $f$ is continuous by Theorem~18.1 since $V$ was an arbitrary neighborhood of $f(x)$ and $x$ was an arbitrary element of $X$.
  }

  We note that the example in part (b) is not locally finite since any neighborhood of $x = 0$ intersects infinitely many $A_n$ in the collection.
  This fact is easy to see and would be easy to prove formally, though a bit tedious.
}

\def\fg{f \times g}
\def\xy{x \times y}
\def\xyp{x' \times y'}
\def\wz{w \times z}
\exercise{10}{
  Let $f: A \to B$ and $g: C \to D$ be continuous functions.
  Let us define a map $f \times g: A \times C \to B \times D$ by the equation
  \gath{
    (f \times g)(a \times c) = f(a) \times g(c) \,.
  }
  Show that $f \times g$ is continuous.
}
\sol{
  \dwhitman

  \qproof{
    Consider any $\xy \in A \times C$ and any neighborhood $V$ of $(\fg)(\xy)$ in $B \times D$.
    Since $V$ is open in $B \times D$, there is a basis element $U_B \times U_D$ of the product topology that contains $(\fg)(\xy)$ where $U_B \times U_D \ss V$.
    Then $U_B$ and $U_D$ are open in $B$ and $D$, respectively.
    Since $f$ is continuous, we then have that $U_A = \inv{f}(U_B)$ is open in $A$.
    Likewise $U_C = \inv{g}(U_D)$ is open in $C$ since $g$ is continuous.
    Then the set $U = U_A \times U_C$ is a basis element of the product topology $A \times C$ and therefore open.

    Since $U_B \times U_D$ contains $(\fg)(\xy) = f(x) \times g(y)$ we have that $f(x) \in U_B$ and $g(y) \in U_D$.
    From this it follows that $x \in \inv{f}(U_B) = U_A$ and $y \in \inv{g}(U_D) = U_C$.
    Therefore $\xy \in U_A \times U_C = U$ so that $U$ is a neighborhood of $\xy$ in $A \times C$.
    Now consider any $\wz \in (\fg)(U)$ so that there is an $\xyp \in U = U_A \times U_C$ where $\wz = (\fg)(\xyp) = f(x') \times g(y')$.
    Hence $w = f(x')$ and $x' \in U_A = \inv{f}(U_B)$ so that $w = f(x') \in U_B$.
    Similarly $z = g(y')$ and $y' \in U_C = \inv{g}(U_D)$ so that $z = g(y') \in U_D$.
    Thus $\wz \in U_B \times U_D$ so that also $\wz \in V$ since $U_B \times U_D \ss V$.
    This shows that $(\fg)(U) \ss V$ since $\wz$ was arbitrary.

    This suffices to show that $\fg$ is continuous by Theorem~18.1 as desired.
  }
}

\exercise{11}{
  Let $F: X \times Y \to Z$.
  We say that $F$ is \boldit{continuous in each variable separately} if for each $y_0$ in $Y$, the map $h: X \to Z$ defined by $h(x) = F(x \times y_0)$ is continuous, and for each $x_0 \in X$, the map $k: Y \to Z$ defined by $k(y) = F(x_0 \times y)$ is continuous.
  Show that if $F$ is continuous, then $F$ is continuous in each variable separately.
}
\sol{
  \dwhitman

  \qproof{
    To show that $F$ is continuous in $x$, consider any $y_0 \in Y$ and define $h: X \to Z$ by $h(x) = F(x \times y_0)$.
    Now consider any $x \in X$ and any neighborhood $V$ of $h(x) = F(x \times y_0)$.
    Then $V$ is an open set containing $h(x) = F(x \times y_0)$ so that it is a neighborhood of $F(x \times y_0)$.
    Since $F$ is continuous, this means that there is neighborhood $U'$ of $x \times y_0$ in $X \times Y$ such that $F(U') \ss V$ by Theorem~18.1.
    It then follows that there is a basis element $U_X \times U_Y$ of $X \times Y$ containing $x \times y_0$ where $U_X \times U_Y \ss U'$.
    Since $X \times Y$ is a product topology, we have that $U_X$ is open in $X$ and $U_Y$ is open in $Y$.
    Then, since $x \times y_0 \in U_X \times U_Y$ we have that $x \in U_X$ and $y_0 \in U_Y$ so that $U_X$ is a neighborhood of $x$ in $X$.

    So consider any $z \in h(U_X)$ so that $z = h(x')$ for some $x' \in U_X$.
    Then $x' \times y_0 \in U_X \times U_Y$ so that also $x' \times y_0$ in $U'$ since $U_X \times U_Y \ss U'$.
    It then also follows that $z = h(x') = F(x' \times y_0) \in F(U')$ so that $z \in V$ since $F(U') \ss V$.
    This shows that $h(U_X) \ss V$ since $z$ was arbitrary.
    It then follows that $h$ is continuous by Theorem~18.1.
    
    The proof that $F$ is continuous in $y$ is directly analogous.
  }
}

\exercise{12}{
  Let $F: \reals \times \reals \to \reals$ by defined by the equation
  \gath{
    F(x \times y) = \begin{cases}
      xy / (x^2 + y^2) & \text{if $x \times y \neq 0 \times 0$.} \\
      0 & \text{if $x \times y = 0 \times 0$.}
    \end{cases}
  }
  \eparts{
  \item Show that $F$ is continuous in each variable separately.
  \item Compute the function $g: \reals \to \reals$ defined by $g(x) = F(x \times x)$.
  \item Show that $F$ is not continuous.
  }
}
\sol{
  \dwhitman

  (a)
  \qproof{
    It is easy to see that $F$ is continuous in $x$.
    For any real $y_0$ we generally have that
    \gath{
      h(x) = F(x \times y_0) = \frac{x y_0}{x^2 + y_0^2}
    }
    so long as one of $x$ and $y_0$ are nonzero.
    If $y_0 = 0$ then $x = 0$ implies that $x \times y_0 = 0 \times 0$ so that $h(x) = F(0 \times0) = 0$ by definition.
    If $x \neq 0$ then  we have $h(x) = 0/x^2 = 0$ again.
    Thus $h$ is the constant function $h(x) = 0$ and so is continuous when $y_0 = 0$.
    If $y_0 \neq 0$ then $y_0^2 > 0$ so that $x^2 + y_0^2 > 0$ since also $x \geq 0$.
    Thus the denominator is never zero that the $h(x)$ is given by the expression above, which is continuous by elementary calculus.
    Hence $h$ is always continuous.
    The same arguments show that $F$ is continuous in $y$ as well.
  }

  (b)
  We clearly have
  \gath{
    g(x) = F(x \times x) = \begin{cases}
      \frac{x^2}{x^2 + x^2} = \frac{x^2}{2x^2} = \frac{1}{2} & x \neq 0 \\
      0 & x = 0 \,.
    \end{cases}
  }

  (c)
  \qproof{
    First consider the function $f: \reals \to \reals \times \reals$ defined simply by $f(x) = x \times x$.
    This function is clearly continuous by Theorem~18.4 since it can be expressed as $f(x) = f_1(x) \times f_2(x)$ where the identical functions $f_1(x) = f_2(x) = x$ are obviously continuous.
    Then $g = F \circ f$, where $g$ is the function from part (b) since we have $g(x) = F(x \times x) = F(f(x))$ for any real $x$.
    Now, clearly $g$ as calculated in part (b) has a discontinuity at $x = 0$ so that it is not continuous.
    It then follows from Theorem~18.2c that either $F$ or $f$ is not continuous since $g = F \circ f$.
    As we know that the trivial function $f$ is continuous, it must then be that $F$ is not as desired.
  }
}

\exercise{13}{
  Let $A \ss X$; let $f: A \to Y$ be continuous; let $Y$ be Hausdorff.
  Show that if $f$ may be extended to a continuous function $g: \clA \to Y$, then $g$ is uniquely determined by $f$.
}
\sol{
  \dwhitman

  \qproof{
    Suppose that $g_1$ and $g_2$ are both continuous functions from $\clA$ to $Y$ that extend $f$ so that $g_1(x) = g_2(x) = f(x)$ for all $x \in A$.
    Clearly $g_1 = g_2$ if and only if $g_1(x) = g_2(x)$ for all $x \in \clA$.
    So suppose that this is \emph{not} the case so that there is an $x_0 \in \clA$ where $g_1(x_0) \neq g_2(x_0)$.
    Since $Y$ is a Hausdorff space and $g_1(x_0)$ and $g_2(x_0)$ are distinct, there are disjoint neighborhoods $V_1$ and $V_2$ of $g_1(x_0)$ and $g_2(x_0)$, respectively.
    Then there are also neighborhoods $U_1$ and $U_2$ of $x_0$ such that $g_1(U_1) \ss V_1$ and $g_2(U_2) \ss V_2$ by Theorem~18.1 since both $g_1$ and $g_2$ are continuous.

    Now let $U = U_1 \cap U_2$ so that $U$ is also a neighborhood of $x_0$.
    Since $x_0 \in \clA$, it follows that $U$ intersects $A$ so that there is a $y \in U$ where also $y \in A$ by Theorem~17.5.
    Since $y \in A$ we have that $g_1(y) = g_2(y) = f(y)$.
    We also have that $y \in U_1$ and $y \in U_2$ since $U = U_1 \cap U_2$.
    Thus $g_1(y) \in g_1(U_1)$ so that $f(y) = g_1(y) \in V_1$ since $g_1(U_1) \ss V_1$.
    Similarly $f(y) = g_2(y) \in V_2$, but then we have that $f(y) \in V_1 \cap V_2$, which contradicts the fact that $V_1$ and $V_2$ are disjoint!
    Hence it must be that $g_1 = g_2$, which shows uniqueness.
  }
}
