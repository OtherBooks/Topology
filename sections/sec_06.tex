\setcounter{subsection}{6-1}
\subsection{Finite Sets}

\exercise{1}{
  \eparts{
  \item Make a list of all the injective maps
    \gath{
      f : \braces{1,2,3} \to \braces{1,2,3,4} \,.
    }
    Show that none is bijective.
    (This constitutes a \emph{direct} proof that a set $A$ of cardinality three does not have cardinality four.)
  \item How many injective maps
    \gath{
      f : \intsfin{8} \to \intsfin{10}
    }
    are there?
    (You can see why one would not wish to try to prove \emph{directly} that there is no bijective correspondence between these sets.)
  }
}
\sol{
  \dwhitman

  \begin{lem}\label{lem:finset:inj}
    The number of injective mappings (i.e. the cardinality of the set of injective functions) from $\intsfin{m}$ to $\intsfin{n}$, where $m \leq n$, is equal to the number of $m$-permutations of $n$, which is
    \gath{
      \frac{n!}{(n-m)!} \,.
    }
  \end{lem}
  \qproof{
    We fix $n$ and show this for all $m \leq n$ by induction.
    First, for $m=1$, the domain of the mappings is simply $\braces{1}$ so that we need only choose a single element to which to map $1$.
    Since there are $n$ elements to choose from (since the range is $\intsfin{n}$) there are clearly
    \gath{
      n = \frac{n!}{(n-1)!} = \frac{n!}{(n-m)!}
    }
    mappings, all of which are trivially injective.

    Now suppose that $m < n$ and that there are $n! / (n-m)!$ injective mappings from $\intsfin{m}$ to $\intsfin{n}$.
    Consider any such mapping $\seqfin{f}{m}$.
    Since this mapping is injective, each $f_i$ is unique so that it uses $m$ of the $n$ available numbers in $\intsfin{n}$.
    Thus there are $n-m$ numbers to choose from to which to set $f_{m+1}$ so that the mapping $\seqfin{f}{m+1}$ is still injective.
    Hence for each injective mapping $\seqfin{f}{m}$ there are $n-m$ injective mappings from $\intsfin{m+1}$ to $\intsfin{n}$.
    Since there are $n!/(n-m)!$ such mappings by the induction hypothesis, the total number of mappings from $\intsfin{m+1}$ to $\intsfin{n}$ will be
    \gath{
      \frac{n!}{(n-m)!}(n-m) = \frac{n!}{(n-m-1)!} = \frac{n!}{[n-(m+1)]!} \,,
    }
    which completes the induction.
  }

  \mainprob
  
  (a) Here we have $n=4$ and $m=3$ in Lemma~\ref{lem:finset:inj} so that we expect $4!/(4-3)! = 4!/1! = 4! = 24$ injective mappings.
  Since the domain of each $f$ is a section of the positive integers, these maps can be written simply as 3-tuples.
  They are enumerated below:
  \begin{multicols}{4}
    \begin{enumerate}[itemsep=0cm]
    \item $(1, 2, 3)$
    \item $(1, 2, 4)$
    \item $(1, 3, 2)$
    \item $(1, 3, 4)$
    \item $(1, 4, 2)$
    \item $(1, 4, 3)$
    \item $(2, 1, 3)$
    \item $(2, 1, 4)$
    \item $(2, 3, 1)$
    \item $(2, 3, 4)$
    \item $(2, 4, 1)$
    \item $(2, 4, 3)$
    \item $(3, 1, 2)$
    \item $(3, 1, 4)$
    \item $(3, 2, 1)$
    \item $(3, 2, 4)$
    \item $(3, 4, 1)$
    \item $(3, 4, 2)$
    \item $(4, 1, 2)$
    \item $(4, 1, 3)$
    \item $(4, 2, 1)$
    \item $(4, 2, 3)$
    \item $(4, 3, 1)$
    \item $(4, 3, 2)$
    \end{enumerate}
  \end{multicols}
  Note that they are all injective since no number is used more than once in each tuple.
  Also none are surjective since it is easily verified that there is always an element of $\braces{1,2,3,4}$ that is not in each tuple.
  Thus none are a bijection since they are not surjective.

  (b) Here we have $n = 10$ and $m = 8$ in Lemma~\ref{lem:finset:inj} so that there are $10! / (10-8)! = 10! / 2! = 1814400$ injective mappings.
  That is nearly two million!
  Certainly a direct proof would be unfeasible by hand, but could be done by computer fairly easily.
}

\exercise{2}{
  Show that if $B$ is not finite and $B \ss A$, then $A$ is not finite.
}
\sol{
  \dwhitman

  \qproof{
    Suppose that $B$ is not finite and $B \ss A$ but that $A$ \emph{is} finite.
    Since $B \ss A$, either $B = A$ or $B$ is a proper subset of $A$.
    In the former case we clearly have a contradiction since $B$ would be finite since $A$ is and $B = A$.
    In the latter case we have that there is a bijection from $A$ to $\intsfin{n}$ for some $n \in \pints$ by definition since $A$ is finite.
    Then, since $B$ is a proper subset of $A$, it follows from Theorem~6.2 that there is a bijection from $B$ to $\intsfin{m}$ for some $m < n$.
    However, then clearly $B$ is finite by definition, which is also a contradiction since we know $B$ is not finite.
    Hence in either case there is a contradiction so that $A$ must not be finite.
  }
}

\exercise{3}{
  Let $X$ be the two-element set $\braces{0,1}$.
  Find a bijective correspondence between $X^\w$ and a proper subset of itself.
}
\sol{
  \dwhitman

  \qproof{
    Let $Y = \braces{\vx \in X^\w \where x_1 = 0}$, which is clearly a proper subset of $X^\w$ since, for example, $(1,1,\ldots)$ is in $X^\w$ but not in $Y$.
    We construct a bijective function $f$ from $X^\w$ to $Y$.
    So consider any $\vx \in X^\w$ and define
    \gath{
      y_i = \begin{cases}
        0 & i = 1 \\
        x_{i-1} & i \neq 1
      \end{cases}
    }
    for $i \in \pints$, noting that when $i \neq 1$ we have $i > 1$ so that $i-1 \geq 1$ so that $y_i = x_{i-1}$ is defined.
    Now define $f(\vx) = \vy = \seqinf{y}$ so that clearly $f$ is a function from $X^\w$ to $Y$, since $y_1 = 0$ for any input $\vx$.

    To show that $f$ is injective, consider $\vx$ and $\vx'$ in $X^\w$ where $\vx \neq \vx'$, and let $\vy = f(\vx)$ and $\vy' = f(\vx')$.
    Now, since $\vx \neq \vx'$, there is an $i \in \pints$ where $x_i \neq x_i'$.
    Since $i > 0$ (since $i \in \pints$) it follows that $i+1 > 1$ so that $i+1 \neq 1$.
    We then have by the definition of $f$ that $y_{i+1} = x_{(i+1)-1} = x_i \neq x_i' = x_{(i+1)-1}' = y_{i+1}'$ so that clearly $f(\vx) = \vy \neq \vy' = f(\vx')$.
    Since $\vx$ and $\vx'$ were arbitrary, this shows that $f$ is indeed injective.

    Now consider any $\vy \in Y$ so that $y_1 = 0$.
    Define $x_i = y_{i+1}$ for any $i \in \pints$ and let $\vx = \seqinf{x}$.
    Then $\vx \in X^\w$ since clearly each $x_i = y_{i+1} \in X$.
    Now let $\vy' = f(\vx)$ and consider any $i \in \pints$.
    If $i = 1$ then clearly $y_i' = y_1' = 0 = y_1 = y_i$ ($y_1' = 0$ since the range of $f$ is $Y$).
    If $i \neq 1$ then $y_i' = x_{i-1}' = y_{(i-1)+1} = y_i$.
    Hence $y_i' = y_i$ in both cases so that $f(\vx) = \vy' = \vy$ since $i$ was arbitrary.
    This shows that $f$ is surjective since $\vy$ was arbitrary.

    Therefore $f$ is bijective as desired.
  }
}

\exercise{4}{
  Let $A$ be a nonempty finite simply ordered set.
  \eparts{
  \item Show that $A$ has a largest element.
    [Hint: Proceed by induction on the cardinality of $A$.]
  \item Show that $A$ has the order type of a section of positive integers.
  }
}
\sol{
  \dwhitman

  (a)
  \qproof{
    We show by induction that, for all $n \in \pints$, any simply ordered set with cardinality $n$ has a largest element.
    This of course shows the result since, by definition, $A \neq \es$ has cardinality $n$ for some $n \in \pints$ when $A$ is finite.

    First, suppose that $A$ is simply ordered and has cardinality 1 so that clearly $A = \braces{a}$ for some element $a$.
    It is also clear that $a$ is trivially the largest element of $A$ since it is the only element.

    Now suppose that any simply ordered set with cardinality $n$ has a largest element.
    Suppose that $A$ is simply ordered by $\prec$ and has cardinality $n+1$.
    Then there is a bijection $f$ from $A$ to $\intsfin{n+1}$, noting that obviously $\inv{f}$ is also a bijection.
    Clearly $A$ is nonempty (since the cardinality of $A$ is $n+1 > n > 0$) so that there is an $a \in A$.
    Let $A' = A-\braces{a}$ so that $A'$ has cardinality $n$ by Lemma~6.1.
    Note also that clearly $A'$ is simply ordered by $\prec$ as well (technically we must restrict $\prec$ to elements of $A'$ so that it is really ordered by $\prec \cap (A' \times A')$).
    It then follows that $A'$ has a largest element $b$ by the induction hypothesis.
    Since $a$ and $b$ must be comparable in $\prec$ by the definition of a simple order we have the following:

    Case: $a = b$.
    This is not possible since $b \in A'$ but clearly $a \notin A - \braces{a} = A'$.
    
    Case: $a \prec b$.
    We claim that $b$ is the largest element of $A$.
    To see this, consider any $x \in A$ so that either $x = a$ or $x \in A'$.
    In the former case clearly $x = a \prece b$, and in the latter $x \prece b$ since $b$ is the largest element of $A'$.
    This shows that $b$ is the largest element of $A$ since $x$ was arbitrary.

    Case: $b \prec a$.
    We claim that $a$ is the largest element of $A$.
    So consider any $x \in A$ so that $x = a$ or $x \in A'$.
    In the first case obviously $x \prece x = a$, and in the second $x \prece b \prece a$ since $b$ is the largest element of $A'$.
    This shows that $a$ is the largest element of $A$ since $x$ was arbitrary.

    Thus in all cases we have shown that $A$ has a largest element, which completes the induction.
  }

  (b)
  \qproof{
    We again show this by induction on the (finite) cardinality of the set.
    First, if $A$ is a simply ordered set with cardinality 1 then clearly $A = \braces{a}$ for some $a$, which is clearly trivially the same order type as the section $\braces{1}$.

    Now suppose that all simply ordered sets of cardinality $n$ have the order type of a section of positive integers.
    Consider then a set $A$ simply ordered by $\prec$ that has cardinality $n+1$.
    Clearly $A \neq \es$ so that it has a largest element $a$ by part (a).
    Then the set $A' = A - \braces{a}$ has cardinality $n$ by Lemma~6.1.
    Since $A'$ is also clearly simply ordered by $\prec$ (with the appropriate restriction) it follows from the induction hypothesis that it has order type of $\intsfin{m}$ for some $m \in \pints$.
    Since this also implies that $A'$ has the cardinality of $m$, it has to be that $m=n$ since this cardinality is unique (by Lemma~6.5).
    So let $f'$ be the order-preserving bijection from $A'$ to $\intsfin{m} = \intsfin{n}$.
    Now define
    \gath{
      f(x) = \begin{cases}
        f'(x) & x \neq a \\
        n+1 & x = a
      \end{cases}
    }
    for any $x \in A$.
    It is clear that $f$ is a function from $A$ to $\intsfin{n+1}$ since obviously $n+1 \in \intsfin{n+1}$ and the range of $f'$ is $\intsfin{n} \ss \intsfin{n+1}$.

    Consider next any $x$ and $x'$ in $A$ where $x \prec x'$.
    Suppose for the moment that $x = a$.
    Then $x' \prece a = x$ since $a$ is the largest element of $A$.
    This contradicts the fact that $x \prec x'$ so that it must be that $x \neq a$.
    Then $f(x) = f'(x)$.
    If also $x' \neq a$ then clearly $f(x) = f'(x) < f'(x') = f(x')$ since $f'$ preserves order.
    If $x' = a$ then $f(x') = n+1$ so that $f(x) = f'(x) \leq n < n+1 = f(x')$ since the range of $f'$ is only $\intsfin{n}$.
    Hence in all cases $f(x) < f(x')$ so that $f$ preserves order since $x$ and $x'$ were arbitrary.
    Note that this also shows that $f$ is injective since, for any $x,x' \in A$ where $x \neq x'$, we can assume without loss of generality that $x \prec x'$ (since it must be that $x \prec x'$ or $x' \prec x$) so that $f(x) < f(x')$, and hence $f(x) \neq f(x')$.

    Lastly consider any $k \in \intsfin{n+1}$.
    If $k = n+1$ then clearly by definition $f(a) = n+1 = k$, noting that obviously $a \in A$.
    On the other hand, if $k \neq n+1$ then it has to be that $k < n+1$ so $k \leq n$.
    Then $k \in \intsfin{n}$, which is the range of $f'$ so that there is an $x \in A'$ where $f'(x) = k$ since $f'$ is bijective (and therefore surjective).
    Since $x \in A'$ we have that $x \in A$ but $x \neq a$ so that $f(x) = f'(x) = k$.
    This shows that $f$ is surjective since $k$ was arbitrary.

    Thus we have shown that $f$ is an order-preserving bijection from $A$ to $\intsfin{n+1}$, which completes the induction since by definition $A$ has order type $\intsfin{n+1}$.
  }
}

\exercise{5}{
  If $A \times B$ is finite, does it follow that $A$ and $B$ are finite?
}
\sol{
  \dwhitman
  
  We claim that in general this does not follow.
  \qproof{
    As a counterexample, let $A = \pints$ and $B = \es$.
    Clearly $A$ is infinite by Corollary~6.4 so that not both $A$ and $B$ are finite.
    It also follows from Exercise~5.3 part (c) that $A \times B = \es$ since $B$ is empty.
    Hence clearly $A \times B$ is finite.
  }

  If we add the additional stipulation that both $A$ and $B$ are nonempty, then the statement becomes true.
  \qproof{
    Since $A \times B$ is finite there is a bijective function $f: A \times B \to \intsfin{n}$ for some $n \in \pints$.
    We then show that $A$ is finite by first constructing an injective function $g$ from $A$ to $A \times B$.
    Since $B \neq \es$, there is a $b \in B$.
    So, for any $x \in A$, set $g(x) = (x, b)$, which is clearly in $A \times B$ so that $g$ is a function from $A$ to $A \times B$.
    Now consider $x$ and $x'$ in $A$ where $x \neq x'$.
    Then clearly $g(x) = (x,b) \neq (x', b) = g(x')$.
    This shows that $g$ is injective since $x$ and $x'$ were arbitrary.

    We then have that the composition $f \circ g$ is an injective function from $A$ to $\intsfin{n}$ by Exercise~2.4 part (b) since $f$ is injective as well (since it is a bijection).
    Therefore $A$ is finite by Corollary~6.7.
    An analogous argument uses the fact that $A \neq \es$ to show that $B$ is also finite.
  }
}

\exercise{6}{
  \eparts{
  \item Let $A = \intsfin{n}$.
    Show there is a bijection of $\pset{A}$ with the cartesian product $X^n$, where $X$ is the two-element set $X = \braces{0,1}$.
  \item Show that if $A$ is finite, then $\pset{A}$ is finite.
  }
}
\sol{
  \dwhitman
  
  (a)
  \qproof{
    We construct a bijection $f : \pset{A} \to X^n$.
    So, for any $Y \in \pset{A}$ we have that clearly $Y \ss A$.
    Then set
    \gath{
      x_i = \begin{cases}
        0 & i \notin Y \\
        1 & i \in Y
      \end{cases}
    }
    for any $i \in \intsfin{n} = A$.
    Now set $f(Y) = \vx = \seqfin{x}{n}$, noting that clearly $f(Y) \in X^n$ since each $x_i \in \braces{0,1} = X$.
    Hence $f$ is a function from $\pset{A}$ to $X^n$.

    To show that $f$ is injective consider $Y$ and $Y'$ in $\pset{A}$ where $Y \neq Y'$.
    Also let $\vx = f(Y)$ and $\vx' = f(Y')$ as defined above.
    Since $Y \neq Y'$, we can without loss of generality assume that there is an $i \in Y$ where $i \notin Y'$.
    It then follows that $x_i = 1 \neq 0 = x_i'$ by the definition of $f$.
    Hence clearly $f(Y) = \vx = \seqfin{x}{n} \neq \seqfin{x'}{n} = \vx' = f(Y')$, which shows that $f$ is injective since $Y$ and $Y'$ were arbitrary.

    Now consider any $\vx \in X^n$ and let $Y = \braces{i \in A \where x_i = 1}$.
    Clearly $Y \ss A$ so that $Y \in \pset{A}$.
    Let $\vx' = f(Y)$ and consider any $i \in \intsfin{n} = A$.
    If $i \in Y$ then $x_i = 1 = x_i'$ by the definitions of $Y$ and $f$.
    It $i \notin Y$ then $x_i \neq 1$ so that it has to be that $x_i = 0$ since $x_i \in X = \braces{0,1}$.
    Also, by the definition of $f$, we have that $x_i' = 0 = x_i$.
    Thus in either case $x_i = x_i'$ so that $\vx = \vx' = f(Y)$ since $i$ was arbitrary.
    Since $\vx$ was arbitrary, this shows that $f$ is surjective.

    Therefore $f$ is a bijection from $A$ to $X^n$ as desired.
  }

  (b)
  \qproof{
    First, if $A = \es$ then clearly $\pset{A} = \pset{\es} = \braces{\es}$ is finite.
    So assume in what follows that $A \neq \es$.
    Since $A$ is finite and nonempty there is a bijection $f$ from $A$ to $B = \intsfin{n}$ for some $n \in \pints$.
    Let $X = \braces{0,1}$ so that by part (a) there is a bijection $g$ from $\pset{B}$ to $X^n$.
    For any $Y \in \pset{A}$ clearly the mapping $h(Y) = \braces{i \in B \where \inv{f}(i) \in Y}$ is a bijection from $\pset{A}$ to $\pset{B}$.
    It then follows that $g \circ h$ is bijection from $\pset{A}$ to $X^n$.
    Since clearly $X^n$ is a finite cartesian product of finite sets, it follows from Corollary~6.8 that $X^n$ is finite so that $\pset{A}$ must be as well since there is a bijection between them.
  }
}

\exercise{7}{
  If $A$ and $B$ are finite, show that the set of all functions $f : A \to B$ is finite.
}
\sol{
  \dwhitman

  \qproof{
    As is customary, denote the set of all functions from $A$ to $B$ by $B^A$.
    First, if $A = \es$, then the only function from $A$ to $B$ is the vacuous function $\es$ so that $B^A = \braces{\es}$, which is clearly finite.
    So assume that $A \neq \es$.
    Then, since $A$ is finite, there is a bijection $f$ from $A$ to $\intsfin{n}$ for some $n \in \pints$, noting that of course $\inv{f}$ is then a bijection from $\intsfin{n}$ to $A$.
    
    We construct a bijection $h$ from $B^A$ to $B^n$.
    So, for any $g \in B^A$ set $h(g) = g \circ \inv{f}$, noting that clearly this is a function from $\intsfin{n}$ to $B$.
    Hence $h$ is a function from $B^A$ to $B^n$.

    To show that $h$ is injective consider $g$ and $g'$ in $B^A$ where $g \neq g'$.
    It then follows that there is an $a \in A$ where $g(a) \neq g'(a)$.
    Then let $k = f(a)$ so that clearly $\inv{f}(k) = a$ and $k \in \intsfin{n}$.
    We then have that $(g \circ \inv{f})(k) = g(\inv{f}(k)) = g(a) \neq g'(a) = g'(\inv{f}(k)) = (g' \circ \inv{f})(k)$ so that it must be that $h(g) = g \circ \inv{f} \neq g' \circ \inv{f} = h(g')$.
    Since $g$ and $g'$ were arbitrary, this shows that $h$ is injective.

    Now consider any function $i \in B^n$ and let $g = i \circ f$ so that clearly $g$ is a function from $A$ to $B$ since $f : A \to \intsfin{n}$ and $i : \intsfin{n} \to B$.
    Hence $g \in B^A$, and $h(g) = g \circ \inv{f} = (i \circ f) \circ \inv{f} = i \circ (f \circ \inv{f}) = i$.
    Since $i$ was arbitrary, this shows that $h$ is surjective as well.

    Hence $h$ is bijection from $B^A$ to $B^n$.
    Now, since $B^n$ is a finite cartesian product of finite sets (since $B$ is finite), it is finite by Corollary~6.8.
    Thus it must be that $B^A$ is also finite since there is bijection between them.
  }
}
