\setcounter{subsection}{13-1}
\subsection{Basis for a Topology}

\def\cTa{\col{T}_\a}
\def\cTb{\col{T}_\b}

\exercise{1}{
  Let $X$ be a topological space; let $A$ be a subset of $X$.
  Suppose that for each $x \in A$ there is an open set $U$ containing $x$ such that $U \ss A$.
  Show that $A$ is open in $X$.
}
\sol{
  \qproof{
    For each $x \in A$ we can choose an open set $U_x$ containing $x$ such that $U_x \ss A$.
    We then claim that $\bigcup_{x \in A} U_x = A$.
    So first consider any $y \in \bigcup_{x \in A} U_x$ so that there is an $x \in A$ such that $y \in U_x$.
    Then clearly also $y \in A$ since $U_x \ss A$.
    Hence $\bigcup_{x \in A} U_x \ss A$ since $y$ was arbitrary.
    Now consider $y \in A$ so that clearly $y \in U_y$.
    Then obviously $y \in \bigcup_{x \in A} U_x$ so that $A \ss \bigcup_{x \in A} U_x$ since $y$ was arbitrary.
    Thus we have shown that $\bigcup_{x \in A} U_x = A$, and since each $U_x$ is open, it follows from the definition of a topology that the union $\bigcup_{x \in A} U_x = A$ is open as well.
  }
}

\exercise{2}{
  Consider the nine topologies on the set $X = \braces{a,b,c}$ indicated in Example~1 of \S 12.
  Compare them; that is, for each pair of topologies, determine whether they are comparable, and if so, which is finer.
}
\sol{
  We label each of the topologies in Figure~12.1 with an ordered pair $(i,j)$ where $1 \leq i,j \leq 3$, $i$ is the row, $j$ is the column, and $(1,1)$ is the upper left corner.
  The following matrix lists which of each pair is \emph{finer}, or ``Inc'' if they are incomparable.
  \begin{center}
    \begin{tabular}{c|c|c|c|c|c|c|c|c|c|}
      & $(1,1)$ & $(1,2)$ & $(1,3)$ & $(2,1)$ & $(2,2)$ & $(2,3)$ & $(3,1)$ & $(3,2)$ & $(3,3)$ \\
      \hline
      $(1,1)$ & $=$ & $(1,2)$ & $(1,3)$ & $(2,1)$ & $(2,2)$ & $(2,3)$ & $(3,1)$ & $(3,2)$ & $(3,3)$ \\
      \hline
      $(1,2)$ &  & $=$ & Inc & Inc & Inc & Inc & $(1,2)$ & $(3,2)$ & $(3,3)$ \\
      \hline
      $(1,3)$ &  &  & $=$ & $(1,3)$ & Inc & $(2,3)$ & $(1,3)$ & Inc & $(3,3)$ \\
      \hline
      $(2,1)$ &  &  &  & $=$ & Inc & $(2,3)$ & Inc & $(3,2)$ & $(3,3)$ \\
      \hline
      $(2,2)$ &  &  &  &  & $=$ & Inc & Inc & Inc & $(3,3)$ \\
      \hline
      $(2,3)$ &  &  &  &  &  & $=$ & $(2,3)$ & Inc & $(3,3)$ \\
      \hline
      $(3,1)$ &  &  &  &  &  &  & $=$ & $(3,2)$ & $(3,3)$ \\
      \hline
      $(3,2)$ &  &  &  &  &  &  &  & $=$ & $(3,3)$ \\
      \hline
      $(3,3)$ &  &  &  &  &  &  &  &  & $=$ \\
      \hline
    \end{tabular}
  \end{center}
  We know that $\pss$ forms a strict partial order on these topologies.
  So we can also list all the maximal simply ordered subsets, each in order:
  \gath{
    (1,1) \pss (2,2) \pss (3,3) \\
    (1,1) \pss (3,1) \pss (1,2) \pss (3,2) \pss (3,3) \\
    (1,1) \pss (3,1) \pss (1,3) \pss (2,3) \pss (3,3) \\
    (1,1) \pss (2,1) \pss (1,3) \pss (2,3) \pss (3,3) \\
    (1,1) \pss (2,1) \pss (3,2) \pss (3,3)
  }
}

\def\cTi{\col{T}_\infty}
\exercise{3}{
  Show that the collection $\cTa$ given in Example~4 of \S 12 is a topology on $X$.
  Is the collection
  \gath{
    \cTi = \braces{U \where X - U \text{ is infinite or empty or all of } X}
  }
  a topology on $X$?
}
\sol{
  Recall that $\cTa$ from Example~12.4 is the set of all subsets $U$ of $X$ such that $X - U$ either is countable or is all of $X$.
  First we show that $\cTa$ is a topology on $X$.
  \qproof{
    First, clearly $\es \in \cTa$ since $X - \es = X$ is all of $X$.
    Also $X \in \cTa$ since $X - X = \es$ is countable.
    Now suppose that $\cA$ is a subcollection of $\cTa$ so that $X - U$ is countable (or all of $X$) for every $U \in \cA$.
    Then we have that
    \gath{
      X - \bigcup \cA = X - \bigcup_{A \in \cA} A = \bigcap_{A \in \cA} (X - A)
    }
    is countable (or all of $X$) since every $X - A$ is countable (or all of $X$).
    Therefore $\bigcup \cA \in \cTa$ by definition.

    Now suppose that $U_1, \ldots, U_n$ are nonempty elements of $\cTa$ so that $X - U_i$ is a countable subset of $X$ or $X$ itself for each $i \in \intsfin{n}$.
    Then we have
    \gath{
      X - \bigcap_{i=1}^n U_i = \bigcup_{i=1}^n (X - U_i)
    }
    is a finite union of sets that are either countable subsets of $X$, or $X$ itself.
    It then follows that the union is countable or $X$ itself so that $\bigcap_{i=1}^n U_i \in \cTa$ by definition.
    This completes the proof that $\cTa$ is a topology on $X$.
  }

  Now we claim that the collection $\cTi$ as defined above is not always a topology on $X$.
  \qproof{
    As a counterexample, let $X = \pints$ and suppose that $\cTi$ is a topology on $X$.
    Clearly if $U$ is a finite subset of $X$, then $X - U$ is infinite since $X$ is infinite so that $U$ is open.
    Now consider the subcollection
    \gath{
      \cA = \braces{\braces{i} \where i \in \pints \text{ and } i > 1} = \braces{\braces{2}, \braces{3}, \ldots} \,.
    }
    Then clearly we have that $\bigcup \cA = \braces{2,3, \ldots}$ so that $X - \bigcup \cA = \braces{1}$ is neither infinite, empty, nor all of $X$.
    Therefore $\bigcup \cA$ cannot be open, which violates property (2) of a topology.
    So it must be that $\cTi$ is not a topology, which of course contradicts our supposition that it is!
  }
}

\def\cTo{\col{T}_1}
\def\cTt{\col{T}_2}
\def\cTs{\col{T}_s}
\def\cTl{\col{T}_l}
\exercise{4}{
  \eparts{
  \item If $\braces{\cTa}$ is a family of topologies on $X$, show that $\bigcap \cTa$ is a topology on $X$.
    Is $\bigcup \cTa$ a topology on $X$?
  \item Let $\braces{\cTa}$ be family of topologies on $X$.
    Show that there is a unique smallest topology on $X$ containing all the collections $\cTa$, and a unique largest topology contained in all $\cTa$.
  \item If $X = \braces{a,b,c}$, let
    \gath{
      \cTo = \braces{\es, X, \braces{a}, \braces{a,b}} \condgap \text{and} \condgap \cTt = \braces{\es, X, \braces{a}, \braces{b,c}} \,.
    }
    Find the smallest topology containing $\cTo$ and $\cTt$, and the largest topology contained in $\cTo$ and $\cTt$.      
  }
}
\sol{
  (a) First we show that $\bigcap \cTa$ is a topology on $X$.
  \qproof{
    First, clearly since $\es$ and $X$ are in every $\cTa$ since they are topologies, they are both in $\bigcap \cTa$ so that property (1).
    Now suppose that $\cA$ is a subcollection of $\bigcap \cTa$.
    Consider any $\cTb$ and any $A \in \cA$.
    Then $A$ is also in $\bigcap \cTa$ since $\cA \ss \bigcap \cTa$.
    It then follows that $A$ is in our specific $\cTb$.
    Since $A$ was arbitrary it follows that $\cA$ is a subcollection of $\cTb$ so that $\bigcup \cA \in \cTb$ also  since $\cTb$ is a topology.
    Since $\cTb$ was also arbitrary it follows that $\bigcup \cA \in \bigcap \cTa$.
    Lastly, since the subcollection $\cA$ was arbitrary, this shows property (2) for $\bigcap \cTa$.

    Finally, suppose that $U_1, \ldots, U_n$ are sets in $\bigcap \cTa$.
    Consider any $\cTb$ so that clearly then $U_i \in \cTb$ for every $i \in \intsfin{n}$.
    It then follows that $\bigcap_{i=1}^n U_i \in \cTb$ since $\cTb$ is a topology.
    Since $\cTb$ was arbitrary, this shows that $\bigcap_{i=1}^n U_i \in \bigcap \cTa$, which shows property (3) for $\bigcap \cTa$.
    This completes the proof that $\bigcap \cTa$ is a topology on $X$ since all three properties have been shown.
  }

  Now we claim that $\bigcup \cTa$ is \emph{not} generally a topology.
  \qproof{
    As a counterexample consider the set $X = \braces{a,b,c}$, the topologies $\cTo = \braces{\es, X, \braces{a}}$ and $\cTt = \braces{\es, X, \braces{b}}$, and the collection of topologies $\col{C} = \braces{\cTo, \cTt}$.
    Then we clearly have that $\bigcup \col{C} = \cTo \cup \cTt = \braces{\es, X, \braces{a}, \braces{b}}$, which is not a topology since $\cA = \braces{\braces{a}, \braces{b}}$ is a subcollection of $\bigcup \col{C}$ but $\bigcup \cA = \braces{a,b}$ is not in $\bigcup \col{C}$.
  }

  (b) First we show that there is a unique smallest topology that contains each $\cTa$.
  \qproof{
    It was proven in part (a) that $\bigcup \cTa$ is not necessarily a topology.
    However, it is clearly always a subbasis for a topology since clearly $X \in \bigcup \cTa$ since it is in each $\cTa$ since they are topologies.
    Hence obviously then $\bigcup \parens{\bigcup \cTa} = X$ so that $\bigcup \cTa$ is a subbasis by definition.
    Then let $\cTs$ be the topology generated by the subbasis $\bigcup \cTa$.
    We claim that $\cTs$ is then the smallest topology that contains all the $\cTa$ as subsets.

    First, from the proof following the definition of a subbasis, we know that the set $\cB$ of finite intersections of elements of $\bigcup \cTa$ is a basis for the topology $\cTs$, and that $\cTs$ is the set of all unions of subcollections of $\cB$.

    We first show that every $\cTa$ is indeed contained as a subset of $\cTs$.
    So consider any specific $\cTb$ and any $U \in \cTb$.
    Then clearly $U \in \bigcup \cTa$ so that $U \in \cB$ since $U = \bigcap \braces{U}$ is a finite intersection of elements of $\bigcup \cTa$.
    It then follows that $U \in \cTs$ since $U = \bigcup \braces{U}$ is the union of a subcollection of $\cB$.
    Since $U$ was arbitrary, this shows that $\cTb \ss \cTs$, which shows the result since $\cTb$ was arbitrary.

    Now we show that $\cTs$ is the smallest such topology as ordered by $\pss$.
    So suppose that $\cT$ is a topology that contains every $\cTa$ as a subset.
    Consider any $U \in \cTs$ so that $U = \bigcup \cC$ for some subcollection $\cC \ss \cB$.
    Now consider any $Y \in \cC$ so that also $Y \in \cB$.
    Then $Y = \bigcap_{i=1}^n Y_i$ where each $Y_i \in \bigcup \cTa$.
    Then each $Y_i$ is in some $\cTb \ss \cT$ so that also $Y_i \in \cT$.
    Since $\cT$ is a topology, it follows that the finite intersection $\bigcap_{i=1}^n Y_i = Y$ is also in $\cT$.
    Since $Y$ was arbitrary, this shows that $\cC \ss \cT$ so that $\cC$ is a subcollection of $\cT$.
    It then follows that $\bigcup \cC = U$ is also in $\cT$ since $\cT$ is a topology.
    Since $U$ was arbitrary, we have that $\cTs \ss \cT$, which shows that $\cTs$ is the smallest topology since $\cT$ was arbitrary.

    It is easy to see that $\cTs$ is unique since, if both $\cTo$ and $\cTs$ are the smallest topologies that contain each $\cTa$ as subsets, then we would have that both $\cTo \ss \cTt$ and $\cTt \ss \cTo$ so that $\cTo = \cTt$.
    Really this follows from the more general fact that smallest elements in any order are always unique.
  }

  Next we show that there is a unique largest topology that is contained in each $\cTa$.
  \qproof{
    It was shown in part (a) that $\cTl = \bigcap \cTa$ is a topology on $X$.
    We claim that in fact this is the unique largest topology contained in all $\cTa$.
    First, clearly $\cTl = \bigcap \cTa$ is contained in each $\cTa$ since the intersection of a collection of sets is always a subset of every set in the collection.
    Now suppose that $\cT$ is a topology that is contained in every $\cTa$, i.e. $\cT \ss \cTa$ for every $\cTa$.
    Then clearly for any $U \in \cT$ we have that $U \in \cTa$ for every $\cTa$ so that $U \in \bigcap \cTa = \cTl$.
    Thus $\cT \ss \cTl$ since $U$ was arbitrary.
    This shows that $\cTl$ is the largest such topology since $\cT$ was arbitrary.

    Clearly also $\cTl$ is unique since, if $\cTo$ and $\cTt$ are two such largest topologies that are contained in every $\cTa$.
    Then we would have $\cTo \ss \cTt$ and $\cTt \ss \cTo$ so that $\cTo = \cTt$.
    This also follows from the fact that the largest element in any ordered set (or collection of sets in this case) is unique.
  }

  (c)
  Note that the proofs in part (b) are constructive so that we can construct these topologies as done in the proof.
  For the smallest topology containing $\cTo$ and $\cTt$ we have that
  \gath{
    \bigcup \braces{\cTo, \cTt} = \cTo \cup \cTt = \braces{\es, X, \braces{a}, \braces{a,b}, \braces{b,c}}
  }
  is a subbasis for the smallest topology $\cTs$.
  Then the collection of all finite intersections of elements of this set is a basis for $\cTs$:
  \gath{
    \cB = \braces{\es, X, \braces{a}, \braces{b}, \braces{a,b}, \braces{b,c}} \,.
  }
  Then the topology $\cTs$ is the set of all unions of subcollections of $\cB$:
  \gath{
    \cTs = \braces{\es, X, \braces{a}, \braces{b}, \braces{a,b}, \braces{b,c}} = \cB
  }
  so that evidently the basis and the topology are the same set here!

  For the largest topology contained in $\cTo$ and $\cTt$ we have simply
  \gath{
    \cTl = \bigcap \braces{\cTo, \cTt} = \cTo \cap \cTt = \braces{\es, X, \braces{a}} \,.
  }
}

\exercise{5}{
  Show that if $\cA$ is a basis for a topology on $X$, then the topology generated by $\cA$ equals the intersection of all topologies on $X$ that contain $\cA$.
  Prove the same if $\cA$ is a subbasis.
}
\sol{
  Suppose that $\cT$ is the topology generated by basis $\cA$, and $\cC$ is the collection of topologies on $X$ that contain $\cA$ as a subset.

  First we show that $\cT = \bigcap \cC$.
  \qproof{
    Consider $U \in \cT$ and any $\cT_c \in \cC$ so that $\cA \ss \cT_c$.
    Then, since $\cA$ generates $\cT$, it follows from Lemma~13.1 that $U$ is the union of elements of $\cA$.
    Clearly then each of these elements of $\cA$ is in $\cT_c$ since $\cA \ss \cT_c$ so that their union is as well since $\cT_c$ is a topology.
    Hence $U \in \cT_c$ so that $\cT \ss \cT_c$ since $U$ was arbitrary.
    Hence $\cT$ is contained in all elements of $\cC$ so that $\cT \ss \bigcap \cC$.
    Also, clearly $\cT$ is a topology that contains $\cA$ so that $\cT \in \cC$.
    Clearly then $\bigcap \cC \ss \cT$ so that $\cT = \bigcap \cC$ as desired.
  }

  Next we show the same thing but when $\cA$ is a subbasis.
  \qproof{
    Let $\cB$ be the set of all finite intersections of elements of $\cA$, which we know is a basis for $\cT$ by the proof after the definition of a subbasis.
    We show that $\cB \ss \cT_c$ for all $\cT_c \in \cC$.
    So consider any set $B \in \cB$ so that $B$ is the finite intersection of elements of $\cA$.
    Also consider any $\cT_c \in \cC$ so that each of these elements is in $\cT_c$ since $\cA \ss \cT_c$.
    Since $\cT_c$ is a topology, clearly the finite intersection of these elements, i.e. $B$, is in $\cT_c$.
    Hence $\cB \ss \cT_c$ since $B$ was arbitrary.

    It then follows from what was shown before that $\cT = \bigcap \cC$ since $\cT$ is the topology generated by the basis $\cB$ and $\cB$ is contained in each topology in $\cC$.
  }
}

\exercise{6}{
  Show that the topologies of $\reals_l$ and $\reals_K$ are not comparable.
}
\sol{
  \qproof{
    Let $\cT_l$ and $\cT_K$ be the topologies of $\reals_l$ and $\reals_K$, respectively.
    Also let $\cB_l$ and $\cB_K$ be the corresponding bases.

    Consider $x = 0 \in \reals$ and $B_l = \clop{0,1}$, which clearly contains $0$ and is a basis element of $\cB_l$.
    Let $B_K$ be any basis element of $\cB_K$ that contains $0$.
    Then $B_K$ is either $(a,b)$ or $(a,b) - K$ for some $a < b$.
    In either case it must be that $a < 0 < b$ so that clearly $a < a/2 < 0 < b$.
    Also $a/2 \notin K$ since $a/2 < 0$ so that we have $a/2 \in (a,b)$ and $a/2 \in (a,b) - K$.
    Clearly also $a/2 \notin \clop{0,1}$ so that it cannot be that $B_K \ss B_l$.
    We have therefore shown that
    \gath{
      \exists x \in \reals \exists B_l \in \cB_l \squares{x \in B_l \land \forall B_K \in \cB_K \parens{x \in B_K \imp B_K \nss B_l}} \\
      \exists x \in \reals \exists B_l \in \cB_l \squares{x \in B_l \land \forall B_K \in \cB_K \parens{x \notin B_K \lor B_K \nss B_l}} \\
      \exists x \in \reals \exists B_l \in \cB_l \squares{x \in B_l \land \lnot \exists B_K \in \cB_K \parens{x \in B_K \land B_K \ss B_l}} \\
      \exists x \in \reals \exists B_l \in \cB_l \lnot \squares{x \notin B_l \lor \exists B_K \in \cB_K \parens{x \in B_K \land B_K \ss B_l}} \\
      \exists x \in \reals \exists B_l \in \cB_l \lnot \squares{x \in B_l \imp \exists B_K \in \cB_K \parens{x \in B_K \land B_K \ss B_l}} \\
      \lnot \forall x \in \reals \forall B_l \in \cB_l \squares{x \in B_l \imp \exists B_K \in \cB_K \parens{x \in B_K \land B_K \ss B_l}} \\
      \lnot \forall x \in \reals \forall B_l \in \cB_l \squares{x \in B_l \imp \exists B_K \in \cB_K \parens{x \in B_K \ss B_l}}
    }
    This shows by the negation of Lemma~13.3 that $\cT_K$ is not finer than $\cT_l$.

    Now consider again $x = 0 \in \reals$ and $B_K = (-1,1) - K$, which clearly contains $0$ and is a basis element of $\cB_K$.
    Let $B_l$ be any basis element of $\cB_l$ that contains $0$ so that $B_l = \clop{a,b}$ where $a \leq 0 < b$.
    Clearly we have that $1/b > 0$ and there is an $n \in \pints$ where $n > 1/b$ since the positive integers have no upper bound.
    We then have
    \ali{
      0 &< 1/b < n \\
      0 &< 1 < bn & \text{(since $b > 0$)} \\
      0 &< 1/n < b & \text{(since $n > 1/b > 0$)}
    }
    so that $1/n \in \clop{0,b} = B_l$.
    However, clearly $1/n \in K$ so that $1/n \notin (-1,1) - K = B_K$.
    Hence it must be that $B_l \nss B_K$.
    This shows that $\cT_l$ is not finer than $\cT_K$ by the negation of Lemma~13.3 as before.

    This completes the proof that $\cT_K$ and $\cT_l$ are not comparable.
  }
}

\exercise{7}{
  Consider the following topologies on $\reals$:
  \ali{
    \cT_1 &= \text{the standard topology,} \\
    \cT_2 &= \text{the topology of $\reals_K$,} \\
    \cT_3 &= \text{the finite compliment topology,} \\
    \cT_4 &= \text{the upper limit topology, having all sets $\opcl{a,b}$ as basis,} \\
    \cT_5 &= \text{the topology having all sets $(-\infty, a) = \braces{x \where x < a}$ as basis.}
  }
  Determine, for each of these topologies, which of the others it contains.
}
\sol{
  We claim that $\cT_3 \pss \cT_1 \pss \cT_2 \pss \cT_4$ and $\cT_5 \pss \cT_1 \pss \cT_2 \pss \cT_4$ but that $\cT_3$ and $\cT_5$ are incomparable.

  Let $\cB_1$, $\cB_2$, $\cB_4$, and $\cB_5$ be the given bases corresponding to the above topologies, noting that $\cT_3$ is defined directly rather than generated from a basis.

  First we show that $\cT_3 \pss \cT_1$.
  \qproof{
    Consider any $U \in \cT_3$ so that $\reals - U$ is finite or $U = \reals$.
    Clearly in the latter case $U \in \cT_1$ since it is a topology.
    In the former case $\reals - U$ is a finite set of real numbers so that its elements can be enumerated as $\braces{x_1, x_2, \ldots, x_n}$ for some $n \in \pints$ where $x_1 < x_2 < \cdots < x_n$.
    Then clearly we have that
    \gath{
      U = (-\infty, x_1) \cup \squares{\bigcup_{k=1}^{n-1} (x_k, x_{k+1})} \cup (x_n, \infty) \,.
    }
    Each of these sets is an interval $(a,b)$ or the union of such intervals.
    For example, the set $(-\infty, x_1)$ can be covered by the countable union of intervals
    \gath{
      \bigcup_{k=1}^\infty (x_1 - k - 1, x_1 - k + 1)
    }
    and similarly for the interval $(x_n, \infty)$.
    Hence the union $U$ is an element of $\cT_1$ by Lemma~13.1.
    Since $U$ was arbitrary, this shows that $\cT_3 \ss \cT_1$.

    Now, clearly the interval $(-1, 1)$ is in $\cT_1$ since it is a basis element.
    However, we also have that $\reals - (-1,1) = \opcl{-\infty, -1} \cup \clop{1, \infty}$ is neither finite nor all of $\reals$.
    Hence $(-1, 1) \notin \cT_3$.
    This shows that $\cT_1$ cannot be a subset of $\cT_3$ so that $\cT_3 \pss \cT_1$ as desired.
  }

  Next we show that $\cT_5 \pss \cT_1$ also.
  \qproof{
    Consider any $x \in \reals$ and any basis element $B_5 \in \cB_5$ containing $x$.
    Then $B_5 = (-\infty, a)$ where $x < a$.
    Let $B_1 = (x-1, a)$, which is a basis element in $\cB_1$.
    Also clearly $B_1$ contains $x$ and is a subset of $B_5$.
    This proves that $\cT_5 \ss \cT_1$ by Lemma~13.3.

    Now consider $x = -1$ and basis element $B_1 = (-2, 0)$ in $\cB_1$, noting that obviously $x \in B_1$, and hence $-2 < x < 0$.
    Let $B_5$ be any element of $\cB_5$ containing $x$ so that $B_5 = (-\infty, a)$ where $x < a$.
    Clearly then $-3 < -2 < x < a$ so that $-3 \in B_5$.
    However, since $-3 \notin (-2, 0) = B_1$, this shows that $B_5 \nss B_1$.
    This suffices to show that $\cT_1 \nss \cT_5$ by the negation of Lemma~13.3.
    Therefore $\cT_5 \pss \cT_1$ as desired.
  }

  Now we show that $\cT_3$ and $\cT_5$ are not comparable.
  \qproof{
    First consider the set $U = \reals - \braces{0}$ so that $U \in \cT_3$ since $\reals - U = \braces{0}$ is obviously finite.
    Now suppose that $U \in \cT_5$ as well.
    Then, since clearly $1 \in U$, there must be a basis element $B_5 \in \cB_5$ where $1 \in B_5$ and $B_5 \ss U$ by the definition of a topological basis.
    Then $B_5 = (-\infty, a)$ where $1 < a$.
    However, since $0 < 1 < a$ as well, it must be that $0 \in B_5$, and hence $0 \in U$ since $B_5 \ss U$.
    As this clearly contradicts the definition of $U$, it has to be that $U$ is not in fact in $\cT_5$ so that $\cT_3 \nss \cT_5$.

    Now consider the set $U = (-\infty, 0)$, which is clearly in $\cT_5$ since it is a basis element.
    However, since $\reals - U = \clop{0, \infty}$ is clearly neither all of $\reals$ nor finite, it follows that $U \notin \cT_3$.
    This shows that $\cT_5 \nss \cT_3$, which completes the proof that the two are incomparable.
  }

  Now, the fact that $\cT_1 \pss \cT_2$ was shown in Lemma~13.4.
  All that remains to be shown is that $\cT_2 \pss \cT_4$ since the rest of the relations follow from the transitivity of proper inclusion.
  \qproof{
    First consider any basis element $B_2 \in \cB_2$ and any $x \in B_2$.
    Either $B_2$ is $(a,b)$ or $(a,b) - K$ for $a < b$ so that $a < x < b$ with $x \notin K$.
    In the former case clearly the set $B_4 = \opcl{a,x}$ is in $\cB_4$, $x \in B_4$, and $B_4 \ss B_2$.
    In the latter case we have the following:

    Case: $x \leq 0$.
    Then here again $B_4 = \opcl{a,x}$ is in $\cB_4$, $x \in B_4$, and $B_4 \ss B_2$ since $y \notin K$ for any $y \in B_4$ since then $a < y \leq x \leq 0$.

    Case: $x > 0$.
    Then let $n$ be the smallest positive integer where $n > 1/x$, which exists since $\pints$ has no upper bound and is well-ordered.
    It then follows that $0 < 1/n < x$ and there are no integers $m$ such that $1/n < 1/m \leq x$.
    So let $a' = \max(a, 1/n)$ and set $B_4 = \opcl{a', x}$ so that, for any $y \in B_4$, both $a \leq a' < y \leq x < b$ and $1/n \leq a' < y < x$, and hence $y \in (a,b)$ and $y \notin K$.
    Therefore $y \in (a,b) - K = B_2$.
    Since $y$ was arbitrary, this shows that $B_4 \ss B_2$, noting that also clearly $x \in B_4$ and $B_4 \in \cB_4$.

    Hence in any case it follows that $\cT_2 \ss \cT_4$ from Lemma~3.13.

    Now let $x = -1$ and $B_4 = \opcl{-2, -1}$ so that clearly $x \in B_4$ and $B_4 \in \cB_4$.
    Then let $B_2$ be any basis element in $\cB_2$ that contains $x$.
    Then we have that $B_2$ is either $(a,b)$ or $(a,b) - K$ where $a < x < b$ and $x \notin K$.

    Case: $0 < b$.
    Then $a < x = -1 < 0 \leq b$ so that $0$ is in both $(a,b)$ and $(a,b) - K$ since clearly $0 \notin K$, and thus $0 \in B_2$.
    However, clearly $0 \notin \opcl{-2,-1} = B_4$.

    Case: $0 \geq b$.
    Then $a < x < (x+b)/2 < b \leq 0$ so that $(x+b)/2 \in B_2$ since $(x+b)/2$ is not in $K$.
    Clearly also though $(x+b)/2 \notin \opcl{-2, x} = B_4$ since $x < (x+b)/2$.

    Thus in either case we have that $B_2 \nss B_4$.
    This shows the negation of Lemma~13.3 so that $\cT_4 \nss \cT_2$.
    Hence $\cT_2 \pss \cT_4$ as desired.
  }

  It is perhaps a rather surprising fact that, though it has been shown that the $K$ and lower limit topology are incomparable (Exercise~13.6), the $K$ topology and the upper limit topology \emph{are} comparable as was just shown.
}

\exercise{8}{
  \eparts{
  \item Apply Lemma~13.2 to show that the countable collection
    \gath{
      \cB = \braces{(a,b) \where a < b, \text{ $a$ and $b$ rational}}
    }
    is a basis that generates the standard topology on $\reals$.
  \item Show that the collection
    \gath{
      \cC = \braces{\clop{a,b} \where a < b, \text{ $a$ and $b$ rational}}
    }
    is a basis that generates a topology different from the lower limit topology on $\reals$.
  }
}
\sol{
  (a)
  \qproof{
    Let $\cT$ be the standard topology on $\reals$.
    First, clearly $\cB$ is a collection of open sets of $\cT$ since each element is a basis element in the standard basis (i.e. an open interval).
    Now consider any $U \in \cT$ and any $x \in U$.
    Then there is a standard basis element $B' = (a',b')$ such that $x \in B'$ and $B' \ss U$ since $\cT$ is generated by the standard basis.
    Then $a' < x < b'$ so that, since the rationals are order-dense in the reals (shown in Exercise~4.9 part (d)), there are rational $a$ and $b$ such that $a' < a < x < b < b'$.
    Let $B = (a, b)$ so that clearly $x \in B$, $B \ss B' \ss U$, and $B \in \cB$.
    This shows that $\cB$ is a basis for $\cT$ by Lemma~13.2 since $U$ and $x \in U$ were arbitrary.
  }

  (b)
  \qproof{
    First we must show that $\cC$ is a basis at all.
    Clearly, for any $x \in \reals$ we have that there is an element in $\cC$ containing $x$, for example $\clop{x, x+1}$.
    Now suppose that $C_1 = \clop{a_1, b_1}$ and $C_2 = \clop{a_2, b_2}$ are two elements of $\cC$ and that $x \in C_1 \cap C_2$.
    Then obviously $a_1 \leq x < b_1$ and $a_2 \leq x < b_2$.
    Let $a = \max(a_1, a_2)$ and $b = \min(b_1, b_2)$ and $C = \clop{a,b}$ so that clearly $C \in \cC$.
    Also clearly $a \leq x < b$ since both $a_1 \leq x < b_1$ and $a_2 \leq x < b_2$, $a$ is $a_1$ or $a_2$, and $b$ is $b_1$ or $b_2$.
    Therefore $C$ contains $x$.
    Now consider any $y \in C$ so that $a_1 \leq a \leq y < b \leq b_1$ and $a_2 \leq a \leq y < b \leq b_2$ and hence $y \in C_1$ and $y \in C_2$.
    This shows that $C \ss C_1 \cap C_2$ since $y$ was arbitrary.
    By definition this suffices to show that $\cC$ is a basis for a topology.

    So let $\cT$ be the topology generated by $\cC$ and $\cT_l$ be the lower limit topology.
    Now consider $U = \clop{x, x+1}$ where $x$ is any irrational number, for example $x = \pi$.
    Let $C$ be any basis element in $\cC$ containing $x$ so that $C = \clop{a,b}$ where $a$ and $b$ are rational.
    It must be that $a \neq x$ since $a$ is rational but $x$ is not.
    Also, since $C$ contains $x$ it has to be that $a \leq x$.
    So it has to be that $a < x$, but then $a \in C$ but $a \notin \clop{x, x+1} = U$.
    This shows that $C$ is not a subset of $U$.
    Hence we have shown
    \gath{
      \exists x \in U \forall C \in \cC \parens{x \in C \imp C \nss U} \\
      \exists x \in U \forall C \in \cC \parens{x \notin C \lor C \nss U} \\
      \lnot \forall x \in U \exists C \in \cC \parens{x \in C \land C \ss U} \,.
    }
    This shows that $U \notin \cT$ by the definition of a generated topology.
    However, clearly we have that $U \in \cT_l$ since it is a lower limit basis element.
    This suffices to show that $\cT$ and $\cT_l$ are different topologies.
  }
}
