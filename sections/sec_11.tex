\setcounter{subsection}{11-1}
\subsection{The Maximum Principle}

\def\cA{\col{A}}
\def\cB{\col{B}}
\def\cC{\col{C}}

\exercise{1}{
  If $a$ and $b$ are real numbers, define $a \prec b$ if $b-a$ is positive and rational.
  Show this is a strict partial order on $\reals$.
  What are the maximal simply ordered subsets?
}
\sol{
  \begin{lem}\label{lem:maximum:nonempty}
    If $B$ is a maximal simply ordered subset of a nonempty partially ordered set $A$, then $B$ is nonempty.
  \end{lem}
  \qproof{
    Since $A$ is nonempty, there is an $a \in A$.
    Clearly $\es$ is vacuously simply ordered.
    However, it cannot be maximal since clearly the set $\braces{a}$ properly contains $\es$ as a subset but is also clearly vacuously simply ordered by $\prec$.
    Hence, since $B$ \emph{is} maximal it must be that $B \neq \es$ as desired.
  }

  \mainprob

  First we show that $\prec$ is a strict partial order.
  \qproof{
    First consider any $a \in \reals$ so that $a-a = 0$, which is not positive and hence it is not true that $a \prec a$.
    Therefore $\prec$ is nonreflexive.
    Now consider $a,b,c \in \reals$ where $a \prec b$ and $b \prec c$.
    Then we have that $x = b-a$ and $y = c-b$ are positive and rational.
    It then clearly follows that
    \gath{
      c - a = (c-b) + (b-a) = y + x
    }
    is also rational and positive since both $x$ and $y$ are.
    Thus $a \prec c$, which shows that $\prec$ is transitive.
    Since $\prec$ was shown to be nonreflexive and transitive, this shows that it is a strict partial order as desired.
  }

  For any element $x \in \reals$, define the set $A_x = \braces{y \in \reals \where x - y \in \rats}$.
  We then claim that the collection $\cA = \braces{A_x}_{x \in \reals}$ is exactly the set of all maximal simply ordered subsets.
  \qproof{
    Suppose that $\cB$ is the set of maximally simply ordered subsets of $\reals$.
    Then we show that $\cA = \cB$.

    To show that $\cA \ss \cB$ consider any $X \in \cA$ so that $X = A_x$ for some $x \in \reals$.
    Now consider any distinct $y$ and $z$ in $X = A_x$ so that by definition $x - y$ and $x - z$ are both rational so that $z - x = -(x-z)$ is also rational.
    Then clearly $z - y = (z-x) + (x-y)$ is rational as is $y - z = -(z-y)$.
    Since $y$ and $z$ are distinct, we have that $z-y$ and $y-z$ are nonzero and that either $y < z$ or $z < y$.
    In the former case we have that $z - y$ is a positive rational number and in the latter $y - z$ is.
    Thus either $y \prec z$ or $z \prec y$, which shows that $X = A_x$ is simply ordered since $y$ and $z$ were arbitrary.
    Now consider any $y \in A_x$ and $z \notin A_x$ so that $x-y$ is rational but $x-z$ is irrational so that $z-x=-(x-z)$ is also irrational.
    Since a rational added to an irrational is also irrational (which is trivially easy to prove), it follows that $z - y = (z-x) + (x-y)$ is irrational as is $y-z = -(z-y)$.
    Hence it cannot be that either $y \prec z$ or $z \prec y$.
    Since $y \in X$ and $z \notin X$ were arbitrary, this show that $X$ is a maximal simply ordered set so that $X \in \cB$.
    This shows that $\cA \ss \cB$ since $X$ was arbitrary.

    Now suppose that $X \in \cB$ so that $X$ is a maximal simply ordered set.
    It follows from Lemma~\ref{lem:maximum:nonempty} that $X$ is nonempty so that there is an $x \in X$, and we claim that in fact $X = A_x$.
    So consider any $y \in X$.
    Clearly if $y = x$ then $x-y = x-x = 0 \in \rats$ so that $y \in A_x$.
    If $y \neq x$ than either $x \prec y$ or $y \prec x$ since $X$ is simply ordered by $\prec$.
    In the former case we have that $y-x$ is positive and rational so that $x-y = -(y-x)$ is negative and rational, and hence $y \in A_x$.
    In the latter case we have that $x-y$ is positive and rational so that clearly again $y \in A_x$.
    Since $y$ was arbitrary this shows that $X \ss A_x$.
    Now consider any $y \in A_x$ so that $x-y \in \rats$.
    If $y = x$ then clearly $y \in X$.
    If $y \neq x$ then either $y-x$ or $x-y$ is positive, and also clearly rational since $x-y$ is rational.
    Hence either $x \prec y$ or $y \prec x$.
    It then follows from the fact that $X$ is \emph{maximally} simply ordered that $y$ must be in $X$ since otherwise $y$ would not be comparable with $x$.
    Since again $y$ was arbitrary this shows that $A_x \ss X$.
    Hence $X = A_x$ so that clearly $X \in \braces{A_x}_{x \in \reals} = \cA$.
    Since $X$ was arbitrary this shows that $\cB \ss \cA$.

    Therefore we have shown that $\cA = \cB$, which shows that $\cA$ is exactly the complete set of maximally simply ordered subsets.
  }

  As an example of a particular maximally well-ordered set we have $\rats = A_0$ itself.
}

\exercise{2}{
  \eparts{
  \item Let $\prec$ be a strict partial order on the set $A$.
    Define a relation on $A$ by letting a $\eprec b$ if either $a \prec b$ or $a = b$.
    Show that this relation has the following properties, which are called the \boldit{partial order axioms}:
    \begin{enumerate}[label=(\roman*)]
    \item $a \eprec a$ for all $a \in A$.
    \item $a \eprec b$ and $b \eprec a$ $\imp$ $a = b$.
    \item $a \eprec b$ and $b \eprec c$ $\imp$ $a \eprec c$.
    \end{enumerate}
  \item Let $P$ be a relation on $A$ that satisfies properties (i)-(iii).
    Define a relation $S$ on $A$ by letting $aSb$ if $aPb$ and $a \neq b$.
    Show that $S$ is a strict partial order on $A$.
  }
}
\sol{
  (a)
  \qproof{
    We show that $\eprec$ satisfies the three partial order axioms:
    
    (i) Consider any $a \in A$.
    Since obviously $a=a$ we have by definition that $a \eprec a$.

    (ii) Suppose that $a \eprec b$ and $b \eprec a$.
    Then either $a \prec b$ or $a  = b$, and either $b \prec a$ or $b = a$.
    So suppose that $a \neq b$ so that it must be that $a \prec b$ and $b \prec a$.
    Since $\prec$ is a strict partial order, it is transitive so that $a \prec a$ since $a \prec b$ and $b \prec a$.
    But this contradicts the nonreflexivity of $\prec$.
    Hence it must be that $a = b$ as desired.

    (iii) Suppose that $a \eprec b$ and $b \eprec c$.
    Hence either $a \prec b$ or $a = b$, and either $b \prec c$ or $b = c$.

    Case: $a \prec b$.
    If $b \prec c$ then clearly $a \prec c$ since $\prec$ is transitive (since it is a strict partial order).
    If $b = c$ then we have that $a \prec b = c$.

    Case: $a = b$.
    If $b \prec c$ then we have that $a = b \prec c$.
    If $b = c$ then we have that $a = b = c$.

    Hence in all cases and sub-cases we have that $a \prec c$ or $a = c$, and thus $a \eprec c$ by definition.
  }

  (b)
  \qproof{
    We show that $S$ satisfies the two strict partial order axioms:

    \emph{Nonreflexivity.}
    Consider any $a \in A$.
    Since $a = a$ it follows that it is not true that $a \neq a$ and hence not true that $aSa$.
    Thus $S$ is nonreflexive since $a$ was arbitrary.

    \emph{Transitivity.}
    Suppose that $aSb$ and $bSc$.
    Hence by definition $aPb$ and $a \neq b$, and $bPc$ and $b \neq c$.
    Then, by the transitivity property of the partial order axioms, which is property (iii), we have that $aPc$.
    Suppose for a moment that $a = c$.
    Then we would have $aPb$ and $bPa$ (since $bPc$ and $c=a$).
    Then by partial order axiom (ii) we have that $a=b$, which contradicts the fact that $a \neq b$.
    So it must be that $a \neq c$.
    Thus $aPc$ and $a \neq c$ so that $aSc$, which shows that $S$ is transitive.
  }
}

\exercise{3}{
  Let $A$ be a set with a strict partial order $\prec$; let $x \in A$.
  Suppose that we wish to find a maximal simply ordered subset $B$ of $A$ that contains $x$.
  One plausible way of attempting to define $B$ is to let $B$ equal the set of all those elements of $A$ that a \emph{comparable} with $x$:
  \gath{
    B = \braces{y \where y \in A \text{ and either } x \prec y \text{ or } y \prec x}\,.
  }
  But this will not always work.
  In which of Examples 1 and 2 will this procedure succeed and in which will it not?
}
\sol{
  First, it seems that, as defined above, $B$ does not actually contain $x$ itself!
  This is because it is not true that $x \prec x$ by the nonreflexivity of the partial order $\prec$.
  We assume that this was an oversight, which is easily remedied by defining
  \gath{
    B' = \braces{y \in A \where \text{either } x \prec y \text{ or } y \prec x}
  }
  and $B = B' \cup \braces{x}$.

  For Example~1, a circular region in $\reals^2$ is clearly
  \gath{
    C_{\vx_0,r} = \braces{\vx \in \reals^2 \where \abs{\vx - \vx_0} < r} \,,
  }
  where the point $\vx_0 \in \reals^2$ is the center of the circle, $r \in \preals$ is the radius, and $\abs{(x,y)} = \sqrt{x^2 + y^2}$ is the standard vector magnitude.
  Then the collection  $\cA$ is the set of all circular regions:
  \gath{
    \cA = \braces{C_{\vx_0,r} \where \vx_0 \in \reals^2 \text{ and } r \in \preals} \,.
  }
  Then let $\cC = \braces{C_{(0,0),r} \where r \in \preals}$ be the set of circles centered at the origin, which is a maximal simply ordered subset according to the example (and this is not difficult to show).
  Arbitrarily choose $X = C_{(0,0),1}$, that is the circular region of radius 1 centered at the origin, so that clearly $X \in \cC$.
  Since the partial order in this is example is ``is a proper subset of'', define
  \gath{
    \cB' = \braces{Y \in \cA \where Y \pss X \text{ or } X \pss Y}
  }
  and $\cB = \cB' \cup \braces{X}$.
  The question is then whether $\cB = \cC$.
  We claim that, for this example, this is not the case.
  \qproof{
    Consider the set $C_{(1,0),2}$ and any $\vx \in X = C_{(0,0),1}$ so that $\abs{\vx - (0,0)} < 1$.
    Then we have
    \gath{
      \abs{\vx - (1,0)} \leq \abs{\vx - (0,0)} + \abs{(0,0) - (1,0)} < 1 + \abs{(-1,0)} = 1 + 1 = 2\,,
    }
    where we have utilized the ever-useful triangle inequality.
    Therefore $\vx \in C_{(1,0),2}$ so that $X \ss C_{(1,0),2}$ since $\vx$ was arbitrary.
    However, clearly the point $(1,0) \in C_{(1,0),2}$ but we have that $(1,0) \notin C_{(0,0),1} = X$ since $\abs{(1,0) - (0,0)} = \abs{(1,0)} = 1 \geq 1$.
    This shows that $X \pss C_{(1,0),2}$ so that by definition $C_{(1,0),2} \in \cB'$ and therefore $C_{(1,0),2} \in \cB = \cB' \cup \braces{X}$.
    But clearly $C_{(1,0),2} \notin \cC$ since it is not centered at the origin.
    This shows that $\cB \neq \cC$ as desired.
  }
  Hence it would seem that this method of attempting to define a maximal simply ordered subset containing $X$ has failed in this example.
  It is easy to come up with an analogous counterexample that shows the same result of the other example of a maximal simply ordered subset of circles tangent to the y-axis at the origin.

  Regarding Example~2, recall that the order $\prec$ is defined by
  \gath{
    (x_0, y_0) \prec (x_1, y_1)
  }
  if $y_0 = y_1$ and $x_0 < x_1$ for $(x_0, y_0)$ and $(x_1, y_1)$ in $\reals^2$.
  It is then claimed (which is again easy to show) that maximal simply ordered subsets are horizontal lines in the plane, that is sets
  \gath{
    L_{y_0} = \braces{(x,y) \in \reals^2 \where y = y_0}
  }
  for some $y_0 \in \reals$.
  So consider any such $y_0 \in \reals$ and let $\vx = (0, y_0)$.
  Now define
  \gath{
    B' = \braces{\vy \in \reals^2 \where \vx \prec \vy \text{ or } \vy \prec \vx}
  }
  and $B = B' \cup \braces{\vx}$.
  In contrast to Example~1, we here claim that $B = L_{y_0}$, which is to say that $B$ \emph{does} define the maximal simply ordered subset.
  \qproof{
    Consider any $(x,y) \in B = B' \cup \braces{\vx}$ so that either $(x,y) \in B'$ or $(x,y) = \vx$.
    Clearly if $(x,y) = \vx = (0, y_0)$ then $(x,y) \in L_{y_0}$ since $y = y_0$.
    On the other hand, if $(x,y) \in B'$ then $(x,y) \prec \vx$ or $\vx \prec (x,y)$.
    In the former case we have that $(x,y) \prec \vx = (0, y_0)$ so that, by definition $y = y_0$ and $x < 0$.
    Clearly then $(x,y) = (x,y_0) \in L_{y_0}$ by definition.
    In the latter case we also have $y = y_0$ (though this time $0 < x$) so that again $(x,y) \in L_{y_0}$.
    Since $(x,y)$ was arbitrary, this shows that $B \ss L_{y_0}$.

    Now consider any $(x,y) \in L_{y_0}$ so that $y = y_0$.
    If $x = 0$ then $(x,y) = (0, y_0) = \vx$ so that obviously $(x,y) \in \braces{\vx}$.
    If $0 < x$ then $(x,y) = (x, y_0) \prec (0, y_0) = \vx$ so that $(x,y) \in B'$.
    Similarly, if $x < 0$, then $\vx = (0, y_0) \prec (x, y_0) = (x,y)$ so that again $(x,y) \in B'$.
    Hence in all cases either $(x,y) \in B'$ or $(x,y) \in \braces{\vx}$ so that $(x,y) \in B' \cup \braces{\vx} = B$.
    This shows that $L_{y_0} \ss B$ since again $(x,y)$ was arbitrary.

    Thus we have shown that $B = L_{y_0}$ as desired.
  }
  So it would seem that, in this example, this naive technique does work!
}

\exercise{4}{
  Given two points $(x_0, y_0)$ and $(x_1, y_1)$ of $\reals^2$, define
  \gath{
    (x_0, y_0) \prec (x_1, y_1)
  }
  if $x_0 < x_1$ and $y_0 \leq y_1$.
  Show that the curves $y = x^3$ and $y=2$ are maximal simply ordered subsets of $\reals^2$, and the curve $y = x^2$ is not.
  Find all maximal simply ordered subsets.
}
\sol{
  First define
  \gath{
    A = \braces{(x,y) \in \reals^2 \where y = x^3} \,.
  }
  We show that it is a maximal simply ordered subset of $\reals^2$.
  \qproof{
    First we show that $A$ is simply ordered by $\prec$.
    Consider distinct $(x_0, y_0)$ and $(x_1, y_1)$ in $A$ so that $y_0 = x_0^3$ and $y_1 = x_1^3$.
    Since they are distinct, it has to be that $x_0 \neq x_1$ or $y_0 \neq y_1$.
    The latter case actually implies the former since the function $f(x) = x^3$ is a well-defined function.
    Hence we can assume that $x_0 \neq x_1$, from which we can also assume without loss of generality that $x_0 < x_1$.
    Since $f(x) = x^3$ is also a monotonically increasing function (which is easy to show), it then follows that $y_0 = x_0^3 < x_1^3 = y_1$.
    Thus we have that $x_0 < x_1$ and $y_0 \leq y_1$ so that $(x_0, y_0) \prec (x_1, y_1)$ by definition.
    Since $(x_0, y_0)$ and $(x_1, y_1)$ were arbitrary, this shows that $\prec$ is a simple order on $A$.

    To show that it is maximal suppose that $B$ is any proper superset of $A$ so that there is an $(x,y) \in B$ where $(x,y) \notin A$.
    Therefore clearly $y \neq x^3$ by definition.
    Now let $z = x^3$ so that $y \neq x^3 = z$ but $(x,z) \in A$.
    Clearly it is not true that $x < x$ so that it can neither be that $(x,y) \prec (x,z)$ nor $(x,z) \prec (z,y)$.
    Hence $(x,y)$ and $(x,z)$ are incomparable in $\prec$.
    This shows that $B$ is not simply ordered and thus that $A$ is maximal since $B$ was an arbitrary superset.
  }

  Now redefine
  \gath{
    A = \braces{(x,y) \in \reals^2 \where y = 2} \,,
  }
  which we also show is a maximal simply ordered subset of $\reals^2$.
  \qproof{
    To show that $A$ is simply ordered consider distinct $(x_0, y_0)$ and $(x_1, y_1)$ in $A$ so that $y_0 = y_1 = 2$.
    Since these points are distinct and $y_0 = y_2$ is must be that $x_0 \neq x_1$, from which we can assume that $x_0 < x_1$ without loss of generality.
    But then clearly it is true that $x_0 < x_1$ and $y_0 \leq y_1$ so that $(x_0, y_0) \prec (x_1, y_1)$.
    Since these points were arbitrary this shows that $A$ is simply ordered by $\prec$.

    To show that it is maximal suppose that $B$ is any proper superset of $A$ so that there is an $(x,y) \in B$ where $(x,y) \notin A$.
    It then follows that $y \neq 2$ so that the point $(x, 2) \in A$ but $(x,2) \neq (x,y)$.
    Clearly it can be that neither $(x,2) \prec (x,y)$ nor $(x,y) \prec (x,2)$ since it is not true that $x < x$.
    Hence $(x,y)$ and $(x,2)$ are incomparable in $\prec$.
    This shows that $B$ is not simply ordered by $\prec$.
    Since $B$ was an arbitrary superset this shows that $A$ is maximal.
  }

  Now let
  \gath{
    A = \braces{(x,y) \in \reals^2 \where y = x^2}\,.
  }
  We claim that this subset is not simply ordered by $\prec$ and therefore cannot be a maximal simply ordered subset.
  \qproof{
    Consider the clearly distinct points $(-1,1)$ and $(0,0)$.
    Clearly since $0 = 0^2$ and $1 = (-1)^2$ these are both in $A$.
    However, since $1 > 0$ it is not true that $-1 < 0$ and $1 \leq 0$, and therefore it is not true that $(-1,1) \prec (0,0)$.
    Similarly since $0 \geq -1$ it is not true that $0 < -1$ and $0 \leq 1$, and therefore it is not true that $(0,0) \prec (-1,1)$.
    Hence the two distinct points are both in $A$ but are not comparable.
    This suffices to show that $A$ is not simply ordered by $\prec$.
  }

  We now claim that the maximal simply ordered subsets of $\reals^2$ as ordered by $\prec$ are exactly the collection of sets of the form
  \gath{
    A_f = \braces{(x,y) \in \reals^2 \where y = f(x)}
  }
  for some function $f: (a,b) \to \reals$, where $(a,b)$ is an open interval of $\reals$, noting that it could be that $a = -\infty$ and/or $b = \infty$.
  The function $f$ must also satisfy the following properties:
  \begin{enumerate}[label=(\roman*)]
  \item It is non-decreasing.
    Recall that this means that $x < y$ implies that $f(x) \leq f(y)$ for any $x,y \in (a,b)$.
  \item If $b < \infty$ then its image is unbounded above.
  \item If $a > -\infty$ then its image is unbounded below.
  \end{enumerate}
  Now, let $\cA$ be the collection of all these subsets and let $\cB$ denote the set of all maximal simply ordered subsets.
  We show that $\cA = \cB$.
  \qproof{
    $(\ss)$ First consider any $A_f \in \cA$ so that $f : (a,b) \to \reals$ with the properties above for some open interval $(a,b)$.
    To show that $A_f$ is simply ordered by $\prec$ consider any distinct $(x,y)$ and $(x',y')$ in $A_f$ so that $y = f(x)$ and $y' = f(x')$.
    Since these are distinct it follows that $x \neq x'$ or $f(x) = y \neq y' = f(x')$.
    In the latter case it also follows that $x \neq x'$ as well for otherwise $f$ would not be a function.
    Hence we can, without loss of generality, assume that $x < x'$.
    Since $f$ is non-decreasing it follows that also $y = f(x) \leq f(x') = y'$, and therefore by definition $(x,y) \prec (x',y')$.
    Since these elements of $A_f$ were arbitrary, it follows that $A_f$ is simply ordered by $\prec$.

    To show that $A_f$ is maximal consider any proper superset $A$ of $A_f$ so that there is an $(x,y) \in A$ where $(x,y) \notin A_f$.
    There are a few possible ways in which $(x,y)$ can fail to be an element of $A_f$.

    Case: $x \in (a,b)$.
    Then it must be that $y \neq f(x)$ since $(x,y) \notin A_f$.
    Since it is not true that $x < x$, it has to be that neither $(x,y) \prec (x,f(x))$ nor $(x,f(x)), (x,y)$.
    Hence $(x,y)$ and $(x,f(x))$ are incomparable elements of $A$ (noting that clearly $(x,(f(x)) \in A_f \ss A$) so that $A$ is not simply ordered by $\prec$.

    Case: $x \geq b$.
    Note that this is only possible if $b < \infty$ so that $b \in \reals$.
    Thus in this case we have that the image of $f$ is unbounded above by property (ii).
    Hence there is a $y_u \in reals$ where $y' > y$ and $y'$ is in the image of $f$.
    Thus there is also an $x' \in (a,b)$ where $y = f(x')$ so that $(x',y') \in A_f \ss A$.
    Now, we have $x' < b \leq x$ but $y' > y$ so that it is not true that $y' \leq y$, and hence it cannot be that $(x',y') \prec (x,y)$.
    Similarly is it is clearly not true that $x < x'$ so that it cannot be that $(x,y) \prec (x',y')$ either.
    This shows that $(x,y)$ and $(x',y')$ are incomparable elements of $A$ so that $A$ is not simply ordered.

    Case: $x \leq a$.
    An argument analogous to the previous case shows that $a > -\infty$ so that the image of $f$ is unbounded below.
    From this it follows again that $A$ is not simply ordered.

    Thus in all cases $A$ is not simply ordered so that $A_f$ is a maximal simply ordered subset of $\reals^2$ since $A$ was an arbitrary proper superset.
    This shows that $A_f \in \cB$ so that $\cA \ss \cB$ since $A_f$ was arbitrary.

    $(\sps)$ Now consider any $B \in \cB$ so that $B$ is a maximal simply ordered set by $\prec$.
    Define
    \gath{
      X = \braces{x \in \reals \where (x,y) \in B \text{ for some } y \in \reals} \,.
    }
    We prove that $B$ has the following properties:
    \lparts{
    \item If $(x_0,y_0)$ and $(x_1,y_1)$ are in $B$ and $x_0 < x_1$ then $y_0 \leq y_1$.
    \item For every $x \in X$ there is a unique $y \in \reals$ where $(x,y) \in B$.
    }
    To show show (1) consider $(x_0,y_0)$ and $(x_1,y_1)$ in $B$ and suppose that $x_0 < x_1$.
    Since $B$ is simply ordered, it must be that either $(x_0,y_0) \prec (x_1,y_1)$ or $(x_1,y_1) \prec (x_0,y_0)$.
    Since $x_0 < x_1$ it clearly must be that $(x_0,y_0) \prec (x_1,y_1)$ and hence also $y_0 \leq y_1$.

    To show (2) consider any $x \in X$.
    Clearly there is a $y \in \reals$ where $(x,y) \in B$ by the definition of $X$.
    To show that this $y$ is unique, suppose that $(x,y_0)$ and $(x,y_1)$ are both in $B$ but that $y_0 \neq y_1$ so that $(x,y_0)$ and $(x,y_1)$ are distinct.
    Since $B$ is simply ordered they must be comparable in $\prec$ but they clearly cannot be since it is not true that $x < x$.
    As this is a contradiction, it must be that $y_0 = y_1$.

    With that out of the way, let $b$ be the least upper bound of $X$ if it is bounded above and $b = \infty$ otherwise.
    Similarly let $a$ be the greatest lower bound if $X$ is bounded below and $a = -\infty$ otherwise.
    Now we claim that $X$ is equal to the open interval $(a,b)$.

    So consider any $x \in X$ so that then clearly $a \leq x \leq b$ since $a$ and $b$ are lower and upper bounds of $X$, respectively.
    Clearly if $b = \infty$ then it cannot be that $x = b$ (since $x \in \reals$) so assume that $b \in \reals$ and $x = b$.
    Then $b = x \in X$ so that by property (2) there is a unique $y \in \reals$ where $(b,y) \in B$.
    Clearly then $(b+1,y) \notin B$, since $b+1 \notin X$,  so that the set $B' = B \cup \braces{(b+1,y)}$ is a proper superset of $B$.
    Now consider any $(x',y') \in B$ so that clearly $x' \in X$ and hence $x' \leq b < b+1$.
    By property (1) above it also follows that $y' \leq y$, and so we have that $(x',y') \prec (b+1, y)$.
    Since $(x',y')$ was arbitrary, this shows that $(b+1,y)$ is comparable to every element of $B$ and hence $B'$ is simply ordered by $\prec$.
    But this is not possible since $B$ is maximal and $B'$ is a proper superset.
    Hence it must be that $x \neq b$.
    An analogous argument shows that $x \neq a$ as well and hence $a < x < b$.
    Since $x$ was arbitrary this shows that $X \ss (a,b)$.

    Now consider any $x \in (a,b)$ so that $a < x < b$.
    Since $b$ is the least upper bound of $X$, it has to be that $x$ is not an upper of $X$ so that there is an $x_g \in X$ where $x < x_g < b$ (clearly the existence of $x_g$ also follows when $b = \infty$ since then $X$ is unbounded above).
    Clearly then there is also a $y_g \in \reals$ where $(x_g,y_g) \in B$.
    It then follows that the set $Y_g = \braces{y \in \reals \where (z,y) \in B \text{ for some } x < z < b}$ is nonempty.
    By an analogous argument there is an $(x_l,y_l) \in B$ where $a < x_l < x$ so that the set $Y_l = \braces{y \in \reals \where (z,y) \in B \text{ for some } a < z < x}$ is nonempty.
    Now, for any $y \in Y_g$, we have that $(z, y) \in B$ for some $x < z < b$.
    Therefore $x_l < x < z$ and by property (1) of $B$ we have that $y_l \leq y$.
    Since $y$ was arbitrary this shows that $y_l$ is a lower bound of $Y_g$ and hence it has a greatest lower bound $y_v$.
    So suppose that $x \notin X$ so that there is not a $y \in \reals$ where $(x,y) \in B$.
    Then we have that the set $B \cup \braces{(x,y_v)}$ is a proper superset of $B$.
    However, consider any $(x',y') \in B$ so that $x' \in X$ but $x' \neq x$.

    Case: $x' < x$.
    Then it has to be that $a < x' < x$ so that $y' \in Y_l$.
    Then, for any $y \in Y_g$ we again have that $(z,y) \in B$ for some $x < z < b$.
    Hence $x' < x < z$ so that $y' \leq y$ by property (1) since $(x',y') \in B$ and $(z,y) \in B$.
    Since $y$ was arbitrary, this shows that $y'$ is a lower bound of $Y_g$.
    Since $y_v$ is the greatest lower bound of $Y_g$, we have that $y' \leq y_v$.
    Then clearly $(x',y') \prec (x, y_v)$ since also $x' < x$.

    Case: $x' > x$.
    Then it has to be that $x < x' < b$ so that $y' \in Y_g$.
    It then follows that $y_v \leq y'$ since $y_v$ is the greatest lower bound of $Y_g$.
    Hence we have that $(x,y_v) \prec (x',y')$  since $x < x'$ as well.

    Therefore in all cases we have that $(x,y_v)$ and $(x',y')$ are comparable in $\prec$.
    Since $(x',y')$ was arbitrary, this clearly shows that $B \cup \braces{(x,y_v)}$ is simply ordered.
    But this cannot be possible since it is a proper superset and $B$ is maximal!
    So it has to be that in fact there \emph{is} a $y \in \reals$ where $(x,y) \in B$, and hence $x \in X$.
    Since $x \in (a,b)$ was arbitrary, this shows that $(a,b) \ss X$.
    This completes the rather long proof that $X = (a,b)$.

    Now, by property (2) there is a unique $y \in \reals$ for every $x \in X = (a,b)$ where $(x,y) \in B$.
    So we define a function $f: (a,b) \to \reals$ by simply setting $f(x) = y$.
    Clearly based on the way this function is defined and the fact that $(a,b) = X$ we have that $B = A_f$.
    We must now show that $f$ has the properties (i) through (iii) above.

    Property (i) follows almost immediately from property (1) of $B$.
    To see this, consider any $x,y \in (a,b)$ where $x < y$.
    Then $(x,f(x))$ and $(y,f(y))$ are in $B$ and hence $f(x) \leq f(y)$ by property (1).
    For property (ii) suppose that $b < \infty$ but that the image of $f$ is bounded above.
    Hence it image has an upper bound, say $y_u \in \reals$, so that clearly $B \cup \braces{(b+1,y_u)}$ is a proper superset of $B$.
    So consider any $(x,y) \in B$ so that $y = f(x)$ for some $x \in (a,b)$.
    Then clearly $f(x)$ is in the image of $f$ so that $y = f(x) \leq y_u$ since $y_u$ is an upper bound of the image.
    Since also we must have $x < b < b+1$, it follows that $(x,y) \prec (b+1, y_u)$.
    Since $(x,y) \in B$ was arbitrary, this shows that $B \cup \braces{(b+1,y_u)}$ is simply ordered, which cannot be possible since it is a proper superset and $B$ is maximal.
    So it has to be that in fact the image of $f$ is unbounded above when $b < \infty$, which shows property (ii).
    An analogous argument shows property (iii).

    Since $f$ has all of the required properties and $B = A_f$, this shows that $B \in \cA$.
    Clearly then $\cB \ss \cA$ since $B$ was arbitrary.
    This shows that $\cA = \cB$ as desired.
  }

  Lastly, note that the example curves $y = x^3$ and $y=2$ are clearly in $\cA = \cB$ since they are non-decreasing functions on $\reals$, ($\reals$ being the same as the open interval $(-\infty, \infty)$), while the curve $y = x^2$ is not since it is decreasing when $x < 0$.
}

\exercise{5}{
  Show that Zorn's Lemma implies the following:
  
  \emph{Lemma (Kuratowski).} Let $\cA$ be a collection of sets.
  Suppose that for every subcollection $\cB$ of $\cA$ that is simply ordered by proper inclusion, the union of the elements of $\cB$ belongs to $\cA$.
  Then $\cA$ has an element that is properly contained in no other element of $\cA$.
}

\sol{
  \qproof{
    First, we know that $\pss$ is a strict partial order on $\cA$, which is trivial to show.
    So consider any simply ordered subset $\cB$ of $\cA$ and let $A = \bigcup \cB$ so that we know that $A \in \cA$.
    Clearly for any set $B \in \cB$ we have that $B \ss \bigcup \cB = A$ so that, since $B$ was arbitrary, $A$ is an upper bound of $\cB$ in the strict partial order $\pss$.
    Since $\cB$ was arbitrary, this shows the hypothesis of Zorn's Lemma so that $\cA$ has a maximal element $A$.
    Then clearly $A$ is not properly contained in any other element of $\cA$.
  }
}

\exercise{6}{
  A collection $\cA$ of subsets of a set $X$ is said to be of \emph{finite type} provided that a subset $B$ of $X$ belongs to $\cA$ if and only if every finite subset of $B$ belongs to $\cA$.
  Show that the Kuratowski lemma implies the following:

  \emph{Lemma (Tukey, 1940).} Let $\cA$ be a collection of sets.
  If $\cA$ is of finite type, then $\cA$ has an element properly contained in no other element of $\cA$.
}
\sol{
  \qproof{
    Suppose that $\cA$ is a collection of sets of finite type.
    Let $\cB$ be a subcollection of $\cA$ that is simply ordered by $\pss$.
    Consider next any finite subset $B$ of $\bigcup \cB$.
    Then, for every $b \in B$, $b \in \bigcup \cB$ so that we can choose a set $B_b \in \cB$ such that $b \in B_b$.
    Note that this does not require the choice axiom since we need to make only a finite number of choices.
    Then the set $\cB' = \braces{B_b \where b \in B}$ is clearly a finite set of elements of $\cB$.
    Since $\cB$ is simply ordered by $\pss$, it follows that $\cB'$ is as well and so has a largest element $C$ since it is finite.

    Hence, for any $b \in B$, we have that $b \in B_b \ss C$ so that $b \in C$, and so $B$ is a finite subset of $C$.
    Since $C \in \cB$ and $\cB \ss \cA$, clearly $C \in \cA$.
    Since $\cA$ is of finite type and $B$ is a finite subset of $C$, it follows that $B \in \cA$ also.
    Since $B$ was an arbitrary finite subset of $\bigcup \cB$, it then follows that $\bigcup \cB$ is also in $\cA$ since it is of finite type.
    It then follows from the Kuratowski lemma (Exercise~11.5) that $\cA$ has an element that is properly contained in no other element of $\cA$ as desired.
  }
}

\exercise{7}{
  Show that the Tukey lemma implies the Hausdorff maximum principle.
  [Hint: If $\prec$ is a strict partial order on $A$, let $\cA$ be the collection of all subsets of $A$ that are simply ordered by $\prec$.
    Show that $\cA$ is of finite type.]
}
\sol{
  \qproof{
    Following the hint, suppose that the set $A$ has strict partial order $\prec$ and let $\cA$ be the collection of all subsets of $A$ that are simply ordered by $\prec$.
    We show that $\cA$ has finite type, i.e. that a subset $B \ss A$ is in $\ca$ if and only if every finite subset of $B$ is.

    $(\imp)$ Suppose that $B \ss A$ is in $\cA$ so that it is simply ordered by $\prec$.
    Clearly any finite subset of $B$ is also simply ordered by $\prec$ so that it is also in $\cA$, which shows the result.

    $(\pmi)$ Now suppose that $B \ss A$ and that every finite subset of $B$ is in $\cA$.
    Now consider two distinct element $x$ and $y$ of $B$.
    Clearly then the set $\braces{x,y}$ is a finite subset of $B$ and hence is in $\cA$.
    Then this means that $\braces{x,y}$ is simply ordered by $\prec$ so that clearly $x$ and $y$ are comparable.
    Since $x$ and $y$ were arbitrary this shows that $B$ is simply ordered by $\prec$ and hence $B \in \cA$.

    We have thus shown that $\cA$ is of finite type so that it has a set $C$ such that is properly contained in no other element of $\cA$.
    Since $C \in \cA$, it is simply ordered by $\prec$.
    It is also maximal since, if $D$ is any proper superset of $C$ then it cannot be that $D$ is simply ordered for then we would have $D \in \cA$ and $C \pss D$, which would contradict the definition of $C$.
    Hence $C$ is the maximal simply ordered subset of $A$ that shows the maximum principle.
  }
}

\exercise{8}{
  A typical use of Zorn's lemma in algebra is the proof that every vector space has a basis.
  Recall that if $A$ is a subset of the vector space $V$, we say a vector belongs to that \emph{span} of $A$ if it equals a finite linear combination of elements of $A$.
  The set $A$ is \emph{independent} if the only finite linear combination of elements of $A$ that equals the zero vector is the trivial one having all coefficients zero.
  If $A$ is independent and if every vector in $V$ belongs to the span of $A$, then $A$ is a \emph{basis} for $V$.
  \eparts{
  \item If $A$ is independent and $v \in V$ does not belong to the span of $A$, show $A \cup \braces{v}$ is independent.
  \item Show the collection of all independent sets in $V$ has a maximal element.
  \item Show that $V$ has a basis.
  }
}
\sol{
  (a)
  \qproof{
    We show this by contradiction.
    Suppose that $A$ is independent and $v \in V$ does not belong to the span of $A$.
    Also let $B = A \cup \braces{v}$ and suppose that $B$ is \emph{not} independent.
    Then
    \gath{
      \sum_{i=1}^n \b_i b_i = 0
    }
    for some nonzero coefficients $\b_i$, where each $b_i$ is in $B$.
    Now, it must be that one of the $b_i$ vectors is $v$ and the rest in $A$ since otherwise they would all be in $A$ and then $A$ would not be independent.
    Hence this can be expressed as
    \gath{
      \sum_{i=1}^{n-1} \a_i a_i + \g v = 0
    }
    for nonzero coefficients $\a_i$ and $\g$ and vectors $a_i \in A$.
    However clearly then we would have
    \gath{
      v = -\frac{1}{\g} \sum_{i=1}^{n-1} \a_1 a_i = \sum_{i=1}^{n-1} \parens{\frac{-\a_i}{\g}} a_i
    }
    so that $v$ is a linear combination of vectors in $A$ and hence is in the span of $A$.
    This is a contradiction so that it must be that in fact $B = A \cup \braces{v}$ is independent as desired.
  }

  (b)
  \qproof{
    Let $\cA$ be the collection of all independent sets in $V$.
    We know that $\pss$ is a strict partial order on $\cA$.
    Now let $\cB$ be any subset of $\cA$ that is simply ordered by $\pss$.
    We claim that $\bigcup \cB$ is an upper bound of $\cB$ that is in $\cA$.
    So first consider any $B \in \cB$ and any $b \in B$ so that clearly then $b \in \bigcup \cB$.
    Hence $B \ss \bigcup \cB$ since $b$ was arbitrary.
    Since $B \in \cB$ was arbitrary, this shows that $\bigcup \cB$ is an upper bound of $\cB$ by $\pss$.

    Next we show that $\bigcup \cB$ is also in $\cA$.
    To this end consider any finite set $B$ of elements of $\bigcup \cB$ so that $B$ is a set of vectors in $V$.
    Now, for each $b \in B$ we have that $b \in \bigcup \cB$ so that we can choose any set $B_b \in \cB$ where $b \in B_b$.
    Note that this does not require the axiom of choice since $B$ is finite.
    Then, since each $B_b$ is in $\cB$, which is simply ordered by $\pss$ and $\braces{B_b \where b \in B}$ is finite, it follows that it has a largest element $C$ so that $B_b \ss C$ for any $b \in B$.
    Hence $B \ss C$ since each $b \in B_b$ and $B_b \ss C$.
    Also $C \in \cA$ since $C \in \cB$ and $\cB \ss \cA$ so that $C$ is independent.
    Hence the only linear combination of the vectors in $B$ that is the zero vector must have all zero coefficients since they are all in the independent set $C$.
    Since $B$ was an arbitrary set of vectors in $\bigcup \cB$, this shows that $\bigcup \cB$ is independent and therefore in $\cA$.

    Since $\cB$ was an arbitrary simply ordered subset of $\cA$, it follows that every such subset has an upper bound in $\cA$.
    Thus by Zorn's Lemma $\cA$ has a maximal element as desired.
  }

  (c)
  \qproof{
    Again let $\cA$ be the collection of all independent sets in $V$, which we know has a maximal element $A$ from part~(b).
    We claim that $A$ is a basis for $V$.
    Suppose to the contrary that it is not so that, since we know that $A$ is independent (since it is in $\cA$), there must be a vector $v \in V$ that is not in the span of $A$.
    Then by part~(a) we have that $A \cup \braces{v}$ is also independent and so in $\cA$.
    We also have that $v \notin A$ since otherwise it would clearly be in the span of $A$.
    Hence $A \pss A \cup \braces{v}$.
    However, this contradicts the fact that $A$ is a maximal element of $\cA$, so that it must be that in fact $A$ is a basis for $V$ as desired.
  }
}
