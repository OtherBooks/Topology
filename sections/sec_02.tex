\setcounter{subsection}{2-1}
\subsection{Functions}

% Some useful macros for this section
\def\ivf{\inv{f}}

\exercise{1}{
  Let $f: A \to B$.
  Let $A_0 \ss A$ and $B_0 \ss B$.
  \eparts{
  \item Show that $A_0 \ss \ivf(f(A_0))$ and that equality holds if $f$ is injective.
  \item Show that $f(\ivf(B_0)) \ss B_0$ and that equality holds if $f$ is surjective.
  }
}
\sol{
  \dwhitman

  (a)
  \qproof{
    Consider any $x \in A_0$ and let $y = f(x)$ so that clearly $y \in f(A_0)$.
    Then, since $f(x) = y \in f(A_0)$, it follows from the definition of the preimage that $x \in \ivf(f(A_0))$.
    Hence $A_0 \ss \ivf(f(A_0))$ as desired since $x$ was arbitrary.
    Now suppose that $f$ is also injective and consider this time any $x \in \ivf(f(A_0))$ so that $y = f(x) \in f(A_0)$ by the definition of a preimage.
    Then there is an $x' \in A_0$ where $f(x') = y = f(x)$ by the definition of an image.
    Since $f$ injective though, it must be that $x = x' \in A_0$.
    This shows that $\ivf(f(A_0)) \ss A_0$ since $x$ was arbitrary.
    The desired equality follows since it was already shown that $A_0 \ss \ivf(f(A_0))$ (whether or not $f$ is injective). 
  }

  (b)
  \qproof{
    First suppose that $y$ is any element of $f(\ivf(B_0))$ so that there is an $x \in \ivf(B_0)$ where $f(x) = y$.
    Since $x \in \ivf(B_0)$, we then have that $y = f(x) \in B_0$ by the definition of a preimage.
    Hence $f(\ivf(B_0)) \ss B_0$ since $y$ was arbitrary.
    Now suppose also that $f$ is surjective and suppose that $y \in B_0$ so that also clearly $y \in B$ since $B_0 \ss B$.
    Since $f$ is surjective, there is an $x \in A$ where $f(x) = y$.
    We then have that $x \in \ivf(B_0)$ since $f(x) = y \in B_0$.
    Clearly then $y = f(x) \in f(\ivf(B_0))$ so that $B_0 \ss f(\ivf(B_0))$ since $y$ was arbitrary.
    This shows equality as desired.
  }
}

\exercise{2}{
  Let $f: A \to B$ and let $A_i \ss A$ and $B_i \ss B$ for $i=0$ and $i=1$.
  Show that $\ivf$ preserves inclusions, unions, intersections, and differences of sets:
  \eparts{
  \item $B_0 \ss B_1 \imp \ivf(B_0) \ss \ivf(B_1)$.
  \item $\ivf(B_0 \cup B_1) = \ivf(B_0) \cup \ivf(B_1)$.
  \item $\ivf(B_0 \cap B_1) = \ivf(B_0) \cap \ivf(B_1)$.
  \item $\ivf(B_0 - B_1) = \ivf(B_0) - \ivf(B_1)$.
  }
  Show that $f$ preserves inclusions and unions only:
  \eparts[4]{
  \item $A_0 \ss A_1 \imp f(A_0) \ss f(A_1)$.
  \item $f(A_0 \cup A_1) = f(A_0) \cup f(A_1)$.
  \item $f(A_0 \cap A_1) \ss f(A_0) \cap f(A_1)$; show that equality holds if $f$ injective.
  \item $f(A_0 - A_1) \sps f(A_0) - f(A_1)$; show that equality holds if $f$ injective.
  }
}
\sol{
  \dwhitman

  (a)
  \qproof{
    Suppose that $B_0 \ss B_1$ and consider any $x \in \ivf(B_0)$.
    Then by the definition of a preimage, we have $f(x) \in B_0$ so that also $f(x) \in B_1$ since $B_0 \ss B_1$.
    This shows that $x \in \ivf(B_1)$ again by the definition of a preimage.
    Thus $\ivf(B_0) \ss \ivf(B_1)$ since $x$ was arbitrary as desired.
  }


  (b)
  \qproof{
    We can show this easily using a string of biconditionals.
    For any $x \in A$ we have
    \ali{
      x \in \ivf(B_0 \cup B_1) &\bic f(x) \in B_0 \cup B_1 \\
      &\bic f(x) \in B_0 \lor f(x) \in B_1 \\
      &\bic x \in \ivf(B_0) \lor x \in \ivf(B_1) \\
      &\bic x \in \ivf(B_0) \cup \ivf(B_1) \,,
    }
    which shows the desired result.
  }

  (c)
  \qproof{
    We can show this in a very similar manner to what was done in part (b).
    We have
    \ali{
      x \in \ivf(B_0 \cap B_1) &\bic f(x) \in B_0 \cap B_1 \\
      &\bic f(x) \in B_0 \land f(x) \in B_1 \\
      &\bic x \in \ivf(B_0) \land x \in \ivf(B_1) \\
      &\bic x \in \ivf(B_0) \cap \ivf(B_1) \,,
    }
    for any $x \in A$.
  }

  (d)
  \qproof{
    This is also shown similarly.
    For $x \in A$ we have
    \ali{
      x \in \ivf(B_0 - B_1) &\bic f(x) \in B_0 - B_1 \\
      &\bic f(x) \in B_0 \land f(x) \notin B_1 \\
      &\bic x \in \ivf(B_0) \land x \notin \ivf(B_1) \\
      &\bic x \in \ivf(B_0) - \ivf(B_1) \,.
    }
  }

  (e)
  \qproof{
    Suppose that $A_0 \ss A_1$ and consider any $y \in f(A_0)$.
    Then there is an $x \in A_0$ where $y = f(x)$ by the definition of an image set.
    Then also $x \in A_1$ since $A_0 \ss A_1$, from which it follows that $y = f(x) \in f(A_1)$.
    Therefore $f(A_0) \ss f(A_1)$ as desired since $y$ was arbitrary.
  }

  (f)
  \qproof{
    We can show this easily using a string of biconditionals.
    For any $x \in A$ we have
    \ali{
      y \in f(A_0 \cup A_1) &\bic \exists x (x \in A_0 \cup A_1 \land y = f(x)) \\
      &\bic \exists x [(x \in A_0 \lor x \in A_1) \land y = f(x)] \\
      &\bic \exists x [(x \in A_0 \land y = f(x)) \lor (x \in A_1 \land y = f(x))] \\
      &\bic \exists x (x \in A_0 \land y = f(x)) \lor \exists x (x \in A_1 \land y = f(x)) \\
      &\bic y \in f(A_0) \lor y \in f(A_1) \\
      &\bic y \in f(A_0) \cup f(A_1) \,,
    }
    which shows the desired result.
  }

  (g)
  \qproof{
    Consider any $y \in f(A_0 \cap A_1)$ so that there is an $x \in A_0 \cap A_1$ where $y = f(x)$.
    Hence of course $x \in A_0$ and $x \in A_1$.
    Since also $y = f(x)$, this suffices to show that $y \in f(A_0)$ and $y \in f(A_1)$, and therefore $y \in f(A_0) \cap f(A_1)$ as desired.

    Now suppose that $f$ is injective and consider any $y \in f(A_0) \cap f(A_1)$.
    Then $y \in f(A_0)$ and $y \in f(A_1)$, from which it follows that there is an $x_0 \in A_0$ where $y = f(x_0)$, and an $x_1 \in A_1$ where $y = f(x_1)$.
    We then have $f(x_0) = y = f(x_1)$ so that $x_0 = x_1$ since $f$ is injective.
    Hence $x_0 \in A_0$ and $x_0 = x_1 \in A_1$, so of course $x_0 \in A_0 \cap A_1$.
    Since also $y = f(x_0)$, this shows by definition that $y \in f(A_0 \cap A_1)$.
    Therefore $f(A_0) \cap f(A_1) \ss f(A_0 \cap A_1)$ since $y$ was arbitrary, which shows the desired equivalence since the other direction was already shown.
  }

  (h)
  \qproof{
    Consider any $y \in f(A_0) - f(A_1)$ so that $y \in f(A_0)$ and $y \notin f(A_1)$.
    Then there is an $x \in A_0$ where $y = f(x)$.
    We also have that there is no $x' \in A_1$ such that $y = f(x')$.
    Since we know that $y = f(x)$ it then has to be that $x \notin A_1$.
    Hence $x \in A_0 - A_1$, so that $y \in f(A_0 - A_1)$ since of course $y = f(x)$.
    This shows that $f(A_0 - A_1) \sps f(A_0) - f(A_1)$ as desired since $y$ was arbitrary.

    Now suppose that $f$ is injective and consider any $y \in f(A_0 - A_1)$.
    Then there is an $x \in A_0 - A_1$ where $y = f(x)$ by the definition of an image set.
    Then $x \in A_0$ but $x \notin A_1$.
    It then follows that $y \in f(A_0)$ since $y = f(x)$ and $x \in A_0$.
    Consider any $x' \in A_1$.
    Then it cannot be that $y = f(x')$, because if this were the case then $f(x) = y = f(x')$ so that $x = x'$ since $f$ is injective.
    But we know that $x' = x \notin A_1$, which would present a contradiction.
    So it must be that there is no $x' \in A_1$ where $y = f(x')$, which suffices to show that $y \notin f(A_1)$.
    Therefore $y \in f(A_0) - f(A_1)$ so that $f(A_0 - A_1) \ss f(A_0) - f(A_1)$ since $y$ was arbitrary.
    This of course shows equivalence as desired.
  }
}
