\setcounter{subsection}{17-1}
\subsection{Closed Sets and Limit Points}

\exercise{1}{
  Let $\cC$ be a collection of subsets of the set $X$.
  Suppose that $\es$ and $X$ are in $C$, and that finite unions and arbitrary intersections of elements of $\cC$ are in $\cC$.
  Show that the collection
  \gath{
    \cT = \braces{X - C \where C \in \cC}
  }
  is a topology on $X$.
}
\sol{
  \qproof{
    First, clearly $\es$ and $X$ are in $\cT$ since $\es = X - X$ and $X = X - \es$ and both $X$ and $\es$ are in $\cC$.
    This shows the first defining property of a topology.

    Now consider an arbitrary sub-collection $\cA$ of $\cT$.
    Then, for each $A \in \cA$, we have that $A = X - B$ for some $B \in \cC$ since also $A \in \cT$.
    So let $\cB = \braces{B \in \cC \where X - B \in \cA}$.
    Then we have that
    \gath{
      \bigcup \cA = \bigcup_{A \in \cA} A = \bigcup_{B \in \cB} (X - B) = X - \bigcap_{B \in \cB} B = X - \bigcap \cB
    }
    by DeMorgan's law.
    By the definition of $\cC$ we have that $\bigcap \cB \in \cC$ since it is an arbitrary intersection of elements of $\cC$.
    It then follows that $\bigcup \cA = X - \bigcap \cB$ is in $\cT$ by definition.
    This shows the second defining property of a topology.

    Lastly, suppose that $\cA$ is a nonempty finite sub-collection of $\cT$, which of course can be expressed as $\cA = \braces{A_k \where k \in \intsfin{n}}$ for some positive integer $n$.
    Then, again we have that that $A_k = X - B_k$ for some $B_k \in \cC$ for all $k \in \intsfin{n}$ since $A_k \in \cT$.
    Then we have
    \gath{
      \bigcap \cA = \bigcap_{k=1}^n A_k = \bigcap_{k=1}^n (X - B_k) = X - \bigcup_{k=1}^n B_k
    }
    by DeMorgan's law.
    Then clearly $\bigcup_{k=1}^n B_k$ is in $\cC$ by definition since it is a finite union of elements of $\cC$.
    It then follows that $\bigcap \cA = X - \bigcup_{k=1}^n B_k$ is in $\cT$ by definition.
    Since $\cA$ was an arbitrary finite sub-collection, this shows the third defining property of a topology.
    Hence $\cT$ is a topology by definition.
  }
}

\exercise{2}{
  Show that if $A$ is closed in $Y$ and $Y$ is closed in $X$, then $A$ is closed in $X$.
}
\sol{
  \qproof{
    Since $A$ is closed in $Y$, it follows from Theorem~17.2 that $A = B \cap Y$ where $B$ is some closed set in $X$.
    Hence by definition $X - B$ is open in $X$.
    Also, since $Y$ is closed in $X$, we have that $X - Y$ is open in $X$ by definition.
    We then have
    \gath{
      X - A = X - (B \cap Y) = (X - B) \cup (X - Y)
    }
    by DeMorgan's law.
    Since both $X - B$ and $X - Y$ are open in $X$, clearly their union must also be open since we are in a topological space.
    Hence $X - A$ is open in $X$ so that $A$ is closed in $X$ by definition.
  }
}

\exercise{3}{
  Show that if $A$ is closed in $X$ and $B$ is closed in $Y$, then $A \times B$ is closed in $X \times Y$.
}
\sol{
  \begin{lem}\label{lem:closedlim:setid}
    If $X$, $Y$, $A$, and $B$ are sets then $X \times Y - A \times B = (X-A) \times Y \cup X \times (Y - B)$.
  \end{lem}
  \qproof{
    We show this via logical equivalences:
    \ali{
      (x,y) \in X \times Y - A \times B &\bic (x,y) \in X \times Y \land (x,y) \notin A \times B \\
      &\bic (x \in X \land y \in Y) \land \lnot(x \in A \land y \in B) \\
      &\bic (x \in X \land y \in Y) \land (x \notin A \lor y \notin B) \\
      &\bic (x \in X \land y \in Y \land x \notin A) \lor (x \in X \land y \in Y \land y \notin B) \\
      &\bic (x \in X - A \land y \in Y) \lor (x \in X \land y \in Y - B) \\
      &\bic (x,y) \in (X-A) \times Y \lor (x,y) \in X \times (Y-B) \\
      &\bic (x,y) \in (X-A) \times Y \cup X \times (Y-B)
    }
    as desired.
  }

  \mainprob
  \qproof{
    Since $A$ is closed we have that $X-A$ is open in $X$.
    Since also $Y$ itself is open in $Y$, we have that $(X-A) \times Y$ is a basis element in the product topology by definition, and is therefore obviously open.
    An analogous argument shows that $X \times (Y-B)$ is also open in the product topology since $B$ is closed in $Y$.
    Hence their union is also open in the product topology, but by Lemma~\ref{lem:closedlim:setid} we have
    \gath{
      (X-A) \times Y \cup X \times (Y-B) = X \times Y - A \times B
    }
    so that $X \times Y - A \times B$ is also open in the product topology.
    It then follows by definition that $A \times B$ is closed as desired.
  }
}

\exercise{4}{
  Show that if $U$ is open in $X$ and $A$ is closed in $X$, then $U-A$ is open in $X$, and $A-U$ is closed in $X$.
}
\sol{
  \begin{lem}\label{lem:closedlim:setmid}
    If $A$, $B$, and $C$ are sets then $A - (B - C) = (A-B) \cup (A \cap C)$.
  \end{lem}
  \qproof{
    We show this by a sequence of logical equivalences:
    \ali{
      x \in A - (B - C) &\bic x \in A \land x \notin B - C \\
      &\bic x \in A \land \lnot(x \in B \land x \notin C) \\
      &\bic x \in A \land (x \notin B \lor x \in C) \\
      &\bic (x \in A \land x \notin B) \lor (x \in A \land x \in C) \\
      &\bic x \in A - B \lor x \in A \cap C \\
      &\bic x \in (A-B) \cup (A \cap C)
    }
    as desired.
  }

  \begin{cor}\label{cor:closedlim:cc}
    If $A \ss X$ and $B = X - A$, then $A = X - B$.
  \end{cor}
  \qproof{
    By Lemma~\ref{lem:closedlim:setmid}, we have that
    \gath{
      X - B = X - (X - A) = (X - X) \cup (X \cap A) = \es \cup (X \cap A) = X \cap A = A
    }
    since $A \ss X$.
  }

  \mainprob
  \qproof{
    First, since $A$ is closed in $X$, we have that $B = X - A$ is open in $X$, and it follows from Corollary~\ref{cor:closedlim:cc} that $A = X - B$.
    Then we have that
    \gath{
      U - A = U - (X - B) = (U - X) \cup (U \cap B)
    }
    by Lemma~\ref{lem:closedlim:setmid}.
    Since $U \ss X$, it follows that $U - X = \es$, and hence
    \gath{
      U - A = \es \cup (U \cap B) = U \cap B \,.
    }
    Then, since both $U$ and $B$ are open, their intersection is as well and therefore $U - A$ is open.

    Next, we have by Lemma~\ref{lem:closedlim:setmid}
    \gath{
      X - (A - U) = (X - A) \cup (X \cap U) = B \cup (X \cap U) = B \cup U.
    }
    since $U \ss X$ so that $X \cap U = U$.
    Since both $B$ and $U$ are open, clearly their union is as well and hence $X - (A - U)$ is open.
    This of course means that $A - U$ is closed by definition.
  }
}

\exercise{5}{
  Let $X$ be an ordered set in the order topology.
  Show that $\closure{(a,b)} \ss [a,b]$.
  Under what conditions does equality hold?
}
\sol{
  \qproof{
    First, the closed interval $[a,b]$ is closed (hence why it is called such!) because clearly its compliment is
    \gath{
      X - [a,b] = (-\infty, a) \cup (b, \infty)
    }
    and we know that open rays are always open so that their union is as well.
    Clearly also $[a,b]$ contains $(a,b)$.
    Hence $[a,b]$ is a closed set containing $(a,b)$.
    Since $\closure{(a,b)}$ is defined as the intersection of closed sets that contain $(a,b)$ clearly we have that $\closure{(a,b)} \ss [a,b]$ as desired.
  }

  The conditions required for equality are such that $[a,b]$ is also a subset of $\closure{(a,b)}$ and, in particular both $a$ and $b$ must be in $\closure{(a,b)}$.
  Since clearly $a,b \notin (a,b)$, it has to be that they are both limit points of $(a,b)$.
  This is equivalent to the condition that $a$ has no immediate successor and $b$ no immediate predecessor.
  We show only the first of these since the second is analogous.
  \qproof{
    $(\imp)$ We show the contrapositive of this.
    So suppose that $a$ \emph{does} have an immediate successor $c$.
    Then the open ray $(-\infty, c)$ is an open set that contains $a$ but does not intersect $(a,b)$.
    This is easy to see, because if they did intersect, there would be an $x \in (a,b)$ where also $x \in (-\infty, c)$.
    From these it follows that $a < x < c$, which contradicts the fact that $c$ is the immediate successor of $a$.
    Hence by definition $a$ is not a limit point of $(a,b)$.

    $(\pmi)$ Suppose that $a$ is not a limit point of $(a,b)$.
    Then there is an open set $U$ containing $a$ that does not intersect $(a,b)$.
    From this it follows that there is a basis element $B$ containing $a$ such that $B \ss U$, and thus $B$ also cannot intersect $(a,b)$ (as, if it did, then so would $U$).
    Suppose that $B$ is the open interval $(c,d)$ so that $c < a < d$.
    It also must be that $d < b$ for otherwise, for any element of $x$ of $(a,b)$, we would have $c < a < x < b \leq d$ so that $x \in (c,d) = B$ and $B$ and $(a,b)$ would not be disjoint.
    We claim that $d$ is the immediate successor of $a$.
    If this is not the case then there would be an $x$ such that $c < a < x < d$ and hence $x \in (c,d) = B$.
    Also $a < x < d < b$ so that also $x \in (a,b)$.
    Therefore $B$ and $(a,b)$ would not be disjoint.
    Similar arguments can be made if $B$ are other types of basis element in the order topology.
    (Actually $B$ cannot be of the form $\opcl{e,f}$ for largest element $f$ of $X$ since then any element of $(a,b)$ would also be in $B$ and they would not be disjoint.)
  }

  It is also worth noting that the Hausdorff axiom (and therefore also the $T_1$ axiom since it is implied by the Hausdorff axiom) is not sufficient for general equivalence of $[a,b]$ and $\closure{(a,b)}$.
  For example the order topology on $\ints$ results in the discrete topology so that every subset is both open and closed.
  Thus for any pair $x_1, x_2$ in $\ints$, the sets $\braces{x_1}$ and $\braces{x_2}$ are neighborhoods of $x_1$ and $x_2$, respectively, that are disjoint.
  This shows that this topology is a Hausdorff space.
  However, the fact that $a$ has an immediate successor in $\pints$ is sufficient to show that $[a,b] \neq \closure{(a,b)}$ per what was just shown above.
}

\exercise{6}{
  Let $A$, $B$, and $A_\a$ denote subsets of a space $X$.
  Prove the following:
  \eparts{
  \item If $A \ss B$, then $\clA \ss \clB$.
  \item $\closure{A \cup B} = \clA \cup \clB$.
  \item $\closure{\bigcup A_a} \sps \bigcup \clA_\a$; give an example where equality fails.
  }
}
\sol{
  (a)
  \qproof{
    Suppose that $A \ss B$ and consider any $x \in \clA$.
    Consider any neighborhood $U$ of $x$ so that $U$ intersects $A$ by Theorem~17.5 part~(a).
    Hence there is a point $y \in U \cap A$ so that $y \in U$ and $y \in A$.
    But then clearly $y \in B$ also since $A \ss B$.
    Therefore $y \in U \cap B$ so that $U$ intersects $B$.
    Since $U$ was an arbitrary neighborhood of $x$, this shows that $x \in \clB$, again by Theorem~17.5 part~(a).
    This of course shows that $\clA \ss \clB$ as desired since $x$ was arbitrary.
  }

  (b)
  \qproof{
    $(\ss)$
    We show this by contrapositive.
    So suppose that $x \notin \clA \cup \clB$.
    Then clearly $x \notin \clA$ and $x \notin \clB$.
    Thus, by Theorem~17.5 part~(a), there is an open set $U_A$ such that $U_A$ does not intersect $A$, and likewise an open $U_B$ that does not intersect $B$.
    Let $U = U_A \cap U_B$, which is clearly open since $U_A$ and $U_B$ are.
    We also note that $U$ contains $x$ since both $U_A$ and $U_B$ do.
    Then it must be that $U$ does not intersect $A$ since, if it did, then $U_A$ would also intersect $A$ since $U \ss U_A$.
    Similarly, $U$ cannot intersect $B$.
    Thus, for all $y \in U$, $y \notin A$ and $y \notin B$.
    This is logically equivalent to saying that there is no $y \in U$ where $y \in A$ or $y \in B$, therefore there is no $y \in U$ where $y \in A \cup B$.
    Hence $U$ and $A \cup B$ do not intersect.
    Since $U$ is open and contains $x$, this shows that $x \notin \closure{A \cup B}$, again by Theorem~17.5 part~(a).
    Therefore, by contrapositive, $x \in \closure{A \cup B}$ implies that $x \in \clA \cup \clB$ so that $\closure{A \cup B} \ss \clA \cup \clB$.

    $(\sps)$
    Consider any $x \in \clA \cup \clB$ and any neighborhood $U$ of $x$.
    If $x \in \clA$ then $U$ intersects $A$ by Theorem~17.5 part~(a).
    Hence there is a $y \in U \cap A$ so that $y \in U$ and $y \in A$.
    Then clearly $y \in A \cup B$ so that $y$ is also in $U \cap (A \cup B)$.
    Hence $U$ intersects $A \cup B$.
    An analogous argument shows that this is also true if $x \in \clB$ instead.
    Since $U$ was an arbitrary neighborhood, this shows that $x \in \closure{A \cup B}$ by Theorem~17.5 part~(a).
    Hence $\clA \cup \clB \ss \closure{A \cup B}$ since $x$ was arbitrary.
  }

  (c)
  \qproof{
    Consider any $x \in \bigcup \clA_\a$ so that there is a particular $\b$ where $x \in \clA_\b$.
    Suppose that $U$ is any open set containing $x$ so that $U$ intersects $A_\b$ by Theorem~17.5 part~(a) since $x \in \clA_\b$.
    Then clearly $U$ also intersects $\bigcup A_\a$ since $A_\b \ss \bigcup A_\a$.
    Since $U$ was an arbitrary open set containing $x$, this shows that $x \in \closure{\bigcup A_\a}$ by Theorem~17.5 part~(a).
    This shows that $\bigcup \clA_\a \ss \closure{\bigcup A_\a}$ since $x$ was arbitrary, which is of course the desired result.
  }

  As an example where equality fails, consider the standard topology on $\reals$ and the sets $A_n = \opcl{1/n, 2}$ for $n \in \pints$.
  It is then trivial to show that $\bigcup A_n = \opcl{0, 2}$ so that clearly $0$ is a limit point of $\bigcup A_n$, and hence $0 \in \closure{\bigcup A_n}$.
  However, for any $n \in \pints$, the open interval $(-1, 1/n)$ is clearly an open set containing $0$ that is disjoint from $\opcl{1/n, 2} = A_n$.
  This shows that $0 \notin \clA_n$ for every $n \in \pints$ by Theorem~17.5 part~(a), from which it follows that $0 \notin \bigcup \clA_n$.
  Hence $\closure{\bigcup A_n}$ is not a subset of $\bigcup \clA_n$ and thus $\closure{\bigcup A_n} \neq \bigcup \clA_n$.
}

\exercise{7}{
  Criticize the following ``proof'' that $\closure{\bigcup A_\a} \ss \bigcup \clA_\a$: if $\braces{A_\a}$ is a collection of sets in $X$ and if $x \in \closure{\bigcup A_\a}$, then every neighborhood $U$ of $x$ intersects $\bigcup A_\a$.
  Thus $U$ must intersect some $A_\a$, so that $x$ must belong to the closure of some $A_\a$.
  Therefore, $x \in \bigcup \clA_\a$.
}
\sol{
  The problem with this ``proof'' is that, just because every neighborhood $U$ intersects \emph{some} $A_\a$, it does not mean that \emph{every} $U$ intersects a single $A_\a$, which is what is required for $x$ to be in $\clA_\a$.
  This is illustrated in the counterexample above at the end of Exercise~17.6 part~(c).
  There, every neighborhood of $0$ clearly intersects \emph{some} set $A_n = \opcl{1/n, 2}$, but, for any given $n \in \pints$, not every neighborhood of $0$ intersects $A_n$, for example the neighborhood $(-1, 1/n)$ does not.
}

\exercise{8}{
  Let $A$, $B$, and $A_\a$ denote subsets of a space $X$.
  Determine whether the following equations hold; if an equality fails, determine whether one of the inclusions $\sps$ or $\ss$ holds.
  \eparts{
  \item $\closure{A \cap B} = \clA \cap \clB$.
  \item $\closure{\bigcap A_\a} = \bigcap \clA_\a$.
  \item $\closure{A - B} = \clA - \clB$.
  }
}
\sol{
  (a) We claim that $\closure{A \cap B} \ss \clA \cap \clB$ but equality is not always true.
  \qproof{
    Consider any $x \in \closure{A \cap B}$ and any open set $U$ containing $x$.
    Then by, Theorem~17.5 part~(a), $U$ intersects $A \cap B$, from which it immediately follows that $U$ intersects both $A$ and $B$.
    However, since $U$ was an arbitrary neighborhood of $x$, it follows from Theorem~17.5 part~(a) again that $x$ is in both $\clA$ and $\clB$.
    Hence $x \in \clA \cap \clB$, which shows that $\closure{A \cap B} \ss \clA \cap \clB$ since $x$ was arbitrary.

    Now consider the standard topology on $\reals$ with $A = \clop{-1,0}$ and $B = \opcl{0,1}$.
    As these are clearly disjoint, we have that $A \cap B = \es$ so that $\closure{A \cap B} = \es$ also.
    However, since we also clearly have that $\clA = [-1,0]$ and $\clB = [0,1]$, it follows that $\clA \cap \clB = \braces{0}$.
    Thus clearly $\closure{A \cap B} = \es \neq \braces{0} = \clA - \clB$ as desired.
  }

  (b) We again claim that $\closure{\bigcap A_\a} \ss \bigcap \clA_\a$ but that equality is not generally true.
  \qproof{
    Consider any $x \in \closure{\bigcap A_\a}$ and any open set $U$ of $x$.
    Then, by Theorem~17.5 part~(a), $U$ intersects $\bigcap A_\a$ so that, for any particular $A_\b$, $U$ intersects $A_\b$.
    This shows that $x \in \clA_\b$ by Theorem~17.5 part~(a) so that $x \in \clA_\a$ for every $\a$ since $\b$ was arbitrary.
    Hence $x \in \bigcap \clA_\a$, which shows that $\closure{\bigcap A_\a} \ss \bigcap \clA_\a$ since $x$ was arbitrary.
    
    As in part~(a), equality fails if we have $A_1 = \clop{-1,0}$ and $A_2 = \opcl{0,1}$ in the standard topology on $\reals$.
    By the same argument as in part~(a) it follows that $\closure{\bigcap_{n=1}^2 A_n} = \es \neq \braces{0} = \bigcap_{n=1}^2 \clA_n$.
  }

  (c) Here we claim that $\closure{A-B} \sps \clA - \clB$ but that the converse does not always hold.
  \qproof{
    Consider any $x \in \clA - \clB$ and any open set $U$ containing $x$.
    Then $x \in \clA$ so that every open set containing $x$ intersects $A$ by Theorem~17.5 part~(a).
    Also $x \notin \clB$ so that there is an open set $V$ containing $x$ that does not intersect $B$, also by Theorem~17.5 part~(a).
    Let $W = U \cap V$ so that $W$ contains $x$ since both $x \in U$ and $x \in V$.
    Now, since $W$ is also an open set containing $x$, $W$ intersects $A$ so that there is a $y \in W$ where also $y \in A$.
    It also cannot be that $y \in B$ since we have $y \in W \ss V$ so that then $V$ would intersect $B$.
    Therefore $y \in A - B$.
    Also we have $y \in W \ss U$ so that also $y \in U$.
    Hence $U$ intersects $A - B$, which shows that $x \in \closure{A-B}$ by Theorem~17.5 part~(a) since $U$ was an arbitrary neighborhood of $x$.
    Therefore $\closure{A-B} \sps \clA - \clB$ as desired since $x$ was arbitrary.

    As a counterexample to equality, consider the standard topology on $\reals$ with $A = [0,2]$ and $B = \opcl{1,3}$.
    Then clearly $\clA = A = [0,2]$ and $\clB = [1,3]$, from which it is easily shown that $\clA - \clB = \clop{0,1}$.
    But we also have $A - B = [0,1]$ so that obviously $\closure{A-B} = [0,1]$ as well.
    Therefore $\closure{A-B} = [0,1] \neq \clop{0,1} = \clA - \clB$ as desired.
  }
}

\exercise{9}{
  Let $A \ss X$ and $B \ss Y$.
  Show that in the space $X \times Y$,
  \gath{
    \closure{A \times B} = \clA \times \clB \,.
  }
}
\sol{
  \qproof{
    $(\ss)$ Consider $(x,y) \in \closure{A \times B}$.
    Also suppose that $U$ and $V$ are any open sets in $X$ and $Y$, respectively, that contain $x$ and $y$, respectively.
    Then $U \times V$ is a basis element of the product topology on $X \times Y$, by definition, that contains $(x,y)$.
    It then follows from Theorem~17.5 part~(b) that $U \times V$ intersects $A \times B$ and hence there is a point $(w,z) \in U \times V$ where also $(w,z) \in A \times B$.
    Then $w \in U$ and $w \in A$ so that $U$ intersects $A$, and hence $x \in \clA$ by Theorem~17.5 part~(a) since $U$ was an arbitrary neighborhood of $x$.
    An analogous argument shows that $y \in \clB$.
    Therefore $(x,y) \in \clA \times \clB$ so that $\closure{A \times B} \ss \clA \times \clB$ since $x$ was arbitrary.

    $(\sps)$ Now suppose that $(x,y)$ is any point in $\clA \times \clB$ so that $x \in \clA$ and $y \in \clB$.
    Suppose also that $U \times V$ is any basis element of $X \times Y$ that contains $(x,y)$ so that by definition $U$ and $V$ are open in $X$ and $Y$, respectively.
    Since $x \in \clA$ and $U$ is an open set where $x \in U$, it follows from Theorem~17.5 part~(a) that $U$ intersects $A$.
    Thus there is $w \in U$ where $w \in A$ as well.
    An analogous argument shows that $V$ intersects $B$ so that there is a $z \in V$ where also $z \in B$.
    We therefore have that $(w,z) \in U \times V$ and $(w,z) \in A \times B$ so that $U \times V$ intersects $A \times B$.
    Since $U \times V$ was an arbitrary basis element containing $(x,y)$, it follows from Theorem~17.5 part~(b) that $(x,y) \in \closure{A \times B}$.
    This shows that $\clA \times \clB \ss \closure{A \times B}$ since the point $(x,y)$ was arbitrary.
  }
}

\exercise{10}{
  Show that every order topology is Hausdorff.
}
\sol{
  \qproof{
    Suppose that $X$ is an ordered set with the order topology.
    Consider a pair of distinct points $x_1$ and $x_2$ in $X$.
    Since $X$ is an order, $x_1$ and $x_2$ must be comparable since they are distinct, so we can assume that $x_1 < x_2$ without loss of generality.

    Case: $x_2$ is the immediate successor of $x_1$.
    Then, if $X$ has a smallest element $a$ then clearly the set $U_1 = \clop{a, x_2}$ is a neighborhood (because it is a basis element) of $x_1$.
    If $X$ has no smallest element then there is an $a < x_1$ so that $U_1 = (a, x_2)$ is a neighborhood of $x_1$.
    Similarly $U_2 = \opcl{x_1, b}$ or $U_2 = (x_1, b)$ is a neighborhood of $x_2$, where $b$ is either the largest element of $X$ or $x_2 < b$, respectively.
    Either way, for any $y \in U_1$ we have that $y < x_2$ so that $y \leq x_1$ since $x_2$ is the immediate successor of $x_1$.
    Hence it is not true that $y > x_1$ so that $y \notin U_2$.
    This shows that $U_1$ and $U_2$ are disjoint.

    Case: $x_2$ is not the immediate successor of $x_1$.
    Then there is an $x \in X$ where $x_1 < x < x_2$.
    So let $U_1 = \clop{a,x}$ (or $U_1 = (a, x)$) for the smallest element $a$ of $X$ (or some $a < x_1$).
    Similarly let $U_2 = \opcl{x,b}$ (or $U_2 = (x,b)$) for the largest element $b$ of $X$ (or some $x_2 < b$).
    Either way $U_1$ and $U_2$ are neighborhoods of $x_1$ and $x_2$, respectively.
    If $y \in U_1$ then $y < x$ so that clearly it is not true that $y > x$ so that $x \notin U_2$.
    Hence again $U_1$ and $U_2$ are disjoint.

    Thus in either case we have shown that $X$ is a Hausdorff space as desired since $x_1$ and $x_2$ were an arbitrary pair.
  }
}

\exercise{11}{
  Show that the product of two Hausdorff spaces is Hausdorff.
}
\sol{
  \qproof{
    Suppose that $X$ and $Y$ are Hausdorff spaces and consider two distinct points $(x_1,y_1)$ and $(x_2,y_2)$ in $X \times Y$.
    Since these points are distinct, it has to be that $x_1 \neq x_2$ or $y_1 \neq y_2$.
    In the first case $x_1$ and $x_2$ are distinct points of $X$ so that there are disjoint neighborhoods $U_1$ and $U_2$ of $x_1$ and $x_2$, respectively.
    This of course follows from the fact that $X$ is a Hausdorff space.
    Then we have that $U_1 \times Y$ and $U_2 \times Y$ are both basis elements, and therefore open sets, in the product space $X \times Y$ since $Y$ itself is obviously an open set of $Y$.
    Clearly also $(x_1,y_1) \in U_1 \times Y$ and $(x_2,y_2) \in U_2 \times Y$ so that $U_1 \times Y$ is a neighborhood of $(x_1,y_1)$ and $U_2 \times Y$ is a neighborhood of $(x_2,y_2)$.

    Then, for any $(x,y) \in U_1 \times Y$ we have that $x \in U_1$ so that $x \notin U_2$ since they are disjoint.
    Then it has to be that $(x,y) \notin U_2 \times Y$.
    This suffices to show that $U_1 \times Y$ and $U_2 \times Y$ are disjoint since $(x,y)$ was arbitrary.
    Thus $X \times Y$ is a Hausdorff space since the points $(x_1,y_1)$ and $(x_2,y_2)$ were arbitrary.
    An analogous argument in the case in which $y_1 \neq y_2$ shows the same result.
  }
}

\exercise{12}{
  Show that a subspace of a Hausdorff space is Hausdorff.
}
\sol{
  \qproof{
    Suppose that $X$ is a Hausdorff space and that $Y$ is a subset of $X$.
    Consider any two distinct points $y_1$ and $y_2$ in $Y$ so that of course also $y_1,y_2 \in X$.
    Then there are neighborhoods $U_1$ and $U_2$ of $y_1$ and $y_2$, respectively, that are disjoint since $X$ is Hausdorff.
    Since $U_1$ is open in $X$, we have that $V_1 = U_1 \cap Y$ is open in $Y$ by the definition of a subspace topology.
    Clearly also $V_1$ contains $y_1$ since $y_1 \in U_1$ and $y_1 \in Y$.
    Similarly $V_2 = U_2 \cap Y$ is an open set of $Y$ that contains $y_2$.
    Then, for any $x \in V_1$ clearly $x \in U_1$ so that $x \notin U_2$ since $U_1$ and $U_2$ are disjoint.
    Then $x \notin U_2 \cap Y = V_2$.
    Since $x$ was arbitrary, this shows that $V_1$ and $V_2$ are disjoint, which then shows that $Y$ is a Hausdorff space as desired.
  }
}

\exercise{13}{
  Show that $X$ is Hausdorff if and only if the \boldit{diagonal} $\D = \braces{x \times x \where x \in X}$ is closed in $X \times X$.
}
\sol{
  \qproof{
    $(\imp)$ Suppose that $X$ is Hausdorff and consider any point $x \times y \in X \times X$ where $x \times y \notin \D$.
    Then it must be that $x \neq y$ so that there are disjoint neighborhood $U$ of $x$ and $V$ of $y$ since $X$ is Hausdorff.
    Then $U \times V$ is a basis element of $X \times X$, by the definition of a product topology, and is therefore open.
    Now consider any point $w \times z \in U \times V$ so that $w \in U$ and $z \in V$.
    Then it has to be that $w \neq z$ since $U$ and $V$ are disjoint, which shows that $w \times z \notin \D$.
    Since $w \times z$ was an arbitrary point of $U \times V$, this shows that $U \times V$ does not intersect $\D$.
    Since also $U \times V$ is open and contains $x \times y$, this shows that $x \times y$ is not a limit point of $\D$.
    Moreover, since $x \times y$ was an arbitrary element of $X \times X$ that is not in $\D$, it follows that $\D$ must contain all of its limit points and is therefore closed by Corollary~17.7.

    $(\pmi)$ Now suppose that $\D$ is closed and suppose that $x$ and $y$ are distinct points in $X$.
    Then $x \times y \notin \D$ so that $x \times y$ cannot be a limit point of $\D$ (since it contains all its limit points by Corollary~17.7).
    Hence there is an open set $T$ in $X \times X$ that contains $x \times y$ and does not intersect $\D$.
    It then follows that there is a basis element $U \times V$ of $X \times X$ containing $x \times y$ where $U \times V \ss T$.
    Then $U$ and $V$ are both open in $X$ by the definition of the product topology, and clearly $x \in U$ and $y \in V$.
    It also follows that $U \times V$ does not intersect $\D$ since, if it did, then $T$ would as well.

    Suppose that $U$ and $V$ are not disjoint so that there is a $z \in U$ where also $z \in V$.
    Then clearly $z \times z \in U \times V$ but we also have that $z \times z \in \D$ so that $U \times V$ intersects $\D$.
    As we know that this cannot be the case, it has to be that $U$ and $V$ are disjoint.
    This shows that $X$ is Hausdorff as desired since $U$ is a neighborhood of $x$, $V$ is a neighborhood of $y$, and $x$ and $y$ were arbitrary distinct points of $X$.
  }
}

\exercise{14}{
  In the finite compliment topology on $\reals$, to what point or points does the sequence $x_n = 1/n$ converge?
}
\sol{
  We claim that this sequence converges to every point in $\reals$.
  \qproof{
    Suppose that this is not the case so that there is point $a \in \reals$ where the sequence does not converge to $A$.
    Then there is an open set $U$ containing $a$ such that, for every $N \in \pints$, there is an $n \geq N$ where $x_n \notin U$.
    It is easy to see that $x_n \notin U$ for an infinite number of $n \in \pints$.
    For, if this were not the case, then there would be an $N \in \pints$ where $x_n \in U$ for every $n \geq N$.
    We know, though, that there must be an $n \geq N$ where $x_n \notin U$.

    Moreover, clearly every $x_n$ in the sequence is distinct so that there are an infinite number of points not in $U$.
    Since each of these points are still in $X$, we have that $X - U$ is infinite.
    As this is the finite compliment topology and $U$ is open, this can only be the case if $X - U = X$ itself, in which case it would have be that $U = \es$ since $U \ss X$.
    This is not possible since $U$ contains $a$.
    So it seems that a contradiction has been reached, which shows the desired result.
  }

  In fact, this is true for any sequence for which the image of the sequence  $\braces{x_n \where n \in \pints}$ is infinite.
  This is to say that any such sequence converges to every point of $\reals$.
  Note also that this shows that the finite compliment topology on $\reals$ is not a Hausdorff space by the contrapositive of Theorem~17.10.
}

\exercise{15}{
  Show the $T_1$ axiom is equivalent to the condition that for each pair of points of $X$, each has a neighborhood not containing the other.
}
\sol{
  Note that, though it does not say so above, the points must be distinct since any neighborhood containing $x$ obviously has to contain $x$.

  \qproof{
    $(\imp)$ Suppose that a space $X$ satisfies the $T_1$ axiom and consider any two distinct points $x$ and $y$ of $X$.
    Then the point $\braces{x}$ is closed since it is finite, and hence it also contains all of its limit points by Corollary~17.7.
    Since the point $y$ is not in $\braces{x}$ (since $y \neq x$), it cannot be a limit point of $\braces{x}$.
    Hence there is a neighborhood $U$ of $y$ that does not intersect $\braces{x}$.
    Hence $x \notin U$.
    An analogous argument involving $\braces{y}$ shows that there is a neighborhood $V$ of $x$ that does not contain $y$.
    Since $x$ and $y$ were arbitrary points, this shows the desired property.

    $(\pmi)$ Now suppose that, for each pair of distinct points in $X$, each point has a neighborhood that does not contain the other point.
    As in the proof of Theorem~17.8, it suffices to show that every one-point set is closed, since any finite set can be expressed as the finite union of such sets, which is also then closed by Theorem~17.1.
    So let $\braces{x}$ be such a one-point set and consider any $y \notin \braces{x}$ so that clearly $y \neq x$.
    Then, since $x$ and $y$ are distinct, there is a neighborhood $U$ of $y$ such that $U$ does not contain $x$.
    Therefore $U$ and $\braces{x}$ are disjoint.
    This shows that $y$ is not a limit point of $\braces{x}$, which shows that $\braces{x}$ contains all its limit points since $y \notin \braces{x}$ was arbitrary.
    Hence $\braces{x}$ is closed as desired by Corollary~17.7.
  }
}

\exercise{16}{
  Consider the five topologies on $\reals$ given in Exercise~7 of \S 13.
  \eparts{
  \item Determine the closure of the set $K = \braces{1/n \where n \in \pints}$ under each of these topologies.
  \item Which of these topologies satisfy the Hausdorff axiom? the $T_1$ axiom?
  }
}
\sol{
  \begin{lem}\label{lem:closedlim:axiomsf}
    Suppose that $\cT$ and $\cT'$ are topologies on $X$ and $\cT'$ is finer than $\cT$.
    If $\cT$ satisfies the $T_1$ axiom, then so does $\cT'$.
    Similarly, if $\cT$ is Hausdorff, then so is $\cT'$.
  \end{lem}
  \qproof{
    First, suppose that $\cT$ satisfies the $T_1$ axiom and consider any finite subset $A$ of $X$.
    Then $A$ is closed in $\cT$ by the $T_1$ axiom so that by definition $X - A$ is open in $\cT$ and hence $X -A \in \cT$.
    Then $X - A \in \cT'$ as well since $\cT \ss \cT'$ so that $X - A$ is open in $\cT'$.
    Hence $A$ is closed in $\cT'$ by definition.
    Since $A$ was an arbitrary finite set, this shows that $\cT'$ also satisfies the $T_1$ axiom.

    Now suppose that $\cT$ is Hausdorff, and consider any two distinct points $x$ and $y$ in $X$.
    Then there are neighborhoods $U$ of $x$ and $V$ of $y$, both in $\cT$, that do not intersect since $\cT$ is Hausdorff.
    Then clearly $U,V \in \cT'$ as well since $\cT \ss \cT'$.
    Hence $U$ and $V$ are neighborhoods of $x$ and $y$, respectively, in $\cT'$ that do not intersect.
    This shows that $\cT'$ is Hausdorff as desired since $x$ and $y$ were arbitrary points of $X$.
  }

  \mainprob

  First we summarize what we claim about these topologies on $\reals$ for both parts:
  \begin{center}
    \begin{tabular}{cl|ccc}
      Topology & Definition &$T_1$ & Hausdorff & $\clK$ \\
      \hline
      $\cT_1$ & Standard & Yes & Yes & $K \cup \braces{0}$ \\
      $\cT_2$ & $\reals_K$ & Yes & Yes & $K$ \\
      $\cT_3$ & Finite complement & Yes & No & $\reals$ \\
      $\cT_4$ & Upper limit & Yes & Yes & $K$ \\
      $\cT_5$ & Basis of $(-\infty, a)$ sets & No & No & $\braces{x \in \reals \where 0 \leq x}$
    \end{tabular}
  \end{center}
  Next we justify these claims for each part.

  (a)
  First we show that $\clK = K \cup \braces{0}$ in $\cT_1$.
  \qproof{
    $(\ss)$ Consider any real number $x$ and suppose that $x \notin K \cup \braces{0}$, hence $x \notin K$ and $x \neq 0$.
    Since $x \neq 0$, we have

    Case: $x < 0$.
    Then clearly the open set $(x-1,0)$ contains $x$ but does not intersect $K$ since $0 < y$ for every $y \in K$, but $y < 0$ for every $y \in (x-1,0)$.

    Case: $x > 0$.
    If $1 < x$, then $(1, x+1)$ contains $x$ but does not intersect $K$ since $y \leq 1$ for every $y \in K$, but $1 < y$ for every $y \in (1, x+1)$.
    If $1 \geq x$ it follows from the fact that $x \notin K$ that there is a positive integer $n$ where $n < 1/x < n+1$, and hence $1/(n+1) < x < 1/n$.
    Then clearly the open set $(1/(n+1), 1/n)$ contains $x$, but we also have that it does not intersect $K$.
    If it did, then there would be an integer $m$ where $1/(n+1) < 1/m < 1/n$ so that $n < m < n+1$, which we know is not possible since $n+1$ is the immediate successor of $n$ in $\pints$.

    Thus in all cases there is a neighborhood of $x$ that does not intersect $K$.
    This of course shows that $x \notin \clK$ by Theorem~17.5 part~(a).
    We have therefore shown that $x \notin K \cup \braces{0}$ implies that $x \notin \clK$.
    By contrapositive, this shows that $\clK \ss K \cup \braces{0}$.

    $(\sps)$ Now consider any neighborhood $U$ of $0$ so that there is a basis element $(a,b)$ containing $0$ that is a subset of $U$.
    Then $a < 0 < b$.
    Clearly there is an $n \in \pints$ large enough where $a < 0 < 1/n < b$ and hence $1/n \in (a,b) \ss U$.
    Since also $1/n \in K$, we have that $U$ intersects $K$.
    Since $U$ was an arbitrary neighborhood, this shows that $0$ is in $\clK$ by Theorem~17.5 part~(a).
    Since also clearly any $x \in K$ is also in $\clK$, it follows that $\clK \sps K \cup \braces{0}$.
  }

  Next we show that $\clK = K$ in $\cT_2$, which is to say that $K$ is already closed.
  \qproof{
    First, clearly $K \ss \clK$ basically by definition.
    Now consider any $x \notin K$.
    Then clearly the set $B = (x-1,x+1) - K$ is a basis element of $\cT_2$.
    Also it clearly contains $x$ since $x \notin K$ and also does not intersect $K$ since $y \in B$ means that $y \notin K$.
    This shows that $x$ is not in $\clK$ by Theorem~17.5 part~(b).
    Since $x$ was arbitrary this shows that $x \notin K$ implies that $x \notin \clK$.
    Thus $\clK \ss K$ by contrapositive.
    This suffices to show that $\clK = K$ as desired.
  }

  Now we show that $\clK = \reals$ in $\cT_3$.
  \qproof{
    Consider any real $x$ and any neighborhood $U$ of $x$.
    Then $U$ is open in $\cT_3$ so that $\reals - U$ must be finite, noting that $\reals - U$ cannot be all of $\reals$ since $U$ would then have to be empty since $U \ss \reals$, whereas we know that $x \in U$.
    It then follows that there are a finite number of real numbers not in $U$.
    However, clearly $K$ is an infinite set so that there must be an element of $K$ that \emph{is} in $U$.
    This shows that $K$ intersects $U$ so that $x$ is in $\clK$ by Theorem~17.5 part~(a) since $U$ was an arbitrary neighborhood.
    Hence $\reals \ss \clK$ since $x$ was arbitrary.
    Clearly also $\clK \ss \reals$ so that $\clK = \reals$.
  }

  Next we show that $\clK = K$ in $\cT_4$ so that $K$ is closed.
  \qproof{
    Clearly $K \ss \clK$.
    So consider any real $x$ where $x \notin K$.

    Case: $x \leq 0$.
    Then the set $B = \opcl{x-1,x}$ is clearly a basis element of $\cT_4$ that contains $x$.
    For any $y \in K$ we have that $x \leq 0 < y$ so that $y \notin B$.
    Hence $B$ does not intersect $K$.

    Case: $x > 0$.
    If $1 \leq x$ then it has to be that $1 < x$ since $1 = 1/1 \in K$ but $x \notin K$, and hence $x \neq 1$.
    Then $B = \opcl{1,x}$ is clearly a basis element of $\cT_4$ and contains $x$.
    This also clearly does not intersect $K$ since $y \leq 1$ for any $y \in K$ so that $y \notin B$.
    On the other hand, if $1 > x$ then there is an integer $n$ where $n < 1/x < n+1$ so that $1/(n+1) < x < 1/n$ since $x \notin K$.
    It then follows that the set $B = \opcl{1/(n+1), x}$ is a basis element of $\cT_4$ that contains $x$ and does not intersect $K$.

    Hence in all cases there is a basis element $B$ containing $x$ that does not intersect $K$.
    This shows that $x \notin \clK$ by Theorem~17.5 part~(b).
    Hence we have shown that $x \notin K$ implies that $x \notin \clK$, which shows by contrapositive that $\clK \ss K$.
    Therefore $\clK = K$ as desired.
  }

  Lastly we show that $\clK = \braces{x \in \reals \where 0 \leq x}$ in $\cT_5$.
  \qproof{
    First, let $A = \braces{x \in \reals \where 0 \leq x}$ and consider any $x \in A$ and any basis element $B = (-\infty,a)$ containing $x$.
    Hence clearly $0 \leq x$ since $x \in A$ and $x < a$ since $x \in B$.
    Thus $0 \leq x < a$ so that there is an integer $n$ large enough that $0 < 1/n < a$.
    Then $1/n \in B$ and also clearly $1/n \in K$.
    Thus $B$ intersects $K$.
    Since $B$ was any neighborhood of $x$ it follows from Theorem~17.5 part~(b) that $x \in \clK$.
    Hence $A \ss \clK$ since $x$ was arbitrary.

    Now suppose that $x \notin A$ so that $x < 0$.
    Then the set $B = (-\infty,0)$ is clearly a basis element of $\cT_5$ that contains $x$.
    Since $0 < y$ for any $y \in K$, it follows that $y \notin B$, and thus $B$ cannot intersect $K$.
    Hence by Theorem~17.5 part~(b) we have that $x \notin \clK$.
    This shows that $\clK \ss A$ by contrapositive, which completes the proof that $\clK = A$.
  }
  
  (b)
  First we show that $\cT_1$, $\cT_2$, and $\cT_4$ are Hausdorff spaces and satisfy the $T_1$ axiom.
  \qproof{
    First consider any two distinct points $x,y \in \reals$.
    Without loss of generality, we can assume that $x < y$.
    Let $z = (x+y)/2$ so that clearly $x < z < y$.
    Then obviously the open intervals $(x-1,z)$ and $(z,y+1)$ are disjoint open sets in $\cT_1$ that contain $x$ and $y$, respectively.
    This shows that $\cT_1$ is a Hausdorff space and therefore also satisfies the $T_1$ axiom by Theorem~17.8.
    
    It then follows that $\cT_2$ and $\cT_4$ are also both Hausdorff and satisfy the $T_1$ axiom.
    This follows from Lemma~\ref{lem:closedlim:axiomsf} since it was shown in Exercise~13.7 that $\cT_1 \ss \cT_2 \ss \cT_4$.
  }

  Next we show that $\cT_3$ satisfies the $T_1$ axiom but is not a Hausdorff space.
  \qproof{
    So first consider any finite subset $A$ of $\reals$.
    Let $U = X - A$ so that clearly $A = X - (X - A) = X - U$.
    Then, since $X - U = A$ is finite, it follows that $U$ is open in $\cT_3$ by the definition of the finite complement topology.
    Hence by definition $A$ is closed in $\cT_3$ since $X - A = U$ is open.
    This shows that $\cT_3$ satisfies the $T_1$ axiom since $A$ was an arbitrary finite subset of $\reals$.

    To show that $\cT_3$ is not Hausdorff, consider any open set $U$ containing $0$ and any open set $V$ containing $1$.
    It then has to be that $\reals - U$ is finite since it cannot be that $\reals - U = \reals$ itself since then $U$ would have to be empty (which we know is not the case since $0 \in U$) since $U \ss \reals$.
    Likewise $\reals - V$ is also finite.
    Thus there are a finite number of real numbers that are not in $U$ and a finite number that are not in $V$.
    From this it clearly follows that there are a finite number of real numbers $x$ where $x \notin U$ or $x \notin V$.
    Since we have
    \gath{
      x \notin U \lor x \notin V
      \bic \lnot (x \in U \land x \in V)
      \bic \lnot (x \in U \cap V)
      \bic x \notin U \cap V \,,
    }
    it has to be that there are a finite number of reals numbers that are not in $U \cap V$.
    But since $\reals$ is infinite, this means that there are an infinite number of real numbers that \emph{are} in $U \cap V$.
    Hence $U \cap V \neq \es$, i.e. they intersect.
    Since $U$ and $V$ were arbitrary neighborhoods, this shows that $\cT_3$ is not Hausdorff by the negation of the definition.
  }

  Lastly we prove that $\cT_5$ is neither a Hausdorff space nor satisfies the $T_1$ axiom.
  \qproof{
    First consider the distinct real numbers $0$ and $1$.
    Consider then any open set $V$ containing $1$ so that there is a basis element $B = (-\infty, a)$ that contains $1$ and is a subset of $U$.
    Clearly we have that $0 \in B$ since $0 < 1 < a$ and hence $0 \in U$ since $B \ss U$.
    Since $U$ was an arbitrary neighborhood of $1$, it follows there is no neighborhood of $1$ that does not contain $0$.
    Hence $\cT_5$ does not satisfy the $T_1$ axiom by the negation of Exercise~17.15.
    It also then follows that $\cT_5$ is not a Hausdorff space by the contrapositive of Theorem~17.8.
  }
}

\exercise{17}{
  Consider the lower limit topology on $\reals$ and the topology given by the basis $\cC$ of Exercise~8 of \S 13.
  Determine the closures of the intervals $A = (0, \sqrt{2})$ and $B = (\sqrt{2}, 3)$ in these two topologies.
}
\sol{
  Recall that $\cC = \braces{\clop{a,b} \where \text{$a < b$, $a$ and $b$ rational}}$ from Exercise~13.8, noting that it was shown there that this basis generates a topology different from the lower limit topology.
  Denote the lower limit topology by $\cT_l$, and denote the topology generated by $\cC$ by $\cT_c$.

  \begin{lem}\label{lem:closedlim:openi}
    The closure of an open interval $(a,b)$ is $\clop{a,b}$ in the lower limit topology on $\reals$.
  \end{lem}
  \qproof{
    First let $A = (a,b)$ and $C = \clop{a, b}$ so that we must show that $\clA = C$.

    $(\sps)$ Consider any $x \in C$.

    Case: $x = a$.
    Consider any basis element $B = \clop{c,d}$ that contains $x = a$ so that $c \leq x = a < d$.
    Let $e = \min(b,d)$ so that $a < e$ since both $a < d$ and $a < b$.
    Then of course there is a real $y$ between $a$ and $e$ so that $a < y < e$.
    Thus we have $c \leq a < y < e \leq d$ so that $y \in B$.
    Also $a < y < e \leq b$ so that $y \in A$.
    Hence $B$ intersects $A$ so that $x = a \in \clA$ by Theorem~17.5 part~(b) since $B$ was an arbitrary basis element.

    Case: $x \neq a$.
    Then it has to be that $x \in (a, b) = A$ so that $x \in \clA$ since obviously $A \ss \clA$.

    This shows that $C \ss \clA$ since $x$ was arbitrary.

    $(\ss)$ Now consider any real $x$ where $x \notin C$ so that either $x < a$ or $x \geq b$.
    If $x < a$ then the basis element $B = \clop{x, a}$ clearly contains $x$ but does not intersect $A$.
    If $x \geq b$ then the basis element $B = \clop{b, x+1}$ contains $x$ and does not intersect $A$.
    Either way it follows from Theorem~17.5 part~(b) that $x \notin \clA$.
    Since $x$ was arbitrary, the contrapositive shows that $\clA \ss C$.
  }

  \begin{lem}\label{lem:closedlim:openir}
    The closure of an open interval $(a,b)$ in $\cT_c$ is $\clop{a,b}$ if $b$ is rational and $[a,b]$ if $b$ is irrational.
  \end{lem}
  \qproof{
    Let $A = (a,b)$.
    Consider any real $x$, and we shall consider an exhaustive list of cases that will show whether $x \in \clA$ or $x \notin \clA$.

    Case: $x < a$.
    Obviously there is a rational $p$ where $p < x$ since the rationals are unbounded below.
    Similarly, there is a rational $q$ where $x < q < a$ since the rationals are order-dense in the reals.
    The set $B = \clop{p,q}$ is then clearly a basis element of $\cT_c$ that contains $x$.
    It is also trivial to show that $B$ does not intersect $A$ since $q < a$, which shows that $x \notin \clA$ by Theorem~17.5 part~(b) whether $b$ is rational or not.

    Case: $x = a$.
    Consider any basis element $B = \clop{p,q}$ (where $p$ and $q$ are rational) that contains $x = a$ so that $p \leq x = a < q$.
    Let $d = \min(b,q)$ so that $a < d$ since both $a < q$ and $a < b$.
    Then of course there is a real $y$ between $a$ and $d$ so that $a < y < d$.
    Thus we have $p \leq a < y < d \leq q$ so that $y \in B$.
    Also $a < y < d \leq b$ so that $y \in A$.
    Hence $B$ intersects $A$ so that $x = a \in \clA$ by Theorem~17.5 part~(b) since $B$ was an arbitrary basis element.
    Note that this is true whether or not $b$ is rational.

    Case: $a < x < b$.
    Then clearly $x \in (a,b) = A$ so that $x \in \clA$ since obviously $A \ss \clA$.

    Case: $x = b$.
    \begin{indpar}
      Case: $b$ is rational.
      Then there is another rational $q$ where $q > b$ since the rationals are unbounded above.
      Then clearly the set $B = \clop{b, q}$ is a basis element of $\cT_c$ that contains $b$.
      Also clearly $B$ does not intersect $A$ since $y \in A$ implies that $y < b$ and hence $y \notin B$.
      This shows that $x = b \notin \clA$ by Theorem~17.5 part~(b).

      Case: $b$ is irrational.
      Then consider any basis element $B = \clop{p,q}$ containing $b$ so that $p$ and $q$ are rational.
      Thus $p \leq b < q$, but since $p$ is rational but $b$ is not, it has to be that $p < b < q$.
      Let $c = \max(p,a)$ so that $c < b$ since both $a < b$ and $p < b$.
      There is then a real $y$ where $c < y < b$ so that $a \leq c < y < b$ and hence $y \in A$.
      Also $p \leq c < y < b < q$ so that also $y \in B$.
      Therefore $B$ and $A$ intersect, which shows that $x = b \in \clA$ by Theorem~17.5 part~(b) since $B$ was arbitrary.
    \end{indpar}

    Case: $x > b$.
    Then there are clearly rationals $p$ and $q$ where $b < p < x$ and $x < q$.
    Then clearly the set $B = \clop{p,q}$ is a basis element that contains $x$ and does not intersect $A$.
    This of course shows that $x \notin \clA$ by Theorem~17.5 part~(b) again, noting that this is true regardless of the rationality of $b$.

    These cases taken together show the desired results.
  }

  \mainprob
  
  First, it follows directly from Lemma~\ref{lem:closedlim:openi} that that $\clA = \clop{0, \sqrt{2}}$ and $\clB = \clop{\sqrt{2},3}$ in $\cT_l$.
  It is worth noting that $\clA$ and $\clB$ are both basis elements of $\cT_l$, which is interesting since they are closures and therefore closed.
  This of course implies that basis elements in $\cT_l$ are both open and closed, which is indeed the case and is easy to see after a little thought.
  
  It also follows directly from Lemma~\ref{lem:closedlim:openir} that $\clA = [0, \sqrt{2}]$ and $\clB = \clop{\sqrt{2}, 3}$ in $\cT_c$ since $\sqrt{2}$ is irrational and $3$ is rational.
}

\exercise{18}{
  Determine the closures of the following subsets of the ordered square:
  \ali{
    A &= \braces{(1/n) \times 0 \where n \in \pints} \,, \\
    B &= \braces{(1-1/n) \times \tfrac{1}{2} \where n \in \pints} \,, \\
    C &= \braces{x \times 0 \where 0 < x < 1} \,, \\
    D &= \braces{x \times \tfrac{1}{2} \where 0 < x < 1} \,, \\
    E &= \braces{\tfrac{1}{2} \times y \where 0 < y < 1} \,.
  }
}
\sol{
  We assume that the ordered square refers to the set $X = [0,1]^2$ with the dictionary order topology.
  Denote the dictionary order on $X$ by $\prec$.

  \begin{defin}
    For a topology on $\reals$ and some subset $A \ss \reals$, consider a point $x \in \reals$.
    We say that $x$ is a limit point of $A$ from above if every neighborhood containing $x$ also contains a point $y$ where $y \in A$ and $x < y$.
    Similarly, a point $x$ is a limit point of $A$ from below if every neighborhood containing $x$ also contains a point $y$ where $y \in A$ and $y < x$.
  \end{defin}

  Note that a point can be a limit point from both below and above.
  
  \begin{lem}\label{lem:closedlim:unitsq}
    Suppose that $A$ is a subset of the real interval $[0,1]$ and that $B = \braces{x \times b \where x \in A}$ for some $b \in [0,1]$ so that $B \ss X = [0,1]^2$.
    Then the point $x \times y$ is a limit point of $B$ in the dictionary order topology on the unit square if and only if either $y = 1$ and $x$ is a limit point of $A$ from above \emph{or} $y = 0$ and $x$ is a limit point of $A$ from below in the order topology on $[0,1]$.
  \end{lem}
  \qproof{
    $(\imp)$ We show this by contrapositive.
    So suppose that $y \neq 1$ or $x$ is not a limit point of $A$ from above \emph{and} that $y \neq 0$ or $x$ is not a limit point from below.

    Case: $y \neq 0$ and $y \neq 1$.
    Clearly then $0 < y < 1$.
    If $y = b$ then the dictionary order interval $(x \times 0, x \times 1)$ is a basis element that contains $x \times y$ and that does not contain any other points of $B$, if indeed $x \in A$ so that $x \times y = x \times b$ is in $B$.
    If $y < b$ then the dictionary order interval $(x \times 0, x \times b)$ is a basis element with the same properties.
    Lastly, if $y > b$ then the dictionary order interval $(x \times b, x \times 1)$ is a basis element that contains $x \times y$ but no points of $B$.

    Case: $y = 0$ or $y = 1$.
    If $y = 0$ then we have
    \begin{indpar}
      Case: $x = 0$.
      Then, if $b = y = 0$, we have that the dictionary order interval $\clop{0 \times 0, 0 \times 1}$ is a basis element containing $x \times y = 0 \times 0$ but no other points of $B$, if indeed $x = 0 \in A$ so that $x \times y \in B$.
      If $b \neq 0$ then $0 < b$ so that the interval $\clop{0 \times 0, 0 \times b}$ is a basis element with the same properties.

      Case: $x \neq 0$.
      Then $0 < x$ and it has to be that $x$ is a not a limit point of $A$ from below.
      Thus there is an interval $(c,d)$ or $\opcl{c,1}$ (or $\clop{0,d}$ in which case let $c=0$ in what follows) that contains $x$ but no other points $y \in A$ where $y < x$.
      If $b = y = 0$ then it is easy to show that $(c \times 1, x \times 1)$ (or $\opcl{c  \times 1, x \times 1}$ if $x = 1$) is a basis element that contains $x \times y$ but no other points of $B$, if indeed $x \in A$ so that $x \times y \in B$.
      If $b \neq y = 0$ then $0 < b$ so that $(c \times 1, x \times b)$ is a basis element with the same property.
    \end{indpar}
    If $y = 1$, then an analogous argument shows analogous results.

    Thus in all cases and sub-cases it follows that $x \times y$ is not a limit point of $B$, which shows the desired result by contrapositive.

    $(\pmi)$ Now suppose that either $y = 1$ and $x$ is a limit point of $A$ from above or $y = 0$ and $x$ is a limit point of $A$ from below.
    In the first case consider any dictionary order interval $C = (a \times c, d \times e)$ that contains $x \times y$.
    Then it has to be that $x < d$ since otherwise it would have to be that $y = 1 < e$ since $x \times y \prec d \times e$, which is of course impossible.
    Then, since $x$ is a limit point of $A$ from above, it follows that the open set $\clop{0,d}$ contains a point $z \in A$ where $x < z$ so that $x < z < d$.
    It then follows that the point $z \times b$ is in both $C$ and $B$, and is of course distinct from $x \times y$ since $x < z$.
    The same argument can be made if $C$ is a basis element in the form of $\clop{0 \times 0, d \times e}$ or $\opcl{a \times c, 1 \times 1}$.
    This suffices to show that $x \times y$ is a limit point of $B$ since $C$ was an arbitrary basis element.

    An analogous argument can be made in the case when $y = 0$ and $x$ is a limit point of $A$ from below, which shows the desired result.
  }

  \mainprob

  First we claim that $\clA = A \cup \braces{0 \times 1}$.
  \qproof{
    First, let $K = \braces{1/n \where n \in \pints} \ss [0,1]$ so that clearly $A = \braces{x \times 0 \where x \in K}$.
    It is easy to show that $0$ is the only limit point of $K$ and it is a limit point from above only.
    It then follows from Lemma~\ref{lem:closedlim:unitsq} that $0 \times 1$ is the only limit point of $A$ so that $\clA = A \cup \braces{0 \times 1}$ since the closure is the union of the set and the set of its limit points.
  }

  Next we claim that $\clB = B \cup \braces{1 \times 0}$.
  \qproof{
    This time let $L = \braces{1-1/n \where n \in \pints}$ so that clearly $B = \braces{x \times \tfrac{1}{2} \where x \in L}$.
    It is trivial to show that $1$ is the only limit point of $L$ and that it is a limit point from below only.
    Hence $1 \times 0$ is the only limit point of $B$ by Lemma~\ref{lem:closedlim:unitsq} so that the result follows.
  }

  Now we claim that $\closure{C} = C \cup \braces{1 \times 0} \cup \braces{x \times 1 \where 0 \leq x < 1}$.
  \qproof{
    First, we clearly have that $C = \braces{x \times 0 \where x \in (0,1)}$.
    It is easy to show that every point of $(0, 1)$ is a limit point both from above and below, that $0$ is a limit point from above only, and that $1$ is a limit point from below only.
    Thus it follows that the set of limit points of $C$ are then $\braces{x \times 0 \where 0 < x \leq 1} \cup \braces{x \times 1 \where 0 \leq x < 1}$ by Lemma~\ref{lem:closedlim:unitsq}.
    As many of these points are contained in $C$ itself, the result follows.
  }

  We claim that $\closure{D} = D \cup \braces{x \times 0 \where 0 < x \leq 1} \cup \braces{x \times 1 \where 0 \leq x < 1}$.
  \qproof{
    The limit points of $D$ are the same as for $C$ above for the same reasons, i.e. $\braces{x \times 0 \where 0 < x \leq 1} \cup \braces{x \times 1 \where 0 \leq x < 1}$.
    The result then follows.
  }

  Lastly, we claim that $\closure{E} = \braces{\tfrac{1}{2} \times y \where 0 \leq y \leq 1} = \braces{\tfrac{1}{2}} \times [0,1]$, noting that clearly $E = \braces{\tfrac{1}{2}} \times (0,1)$.
  \qproof{
    Let $F = \braces{\tfrac{1}{2}} \times [0,1]$ so that we must show that $\closure{E} = F$.

    $(\ss)$ Consider any $x \times y$ where $x \times y \notin F$ so that simply $x \neq \tfrac{1}{2}$ since it has to be that $y \in [0,1]$.
    If $x < \tfrac{1}{2}$ then the basis element $\clop{0 \times 0, \tfrac{1}{2} \times 0}$ clearly contains $x \times y$ but no elements of $E$.
    If $x > \tfrac{1}{2}$ then the basis element $\opcl{\tfrac{1}{2} \times 1, 1 \times 1}$ clearly contains $x \times y$ but no elements of $E$ either.
    This shows that $x \times y$ is a not in $\closure{E}$ by Theorem~17.5 part~(b).
    Hence $\closure{E} \ss F$ by contrapositive.

    $(\sps)$ Consider any $x \times y \in F$ so that $x = \tfrac{1}{2}$ and $y \in [0,1]$.
    If $y \in (0,1)$ then $x \times y \in E$ so that $x \times y \in \closure{E}$ since obviously $E \ss \closure{E}$.
    If $y = 0$ then consider any dictionary order interval $F = (a \times c, b \times d)$ containing $x \times y = \tfrac{1}{2} \times 0$.
    In particular we have that $\tfrac{1}{2} \times 0 \prec b \times d$ so that either $\tfrac{1}{2} < b$, or $b = \tfrac{1}{2}$ and $0 < d$.
    In the first case we have that $\tfrac{1}{2} \times \tfrac{1}{2}$ is in both $F$ and $E$.
    In the second case let $z = d/2$ so that we have $0 < z < d \leq 1$.
    Then clearly the point $\tfrac{1}{2} \times z$ is in $F$, but we also have that $\tfrac{1}{2} \times z$ is in $E$ since $0 < z < 1$.
    The same argument applies if the basis element $F$ is of the form $\clop{0 \times 0, b \times d}$ or $\opcl{a \times c, 1 \times 1}$.
    A similar argument shows an analogous result in the case when $y = 1$.
    This shows by Theorem~17.5 part~(b) that $x \times y \in \closure{E}$ since $F$ was an arbitrary basis element, which of course shows that $\closure{E} \sps F$ since $x \times y$ was arbitrary.
  }
}

\exercise{19}{
  If $A \ss X$, we define the \boldit{boundary} of $A$ by the equation
  \gath{
    \Bd{A} = \clA \cap \closure{(X-A)} \,.
  }
  \eparts{
  \item Show that $\Int{A}$ and $\Bd{A}$ are disjoint, and $\clA = \Int{A} \cup \Bd{A}$.
  \item Show that $\Bd{A} = \es \bic A$ is both open and closed.
  \item Show that $U$ is open $\bic \Bd{U} = \clU - U$.
  \item If $U$ is open, is it true that $U = \Int(\clU)$?
    Justify your answer.
  }
}
\sol{
  (a)
  \qproof{
    Consider any $x \in \Int{A}$ so that there is a neighborhood of $x$ that is entirely contained in $A$.
    Then, for any $y \in U$, we have that $y \in A$ and hence $y \notin X - A$.
    This shows that $U$ does not intersect $X - A$, which suffices to show that $x$ is not in the closure of $X - A$ by Theorem~17.5 part~(a).
    Thus $x$ is not in the boundary of $A$ since $\Bd{A} = \clA \cap \closure{(X-A)}$.
    This of course shows that $\Int{A}$ and $\Bd{A}$ are disjoint since $x$ was arbitrary.

    To show that $\clA = \Int{A} \cup \Bd{A}$, first consider any $x \in \clA$.
    If $x \in \Int{A}$ then clearly $x \in \Int{A} \cup \Bd{A}$, so assume that $x \notin \Int{A}$.
    Consider any neighborhood $U$ of $X$.
    Then it has to be that $U$ is not a subset of $A$ since otherwise $x$ would be in the union of open subsets of $A$ and hence in the interior.
    It then follows that there is a point $y \in U$ where $y \notin A$ and therefore $y \in X - A$.
    This shows that $U$ intersects $X - A$ so that $x$ is in the closure of $X - A$ since $U$ was an arbitrary neighborhood.
    Since also $x \in \clA$, we have that $x \in \clA \cap \closure{(X-A)} = \Bd{A}$.
    Hence clearly $x \in \Int{A} \cup \Bd{A}$ so that $\clA \ss \Int{A} \cup \Bd{A}$ since $x$ was arbitrary.

    Now consider any $x \in \Int{A} \cup \Bd{A}$.
    If $x \in \Int{A}$ then also $x \in \clA$ since we have that $\Int{A} \ss A \ss \clA$.
    On the other hand, if $x \in \Bd{A} = \clA \cap \closure{(X-A)}$ then of course $x \in \clA$.
    This shows that $\Int{A} \cup \Bd{A} \ss \clA$ in either case since $x$ was arbitrary.
    Since both directions have been shown, it follows that $\clA = \Int{A} \cup \Bd{A}$ as desired.
  }

  (b)
  \qproof{
    $(\imp)$ First suppose that $\Bd{A} = \es$.
    Then by part~(a) we have that $\clA = \Int{A} \cup \Bd{A} = \Int{A} \cup \es = \Int{A}$.
    Hence $A \ss \clA = \Int{A}$ so that $A = \Int{A}$ since it is also always the case that $\Int{A} \ss A$.
    This shows that $A$ is open since $\Int{A}$ is always open.
    We also have $\clA = \Int{A} \ss A$ so that $A = \clA$ since it is always also the case that $A \ss \clA$.
    This of course shows that $A$ is also closed since $\clA$ is always closed.

    $(\pmi)$ Now suppose that $A$ is both open and closed.
    It then follows that $\clA = A = \Int{A}$.
    So consider any $x \in \clA$ so that also $x \in \Int{A}$.
    Then there is a neighborhood $U$ of $x$ contained entirely in $A$.
    Thus, for any point $y \in U$, we have that $y \in A$ so that $y \notin X - A$, which shows that $U$ does not intersect $X - A$.
    Since $U$ is a neighborhood of $x$, this shows that $x \notin \closure{X-A}$ by Theorem~17.5 part~(a).
    Then, since $x$ was an arbitrary element of $\clA$, it follows that $\clA$ and $\closure{X-A}$ are disjoint so that $\Bd{A} = \clA \cap \closure{(X-A)} = \es$ as desired.
  }

  (c)
  \qproof{
    $(\imp)$ First suppose that $U$ is open and consider any $x \in \Bd{U}$.
    Then we have that $x \in \clU$ and $x \in \closure{X-U}$ since $\Bd{U} = \clU \cap \closure{(X-U)}$ by definition.
    Suppose for the moment that $x \in U$ so that $U$ itself is a neighborhood of $x$ since it is open.
    For any $y \in U$ we have that $y \notin X - U$, and hence $U$ does not intersect $X - U$.
    This shows that $x$ is not in $\closure{X-U}$ by Theorem~17.5 part~(a), which is a contradiction since we know it is.
    Thus it must be that $x \notin U$ so that $x \in \clU - U$.
    This of course shows that $\Bd{U} \ss \clU - U$ since $x$ was arbitrary.

    Now consider any $x \in \clU - U$ so that clearly $x \in \clU$.
    Since also $x \notin U$, it follows that $x \in X-U$ so that of course $x \in \closure{X-U}$ as well.
    Hence $x \in \clU \cap \closure{(X-U)} = \Bd{U}$, which shows that $\clU - U \ss \Bd{U}$ since $x$ was arbitrary.
    This suffices to show that $\Bd{U} = \clU - U$ as desired.

    $(\pmi)$ Now suppose that $\Bd{U} = \clU - U$ and consider any $x \in U$.
    Then we have that $x \notin \clU - U = \Bd{U} = \clU \cap \closure{(X-U)}$.
    Since we know that $x \in \clU$ (since $U \ss \clU$), it must be that $x \notin \closure{X-U}$.
    Thus, by Theorem~17.5 part~(a), there is a neighborhood of $V$ of $x$ that does not intersect $X - U$.
    This means that, for any point $y \in V$, we have that $y \notin X - U$.
    Since of course $y \in X$, it follows that $y$ must be in $U$.
    This shows that $V \ss U$ since $y$ was arbitrary.
    Hence $V$ is a neighborhood of $x$ that is entirely contained in $U$ so that $x$ is in the union of open sets contained in $U$, hence $x \in \Int{U}$.
    Since $x$ was an arbitrary element of $U$, this shows that $U \ss \Int{U}$.
    As it is always the case that $\Int{U} \ss U$ as well, we have that $U = \Int{U}$ so that $U$ is open since $\Int{U}$ is always open.
  }

  (d)
  We claim that this is not generally true.
  \qproof{
    As a counterexample consider the set $U = \realsnz$ in the finite complement topology on $\reals$.
    Clearly $U$ is open as its complement $\reals - U = \braces{0}$ is finite.
    It is also obvious that $U$ is an infinite set.

    Now consider any real number $x$ and any neighborhood $V$ of $x$.
    It cannot be that $\reals - V$ is all of $\reals$ since then $V$ would be empty, and we know that $x \in V$.
    So it must be that $\reals - V$ is finite since $V$ is open, which means that there are only a finite number of real numbers that are \emph{not} in $V$.
    However, since $U$ is infinite, there must be an element of $U$ that \emph{is} in $V$ (in fact there are an infinite number of such elements).
    Hence $V$ intersects $U$ so that $x \in \clU$ by Theorem~17.5 part~(a).
    Since $x \in \reals$ was arbitrary, it must be that $\clU$ is all of $\reals$.

    Clearly $\reals$ is open (since the a set is always open in any topology on that set) so that $\Int(\clU) = \Int{\reals} = \reals$.
    Then, since $0 \in \reals = \Int(\clU)$ but $0 \notin U$, we have that $U \neq \Int(\clU)$.
  }
}

\exercise{20}{
  Find the boundary and the interior of each of the following subsets of $\reals^2$:
  \eparts{
  \item $A = \braces{x \times y \where y = 0}$
  \item $B = \braces{x \times y \where x > 0 \text{ and } y \neq 0}$
  \item $C = A \cup B$
  \item $D = \braces{x \times y \where x \text{ is rational}}$
  \item $E = \braces{x \times y \where 0 < x^2 - y^2 \leq 1}$
  \item $F = \braces{x \times y \where x \neq 0 \text{ and } y \leq 1/x}$
  }
}
\sol{
  (a) It is easy to show that $A$ is closed so that $\clA = A$, and that also $\closure{\reals - A} = A$ so that $\Bd{A} = A$.
  It is also easy to see that no basis element and therefore no neighborhood of any point in $A$ is contained entirely within $A$.
  From this it follows that $\Int{A} = \es$.

  (b) It is easy to show that $B$ is open so that $\Int{B} = B$.
  It is likewise not difficult to prove that $\clB = \braces{x \times y \where x \geq  0}$.
  We then have from Exercise~17.19 part~(c) that $\Bd{B} = \clB - B = \braces{x \times y \where x = 0} \cup \braces{x \times y \where x > 0 \text{ and } y = 0}$.

  (c) Here we have that $C = A \cup B = \braces{x \times y \where y = 0} \cup \braces{x \times y \where x > 0}$.
  It is then easy to show that the closure is $\closure{C} = \braces{x \times y \where y = 0} \cup \braces{x \times y \where x \geq 0}$.
  We also have that $\reals - C = \braces{x \times y \where x \leq 0 \text{ and } y \neq 0}$ so that $\closure{\reals - C} = \braces{x \times y \where x \leq 0}$.
  From these we clearly then have
  \gath{
    \Bd{C} = \closure{C} \cap \closure{(\reals - C)} = \braces{x \times y \where x < 0 \text{ and } y = 0} \cup \braces{x \times y \where x = 0} \,.
  }
  It is also not difficult to show that $\Int{C} = \braces{x \times y \where x > 0}$.

  (d) Clearly we have that $\closure{D}$ is all of $\reals^2$ as a consequence of the fact that the rationals are order-dense in the reals.
  Also, since any neighborhood of any point in $D$ will intersect a point $x \times y$ with irrational $x$, it follows that no point of $D$ is in its interior.
  Thus $\Int{D} = \es$ so that $\closure{D} = \Int{D} \cup \Bd{D} = \es \cup \Bd{D} = \Bd{D}$ by Exercise~17.19 part~(a), and hence $\Bd{D} = \closure{D} = \reals^2$.

  (e) It should be fairly obvious by this point that
  \gath{
    \Bd{E} = \braces{x \times y \where \abs{y} = \abs{x}} \cup \braces{x \times y \where x^2 - y^2 = 1}
  }
  and $\Int{E} = \braces{x \times y \where 0 < x^2 - y^2 < 1}$.
  This would be easy but tedious to prove rigorously.

  (f) First we clearly have that $\Int{F} = \braces{x \times y \where x \neq 0 \text{ and } y < 1/x}$.
  We also have that $\closure{F} = \braces{x \times y \where x = 0} \cup \braces{x \times y \where x \neq 0 \text{ and } y \leq 1/x}$.
  By Exercise~17.19 part~(a) we have that $\closure{F} = \Int{F} \cup \Bd{F}$ and that $\Int{F} \cap \Bd{F} = \es$ so that $\Bd{F} = \closure{F} - \Int{F}$.
  Thus we have that $\Bd{F} = \braces{x \times y \where x = 0} \cup \braces{x \times y \where x \neq 0 \text{ and } y = 1/x}$.
  Again these facts are not difficult to show rigorously but would be tedious.
}

\exercise{21}{
  (Kuratowski) Consider the collection of all subsets $A$ of the topological space $X$.
  The operations of closure $A \to \clA$ and complementation $A \to X - A$ are functions from the collection to itself.
  \eparts{
  \item Show that starting with a given set $A$, one can form no more than 14 distinct sets by applying these two operations successively.
  \item Find a subset $A$ of $\reals$ (in its usual topology) for which the maximum of 14 is obtained.
  }
}
\sol{
  For the following we introduce the following notation to make things simpler. If $A$ is a subset of a topological space $X$ then denote
  \ali{
    cA &= \clA &
    xA &= X - C \\
    iA &= \Int{A} &
    bA &= \Bd{A} \,.
  }
  We can consider these ($c$, $x$, $i$, and $b$) as operators on sets that can be chained together in the obvious way so that, for example, $cxiA = \closure{X - \Int{A}}$.

  \begin{lem}\label{lem:closedlim:part}
    For a subset $A$ of topological space $X$, $X = cA \cup ixA$, and $cA$ and $ixA$ are disjoint
  \end{lem}
  \qproof{
    First, it is obvious that $cA \cup ixA \ss X$ since each of the sets in the union is a subset of $X$.
    Now consider any $x \in X$ and suppose that $x \notin cA = \clA$.
    Then by Lemma~17.5 part~(a) there is an open set $U$ containing $x$ where $U$ does not intersect $A$.
    For any $y \in U$ we thus have that $y \notin A$ and hence $y \in X - A = xA$.
    This shows that $U \ss xA$ since $y$ was arbitrary, which suffices to show that $x \in \Int(xA) = ixA$ since $U$ is a neighborhood of $x$.
    This of course shows that $x \in cA \cup ixA$ so that $X \ss cA \cup ixA$ since $x$ was arbitrary.
    This completes the proof that $X = cA \cup ixA$.

    To show that $cA$ and $ixA$ are disjoint, consider any $x \in cA$.
    Consider any neighborhood $U$ of $x$ so that $U$ intersects $A$ by Lemma~17.5 part~(a).
    Hence there is a point $y \in U$ where also $y \in A$, from which it follows that $y \notin X - A = xA$.
    This suffices to show that $U$ is not a subset of $xA$.
    Since $U$ is an arbitrary neighborhood, this shows that $x \notin \Int(xA) = ixA$.
    This of course shows that $cA$ and $ixA$ are disjoint as desired.
  }

  \begin{lem}\label{lem:closedlim:idents}
    For a subsets $A$ and $B$ of topological space $X$ where $A \ss B$, we have the following:
    \begin{multicols}{3}
      \eparts{
      \item $cA \ss cB$
      \item $iA \ss iB$
      \item $ccA = cA$
      \item $iiA = iA$
      \item $xxA = A$
      \item $xcA = ixA$
      \item $xiA = cxA$
      \item $icicA = icA$
      \item $ciciA = ciA$.
      }
    \end{multicols}
  \end{lem}
  \qproof{
    (a) This was shown in Exercise~17.6 part~(a).

    (b) Consider any $x \in iA$ so that there is a neighborhood $U$ of $x$ that is totally contained in $A$.
    Then clearly $U$ is also totally contained in $B$ as well since, for any $x \in U$, we have that $x \in A$ and hence $x \in B$ since $A \ss B$.
    This shows that $x \in iB$ since $U$ is a neighborhood of $x$.
    Hence $iA \ss iB$ since $x$ was arbitrary.

    (c) Since $cA = \clA$ is closed, we clearly have $ccA = cA$.

    (d) Since $iA = \Int{A}$ is open, its interior is itself, i.e.  $iiA = iA$.

    (e) Obviously $xxA = X - (X - A) = A$ since $A \ss X$.

    (f) We have by Lemma~\ref{lem:closedlim:part} that $X = cA \cup ixA$ where $cA$, and $ixA$ are mutually disjoint.
    From this it follows that $ixA = X - cA = xcA$.

    (g) We have
    \ali{
      cxA &= xxcxA & \text{(by (e))} \\
      &= xixxA & \text{(by (f))} \\
      &= xiA & \text{(by (e) again)}
    }
    as desired.

    (h) First we have that $icA = iicA$ by (d).
    Also clearly $icA = c(icA) = cicA$ since a set is always a subset of its closure.
    Hence by (b) we have that $icA = iicA = i(icA) \ss i(cicA) = icicA$.
    Now, we also have that $icA = i(cA) \ss cA$ since the interior of a set is always a subset of the set.
    Hence by (a) and (c) we have $cicA = c(icA) \ss c(cA) = ccA = cA$.
    It then follows from (b) that $icicA = i(cicA) \ss i(cA) = icA$ as well.
    This of course shows that $icicA = icA$ as desired.

    (i) Lastly, we have
    \ali{
      ciciA &= cicixxA & \text{(by (e))} \\
      &= cicxcxA & \text{(by (f))} \\
      &= cixicxA & \text{(by (g))} \\
      &= cxcicxA & \text{(by (f))} \\
      &= xicicxA & \text{(by (g))} \\
      &= xicxA & \text{(by (h))} \\
      &= cxcxA & \text{(by (g))} \\
      &= cixxA & \text{(by (f))} \\
      &= ciA & \text{(by (e))}
    }
    as desired.
  }

  \mainprob

  (a)
  \def\pres{\text{(previous result)}}
  \newcommand\lemref[1]{\text{(by Lemma~\ref{lem:closedlim:idents}#1)}}
  \qproof{
    We are interested in sequences applying the operators $c$ and $x$ to a subset $A$.
    By Lemma~\ref{lem:closedlim:idents} (c) and (e) we have that $ccA = cA$ and $xxA = A$.
    Thus there is no point in ever applying $c$ or $x$ twice in a row since that would clearly result in a set that we have seen before.
    We are then interested only in sequences that apply alternating $c$ and $x$.
    If we apply the closure $c$ first, we obtain the following sequence:
    \ali{
      A &= A \\
      cA &= cA \\
      xcA &= ixA & \lemref{f} \\
      cxcA &= cixA & \pres \\
      xcxcA &= xcixA & \pres \\
      &= ixixA & \lemref{f} \\
      &= icxxA & \lemref{g} \\
      &= icA & \lemref{e} \\
      cxcxcA &= cicA & \pres \\
      xcxcxcA &= xcicA & \pres \\
      &= ixicA & \lemref{f} \\
      &= icxcA & \lemref{g} \\
      &= icixA & \lemref{f}
    }
    If we apply the next operation we obtain
    \ali{
      cxcxcxcA &= cicixA & \pres \\
      &= cixA \,, & \lemref{i}
    }
    which is the same as the fourth set above.
    Therefore we can get at most 7 distinct sets by applying $c$ first, including $A$ itself.
    If we instead apply $x$ first then we get the following sequence:
    \ali{
      xA &= xA \\
      cxA &= cxA \\
      xcxA &= ixxA & \text{(corresponding result above)} \\
      &= iA & \lemref{e} \\
      cxcxA &= ciA & \pres \\
      xcxcxA &= xciA & \pres \\
      &= ixiA & \lemref{f} \\
      &= icxA & \lemref{g} \\
      cxcxcxA &= cicxA & \pres \\
      xcxcxcxA &= xcicxA & \pres \\
      &= ixicxA & \lemref{f} \\
      &= icxcxA & \lemref{g} \\
      &= icixxA & \lemref{f} \\
      &= iciA & \lemref{e}
    }
    Again if we try to apply the next operation we get
    \ali{
      cxcxcxcxA &= ciciA & \pres \\
      &= ciA\, & \lemref{i}
    }
    which as before is the same as the fourth set in the sequence.
    Hence we have at most 7 distinct sets in this sequence for a total of 14 potentially distinct sets as desired.
  }

  Note that this only shows that there can be \emph{no more than} 14 distinct sets.
  It could be that there are always less than 14 in general.
  While there are certainly sets that generate less than 14 distinct sets, the next part shows the existence of a topology and a set that does result in 14 distinct sets.
  This of course shows that 14 is the lowest possible bound in general.

  (b) We claim that $A = (-3, -2) \cup (-2, -1)  \cup ([0,1] \cap \rats) \cup \braces{2}$ in the standard topology on $\reals$ is a set that results in 14 distinct sets when the operational sequences from part~(a) are applied.
  We do not prove each sequential operation as this is easy but would be prohibitively tedious.
  First we enumerate the first sequence, starting with $A$.
  \begin{center}
    \begin{tabular}{c|c}
      Operations & Set \\
      \hline
      $A$ & $(-3, -2) \cup (-2, -1) \cup ([0,1] \cap \rats) \cup \braces{2}$ \\
      $cA$ & $[-3,-1] \cup [0,1] \cup \braces{2}$ \\
      $xcA = ixA$ & $(-\infty, -3) \cup (-1, 0) \cup (1,2) \cup (2, \infty)$ \\
      $cxcA = cixA$ & $\opcl{-\infty, -3} \cup [-1,0] \cup \clop{1, \infty}$ \\
      $xcxcA = icA$ & $(-3, -1) \cup (0, 1)$ \\
      $cxcxcA = cicA$ & $[-3, -1] \cup [0, 1]$ \\
      $xcxcxcA = icixA$ & $(-\infty, -3) \cup (-1,0) \cup (1, \infty)$
    \end{tabular}
  \end{center}
  Next we enumerate the next sequence of 7 sets, starting with $xA$:
  \begin{center}
    \begin{tabular}{c|c}
      Operations & Set \\
      \hline
      $xA$ & $\opcl{-\infty,-3} \cup \braces{-2} \cup \clop{-1,0} \cup \parens{(0,1) - \rats} \cup (1,2) \cup (2, \infty)$ \\
      $cxA$ & $\opcl{-\infty -3} \cup \braces{-2} \cup \clop{-1, \infty}$ \\
      $xcxA = iA$ & $(-3, -2) \cup (-2, -1)$ \\
      $cxcxA = ciA$ & $[-3, -1]$ \\
      $xcxcxA = icxA$ & $(-\infty, -3) \cup (-1, \infty)$ \\
      $cxcxcxA = cicxA$ & $\opcl{-\infty, -3} \cup \clop{-1, \infty}$ \\
      $xcxcxcxA = iciA$ & $(-3, 1)$
    \end{tabular}
  \end{center}
  It is easy to see that these are 14 distinct sets.

  We do note that, in an interval containing only rationals (or only irrationals), such as $[0,1] \cap \rats$ used as part of $A$, clearly every point in the interval is a limit point, including any irrational (or rational) points.
  This is because any open interval containing any real always contains both rationals and irrationals on account of $\rats$ being order-dense in $\reals$.
  For the same reason no point of such an interval of rationals (or irrationals) is in its interior.
  If, for example $C = [0,1] \cap \rats$, this clearly then results in $cC = [0,1]$ and $iC = \es$.
  Indeed this property of this part of $A$ is crucial in its success in generating 14 distinct sets.
}
