\setcounter{subsection}{9-1}
\subsection{Infinite Sets and the Axiom of Choice}

\exercise{1}{
  Define an injective map $f: \pints \to X^\w$, where $X$ is the two-element set $\braces{0,1}$, without using the choice axiom.
}
\sol{
  \dwhitman

  For any $n \in \pints$, define
  \gath{
    x_i = \begin{cases}
      0 & i \neq n \\
      1 & i = n
    \end{cases}
  }
  for $i \in \pints$.
  Then set $f(n) = \vx = (x_1, x_2, \ldots)$ so that clearly $f$ is a function from $\pints$ to $X^\w$.
  It is easy to show that $f$ is injective.
  \qproof{
    Consider $n,m \in \pints$ where $n \neq m$.
    Then let $\vx = f(n)$ and $\vy = f(m)$.
    Then we have that $x_n = 1$ while $y_n = 0$ by the definition of $f$ since $n \neq m$.
    It thus follows that $f(n) = \vx \neq \vy = f(m)$, which shows that $f$ is injective since $n$ and $m$ were arbitrary.
  }
}

\newcommand\col[1]{\mathcal{#1}}
\exercise{2}{
  Find if possible a choice function for each of the following collections, without using the choice axiom:
  \eparts{
  \item The collection $\col{A}$ of nonempty subsets of $\pints$.
  \item The collection $\col{B}$ of nonempty subsets of $\ints$.
  \item The collection $\col{C}$ of nonempty subsets of the rational numbers $\rats$.
  \item The collection $\col{D}$ of nonempty subsets of $X^\w$, where $X = \braces{0,1}$.
  }
}
\sol{
  \dwhitman

  \begin{lem}\label{lem:choice:countable}
    If $A$ is a countable set and $\col{A}$ is the collection of nonempty subsets of $A$ then $\col{A}$ has a choice function.
  \end{lem}
  \qproof{
    Since $A$ is countable, there is an injective function $A \to \pints$ by Theorem~7.1.
    We define a choice function $c : \col{A} \to \bigcup_{B \in \col{A}} B$.
    Consider any $X \in \col{A}$ so that $X$ is a nonempty subset of $A$.
    Then $f(X)$ is a nonempty subset of $\pints$ so that it has a unique smallest element $n$ since $\pints$ is well-ordered.
    Now, since $n \in f(X)$, clearly there is an $x \in X$ such that $f(x) = n$.
    Moreover, it follows from the fact that $f$ is injective that this $x$ is unique.
    So set $c(X) = x$ so that clearly $x$ is a choice function on $\col{A}$ since $c(X) = x \in X$.
  }

  \mainprob
  
  (a) Since $\pints$ is countable, a choice function can be constructed as in Lemma~\ref{lem:choice:countable}.

  (b) Since $\ints$ is countable (by Example~7.1), a choice function can be constructed as in Lemma~\ref{lem:choice:countable}.

  (c) Since $\rats$ is countable (by Exercise~7.1), a choice function can be constructed as in Lemma~\ref{lem:choice:countable}.

  (d) First, there is an injective function $f$ from the real interval $[0,1]$ to $X^\w$.
  The most straightforward such function is, for each $x \in [0,1]$ let $0.x_1 x_2 x_3 \ldots$ be a unique binary expansion of $x$ (these can be made unique by avoiding binary expansions that end in all 1's, noting though that the expansion of $1$ itself must be $0.111\ldots$).
  So suppose that $c$ were a choice function on $\col{D}$ (that is presumably constructed without the choice axiom).
  If $X$ is a nonempty subset of $[0,1]$ then $f(X)$ is a set in $\col{D}$ so that we can choose $c(f(X)) \in f(X)$.
  Since $f$ is injective, there is a unique $x \in X$ where $f(x) = c(f(X))$, and so choosing $x$ results in a choice function on the collection of nonempty subsets of $[0,1]$ since $X$ was arbitrary.

  This would allow one to then well-order $[0,1]$ without using the choice axiom, which evidently nobody has done.
  As far as I have been able to determine, this has not yet been proven impossible, it is just that nobody has been able to do it.
  So it would seem that such an explicit construction of a choice function on $\col{D}$ would at least make one famous.
  Or else it is impossible, which is what we assume to be the case here.
}

\exercise{3}{
  Suppose that $A$ is a set and $\braces{f_n}_{n \in \pints}$ is a given indexed family of injective functions
  \gath{
    f_n : \intsfin{n} \to A \,.
  }
  Show that $A$ is infinite.
  Can you define an injective function $f : \pints \to A$ without using the choice axiom?
}
\sol{
  \dwhitman

  We defer the proof that $A$ is infinite until we define an injective $f : \pints \to A$, which we can do without using the choice axiom by using the principle of recursive definition.
  \qproof{
    First, let $a_0 = f_1(1) \in A$.
    Now consider any function $g : S_n \to A$.
    If $g = \es$ then set $\r(\es) = \r(g) = a_0$.
    Otherwise let $I_g = \braces{i \in S_{n+1} \where f_n(i) \notin g(S_n)}$.
    Suppose for the moment that $I_g = \es$.
    Consider any $x \in f_n(S_{n+1})$ so that there is a $k \in S_{n+1}$ where $f_n(k) = x$.
    Then it has to be that $x = f_n(k) \in g(S_n)$ since otherwise we would have $k \in I_g$.
    Since $x$ was arbitrary, this shows that $f_n(S_{n+1}) \ss g(S_n)$.
    Thus the identity function $h_1 : f_n(S_{n+1}) \to g(S_n)$ is an injection.
    Clearly $g$ is a surjection from $S_n$ to its image $g(S_n)$ so that there is we can construct a particular injection $h_2 : g(S_n) \to S_n$ by Corollary~6.7.
    Lastly, $f_n$ is an injection from $S_{n+1}$ to $f_n(S_{n+1})$.
    Therefore $h = h_2 \circ h_1 \circ f_n$ is an injection from $S_{n+1}$ to $S_n$.
    Hence $h$ is a bijection from $S_{n+1}$ to $h(S_{n+1})$, which is clearly a subset of $S_n$ since $S_n$ is the range of $h$.
    But, since $S_n \pss S_{n+1}$, clearly $h(S_{n+1}) \pss S_{n+1}$ as well so that $h$ is a bijection from $S_{n+1}$ onto a proper subset of itself.
    As $S_{n+1}$ is clearly finite, this violates Corollary~6.3 so that we have a contradiction.
    
    So it must be that $I_g \neq \es$ so that it is a nonempty set of positive integers, and hence has a smallest element $i$.
    So simply set $\r(g) = f_n(i)$.
    Now, it then follows from the principle of recursive definition that there is a unique $f: \pints \to A$ such that
    \ali{
      f(1) &= a_0 \,, \\
      f(n) &= \r(f \rest S_n) \condgap \text{for $n > 1$.} 
    }
    We claim that this $f$ is injective.

    To see this we first show that $f(n) \notin f(S_n)$ for all $n \in \pints$.
    If $n = 1$ we have that $f(n) = f(1) = a_0 = f_1(1)$ and $f(S_n) = f(\es) = \es$ so that clearly the result holds.
    If $n > 1$ then $f(n) = \r(f \rest S_n) = f_n(i)$ for some $i \in I_{f \rest S_n}$ since clearly $S_n \neq \es$ so that $f \rest S_n \neq \es$.
    Since $i \in I_{f \rest S_n}$ we have that $f(n) = f_n(i) \notin (f \rest S_n)(S_n) = f(S_n)$ as desired.
    This shows that $f$ is injective.
    For consider any $n,m \in \pints$ where $n \neq m$.
    Without loss of generality we can assume that $n < m$.
    Then clearly $f(n) \in f(S_m)$ since $n \in S_m$ since $n < m$.
    However, by what was just shown, we have $f(m) \notin f(S_m)$ so that it has to be that $f(n) \neq f(m)$.
    This shows $f$ to be injective since $n$ and $m$ were arbitrary.

    Lastly, since $f: \pints \to A$ is injective, it follows that $f$ is a bijection from $\pints$ to $f(\pints) \ss A$.
    Hence $f(\pints)$ is infinite since $\pints$ is, and since it is a subset of $A$, it has to be that $A$ is infinite as well.
  }
}
