\setcounter{subsection}{9-1}
\subsection{Infinite Sets and the Axiom of Choice}

\newcommand\col[1]{\mathcal{#1}}

\exercise{1}{
  Define an injective map $f: \pints \to X^\w$, where $X$ is the two-element set $\braces{0,1}$, without using the choice axiom.
}
\sol{
  \dwhitman

  For any $n \in \pints$, define
  \gath{
    x_i = \begin{cases}
      0 & i \neq n \\
      1 & i = n
    \end{cases}
  }
  for $i \in \pints$.
  Then set $f(n) = \vx = (x_1, x_2, \ldots)$ so that clearly $f$ is a function from $\pints$ to $X^\w$.
  It is easy to show that $f$ is injective.
  \qproof{
    Consider $n,m \in \pints$ where $n \neq m$.
    Then let $\vx = f(n)$ and $\vy = f(m)$.
    Then we have that $x_n = 1$ while $y_n = 0$ by the definition of $f$ since $n \neq m$.
    It thus follows that $f(n) = \vx \neq \vy = f(m)$, which shows that $f$ is injective since $n$ and $m$ were arbitrary.
  }
}

\exercise{2}{
  Find if possible a choice function for each of the following collections, without using the choice axiom:
  \eparts{
  \item The collection $\col{A}$ of nonempty subsets of $\pints$.
  \item The collection $\col{B}$ of nonempty subsets of $\ints$.
  \item The collection $\col{C}$ of nonempty subsets of the rational numbers $\rats$.
  \item The collection $\col{D}$ of nonempty subsets of $X^\w$, where $X = \braces{0,1}$.
  }
}
\sol{
  \dwhitman

  \begin{lem}\label{lem:choice:countable}
    If $A$ is a countable set and $\col{A}$ is the collection of nonempty subsets of $A$ then $\col{A}$ has a choice function.
  \end{lem}
  \qproof{
    Since $A$ is countable, there is an injective function $A \to \pints$ by Theorem~7.1.
    We define a choice function $c : \col{A} \to \bigcup_{B \in \col{A}} B$.
    Consider any $X \in \col{A}$ so that $X$ is a nonempty subset of $A$.
    Then $f(X)$ is a nonempty subset of $\pints$ so that it has a unique smallest element $n$ since $\pints$ is well-ordered.
    Now, since $n \in f(X)$, clearly there is an $x \in X$ such that $f(x) = n$.
    Moreover, it follows from the fact that $f$ is injective that this $x$ is unique.
    So set $c(X) = x$ so that clearly $x$ is a choice function on $\col{A}$ since $c(X) = x \in X$.
  }

  \mainprob
  
  (a) Since $\pints$ is countable, a choice function can be constructed as in Lemma~\ref{lem:choice:countable}.

  (b) Since $\ints$ is countable (by Example~7.1), a choice function can be constructed as in Lemma~\ref{lem:choice:countable}.

  (c) Since $\rats$ is countable (by Exercise~7.1), a choice function can be constructed as in Lemma~\ref{lem:choice:countable}.

  (d) First, there is an injective function $f$ from the real interval $[0,1]$ to $X^\w$.
  The most straightforward such function is, for each $x \in [0,1]$ let $0.x_1 x_2 x_3 \ldots$ be a unique binary expansion of $x$ (these can be made unique by avoiding binary expansions that end in all 1's, noting though that the expansion of $1$ itself must be $0.111\ldots$).
  So suppose that $c$ were a choice function on $\col{D}$ (that is presumably constructed without the choice axiom).
  If $X$ is a nonempty subset of $[0,1]$ then $f(X)$ is a set in $\col{D}$ so that we can choose $c(f(X)) \in f(X)$.
  Since $f$ is injective, there is a unique $x \in X$ where $f(x) = c(f(X))$, and so choosing $x$ results in a choice function on the collection of nonempty subsets of $[0,1]$ since $X$ was arbitrary.

  This would allow one to then well-order $[0,1]$ without using the choice axiom, which evidently nobody has done.
  As far as I have been able to determine, this has not yet been proven impossible, it is just that nobody has been able to do it.
  So it would seem that such an explicit construction of a choice function on $\col{D}$ would at least make one famous.
  Or else it is impossible, which is what we assume to be the case here.
}

\exercise{3}{
  Suppose that $A$ is a set and $\braces{f_n}_{n \in \pints}$ is a given indexed family of injective functions
  \gath{
    f_n : \intsfin{n} \to A \,.
  }
  Show that $A$ is infinite.
  Can you define an injective function $f : \pints \to A$ without using the choice axiom?
}
\sol{
  \dwhitman

  We defer the proof that $A$ is infinite until we define an injective $f : \pints \to A$, which we can do without using the choice axiom by using the principle of recursive definition.
  \qproof{
    First, let $a_0 = f_1(1) \in A$.
    Now consider any function $g : S_n \to A$.
    If $g = \es$ then set $\r(\es) = \r(g) = a_0$.
    Otherwise let $I_g = \braces{i \in S_{n+1} \where f_n(i) \notin g(S_n)}$.
    Suppose for the moment that $I_g = \es$.
    Consider any $x \in f_n(S_{n+1})$ so that there is a $k \in S_{n+1}$ where $f_n(k) = x$.
    Then it has to be that $x = f_n(k) \in g(S_n)$ since otherwise we would have $k \in I_g$.
    Since $x$ was arbitrary, this shows that $f_n(S_{n+1}) \ss g(S_n)$.
    Thus the identity function $h_1 : f_n(S_{n+1}) \to g(S_n)$ is an injection.
    Clearly $g$ is a surjection from $S_n$ to its image $g(S_n)$ so that there is we can construct a particular injection $h_2 : g(S_n) \to S_n$ by Corollary~6.7.
    Lastly, $f_n$ is an injection from $S_{n+1}$ to $f_n(S_{n+1})$.
    Therefore $h = h_2 \circ h_1 \circ f_n$ is an injection from $S_{n+1}$ to $S_n$.
    Hence $h$ is a bijection from $S_{n+1}$ to $h(S_{n+1})$, which is clearly a subset of $S_n$ since $S_n$ is the range of $h$.
    But, since $S_n \pss S_{n+1}$, clearly $h(S_{n+1}) \pss S_{n+1}$ as well so that $h$ is a bijection from $S_{n+1}$ onto a proper subset of itself.
    As $S_{n+1}$ is clearly finite, this violates Corollary~6.3 so that we have a contradiction.
    
    So it must be that $I_g \neq \es$ so that it is a nonempty set of positive integers, and hence has a smallest element $i$.
    So simply set $\r(g) = f_n(i)$.
    Now, it then follows from the principle of recursive definition that there is a unique $f: \pints \to A$ such that
    \ali{
      f(1) &= a_0 \,, \\
      f(n) &= \r(f \rest S_n) \condgap \text{for $n > 1$.} 
    }
    We claim that this $f$ is injective.

    To see this we first show that $f(n) \notin f(S_n)$ for all $n \in \pints$.
    If $n = 1$ we have that $f(n) = f(1) = a_0 = f_1(1)$ and $f(S_n) = f(\es) = \es$ so that clearly the result holds.
    If $n > 1$ then $f(n) = \r(f \rest S_n) = f_n(i)$ for some $i \in I_{f \rest S_n}$ since clearly $S_n \neq \es$ so that $f \rest S_n \neq \es$.
    Since $i \in I_{f \rest S_n}$ we have that $f(n) = f_n(i) \notin (f \rest S_n)(S_n) = f(S_n)$ as desired.
    This shows that $f$ is injective.
    For consider any $n,m \in \pints$ where $n \neq m$.
    Without loss of generality we can assume that $n < m$.
    Then clearly $f(n) \in f(S_m)$ since $n \in S_m$ since $n < m$.
    However, by what was just shown, we have $f(m) \notin f(S_m)$ so that it has to be that $f(n) \neq f(m)$.
    This shows $f$ to be injective since $n$ and $m$ were arbitrary.

    Lastly, since $f: \pints \to A$ is injective, it follows that $f$ is a bijection from $\pints$ to $f(\pints) \ss A$.
    Hence $f(\pints)$ is infinite since $\pints$ is, and since it is a subset of $A$, it has to be that $A$ is infinite as well.
  }
}

\exercise{4}{
  There was a theorem in \S 7 whose proof involved an infinite number of arbitrary choices.
  Which one was it?
  Rewrite the proof so as to make explicit use of the choice axiom.
  (Several of the earlier exercises have used the choice axiom also.)
}
\sol{
  \dwhitman

  This was the proof of Theorem~7.5, which asserts that a countable union of countable sets is also countable.
  The following rewritten proof makes explicit use of the choice axiom and so points out where it is needed.
  \qproof{
    Let $\braces{A_n}_{n \in J}$ be an indexed family of countable sets, where the index set $J$ is $\intsfin{N}$ or $\pints$.
    Assume that each set $A_n$ is nonempty for convenience since this does not change anything.
    Now, for each $n \in J$, let $B_n$ be the set of surjective functions from $\pints$ to $A_n$.
    Since each $A_n$ is countable, it follows from Theorem~7.1 that $B_n \neq \es$.
    Then, by the axiom of choice, the collection $\braces{B_n}_{n \in J}$ has a choice function $c$ such that $c(B_n) \in B_n$ for every $n \in J$.

    Now set $f_n = c(B_n)$ for every $n \in J$ so that $f_n \in B_n$ and hence is a surjection from $\pints$ into $A_n$.
    Since $J$ is countable, there is also a surjection $g : \pints \to J$ by Theorem~7.1.
    Then define $h : \pints \times \pints \to \bigcup_{n \in J} A_n$ by $h(k,m) = f_{g(k)}(m)$ for $k,m \in \pints$.

    We now show that $h$ is surjective.
    So consider any $a \in \bigcup_{n \in J} A_n$ so that $a \in A_n$ for some $n \in J$.
    Since $g : \pints \to J$ is surjective, there is a $k \in \pints$ where $g(k) = n$.
    Also, since $f_n : \pints \to A_n$ is surjective, there is an $m \in \pints$ where $f_n(m) = a$.
    We then have by definition that
    \gath{
      h(k,m) = f_{g(k)}(m) = f_n(m) = a \,,
    }
    which shows that $h$ is surjective since $a$ was arbitrary.

    Lastly, since $\pints \times \pints$ is countable by Example~7.2, there is a bijection $h' : \pints \to \pints \times \pints$.
    It then follows that $h \circ h'$ is a surjection from $\pints$ to $\bigcup_{n \in J} A_n$, which shows that $\bigcup_{n \in J} A_n$ is countable again by Theorem~7.1.
  }
}

\exercise{5}{
  \eparts{
  \item Use the choice axiom to show that if $f : A \to B$ is surjective, then $f$ has a right inverse $h : B \to A$.
  \item Show that if $f : A \to B$ is injective and $A$ is not empty, then $f$ has a left inverse.
    Is the axiom of choice needed?
  }
}
\sol{
  \dwhitman

  (a)
  \qproof{
    Suppose that $f: A \to B$ is surjective.
    Now, by the choice axiom, the collection $\col{A} = \pset{A} - \braces{\es}$ is a collection of nonempty sets and thus has a choice function $c$.
    Consider any $b \in B$ and the set $A_b = \braces{x \in A \where f(x) = b}$.
    Then $A_b \neq \es$ since $f$ is surjective, and hence $A_b \in \col{A}$ since clearly also $A_b \ss A$ so that $A_b \in \pset{A}$.
    So set $h(b) = c(A_b) \in A_b$ so that $h(b) \in A$ since $A_b \ss A$.
    Hence $h$ is a function from $B$ to $A$.

    Recall that, by definition, $h$ is a right inverse if and only if $f \circ h = i_B$, which we show presently.
    So consider any $b \in B$ and let $a = h(b) = c(A_b) \in A_b$ so that $f(a) = b$.
    Then clearly
    \gath{
      (f \circ h)(b) = f(h(b)) = f(a) = b \,,
    }
    which shows that $f \circ h = i_B$ since $b$ was arbitrary.
    Hence $h$ is a right inverse of $f$.
  }

  (b)
  \qproof{
    Suppose that $f : A \to B$ is injective and $A \neq \es$.
    Then $f$ is a bijection from $A$ to its image $f(A) \ss B$ and hence its inverse $\inv{f}$ is a function from $f(A)$ to $A$.
    Now, since $A$ is nonempty, there is an $a_0 \in A$.
    So define $h: B \to A$ by
    \gath{
      h(b) = \begin{cases}
        \inv{f}(b) & b \in f(A) \\
        a_0 & b \notin f(A)
      \end{cases}
    }
    for any $b \in B$.
    Recall that $h$ is a left inverse of $f$ if and only if $h \circ f = i_A$ by definition, which we show now.

    So consider any $a \in A$ and let $b = f(a)$ so that clearly $b \in f(A)$.
    Hence by definition $h(b) = \inv{f}(b) = \inv{f}(f(a)) = a$.
    Finally, we have
    \gath{
      (h \circ f)(a) = h(f(a)) = h(b) = a \,.
    }
    This shows that $h \circ f = i_A$ since $a$ was arbitrary.
    Therefore $h$ is a left inverse of $f$ as desired.
  }

  Note that this proof does not require the axiom of choice as we did not need to make a choice for each $b \in B$ in order to define $h$ as we did in part $A$.
}

\def\ca{\col{A}}
\def\cb{\col{B}}
\exercise{6}{
  Most of the famous paradoxes of naive set theory are associated in some way or another with the concept of the ``set of all sets.''
  None of the rules we have given for forming sets allows us to consider such a set.
  And for good reason -- the concept itself is self-contradictory.
  For suppose that $\ca$ denotes the ``set of all sets.''
  \eparts{
  \item Show that $\pset{\ca} \ss \ca$; derive a contradiction.
  \item (\emph{Russell's paradox.}) Let $\cb$ be the subset of $\ca$ consisting of all sets that are not elements of themselves:
    \gath{
      \cb = \braces{A \where \text{$A \in \ca$ and $A \notin A$}} \,.
    }
    (Of course, there may be \emph{no} set $A$ such that $A \in A$;
    If such is the case, then $\cb = \ca$.)
    Is $\cb$ an element of itself or not?
  }
}
\sol{
  \dwhitman

  (a) We show that $\pset{\ca} \ss \ca$ and that a contradiction results.
  \qproof{
    Consider any set $A \in \pset{\ca}$.
    Since $A$ is a set and $\ca$ is the set of all sets, clearly $A \in \ca$ and hence $\pset{\ca} \ss \ca$ since $A$ was arbitrary.
    Therefore the identity function $i_{\pset{\ca}}$ is clearly an injection from $\pset{\ca}$ to $\ca$.
    However, this is impossible by Theorem~7.8!
    Hence we have reached a contradiction.
  }

  (b) We show that the existence of $\cb$ is a contradiction by showing that supposing either $\cb \in \cb$ or $\cb \notin \cb$ results in a contradiction.
  \qproof{
    Suppose that $\cb \in \cb$ so that by definition we have $\cb \in \ca$ and $\cb \notin \cb$, the latter of which clearly contradicts our initial supposition.
    On the other hand, suppose that $\cb \notin \cb$.
    Then, since clearly also $\cb \in \ca$ since it is a set, it follows that $\cb \in \cb$ by definition.
    This again contradicts the initial supposition.
    Since one or the other ($\cb \in \cb$ or $\cb \notin \cb$) must be true, we are then guaranteed to have a contradiction.
  }
}

\exercise{7}{
  Let $A$ and $B$ be two nonempty sets.
  If there is an injection of $B$ into $A$, but no injection of $A$ into $B$, we say that $A$ has \boldit{greater cardinality} than $B$.
  \eparts{
  \item Conclude from Theorem~9.1 that every uncountable set has greater cardinality than $\pints$.
  \item Show that if $A$ has greater cardinality than $B$, and $B$ has greater cardinality than $C$, then $A$ has greater cardinality than $C$.
  \item Find a sequence $A_1, A_2, \ldots$ of infinite sets, such that for each $n \in \pints$, the set $A_{n+1}$ has greater cardinality than $A_n$.
  \item Find a set that for every $n$ has cardinality greater than $A_n$.
  }
}
\sol{
  \dwhitman

  \begin{lem}\label{lem:choice:pset}
    For any set $A$, $\pset{A}$ has greater cardinality than $A$.
  \end{lem}
  \qproof{
    Clearly the function that maps $a \in A$ to $\braces{a} \in \pset{A}$ is an injection.
    However, we know from Theorem~7.8 that there is no injection from $\pset{A}$ to $A$.
    Together these show that $\pset{A}$ has greater cardinality than $A$ as desired.
  }

  \mainprob

  (a)
  \qproof{
    Suppose that $A$ is any uncountable set.
    Clearly $A$ is not finite for then it would be countable.
    Hence it is infinite and so there is a injection from $\pints$ to $A$ by Theorem~9.1.
    There also cannot be an injection from $A$ to $\pints$, for if there were then $A$ would be countable by Theorem~7.1.
    This shows that $A$ has greater cardinality than $\pints$ by definition.
  }

  (b)
  \qproof{
    Since $A$ has greater cardinality than $B$, there is an injection $f: B \to A$.
    Likewise, since $B$ has greater cardinality than $C$, there is an injection $g : C \to B$.
    It then follows that $f \circ g$ is an injection of $C$ into $A$.
    Now suppose that $h : A \to C$ is injective.
    Then $g \circ h$ would be an injection of $A$ into $B$, which we know cannot exist since $A$ has greater cardinality than $B$.
    Hence it must be that no such injection $h$ exists, which shows that $A$ has greater cardinality than $C$ as desired.
  }

  (c)
  We define a sequence of sets recursively:
  \ali{
    A_1 &= \pints \,, \\
    A_n &= \pset{A_{n-1}} \condgap \text{for $n > 1$.}
  }
  We show that this meets the requirements.
  \qproof{
    First we show that each $A_n$ is infinite by induction.
    Clearly $A_1 = \pints$ is infinite.
    Now assume that $A_n$ is infinite for $n \in \pints$ so that there is an injection $f: \pints \to A_n$ by Theorem~9.1.
    Then, by Lemma~\ref{lem:choice:pset}, $A_{n+1} = \pset{A_n}$ has greater cardinality than $A_n$ so that there is an injection $g: A_n \to A_{n+1}$.
    Then $g \circ f$ is an injection from $\pints$ to $A_{n+1}$ so that $A_{n+1}$ is infinite as well by Theorem~9.1.
    This completes the induction.

    Finally, for any $n \in \pints$ we have that $n+1 > 1$ so that $A_{n+1} = \pset{A_{(n+1)-1}} = \pset{A_n}$.
    Then clearly $A_{n+1}$ has greater cardinality than $A_{n}$ by Lemma~\ref{lem:choice:pset}.
    This shows the desired result.
  }

  (d)
  Let $A = \bigcup_{n \in \pints} A_n$, which we claim has the required property.
  \qproof{
    Consider any $n \in \pints$.
    Clearly $A_n \ss A$ so that the identity function $i_{A_n}$ is an injection of $A_n$ into $A$.
    Now suppose for the moment that $g : A \to A_n$ is injective.
    Since clearly also $A_{n+1} \ss A$, it follows that $g \rest A_{n+1}$ is then an injection of $A_{n+1}$ into $A_n$.
    However this contradicts the proven fact that $A_{n+1}$ has greater cardinality than $A_n$.
    Hence it has to be that no such injection $g$ exists, which shows that $A$ has greater cardinality than $A_n$.
    Since $n$ was arbitrary, this shows the desired result.
  }
}

\exercise{8}{
  Show that $\pset{\pints}$ and $\reals$ have the same cardinality.
  [Hint: You may use the fact that every real number has a decimal expansion, which is unique if expansions that end in an infinite string of 9's are forbidden.]

  A famous conjecture of set theory, called the \emph{continuum hypothesis}, asserts that there exists no set having cardinality greater than $\pints$ and lesser cardinality than $\reals$.
  The \emph{generalized continuum hypothesis} asserts that, given the infinite set $A$, there is no set having greater cardinality than $A$ and lesser cardinality than $\pset{A}$.
  Surprisingly enough, both of these assertions have been shown to be independent of the usual axioms of set theory.
  For a readable expository account, see [Sm].
}
\sol{
  \dwhitman

  \begin{lem}\label{lem:choice:psetsame}
    If $A$ and $B$ are sets with the same cardinality, then $\pset{A}$ and $\pset{B}$ have the same cardinality.
  \end{lem}
  \qproof{
    Since $A$ and $B$ have the same cardinality there is a bijection $f : A \to B$.
    We define a bijection $g : \pset{A} \to \pset{B}$.
    So, for any $X \in \pset{A}$, set $g(X) = f(X)$.
    Clearly $f(X) \ss B$, since $B$ is the range of $f$, so that $g(X) = f(X) \in \pset{B}$ and hence $\pset{B}$ can be the range of $g$.

    To show that $g$ is injective, consider sets $X$ and $Y$ in $\pset{A}$ so that $X,Y \ss A$.
    Also suppose that $X \neq Y$ so, without loss of generality, we can assume that there is an $x \in X$ where $x \notin Y$.
    Clearly $f(x) \in f(X)$ since $x \in X$.
    Were it the case that $f(x) \in f(Y)$ then there would be a $y \in Y$ such that $f(y) = f(x)$.
    But then we would have that $y = x$ since $f$ is injective and hence $x = y \in Y$, which we know not to be the case.
    Hence $f(x) \notin f(Y)$ so that it has to be that $g(X) = f(X) \neq f(Y) = g(Y)$ since $f(x) \in f(X)$.
    Since $X$ and $Y$ were arbitrary this shows that $g$ is injective.

    To show that $g$ is surjective consider any $Y \in \pset{B}$ so that $Y \ss B$.
    Let $X = \inv{f}(Y)$, noting that $\inv{f}$ is a bijection from $B$ to $A$ since $f$ is bijective.
    Clearly $X \ss A$ since $A$ is the range of $\inv{f}$ so that $X \in \pset{A}$.
    Now consider any $y \in f(X)$ so that there is an $x \in X$ where $f(x) = y$.
    Then, since $X = \inv{f}(Y)$, there is a $y' \in Y$ where $x = \inv{f}(y')$, and hence $y = f(x) = f(\inv{f}(y')) = y'$.
    Thus $y = y' \in Y$ so that $f(X) \ss Y$ since $y$ was arbitrary.
    Now consider $y \in Y$ and let $x = \inv{f}(y)$ so that clearly $x = \inv{f}(y) \in \inv{f}(Y) = X$.
    Moreover, $f(x) = f(\inv{f}(y)) = y$ so that $y \in f(X)$.
    Thus $Y \ss f(X)$ as well since $y$ was arbitrary.
    This shows that $g(X) = f(X) = Y$, from which we conclude that $g$ is surjective since $Y$ was arbitrary.

    Hence $g: \pset{A} \to \pset{B}$ is a bijection so that $\pset{A}$ and $\pset{B}$ have the same cardinality by definition.
  }

  \mainprob

  \qproof{
    We show this using the Cantor-Schroeder-Bernstein (CSB) Theorem, which was proven in Exercise~7.6b.

    First, we construct an injective function $f$ from $\reals$ to $\pset{\rats}$.
    For any $x \in \reals$ let $Q = \braces{q \in \rats \where q < x}$ so that clearly $Q \ss \rats$ and hence $Q \in \pset{\rats}$.
    Therefore setting $f(x) = Q$ means that $f$ is a function from $\reals$ to $\pset{\rats}$.
    To show that $f$ is injective consider $x,y \in \reals$ where $x \neq y$.
    Without loss of generality we can assume that $x < y$ so that there is a $q \in \rats$ where $x < q < y$ since the rationals are order-dense in the reals.
    Also set $Q = f(x)$ and $P = f(y)$.
    Since $q > x$ we have that $q \notin Q$.
    Analogously, since $q < y$ we have that $q \in P$.
    Thus it has to be that $f(x) = Q \neq P = f(y)$, which shows that $f$ is injective since $x$ and $y$ were arbitrary.

    Now, it was shown in Exercise~7.1 that $\rats$ is countably infinite and thus has the same cardinality as $\pints$.
    From Lemma~\ref{lem:choice:psetsame} it then follows that $\pset{\rats}$ has the same cardinality as $\pset{\pints}$ so that there is a bijection $g: \pset{\rats} \to \pset{\pints}$.
    Clearly then $g \circ f$ is an injection from $\reals$ to $\pset{\pints}$.

    Now let $X = \braces{0,1}$, and we construct an injection $h : X^\w \to \reals$.
    For any sequence $\vx = (x_1, x_2, \ldots) \in X^\w$ set $h(\vx)$ to the decimal expansion $0.x_1 x_2 x_3 \ldots$, where clearly each $x_n$ is the digit 0 or 1.
    Clearly $h(\vx)$ is a real number so that $h$ is a function from $X^\w$ to $\reals$.
    It is easy to see that $h$ is injective since different sequences will result in different decimal expansions.
    Since none of the expansions end in an infinite sequence of 9's, clearly the corresponding real numbers will be different.

    Now, it was shown in Exercise~7.3 that $\pset{\pints}$ and $X^\w$ have the same cardinality so that there is a bijection $i : \pset{\pints} \to X^\w$.
    It then follows that $h \circ i$ is an injection of $\pset{\pints}$ into $\reals$.
    Since we have shown the existence of both injections, the result follows from the CSB Theorem.
  }
}
