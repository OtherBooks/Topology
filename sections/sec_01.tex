\setcounter{subsection}{1-1}
\subsection{Fundamental Concepts}

\exercise{1}{
  Check the distributive laws for $\cup$ and $\cap$ and DeMorgan's laws.
}
\sol{
  \dwhitman

  Suppose that $A$, $B$, and $C$ are sets.
  First we show that $A \cap (B \cup C) = (A \cap B) \cup (A \cap C)$.
  \qproof{
    We show this as a series of logical equivalences:
    \ali{
      x \in A \cap (B \cup C) &\bic x \in A \land x \in B \cup C \\
      &\bic x \in A \land (x \in B \lor x \in C) \\
      &\bic (x \in A \land x \in B) \lor (x \in A \land x \in C) \\
      &\bic x \in A \cap B \lor x \in A \cap C \\
      &\bic x \in (A \cap B) \cup (A \cap C) \,,
    }
    which of course shows the desired result.
  }

  Next we show that $A \cup (B \cap C) = (A \cup B) \cap (A \cup C)$.
  \qproof{
    We show this in the same way:
    \ali{
      x \in A \cup (B \cap C) &\bic x \in A \lor x \in B \cap C \\
      &\bic x \in A \lor (x \in B \land x \in C) \\
      &\bic (x \in A \lor x \in B) \land (x \in A \lor x \in C) \\
      &\bic x \in A \cup B \land x \in A \cup C \\
      &\bic x \in (A \cup B) \cap (A \cup C) \,,
    }
    which of course shows the desired result.
  }

  Now we show the first DeMorgan's law that $A - (B \cup C) = (A - B) \cap (A - C)$.
  \qproof{
    We show this in the same way:
    \ali{
      x \in A - (B \cup C) &\bic x \in A \land x \notin B \cup C \\
      &\bic x \in A \land \lnot (x \in B \lor x \in C) \\
      &\bic x \in A \land (x \notin B \land x \notin C) \\
      &\bic (x \in A \land x \notin B) \land (x \in A \land x \notin C) \\
      &\bic x \in A - B \land x \in A - C \\
      &\bic x \in (A - B) \cap (A - C) \,,
    }
    which is the desired result.
  }

  Lastly we show that $A - (B \cap C) = (A - B) \cup (A - C)$.
  \qproof{
    Again we use a sequence of logical equivalences:
    \ali{
      x \in A - (B \cap C) &\bic x \in A \land x \notin B \cap C \\
      &\bic x \in A \land \lnot (x \in B \land x \in C) \\
      &\bic x \in A \land (x \notin B \lor x \notin C) \\
      &\bic (x \in A \land x \notin B) \lor (x \in A \land x \notin C) \\
      &\bic x \in A - B \lor x \in A - C \\
      &\bic x \in (A - B) \cup (A - C) \,,
    }
    as desired.
  }
}

\exercise{2}{
  Determine which of the following statements are true for all sets $A$, $B$, $C$, and $D$.
  If a double implication fails, determine whether one or the other of the possible implications holds.
  If an equality fails, determine whether the statement becomes true if the ``equals'' symbol is replaced by one or the other of the inclusion symbols $\ss$ or $\sps$.
  \begin{multicols}{2}
    \eparts{
    \item $A \ss B$ and $A \ss C \bic A \ss (B \cup C)$.
    \item $A \ss B$ or $A \ss C \bic A \ss (B \cup C)$.
    \item $A \ss B$ and $A \ss C \bic A \ss (B \cap C)$.
    \item $A \ss B$ or $A \ss C \bic A \ss (B \cap C)$.
    \item $A - (A - B) = B$.
    \item $A - (B - A) = A - B$.
    \item $A \cap (B - C) = (A \cap B) - (A \cap C)$.
    \item $A \cup (B - C) = (A \cup B) - (A \cup C)$.
    \item $(A \cap B) \cup (A - B) = A$.
    \item $A \ss C$ and $B \ss D \imp (A \times B) \ss (C \times D)$.
    \item The converse of (j).
    \item The converse of (j), assuming that $A$ and $B$ are nonempty.
    \item $(A \times B) \cup (C \times D) = (A \cup C) \times (B \cup D)$.
    \item $(A \times B) \cap (C \times D) = (A \cap C) \times (B \cap D)$.
    \item $A \times (B - C) = (A \times B) - (A \times C)$.
    \item $(A - B) \times (C - D) = (A \times C - B \times C) - A \times D$.
    \item $(A \times B) - (C \times D) = (A - C) \times (B - D)$.
    }
  \end{multicols}
}
\sol{
  \dwhitman

  (a) We claim that $A \ss B$ and $A \ss C \imp A \ss (B \cup C)$ but that the converse is not generally true.
  \qproof{
    Suppose that $A \ss B$ and $A \ss C$ and consider any $x \in A$.
    Then clearly also $x \in B$ since $A \ss B$ so that $x \in B \cup C$.
    Since $x$ was arbitrary, this shows that $A \ss (B \cup C)$ as desired.

    To show that the converse is not true, suppose that $A = \braces{1,2,3}$, $B = \braces{1,2}$, and $C = \braces{3, 4}$.
    Then clearly $A \ss \braces{1,2,3,4} = B \cup C$ but it neither true that $A \ss B$ (since $3 \in A$ but $3 \notin B$) nor $A \ss C$ (since $1 \in A$ but $1 \notin C$).
  }

  (b) We claim that $A \ss B$ or $A \ss C \imp A \ss (B \cup C)$ but that the converse is not generally true.
  \qproof{
    Suppose that $A \ss B$ or $A \ss C$ and consider any $x \in A$.
    If $A \ss B$ then clearly $x \in B$ so that $x \in B \cup C$.
    If $A \ss C$ then clearly $x \in C$ so that again $x \in B \cup C$.
    Since $x$ was arbitrary, this shows that $A \ss (B \cup C)$ as desired.

    The counterexample that disproves the converse of part (a), also serves as a counterexample to the converse here.
    Again this is because $A \ss B \cup C$ but neither $A \ss B$ nor $A \ss C$, which is to say that $A \nss B$ and $A \nss C$.
    Hence it is not true that $A \ss B$ or $A \ss C$.
  }

  (c) We claim that this biconditional is true.
  \qproof{
    $(\imp)$ Suppose that $A \ss B$ and $A \ss C$ and consider any $x \in A$.
    Then clearly also $x \in B$ and $x \in C$ since both $A \ss B$ and $A \ss C$.
    Hence $x \in B \cap C$, which proves that $A \ss B \cap C$ since $x$ was arbitrary.

    $(\pmi)$ Now suppose that $A \ss B \cap C$ and consider any $x \in A$.
    Then $x \in B \cap C$ as well so that $x \in B$ and $x \in C$.
    Since $x$ was an arbitrary element of $A$, this of course shows that both $A \ss B$ and $A \ss C$ as desired.
  }

  (d) We claim that only the converse is true here.
  \qproof{
    To show the converse, suppose that $A \ss B \cap C$.
    It was shown in part (c) that this implies that both $A \ss B$ and $A \ss C$.
    Thus it is clearly true that $A \ss B$ or $A \ss C$.

    As a counterexample to the forward implication, let $A = \braces{1}$, $B = \braces{1,2}$, and $C = \braces{3,4}$ so that clearly $A \ss B$ and hence $A \ss B$ or $A \ss C$ is true.
    However we have that $B$ and $C$ are disjoint so that $B \cap C = \es$, therefore $A \nss \es = B \cap C$ since $A \neq \es$.
  }

  (e) We claim that $A - (A - B) \ss B$ but that the other direction is not generally true.
  \qproof{
    First consider any $x \in A - (A - B)$ so that $x \in A$ but $x \notin A - B$.
    Hence it is not true that $x \in A$ and $x \notin B$.
    So it must be that $x \notin A$ or $x \in B$.
    However, since we know that $x \in A$, it has to be that $x \in B$.
    Thus $A - (A - B) \ss B$ since $x$ was arbitrary.

    Now let $A = \braces{1,2}$ and $B = \braces{2,3}$.
    Then we clearly have $A - B = \braces{1}$, and thus $A - (A - B) = \braces{2}$.
    So clearly $B$ is not a subset of $A - (A - B)$ since $3 \in B$ but $3 \notin A - (A - B)$.
  }

  (f) Here we claim that $A - (B - A) \sps A - B$ but that the other direction is not generally true.
  \qproof{
    First suppose that $x \in A - B$ so that $x \in A$ but $x \notin B$.
    Then it is certainly true that $x \notin B$ or $x \in A$ so that, by logical equivalence, it is not true that $x \in B$ and $x \notin A$.
    That is, it is not true that $x \in B - A$, which is to say that $x \notin B - A$.
    Since also $x \in A$, it follows that $x \in A - (B - A)$, which shows the desired result  since $x$ was arbitrary.

    To show that the other direction does not hold consider the counterexample $A = \braces{1,2}$ and $B = \braces{2, 3}$.
    Then $B - A = \braces{3}$ so that $A - (B - A) = \braces{1, 2} = A$.
    We also have that $A - B = \braces{1}$ so that $2 \in A - (B - A)$ but $2 \notin A - B$.
    This suffices to show that $A - (B - A) \nss A - B$.
  }

  (g) We claim that equality holds here, i.e. that $A \cap (B - C) = (A \cap B) - (A \cap C)$.
  \qproof{
    $(\ss)$ Suppose that $x \in A \cap (B - C)$ so that $x \in A$ and $x \in B - C$.
    Thus $x \in B$ but $x \notin C$.
    Since both $x \in A$ and $x \in B$ we have that $x \in A \cap B$.
    Also since $x \notin C$ it clearly must be that $x \notin A \cap C$.
    Hence $x \in (A \cap B) - (A \cap C)$, which shows the forward direction since $x$ was arbitrary.

    $(\sps)$ Now suppose that $x \in (A \cap B) - (A \cap C)$.
    Hence $x \in A \cap B$ but $x \notin A \cap C$.
    From the former of these we have that $x \in A$ and $x \in B$, and from the latter it follows that either $x \notin A$ or $x \notin C$.
    Since we know that $x \in A$, it must therefore be that $x \notin C$.
    Hence $x \in B - C$ since $x \in B$ but $x \notin C$.
    Since also $x \in A$ we have that $x \in A \cap (B - C)$, which shows the desired result since $x$ was arbitrary.
  }

  (h) Here we claim that $A \cup (B - C) \sps (A \cup B) - (A \cup C)$ but that the forward direction is not generally true.
  \qproof{
    First consider any $x \in (A \cup B) - (A \cup C)$ so that $x \in A \cup B$ and $x \notin A \cup C$.
    From the latter, it follows that $x \notin A$ and $x \notin C$ since otherwise we would have $x \in A \cup C$.
    From the former, we have that $x \in A$ or $x \in B$ so that it must be that $x \in B$ since $x \notin A$.
    Therefore we have that $x \in B$ and $x \notin C$ so that $x \in B - C$.
    From this it obviously follows that $x \in A \cup (B - C)$, which shows that $A \cup (B - C) \sps (A \cup B) - (A \cup C)$ since $x$ was arbitrary.

    To show that the forward direction does not always hold, consider the sets $A = \braces{1,2}$, $B = \braces{2, 3}$, and $C = \braces{2}$.
    Then we clearly have that $B - C = \braces{3}$, and hence $A \cup (B - C) = \braces{1,2,3}$.
    On the other hand, we have $A \cup B = \braces{1,2,3}$ and $A \cup C = \braces{1,2}$ so that $(A \cup B) - (A \cup C) = \braces{3}$.
    Hence, for example, $1 \in A \cup (B - C)$ but $1 \notin (A \cup B) - (A \cup C)$, which suffices to show that $A \cup (B - C) \nss (A \cup B) - (A \cup C)$ as desired.
  }

  (i) We claim that equality holds here.
  \qproof{
    We show this with a chain of logical equivalences:
    \ali{
      x \in (A \cap B) \cup (A - B) &\bic x \in A \cap B \lor x \in A - B \\
      &\bic (x \in A \land x \in B) \lor (x \in A \land x \notin B) \\
      &\bic x \in A \land (x \in B \lor x \notin B) \\
      &\bic x \in A \land \mathrm{True} \\
      &\bic x \in A \,,
    }
    where we note that ``True'' denotes the fact that $x \in B \lor x \notin B$ is always true by the excluded middle property of logic.
  }

  (j) We claim that this implication is true.
  \qproof{
    Suppose that $A \ss C$ and $B \ss D$.
    Consider any $(x,y) \in A \times B$ so that $x \in A$ and $y \in B$ by the definition of the cartesian product.
    Then also clearly $x \in C$ and $y \in D$ since $A \ss C$ and $B \ss D$.
    Hence $(x,y) \in C \times D$, which shows the result since the ordered pair $(x,y)$ was arbitrary.
  }

  (k) We claim that the converse of (j) is not always true.
  \qproof{
    Consider the following sets:
    \ali{
      A &= \es & C &= \braces{1} \\
      B &= \braces{1,2} & D &= \braces{2} \,.
    }
    Then we have that $A \times B = \es$ since there are no ordered pairs $(x,y)$ such that $x \in A$ (since $A = \es$).
    Hence it is vacuously true that $(A \times B) \ss (C \times D)$.
    However, clearly it is not the case that $B \ss D$, and so, even though $A \ss C$, it is not true that $A \ss C$ and $B \ss D$.
  }

  (l) We claim that the converse of (j) is true with the stipulation that $A$ and $B$ are both nonempty.
  \qproof{
    Suppose that $(A \times B) \ss (C \times D)$.
    First consider any $x \in A$.
    Then, since $B \neq \es$, there is a $b \in B$.
    Then $(x,b) \in A \times B$ so that clearly also $(x,b) \in C \times D$.
    Hence $x \in C$ so that $A \ss C$ since $x$ was arbitrary.
    An analogous argument shows that $B \ss D$ since $A$ is nonempty.
    Hence it is true that $A \ss C$ and $B \ss D$ as desired.
  }

  (m) Here we claim that $(A \times B) \cup (C \times D) \ss (A \cup C) \times (B \cup D)$ but that the other direction is not always true.
  \qproof{
    First consider any $(x,y) \in (A \times B) \cup (C \times D)$ so that either $(x,y) \in A \times B$ or $(x,y) \in C \times D$.
    In the first case $x \in A$ and $y \in B$ so that clearly $x \in A \cup C$ and $y \in B \cup D$.
    Hence $(x, y) \in (A \cup C) \times (B \cup D)$.
    In the second case we have $x \in C$ and $y \in D$ so that again $x \in A \cup C$ and $y \in B \cup D$ are still both true.
    Hence of course $(x, y) \in (A \cup C) \times (B \cup D)$ here also.
    This shows the result in either case since $(x,y)$ was an arbitrary ordered pair.

    To show that the other direction does not always hold, consider $A = B = \braces{1}$ and $C = D = \braces{2}$.
    Then we clearly have $A \times B = \braces{(1,1)}$ and $C \times D = \braces{(2,2)}$ so that $(A \times B) \cup (C \times D) = \braces{(1,1), (2,2)}$.
    We also have $A \cup C = B \cup D = \braces{1,2}$ so that $(A \cup C) \times (B \cup D) = \braces{(1,1), (1,2), (2,1), (2,2)}$.
    This clearly shows that $(A \times B) \cup (C \times D) \not \sps (A \cup C) \times (B \cup D)$ as desired.
  }

  (n) We claim that the equality holds here.
  \qproof{
    We can show this by a series of logical equivalences:
    \ali{
      (x,y) \in (A \times B) \cap (C \times D) &\bic (x,y) \in A \times B \land (x,y) \in C \times D \\
      &\bic (x \in A \land y \in B) \land (x \in C \land y \in D) \\
      &\bic (x \in A \land x \in C) \land (y \in B \land y \in D) \\
      &\bic x \in A \cap C \land y \in B \cap D \\
      &\bic (x,y) \in (A \cap C) \times (B \cap D)
    }
    as desired.
  }

  (o) We claim that equivalence holds here as well.
  \qproof{
    $(\ss)$ First consider any $(x,y) \in A \times (B - C)$ so that $x \in A$ and $y \in B - C$.
    From the latter of these we have that $y \in B$ but $y \notin C$.
    We clearly then have that $(x,y) \in A \times B$ since $x\in A$ and $y \in B$.
    It also has to be that $(x,y) \notin A \times C$ since $y \notin C$ even though it is true that $x\ in A$.
    Therefore $(x,y) \in (A \times B) - (A \times C)$ as desired.

    $(\sps)$ Now suppose that $(x,y) \in (A \times B) - (A \times C)$ so that $(x,y) \in A \times B$ but $(x,y) \notin (A \times C)$.
    From the former we have that $x \in A$ and $y \in B$.
    It then must be that $y \notin C$ since $(x,y) \notin (A \times C)$ but we know that $x \in A$.
    Then we have $y \in B - C$ since $y \in B$ but $y \notin C$.
    Since also $x \in A$, it follows that $(x,y) \in A \times (B-C)$ as desired.
  }

  (p) We claim the equivalence hold for this statement.
  \qproof{
    $(\ss)$ Suppose that $(x,y) \in (A - B) \times (C - D)$ so that $x \in A - B$ and $y \in C - D$.
    Then we have that $x \in A$, $x \notin B$, $y \in C$, and $y \notin D$.
    So first, clearly $(x,y) \in A \times C$.
    Then, since $x \notin B$, we have that $(x,y) \notin B \times C$, and hence $(x,y) \in A \times C - B \times C$.
    Since $y \notin D$, we also have that $(x,y) \notin A \times D$, and thus $(x,y) \in (A \times C - B \times C) - A \times D$.
    This clearly shows the desired result since $(x,y)$ was arbitrary.

    $(\sps)$ Now suppose that $(x,y) \in (A \times C - B \times C) - A \times D$ so that $(x,y) \in A \times C - B \times C$ but $(x,y) \notin A \times D$.
    From the former we have that $(x,y) \in A \times C$ and $(x,y) \notin B \times C$.
    Thus $x \in A$ and $y \in C$ so that it has to be that $x \notin B$ since $(x,y) \notin B \times C$ but we know that $y \in C$.
    It also must be that $y \notin D$ since $(x,y) \notin A \times D$ but $x \in A$.
    Therefore we have that $x \in A$, $x \notin B$, $y \in C$, and $y \notin D$, from which it readily follows that $x \in A - B$ and $y \in C - D$.
    Thus clearly $(x,y) \in (A-B) \times (C-D)$, which shows the desired result since $(x,y)$ was arbitrary.
  }

  (q) Here we claim that $(A \times B) - (C \times D) \sps (A - C) \times (B - D)$ but that the forward direction is not true in general.
  \qproof{
    First consider any $(x,y) \in (A - C) \times (B - D)$ so that $x \in A - C$ and $y \in B - D$.
    Thus we have $x \in A$, $x \notin C$, $y \in B$, and $y \notin D$.
    From this clearly $(x,y) \in A \times B$ but $(x,y) \notin C \times D$.
    Hence $(x,y) \in (A \times B) - (C \times D)$, which clearly shows the desired result since $(x,y)$ was arbitrary.

    To show that the forward direction does not hold, consider $A = \braces{1,2}$, $B = \braces{a,b}$, $C = \braces{2,3}$, and $D = \braces{b,c}$.
    We then clearly have the following sets:
    \ali{
      A \times B &= \braces{(1,a), (1,b), (2,a), (2,b)} & A - C &= \braces{1} \\
      C \times D &= \braces{(2,b), (2,c), (3,b), (3,c)} & B - D &= \braces{a} \\
      (A \times B) - (C \times D) &= \braces{(1,a), (1,b), (2,a)} & (A - C) \times (B - D) &= \braces{(1,a)} \,.
    }
    This clearly shows that $(A \times B) - (C \times D)$ is not a subset of $(A - C) \times (B - D)$.
  }
}

\exercise{3}{
  \eparts{
  \item Write the contrapositive and converse of the following statement: ``If $x < 0$, then $x^2 - x > 0$,'' and determine which (if any) of the three statements are true.
  \item Do the same for the statement ``If $x > 0$, then $x^2 - x > 0$.''
  }
}
\sol{
  \dwhitman

  (a) First we claim that the original statement is true.
  \qproof{
    Since $x < 0$ we clearly have that $x-1 < x < 0$ as well.
    Then, since the product of two negative numbers is positive, we have that $x^2 - x = x(x-1) > 0$ as desired.
  }
  The contrapositive of this is, ``If $x^2 - x \leq 0$, then $x \geq 0$.''
  This is of course also true by virtue of the fact that the contrapositive is logically equivalent to the original implication.

  Lastly, the converse of this statement is, ``If $x^2 - x > 0$, then $x < 0$.''
  We claim that this is not generally true.
  \qproof{
    A simple counterexample of $x = 2$ shows this.
    We have $x^2 - x = 2^2 - 2 = 4 - 2 = 2 > 0$, but also clearly $x = 2 > 0$ as well so that $x < 0$ is clearly false.
  }

  (b) First we claim that this statement is false.
  \qproof{
    As a counterexample, let $x = 1/2$.
    Then clearly $x > 0$, but we also have $x^2 - x = (1/2)^2 - 1/2 = 1/4 - 1/2 = -1/4 < 0$ so that $x^2-x > 0$ is obviously not true.
  }
  The contrapositive is then ``If $x^2 - x \leq 0$, then $x \leq 0$,'' which is false since it is logically equivalent to the original statement.

  The converse is ``If $x^2 - x > 0$, then $x > 0$,'' which we claim is false.
  \qproof{
    As a counterexample, consider $x = -1$ so that $x^2 - x = (-1)^2 - (-1) = 1 + 1 = 2 > 0$.
    However, we also clearly have $x = -1 < 0$ so that $x > 0$ is not true.
  }
}

\exercise{4}{
  Let $A$ and $B$ be sets of real numbers.
  Write the negation of each of the following statements:
  \eparts{
  \item For every $a \in A$, it is true that $a^2 \in B$.
  \item For at least one $a \in A$, it is true that $a^2 \in B$.
  \item For every $a \in A$, it is true that $a^2 \notin B$.
  \item For at least one $a \notin A$, it is true that $a^2 \in B$.
  }
}
\sol{
  \dwhitman

  These are all basic logical negations using existential quantifiers:

  (a) There is an $a \in A$ where $a^2 \notin B$.
  
  (b) For every $a \in A$, $a^2 \notin B$.

  (c) There is an $a \in A$ where $a^2 \in B$.

  (d) For every $a \notin A$, $a^2 \notin B$.
}

\exercise{5}{
  Let $\cA$ be a nonempty collection of sets.
  Determine the truth of each of the following statements and of their converses:
  \eparts{
  \item $x \in \bigcup_{A \in \cA} A \imp x \in A$ for at least one $A \in \cA$.
  \item $x \in \bigcup_{A \in \cA} A \imp x \in A$ for every $A \in \cA$.
  \item $x \in \bigcap_{A \in \cA} A \imp x \in A$ for at least one $A \in \cA$.
  \item $x \in \bigcap_{A \in \cA} A \imp x \in A$ for every $A \in \cA$.
  }
}
\sol{
  \dwhitman

  (a) The statement on the right is the definition of the statement on the left so of course the implication and its converse are true.

  (b) The implication is generally false.
  \qproof{
    As a counterexample, consider $\cA = \braces{\braces{1}, \braces{2}}$.
    Then clearly $\bigcup_{A \in \cA} A = \braces{1, 2}$ so that $1 \in \bigcup_{A \in \cA} A$, but $1$ is not in $A$ for every $A \in \cA$ since $1 \notin \braces{2}$.
  }
  However, the converse \emph{is} true.
  \qproof{
    Suppose that $x \in A$ for every $A \in \cA$.
    Since $\cA$ is nonempty there is an $A_0 \in \cA$.
    Then $x \in A_0$ since $A_0 \in \cA$.
    Hence by definition $x \in \bigcup_{A \in \cA} A$ since $x \in A_0$ and $A_0 \in \cA$.
  }

  (c) The implication here is true.
  \qproof{
    Suppose that $x \in \bigcap_{A \in \cA} A$ so that by definition $x \in A$ for every $A \in \cA$.
    Since $\cA$ is nonempty there is an $A_0 \in \cA$ so that in particular $x \in A_0$.
    This shows the desired result since $A_0 \in \cA$.
  }
  The converse is not generally true.
  \qproof{
    As a counterexample consider $\cA = \braces{\braces{1,2}, \braces{2,3}}$.
    Then $1 \in \braces{1,2}$ and $\braces{1,2} \in \cA$, but $1 \notin \bigcap_{A \in \cA} A$ since clearly $\bigcap_{A \in \cA} A = \braces{2}$.
  }

  (d) The statement on the right is the definition of the statement on the left so of course the implication and its converse are true.
}

\exercise{6}{
  Write the contrapositive of each of the statements of Exercise~5.
}
\sol{
  \dwhitman

  Again these involve simple logical negations of both sides of the implications:

  (a) $x \notin A$ for every $A \in \cA \imp x \notin \bigcup_{A \in \cA} A$.

  (b) $x \notin A$ for at least one $A \in \cA \imp x \notin \bigcup_{A \in \cA} A$.

  (c) $x \notin A$ for every $A \in \cA \imp x \notin \bigcap_{A \in \cA} A$.

  (d) $x \notin A$ for at least one $A \in \cA \imp x \notin \bigcap_{A \in \cA} A$.
}

\exercise{7}{
  Given sets $A$, $B$, and $C$, express each of the following sets in terms of $A$, $B$, and $C$, using the symbols $\cup$, $\cap$ and $-$.
  \ali{
    D &= \braces{x \where x \in A \text{ and } (x \in B \text{ or } x \in C} \,, \\
    E &= \braces{x \where (x \in A \text{ and } x \in B) \text{ or } x \in C} \,, \\
    F &= \braces{x \in A \text{ and } (x \in B \imp x \in C)} \,.
  }
}
\sol{
  \dwhitman

  First, we obviously have
  \ali{
    D &= A \cap (B \cup C) \\
    E &= (A \cap B) \cup C \,,
  }
  noting that $D \neq E$ generally though they appear similar.
  Regarding $F$ we have the following sequence of logical equivalences:
  \ali{
    x \in F &\bic x \in A \land (x \in B \imp x \in C) \\
    &\bic x \in A \land (x \notin B \lor x \in C) \\
    &\bic x \in A \land \lnot(x \in B \land x \notin C) \\
    &\bic x \in A \land \lnot(x \in B - C) \\
    &\bic x \in A \land x \notin B - C \\
    &\bic x \in A - (B - C)
  }
  so that of course $F = A - (B - C)$.
}
